\section{Dirac delta and impulse response}
\label{diracdelta:section}

\LAtt{6.4}

\LO{
\item Understand the Dirac delta function as an impulse test for a differential equation, and
\item Use the Laplace transform to solve differential equations using the Dirac delta function.
}

% \sectionnotes{Verbatim from Lebl}

%\sectionnotes{1 or 1.5 lecture\EPref{, \S7.6 in \cite{EP}}\BDref{,
%\S6.5 in \cite{BD}}}

\subsection{Rectangular pulse}

Often in applications we study a physical system by putting in a short pulse 
and then seeing what the system does.  The resulting behavior is
often called \emph{\myindex{impulse response}}.
Let us see what we mean by a pulse.
The simplest kind of a pulse is a simple rectangular pulse defined by
\begin{equation*}
\varphi(t) = 
\begin{cases}
0 & \text{if } \; \phantom{a \leq {}} t < a , \\
M & \text{if } \; a \leq t < b , \\
0 & \text{if } \; b \leq t .
\end{cases}
\end{equation*}
See \figurevref{lt:sqpulse} for a graph.

%15 is number of lines, must be adjusted
\begin{mywrapfig}[15]{3.25in}
\capstart
\diffyincludegraphics{width=3in}{width=4.5in}{lt-sqpulse}
\caption{Sample square pulse with $a=0.5$, $b=1$ and $M = 2$.\label{lt:sqpulse}}
\end{mywrapfig}

Notice that
\begin{equation*}
\varphi(t) = M \bigl( u(t-a) - u(t-b) \bigr) ,
\end{equation*}
where $u(t)$ is the unit step function.

Let us take the Laplace transform of a square pulse,
\begin{equation*}
\begin{split}
{\mathcal{L}} \bigl\{ \varphi(t) \bigr\}
& =
{\mathcal{L}} \bigl\{ M \bigl( u(t-a) - u(t-b) \bigr)  \bigr\}
\\
& =
M
\frac{e^{-as} - e^{-bs}}{s} .
\end{split}
\end{equation*}

For simplicity we let $a=0$, and it is convenient
to set $M = \nicefrac{1}{b}$ to have
\begin{equation*}
\int_0^\infty \varphi(t) \,dt = 1 .
\end{equation*}
That is, to have the pulse have \myquote{unit mass.}
For such a pulse we
compute
\begin{equation*}
{\mathcal{L}} \bigl\{ \varphi(t) \bigr\}
=
{\mathcal{L}} \left\{ \frac{u(t) - u(t-b)}{b}  \right\}
=
\frac{1 - e^{-bs}}{bs} .
\end{equation*}
We generally want $b$ to be very small.  That is, we wish to have
the pulse be very short and very tall.  By letting $b$ go to zero we arrive
at the concept of the Dirac delta function.

\subsection{The delta function}

The \emph{\myindex{Dirac delta function}}\index{delta function}%
\footnote{Named after the English physicist and mathematician
\href{https://en.wikipedia.org/wiki/Paul_Dirac}{Paul Adrien Maurice Dirac}
(1902--1984).}
is not exactly a function; it is sometimes called a
\emph{\myindex{generalized function}}.  We
avoid unnecessary details and simply say that it is an object
that does not really make sense unless we integrate it.  The motivation is
that we would like a \myquote{function} $\delta(t)$
such that 
for any continuous function $f(t)$ we have
\begin{equation*}
\mybxbg{~~
\int_{-\infty}^\infty \delta(t) f(t) \,dt = f(0) .
~~}
\end{equation*}
The formula should hold if we integrate over any interval that contains 0,
not just $(-\infty,\infty)$.
So $\delta(t)$ is a \myquote{function} 
with all its \myquote{mass} at the single point $t=0$.  In other words, for any
interval $[c,d]$
\begin{equation*}
\int_c^d \delta(t) \,dt = 
\begin{cases}
1 & \text{if the interval $[c,d]$ contains 0, i.e.\ } c \leq 0 \leq d, \\
0 & \text{otherwise.}
\end{cases}
\end{equation*}
Unfortunately there is no such function in the classical sense.  You could
informally think that $\delta(t)$ is zero for $t\not=0$ and somehow
infinite at $t=0$.

A good way to think about $\delta(t)$ is as a limit of short pulses
whose integral is $1$.  For example, suppose that
we have a square pulse $\varphi(t)$ as above with $a=0$,
$M=\nicefrac{1}{b}$, that is $\varphi(t) = \frac{u(t) - u(t-b)}{b}$.
Compute
\begin{equation*}
\int_{-\infty}^\infty \varphi(t) f(t) \,dt =
\int_{-\infty}^\infty \frac{u(t) - u(t-b)}{b} f(t) \,dt =
\frac{1}{b} \int_{0}^b f(t) \,dt .
\end{equation*}
If $f(t)$ is continuous at $t=0$, then
for very small $b$, the function $f(t)$ is approximately equal to $f(0)$ on
the interval $[0,b]$.  We approximate the integral
\begin{equation*}
\frac{1}{b} \int_{0}^b f(t) \,dt \approx
\frac{1}{b} \int_{0}^b f(0) \,dt = f(0) .
\end{equation*}
Hence,
\begin{equation*}
\lim_{b\to 0}
\int_{-\infty}^\infty \varphi(t) f(t) \,dt =
\lim_{b\to 0}
\frac{1}{b} \int_{0}^b f(t) \,dt  = f(0) .
\end{equation*}

Let us therefore accept $\delta(t)$ as an object that is possible to
integrate.  We often want to shift $\delta$ to another point, for example
$\delta(t-a)$.  In that case we have
\begin{equation*}
\int_{-\infty}^\infty \delta(t-a) f(t) \,dt = f(a) .
\end{equation*}
Note that $\delta(a-t)$ is the same object as $\delta(t-a)$.
In other words, the convolution of $\delta(t)$ with $f(t)$ is again $f(t)$,
\begin{equation*}
(f * \delta) (t) = 
\int_{0}^t \delta(t-s) f(s) \,ds
= f(t) .
\end{equation*}

As we can integrate $\delta(t)$, let us compute its Laplace transform.
\begin{equation*}
\mybxbg{~~
{\mathcal{L}} \bigl\{ \delta(t-a) \bigr\}
=
\int_{0}^\infty e^{-st} \delta(t-a) \,dt = e^{-as} .
~~}
\end{equation*}
In particular,
\begin{equation*}
{\mathcal{L}} \bigl\{ \delta(t) \bigr\} = 1 .
\end{equation*}

\begin{remark}
Notice that the Laplace transform of $\delta(t-a)$ looks like
the Laplace transform of the derivative of the Heaviside function
$u(t-a)$, if we could differentiate the Heaviside function.
First notice
\begin{equation*}
{\mathcal{L}} \bigl\{ u(t-a) \bigr\} = \frac{e^{-as}}{s}.
\end{equation*}
To obtain what the Laplace transform of the derivative would be
we multiply by $s$, to obtain $e^{-as}$, which is the Laplace transform
of $\delta(t-a)$.
We see the same thing using integration,
\begin{equation*}
\int_0^t \delta(s-a)\,ds = u(t-a) .
\end{equation*}
So in a certain sense
\begin{equation*}
%mbxSTARTIGNORE
\text{``}
%mbxENDIGNORE
%mbxlatex \text{"}
\quad \frac{d}{dt} \Bigl[ u(t-a) \Bigr] = \delta(t-a) . \quad
%mbxSTARTIGNORE
\text{''}
%mbxENDIGNORE
%mbxlatex \text{"}
\end{equation*}
This line of reasoning allows us to talk about derivatives of functions with jump
discontinuities.
We can think of
the derivative of the Heaviside function $u(t-a)$ as being somehow infinite
at $a$, which is precisely our intuitive understanding of the delta
function.
\end{remark}

\begin{example}
Let us compute ${\mathcal{L}}^{-1} \left\{ \frac{s+1}{s} \right\}$.  So
far we have always looked at proper rational functions in the $s$ variable.
That is, the numerator was always of lower degree than the denominator.
Not so with $\frac{s+1}{s}$.
We write,
\begin{equation*}
{\mathcal{L}}^{-1} \left\{ \frac{s+1}{s} \right\}
=
{\mathcal{L}}^{-1} \left\{ 1 + \frac{1}{s} \right\}
=
{\mathcal{L}}^{-1} \{ 1 \}
+
{\mathcal{L}}^{-1} \left\{ \frac{1}{s} \right\}
=
\delta(t) + 1 .
\end{equation*}
The resulting object is a generalized
function and only makes sense when put underneath an integral.
\end{example}

\subsection{Impulse response}

As we said before, in the differential equation
$L x = f(t)$,
we think of $f(t)$ as input, and $x(t)$ as the output.  Often it is important
to find the response to an impulse, and then we use
the delta function in place of $f(t)$.
The solution to
\begin{equation*}
L x = \delta(t)
\end{equation*}
is called the
\emph{\myindex{impulse response}}.

\begin{example}
Solve (find the impulse response)
\begin{equation} \label{eq:lteximpulseresp}
x'' + \omega_0^2 x = \delta(t) , \quad x(0) = 0, \quad x'(0) = 0 .
\end{equation}
\end{example}

\begin{exampleSol}
We first apply the Laplace transform to the equation.  Denote
the transform of $x(t)$ by $X(s)$.
\begin{equation*}
s^2 X(s) + \omega_0^2 X(s) = 1 ,
\qquad \text{and so} \qquad
X(s) = \frac{1}{s^2+ \omega_0^2} .
\end{equation*}
Taking the inverse Laplace transform we obtain
\begin{equation*}
x(t) = 
\frac{\sin (\omega_0 t)}{\omega_0} .
\end{equation*}
\end{exampleSol}

Let us notice something about the example above.  We showed before that
when the input is $f(t)$, then the solution to $Lx = f(t)$
is given by
\begin{equation*}
x(t) = 
\int_0^t
f(\tau) 
\frac{\sin \bigl( \omega_0 (t-\tau) \bigr)}{\omega_0} ~ d\tau .
\end{equation*}
That is, the solution for an arbitrary input is given as
convolution with the impulse response.  Let us see why.
The key is to notice that for functions $x(t)$ and $f(t)$,
\begin{equation*}
(x * f)''(t) =
\frac{d^2}{dt^2}\left[
\int_0^t
f(\tau) 
x(t-\tau) ~ d\tau \right]
=
\int_0^t
f(\tau) 
x''(t-\tau) ~ d\tau
= (x'' * f)(t) .
\end{equation*}
We simply differentiate twice under the
integral\footnote{You should really think of the integral going over
$(-\infty,\infty)$ rather than over $[0,t]$ and simply assume that $f(t)$ and
$x(t)$ are continuous and zero for negative $t$.}, the details are
left as an exercise.
If we convolve the entire equation \eqref{eq:lteximpulseresp},
the left-hand side becomes
\begin{equation*}
(x'' + \omega_0^2 x) * f =
(x'' * f) + \omega_0^2 (x * f) =
(x * f)'' + \omega_0^2 (x * f) .
\end{equation*}
The right-hand side becomes
\begin{equation*}
(\delta * f)(t) = f(t).
\end{equation*}
Therefore $y(t) = (x * f)(t)$ is the solution to
\begin{equation*}
y'' + \omega_0^2 y = f(t) .
\end{equation*}
This procedure works in general for other linear
equations $Lx = f(t)$.  If you determine the impulse response,
you also know how to obtain the output $x(t)$ for any input $f(t)$
by simply convolving
the impulse response and the input $f(t)$.

\subsection{Three-point beam bending}
\index{three-point beam bending}

Let us give another quite different
example where delta functions turn up.  In this case 
representing point loads on a steel beam.  Suppose we have a beam
of length $L$, resting on two simple supports at the ends.  Let $x$ denote
the position on the beam, and let $y(x)$ denote the deflection of the beam in
the vertical direction.  The deflection $y(x)$ satisfies the
\emph{\myindex{Euler--Bernoulli equation}}%
\footnote{Named for the Swiss mathematicians
\href{https://en.wikipedia.org/wiki/Jacob_Bernoulli}{Jacob Bernoulli}
(1654--1705),
\href{https://en.wikipedia.org/wiki/Daniel_Bernoulli}{Daniel Bernoulli}
(1700--1782), the nephew of Jacob,
and
\href{https://en.wikipedia.org/wiki/Euler}{Leonhard Paul Euler}
(1707--1783).},
\begin{equation*}
EI \frac{d^4 y}{dx^4} = F(x) ,
\end{equation*}
where $E$ and $I$ are constants\footnote{$E$ is the elastic modulus and $I$
is the second moment of area.  Let us not worry about the details and simply
think of these as some given constants.} and
$F(x)$ is the force applied per unit length at position $x$.  The situation
we are interested in is when the force is applied at a single point as in
\figurevref{lt:beambendingfig}.

\begin{myfig}
\capstart
\inputpdft{beam-bending}
\caption{Three-point bending.\label{lt:beambendingfig}}
\end{myfig}

In this case the equation becomes
\begin{equation*}
EI \frac{d^4 y}{dx^4} = -F \delta(x-a) ,
\end{equation*}
where $x=a$ is the point where the mass is applied.  $F$ is the force
applied and the minus sign indicates that the force is downward, that is, in the
negative $y$ direction.  The end points of the
beam satisfy the conditions,
\begin{align*}
& y(0) = 0, \qquad y''(0) = 0, \\
& y(L) = 0, \qquad y''(L) = 0.
\end{align*}
% See \sectionref{sec:appeig} for further information about endpoint
% conditions applied to beams.


\begin{example} \label{lt:examplebeam}
Suppose that length of the beam is 2, and suppose that $EI=1$ for
simplicity.  Further suppose that the force $F=1$ is applied at $x=1$.
That is, we have the equation
\begin{equation*}
\frac{d^4 y}{dx^4} = -\delta(x-1) ,
\end{equation*}
and the endpoint conditions are
\begin{equation*}
y(0) = 0, \qquad y''(0) = 0, \qquad
y(2) = 0, \qquad y''(2) = 0.
\end{equation*}
\end{example}

\begin{exampleSol}
We could integrate, but using the Laplace transform
is even easier.
We apply the transform
in the $x$ variable rather than the $t$ variable.  Let us again denote the
transform of $y(x)$ as $Y(s)$.
\begin{equation*}
s^4Y(s)-s^3y(0)-s^2y'(0)-sy''(0)-y'''(0)
= -e^{-s}.
\end{equation*}
We notice that $y(0) = 0$ and $y''(0) = 0$.  Let us
call $C_1 = y'(0)$ and $C_2=y'''(0)$.
We solve for $Y(s)$,
\begin{equation*}
Y(s) = \frac{-e^{-s}}{s^4} + \frac{C_1}{s^2}+ \frac{C_2}{s^4} .
\end{equation*}
We take the inverse Laplace transform utilizing the 
second shifting property \eqref{ltd:sseq} to take the inverse of the first
term.
\begin{equation*}
y(x) = \frac{-{(x-1)}^3}{6} u(x-1) + C_1 x + \frac{C_2}{6} x^3 .
\end{equation*}
We still need to apply two of the endpoint conditions.  As the conditions
are at $x=2$ we can simply replace $u(x-1) = 1$ when taking
the derivatives.  Therefore,
\begin{equation*}
0 = y(2) = \frac{-{(2-1)}^3}{6} + C_1 (2) + \frac{C_2}{6} 2^3 =
\frac{-1}{6} + 2 C_1 + \frac{4}{3} C_2 ,
\end{equation*}
and
\begin{equation*}
0 = y''(2) = \frac{-3\cdot 2 \cdot (2-1)}{6} + \frac{C_2}{6} 3\cdot 2 \cdot 2
 = -1 + 2 C_2 .
\end{equation*}
Hence $C_2 = \frac{1}{2}$ and solving for $C_1$ using the first
equation we obtain
$C_1 = \frac{-1}{4}$.  Our solution for the beam deflection is
\begin{equation*}
y(x) = \frac{-{(x-1)}^3}{6} u(x-1) - \frac{x}{4} + \frac{x^3}{12} .
\end{equation*}
\end{exampleSol}

\subsection{Exercises}

\begin{exercise}
Solve (find the impulse response)
$x'' + x' + x = \delta(t)$, $x(0) = 0$, $x'(0)=0$.
\end{exercise}

\begin{exercise}
Solve (find the impulse response)
$x'' + 2 x' + x = \delta(t)$, $x(0) = 0$, $x'(0)=0$.
\end{exercise}

\begin{exercise}\ansMark%
Solve (find the impulse response)
$x'' = \delta(t)$, $x(0) = 0$, $x'(0)=0$.
\end{exercise}
\exsol{%
$x(t) = t$
}

\begin{exercise}\ansMark%
Solve (find the impulse response)
$x' + a x = \delta(t)$, $x(0) = 0$, $x'(0)=0$.
\end{exercise}
\exsol{%
$x(t) = e^{-at}$
}
%$sX+aX = 1$
%$X = \frac{1}{s+a}$

\begin{exercise}
A pulse can come later and can be bigger.
Solve 
$x'' + 4 x = 4\delta(t-1)$, $x(0) = 0$, $x'(0)=0$.
\end{exercise}

\begin{exercise}
Suppose that $f(t)$ and $g(t)$ are differentiable functions
and suppose that $f(t) = g(t) = 0$ for all $t \leq 0$.  Show that
\begin{equation*}
(f * g)'(t) = (f' * g)(t) = (f * g')(t) .
\end{equation*}
\end{exercise}

\begin{exercise}
Suppose that $L x = \delta(t)$, $x(0) = 0$, $x'(0) = 0$, has the solution
$x = e^{-t}$ for $t > 0$.  Find the solution to
$Lx = t^2$, $x(0) = 0$, $x'(0) = 0$ for $t > 0$.
\end{exercise}

\begin{exercise}\ansMark%
Suppose that $L x = \delta(t)$, $x(0) = 0$, $x'(0) = 0$, has the solution
$x(t) = \cos(t)$ for $t > 0$.  Find (in closed form) the solution to
$Lx = \sin(t)$, $x(0) = 0$, $x'(0) = 0$ for $t > 0$.
\end{exercise}
\exsol{%
$x(t) = (\cos * \sin)(t) = \frac{1}{2} t \sin(t)$
}

\begin{exercise}
Compute
${\mathcal{L}}^{-1} \left\{ \frac{s^2+s+1}{s^2} \right\}$.
\end{exercise}

\begin{exercise}\ansMark%
Compute
${\mathcal{L}}^{-1} \left\{ \frac{s^2}{s^2+1} \right\}$.
\end{exercise}
\exsol{%
$\delta(t) - \sin(t)$
}
%1 - 1/(s^2+1)

\begin{exercise}\ansMark%
Compute
${\mathcal{L}}^{-1} \left\{ \frac{3 s^2 e^{-s} + 2}{s^2} \right\}$.
\end{exercise}
\exsol{%
$3 \delta(t-1) + 2 t$
}

\begin{exercise}[challenging]
Solve \exampleref{lt:examplebeam} via integrating 4 times in the $x$ variable.
\end{exercise}

\begin{exercise}
Suppose we have a beam of length $1$ simply supported at the ends and
suppose that force $F=1$ is applied at $x=\frac{3}{4}$ in the downward
direction.  Suppose that $EI=1$ for simplicity.  Find the beam deflection
$y(x)$.
\end{exercise}

\setcounter{exercise}{100}

