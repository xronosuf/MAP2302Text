\section{Transforms of derivatives and ODEs}
\label{transformsofders:section}

\LAtt{6.2}

\LO{
\item Relate the Laplace transform of a derivative to the transform of the original function,
\item Use the Laplace transform to solve ordinary differential equations, and
\item Find the Laplace transform of an integral using the properties of the transform.
}

% \sectionnotes{Verbatim from Lebl}

%\sectionnotes{2 lectures\EPref{, \S7.2--7.3 in \cite{EP}}\BDref{,
% \S6.2 and \S6.3 in \cite{BD}}}

\subsection{Transforms of derivatives}

Let us see how the Laplace transform is used for differential equations.
First let us try to find the Laplace transform of a function that is a
derivative.  Suppose $g(t)$ is a differentiable function
of exponential order, that is, $\lvert g(t) \rvert \leq M e^{ct}$ for some
$M$ and $c$.  So $\mathcal{L} \bigl\{ g(t) \bigr\}$ exists, and what is more,
$\lim_{t\to\infty} e^{-st}g(t) = 0$ when $s > c$.  Then
\begin{equation*}
\mathcal{L} \bigl\{ g'(t) \bigr\}
=
\int_0^\infty
e^{-st}
g'(t) \,dt
=
\Bigl[e^{-st} g(t) \Bigr]_{t=0}^\infty
-
\int_0^\infty
(-s)\,
e^{-st}
g(t) \,dt
=
-g(0) + s \mathcal{L} \bigl\{ g(t) \bigr\} .
\end{equation*}
We repeat this procedure for higher derivatives.
The results are
listed in \tablevref{ltd:table}.  The procedure also works for piecewise
smooth functions, that is functions that are piecewise continuous with a
piecewise continuous derivative.
%The fact that the function is of
%exponential order is used to show that the limits appearing above 
%exist.  We will not worry much about this fact.

\begin{table}[h!t]
\mybeginframe
\capstart
\begin{center}
\begin{tabular}{@{}ll@{}}
\toprule
$f(t)$ & $\mathcal{L} \bigl\{ f(t) \bigr\} = F(s)$ \\
\midrule
$g'(t)$ & $sG(s)-g(0)$ \\[4pt]
$g''(t)$ & $s^2G(s)-sg(0)-g'(0)$ \\[4pt]
$g'''(t)$ & $s^3G(s)-s^2g(0)-sg'(0)-g''(0)$ \\[4pt]
\bottomrule
\end{tabular}
\end{center}
\caption{Laplace transforms of derivatives
($G(s) = \mathcal{L} \bigl\{ g(t) \bigr\}$
as usual).\label{ltd:table}}
\myendframe
\end{table}

\begin{exercise}
Verify \tablevref{ltd:table}.
\end{exercise}

\subsection{Solving ODEs with the Laplace transform}

Notice that the Laplace transform turns differentiation into
multiplication by $s$.  Let us see how to apply this fact to differential
equations.

\begin{example}
Solve the differential equation
\begin{equation*}
x''(t) + x(t) = \cos (2t), \quad x(0) = 0, \quad x'(0) = 1 .
\end{equation*}
using the Laplace transform.
\end{example}

\begin{exampleSol}
We will take the Laplace transform of both sides.
By $X(s)$ we will, as usual, denote the Laplace transform of
$x(t)$.
\begin{align*}
\mathcal{L} \bigl\{ x''(t) + x(t) \bigr\} & = \mathcal{L} \bigl\{ \cos (2t) \bigr\} , \\
s^2 X(s) -sx(0)-x'(0) + X(s) & = \frac{s}{s^2 + 4} .
\end{align*}
We plug in the initial conditions now---this makes the computations more
streamlined---to obtain
\begin{equation*}
s^2 X(s) -1 + X(s) = \frac{s}{s^2 + 4} .
\end{equation*}
We solve for $X(s)$,
\begin{equation*}
X(s) = \frac{s}{(s^2+1)(s^2 + 4)} + \frac{1}{s^2+1} .
\end{equation*}
We use partial fractions (exercise) to write
\begin{equation*}
X(s) =\frac{1}{3} \, \frac{s}{s^2+1} - 
\frac{1}{3}\, \frac{s}{s^2+4} + \frac{1}{s^2+1} .
\end{equation*}
Now take the inverse Laplace transform to obtain
\begin{equation*}
x(t) =\frac{1}{3}  \cos (t) -
\frac{1}{3} \cos (2t) + \sin (t) .
\end{equation*}
\end{exampleSol}

The procedure for linear constant coefficient equations is as follows.
We take an ordinary differential
equation in the time variable $t$.  We apply the Laplace transform
to transform the equation into an algebraic (non differential) equation in
the frequency domain.  All the $x(t)$, $x'(t)$, $x''(t)$, and so on, will
be converted to $X(s)$, $sX(s) - x(0)$, $s^2X(s) - sx(0) - x'(0)$,
and so on.
We solve the equation for $X(s)$.
Then taking the inverse transform, if possible, we find $x(t)$.

It should be noted that since not every function has a Laplace transform,
not every equation can be solved in this manner.  Also if the equation
is not a linear constant coefficient ODE\@,
then by applying the Laplace transform we may not
obtain an algebraic equation.

\subsection{Using the Heaviside function}

Before we move on to more general equations
than those we could solve before,
we want to consider the Heaviside function.  See \figurevref{lt:heavisidefig}
for the graph.
\begin{equation*}
u(t) =
\begin{cases}
0 & \text{if } \; t < 0 , \\ 
1 & \text{if } \; t \geq 0 .
\end{cases}
\end{equation*}

\begin{myfig}
\capstart
\diffyincludegraphics{width=3in}{width=4.5in}{lt-heaviside}
\caption{Plot of the Heaviside (unit step) function
$u(t)$.\label{lt:heavisidefig}}
\end{myfig}

This function is useful for
putting together functions, or cutting functions off.  Most commonly it is
used as $u(t-a)$ for some constant $a$.  This just shifts the graph to the
right by $a$.  That is, it is a function that is 0 when $t < a$ and 1
when $t \geq a$.  Suppose for example that $f(t)$ is a \myquote{signal} and
you started receiving the signal
$\sin t$ at time $t=\pi$.  The function $f(t)$ should then be defined as
\begin{equation*}
f(t) =
\begin{cases}
0 & \text{if } \; t < \pi , \\ 
\sin t & \text{if } \; t \geq \pi .
\end{cases}
\end{equation*}
Using the Heaviside function, $f(t)$ can
be written as
\begin{equation*}
f(t) = u(t - \pi) \, \sin t .
\end{equation*}
Similarly the step function that is 1 on the interval $[1,2)$ and zero
everywhere else can be written as
\begin{equation*}
u(t - 1) - u(t-2) .
\end{equation*}
The Heaviside function is useful to define functions defined piecewise.  If
you want to define $f(t)$ such that $f(t) = t$ when $t$ is in $[0,1]$,
$f(t) = -t+2$
when $t$ is in $[1,2]$, and $f(t) = 0$ otherwise, then you can use the expression
\begin{equation*}
f(t) = t \, \bigl( u(t) - u(t-1) \bigr) + 
(-t+2) \, \bigl( u(t - 1) - u(t-2) \bigr) .
\end{equation*}

Hence it is 
useful to know how the Heaviside function interacts with the Laplace
transform.  We have already seen that
\begin{equation*}
\mathcal{L} \bigl\{ u(t-a) \bigr\} = \frac{e^{-as}}{s} .
\end{equation*}
This can be generalized into a \emph{\myindex{shifting property}}
or \emph{\myindex{second shifting property}}.
\begin{equation} \label{ltd:sseq}
\mybxbg{~~
\mathcal{L} \bigl\{ f(t-a) \, u(t-a) \bigr\}
= e^{-as} \mathcal{L} \bigl\{ f(t) \bigr\} .
~~}
\end{equation}

\begin{example} \label{lt:rocketex}
Suppose that the forcing function is not periodic.  For example,
suppose that we had a mass-spring system
\begin{equation*}
x''(t) + x(t) = f(t) , \quad x(0) = 0, \quad x'(0) = 0,
\end{equation*}
where $f(t) = 1$ if $1 \leq t < 5$ and zero otherwise.  We could imagine a
mass-spring system, where a rocket is fired for 4 seconds starting at
$t=1$.  Or perhaps an RLC circuit, where the voltage is raised
at a constant rate for 4 seconds starting at $t=1$, and then held steady 
again
starting at $t=5$. Solve this differential equation using the Laplace transform.
\end{example}

\begin{exampleSol}
We can
write $f(t) = u(t-1) - u(t-5)$.  We transform the equation and we plug in
the initial conditions as before to obtain
\begin{equation*}
s^2 X(s) + X(s) = \frac{e^{-s}}{s} - \frac{e^{-5s}}{s} .
\end{equation*}
We solve for $X(s)$ to obtain
\begin{equation*}
X(s) = \frac{e^{-s}}{s(s^2+1)} - \frac{e^{-5s}}{s(s^2+1)} .
\end{equation*}
We leave it as an exercise to the reader to show that
\begin{equation*}
{\mathcal{L}}^{-1} \left\{ \frac{1}{s(s^2+1)} \right\}
= 1 - \cos t .
\end{equation*}
In other words 
$\mathcal{L} \{ 1 - \cos t  \} = 
\frac{1}{s(s^2+1)}$.  So using \eqref{ltd:sseq} we find
\begin{equation*}
{\mathcal{L}}^{-1} \left\{ \frac{e^{-s}}{s(s^2+1)} \right\}
=
{\mathcal{L}}^{-1} \left\{
e^{-s}
\mathcal{L} \{ 1 - \cos t \}
\right\}
=
\bigl( 1 - \cos (t-1) \bigr) \, u(t-1) .
\end{equation*}
Similarly
\begin{equation*}
{\mathcal{L}}^{-1} \left\{ \frac{e^{-5s}}{s(s^2+1)} \right\}
=
{\mathcal{L}}^{-1} \left\{
e^{-5s}
\mathcal{L} \{ 1 - \cos t \}
\right\}
=
\bigl( 1 - \cos (t-5) \bigr) \, u(t-5) .
\end{equation*}
Hence, the solution is
\begin{equation*}
x(t) = 
\bigl( 1 - \cos (t-1) \bigr) \, u(t-1) -
\bigl( 1 - \cos (t-5) \bigr) \, u(t-5) .
\end{equation*}
The plot of this solution is given in \figurevref{lt:heavisideexfig}.

\begin{myfig}
\capstart
\diffyincludegraphics{width=3in}{width=4.5in}{lt-heavisideex}
\caption{Plot of $x(t)$.\label{lt:heavisideexfig}}
\end{myfig}
\end{exampleSol}

\subsection{Transfer functions}

The Laplace transform leads to the following useful concept for studying the
steady state behavior of a linear system.  Consider an equation of the
form
\begin{equation*}
L x = f(t) ,
\end{equation*}
where $L$ is a linear constant coefficient differential operator.
Then $f(t)$ is usually thought of as input of the system and $x(t)$ is
thought of as the output of the system.  For example, for a mass-spring
system the input is the forcing function and the output is the behavior of the
mass.  We would like to have a convenient way to study the behavior of
the system for different inputs.

Let us suppose that
all the initial conditions are zero and take the Laplace transform
of the equation, we obtain the equation
\begin{equation*}
A(s) X(s) = F(s) .
\end{equation*}
Solving for the ratio $\nicefrac{X(s)}{F(s)}$ we obtain the so-called
\emph{\myindex{transfer function}}
$H(s) = \nicefrac{1}{A(s)}$,
that is,
\begin{equation*}
H(s) = \frac{X(s)}{F(s)} .
\end{equation*}
In other words, $X(s) = H(s) F(s)$.  We obtain an algebraic dependence of
the output of the system based on the input.  We can now easily study the
steady state behavior of the system given different inputs by simply
multiplying by the transfer function.

\begin{example}
Find the transfer function for the differnetial equation $x'' + \omega_0^2 x = f(t)$
(assuming the initial conditions are zero).
\end{example}

\begin{exampleSol}
First, we take the Laplace transform of the equation.
\begin{equation*}
s^2 X(s) + \omega_0^2 X(s) = F(s) .
\end{equation*}
Now we solve for the transfer function $\nicefrac{X(s)}{F(s)}$.
\begin{equation*}
H(s) = \frac{X(s)}{F(s)} = \frac{1}{s^2 + \omega_0^2} .
\end{equation*}

Let us see how to use the transfer function.  Suppose we have the constant input
$f(t) = 1$.  Hence $F(s) = \nicefrac{1}{s}$, and
\begin{equation*}
X(s) = H(s) F(s) = \frac{1}{s^2+\omega_0^2} \frac{1}{s} .
\end{equation*}
Taking the inverse Laplace transform of $X(s)$ we obtain
\begin{equation*}
x(t) = \frac{1-\cos(\omega_0 t)}{\omega_0^2} .
\end{equation*}
\end{exampleSol}

\subsection{Transforms of integrals}

A feature of Laplace transforms is that it is also able to easily deal
with integral equations.  That is, equations in which integrals rather than
derivatives of functions appear.  The basic property, which can be proved
by applying the definition and doing integration by parts, is 
\begin{equation*}
\mybxbg{~~
\mathcal{L} \left\{
\int_0^t f(\tau) ~ d\tau
\right\} = \frac{1}{s} \, F(s) .
~~}
\end{equation*}
It is sometimes useful (e.g.\ for computing the inverse transform) to write
this as
\begin{equation*}
\int_0^t f(\tau) ~ d\tau
=
{\mathcal{L}}^{-1} \left\{
\frac{1}{s} \, F(s) \right\} .
\end{equation*}

\begin{example}
To compute ${\mathcal{L}}^{-1} \left\{\frac{1}{s(s^2+1)}\right\}$ we could
proceed by applying this integration rule.  
\begin{equation*}
{\mathcal{L}}^{-1} \left\{
\frac{1}{s} \, \frac{1}{s^2+1} \right\} 
=
\int_0^t 
{\mathcal{L}}^{-1} \left\{
\frac{1}{s^2+1} \right\} ~ d\tau
=
\int_0^t 
\sin \tau ~ d\tau
=
1 - \cos t .
\end{equation*}
\end{example}

\begin{example}
An equation containing an integral of the unknown function is
called an \emph{\myindex{integral equation}}.  For example, take
\begin{equation*}
t^2 = \int_0^t e^{\tau} x(\tau) ~d\tau ,
\end{equation*}
where we wish to solve for $x(t)$.
We apply the Laplace transform and the shifting property to get
\begin{equation*}
\frac{2}{s^3} = \frac{1}{s} \, \mathcal{L} \bigl\{ e^{t} x(t) \bigr\}
=
\frac{1}{s} \, X(s-1) ,
\end{equation*}
where $X(s) = \mathcal{L} \bigl\{ x(t) \bigr\}$.  Thus
\begin{equation*}
X(s-1) =
\frac{2}{s^2} \qquad \text{or} \qquad
X(s) =
\frac{2}{{(s+1)}^2}.
\end{equation*}
We use the shifting property again
\begin{equation*}
x(t) = 2 e^{-t} t .
\end{equation*}
%More complicated integral
%equations can also be solved using convolution, about which we will
%learn in the next section.
\end{example}

\subsection{Exercises}

\begin{exercise}
Using the Heaviside function write down the piecewise function
that is 0 for $t < 0$, $t^2$ for $t$ in $[0,1]$ and $t$ for $t > 1$.
\end{exercise}

\begin{exercise}\ansMark%
Using the Heaviside function $u(t)$, write down the function
\begin{equation*}
f(t) =
\begin{cases}
0 & \text{if } \; \phantom{1 \leq {}} t < 1  , \\
t-1 & \text{if } \; 1 \leq t < 2 , \\
1 & \text{if } \; 2 \leq t .
\end{cases}
\end{equation*}
\end{exercise}
\exsol{%
$f(t) =
(t-1)\bigl(u(t-1) - u(t-2)\bigr) + 
u(t-2)$
}

\begin{exercise}
Using the Laplace transform solve
\begin{equation*}
m x'' + c x' + k x = 0 , \quad x(0) = a, \quad x'(0) = b ,
\end{equation*}
where $m > 0$, $c > 0$, $k > 0$, and
$c^2 - 4km > 0$ (system is overdamped).
\end{exercise}

\begin{exercise}
Using the Laplace transform solve
\begin{equation*}
m x'' + c x' + k x = 0 , \quad x(0) = a, \quad x'(0) = b ,
\end{equation*}
where $m > 0$, $c > 0$, $k > 0$, and
$c^2 - 4km < 0$ (system is underdamped).
\end{exercise}

\begin{exercise}
Using the Laplace transform solve
\begin{equation*}
m x'' + c x' + k x = 0 , \quad x(0) = a, \quad x'(0) = b ,
\end{equation*}
where $m > 0$, $c > 0$, $k > 0$, and
$c^2 = 4km$ (system is critically damped).
\end{exercise}

\begin{exercise}
Solve $x'' + x = u(t-1)$ for initial conditions $x(0) = 0$ and $x'(0) = 0$.
\end{exercise}

\begin{exercise}\ansMark%
Solve $x''-x = (t^2-1) u(t-1)$ for initial conditions $x(0)=1$, $x'(0) = 2$
using the Laplace transform.
\end{exercise}
\exsol{%
$x(t) = (2e^{t-1}-t^2-1) u(t-1) -\frac{1}{2}e^{-t}+\frac{3}{2}e^t$
}
 

\begin{exercise}
Show the differentiation of the transform property.  Suppose
$\mathcal{L} \bigl\{ f(t) \bigr\} = F(s)$, then show
\begin{equation*}
\mathcal{L} \bigl\{ -t f(t) \bigr\} = F'(s) .
\end{equation*}
Hint: Differentiate under the integral sign.
\end{exercise}

\begin{exercise}
Solve $x''' + x = t^3 u(t-1)$ for initial conditions $x(0) = 1$ and $x'(0) =
0$, $x''(0) = 0$.
\end{exercise}

\begin{exercise}
Show the second shifting property: 
$\mathcal{L} \bigl\{ f(t-a) \, u(t-a) \bigr\}
= e^{-as} \mathcal{L} \bigl\{ f(t) \bigr\}$.
\end{exercise}

\begin{exercise}
Let us think of the mass-spring system with a rocket from
\exampleref{lt:rocketex}.  We noticed that the solution kept oscillating
after the rocket stopped running.  The amplitude of the oscillation depends
on the time that the rocket was fired (for 4 seconds in the example).
\begin{tasks}
\task
Find a formula for the amplitude of the resulting oscillation
in terms of the amount of time the rocket is fired.
\task
 Is there
a nonzero time (if so what is it?)
for which the rocket fires and the resulting oscillation
has amplitude 0 (the mass is not moving)?
\end{tasks}
\end{exercise}

\begin{exercise}
Define
\begin{equation*}
f(t) =
\begin{cases}
{(t-1)}^2 & \text{if } \; 1 \leq t < 2, \\
3-t & \text{if } \; 2 \leq t < 3, \\
0 & \text{otherwise} .
\end{cases}
\end{equation*}
\begin{tasks}
\task Sketch the graph of $f(t)$.
\task Write down $f(t)$ using the Heaviside function.
\task Solve $x''+x=f(t)$, $x(0)=0$, $x'(0) = 0$ using Laplace transform.
\end{tasks}
\end{exercise}

\begin{exercise}
Find the transfer function for 
$m x'' + c x' + kx = f(t)$
(assuming the initial conditions are zero).
\end{exercise}

\begin{exercise}\ansMark%
Find the transfer function for 
$x' + x = f(t)$
(assuming the initial conditions are zero).
\end{exercise}
\exsol{%
$H(s) = \frac{1}{s+1}$
}

\setcounter{exercise}{100}

