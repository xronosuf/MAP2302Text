
\section{Convolution}
\label{convolution:section}

\LAtt{6.3}

\LO{
\item Compute the convolution of two functions, 
\item Relate the convolution of two functions to the Laplace transform of those functions, and 
\item Use the Laplace transform to solve more complicated differential equations involving products.
}

% \sectionnotes{Verbatim from Lebl}

% \sectionnotes{1 or 1.5 lectures\EPref{, \S7.2 in \cite{EP}}\BDref{,
% \S6.6 in \cite{BD}}}

\subsection{The convolution}

The Laplace transformation of a product is not the product
of the transforms.  All hope is not lost however.  We simply have to use
a different type of a \myquote{product.}
Take
two functions $f(t)$ and $g(t)$ defined for $t \geq 0$,
and define the \emph{\myindex{convolution}}%
\footnote{%
For those that have seen convolution before, you may have
seen it defined as
$(f * g)(t) =
\int_{-\infty}^\infty f(\tau) g(t-\tau) ~ d\tau$.  This definition
agrees with \eqref{ltc:convdef} if you define $f(t)$ and $g(t)$
to be zero for $t < 0$.
When discussing the Laplace transform the definition we gave is
sufficient.  Convolution does occur in many other applications, however,
where you may have to use the more general definition with infinities.
}
of $f(t)$ and $g(t)$ as
\begin{equation} \label{ltc:convdef}
\mybxbg{~~
(f * g)(t) \overset{\text{def}}{=}
\int_0^t f(\tau) g(t-\tau) ~ d\tau .
~~}
\end{equation}
As you can see, the convolution of two functions of $t$ is another function of $t$.


\begin{example}
Take $f(t) = e^t$ and $g(t) = t$ for $t \geq 0$.  Then 
\begin{equation*}
(f*g)(t)
=
\int_0^t e^\tau (t-\tau) ~ d\tau
=
e^t - t - 1 .
\end{equation*}
To solve the integral we
did one integration by parts.
\end{example}

\begin{example} \label{ltc:convsincosex}
Take $f(t) = \sin (\omega t)$ and $g(t) = \cos (\omega t)$ for $t \geq 0$.
Then 
\begin{equation*}
(f*g)(t)
=
\int_0^t  \sin ( \omega \tau ) \,
\cos \bigl( \omega (t-\tau) \bigr) ~ d\tau .
\end{equation*}
Apply the identity
\begin{equation*}
\cos (\theta) \sin (\psi) =
\frac{1}{2} \, \bigl( \sin (\theta + \psi) - \sin (\theta - \psi) \bigr) ,
\end{equation*}
to get
\begin{equation*}
\begin{split}
(f*g)(t)
& =
\int_0^t
\frac{1}{2} \, \bigl( \sin (\omega t) - \sin (\omega t - 2 \omega \tau
) \bigr) ~ d\tau
\\
& =
\left[ \frac{1}{2} \, \tau  \sin (\omega t) + \frac{1}{4\omega} \, \cos (2 \omega \tau -
\omega t) \right]_{\tau=0}^t
\\
& = \frac{1}{2} \, t \sin (\omega t) .
\end{split}
\end{equation*}
The formula holds only for $t \geq 0$.  The functions $f$, $g$,
and $f*g$ are undefined for $t < 0$.
\end{example}

Convolution has many properties that make it behave like a product.
Let $c$ be a constant and $f$, $g$, and $h$ be functions.  Then
\begin{align*}
& f * g = g * f , \\
& (c f) * g = f * (c g) = c (f*g) , \\
& ( f * g ) * h = f * ( g * h ) .
\end{align*}
The most interesting property for us is the following theorem.

\begin{theorem1}{}
Let $f(t)$ and $g(t)$ be of exponential order, then
\begin{equation*}
%\mybxbg{~~
\mathcal{L} \bigl\{ (f*g)(t) \bigr\}
=
\mathcal{L} \left\{ \int_0^t f(\tau) g(t-\tau) ~ d\tau \right\}
=
\mathcal{L} \bigl\{ f(t) \bigr\} \mathcal{L} \bigl\{ g(t) \bigr\} .
%~~}
\end{equation*}
\end{theorem1}

In other words, the Laplace transform of a convolution is the product
of the Laplace transforms.  The simplest way to use this result is in
reverse.

\begin{example}
Suppose we have the function of $s$
defined by
\begin{equation*}
\frac{1}{(s+1)s^2} = 
\frac{1}{s+1}\,
\frac{1}{s^2} .
\end{equation*}
We recognize the two entries of \tableref{ltd:table}.  That is,
\begin{equation*}
\mathcal{L}^{-1} 
\left\{
\frac{1}{s+1} \right\}
= e^{-t}
\qquad \text{and} \qquad
\mathcal{L}^{-1} 
\left\{
\frac{1}{s^2} \right\} 
= t.
\end{equation*}
Therefore,
\begin{equation*}
\mathcal{L}^{-1}
\left\{
\frac{1}{s+1}\,
\frac{1}{s^2} \right\}
=
\int_0^t
\tau e^{-(t-\tau)} ~d\tau
=
e^{-t}+t-1 .
\end{equation*}
The calculation of the integral involved an integration by parts.
\end{example}

\subsection{Solving ODEs}

The next example demonstrates the full power of the convolution and
the Laplace transform.  We can give the solution to
the forced oscillation problem for any forcing function as a definite
integral.

\begin{example}
Find the solution to
\begin{equation*}
x'' + \omega_0^2 x = f(t) , \quad x(0) = 0, \quad x'(0) = 0 ,
\end{equation*}
for an arbitrary function $f(t)$.
\end{example}

\begin{exampleSol}
We first apply the Laplace transform to the equation.  Denote
the transform of $x(t)$ by $X(s)$ and the transform of $f(t)$ by
$F(s)$ as usual.
\begin{equation*}
s^2 X(s) + \omega_0^2 X(s) = F(s) ,
\end{equation*}
or in other words
\begin{equation*}
X(s) = F(s) \frac{1}{s^2+ \omega_0^2} .
\end{equation*}
We know
\begin{equation*}
{\mathcal{L}}^{-1} \left\{
\frac{1}{s^2+ \omega_0^2}
\right\} = 
\frac{\sin (\omega_0 t)}{\omega_0} .
\end{equation*}
Therefore,
\begin{equation*}
x(t) = 
\int_0^t
f(\tau) 
\frac{\sin \bigl( \omega_0 (t-\tau) \bigr)}{\omega_0} ~ d\tau ,
\end{equation*}
or if we reverse the order
\begin{equation*}
x(t) = 
\int_0^t
\frac{\sin (\omega_0 \tau)}{\omega_0}
f(t-\tau) ~ d\tau .
\end{equation*}
\end{exampleSol}

Notice one more feature of this example.
We can now see how Laplace transform
handles \myindex{resonance}.  Suppose that $f(t) =
\cos (\omega_0 t)$.  Then
\begin{equation*}
x(t) = 
\int_0^t
\frac{\sin (\omega_0 \tau)}{\omega_0} \,
\cos \bigl( \omega_0 (t-\tau) \bigr) ~ d\tau
=
\frac{1}{\omega_0}
\int_0^t
\sin ( \omega_0 \tau ) \,
\cos \bigl(\omega_0 (t-\tau) \bigr) ~ d\tau .
\end{equation*}
We have computed the convolution of sine and cosine in
\exampleref{ltc:convsincosex}.  Hence
\begin{equation*}
x(t) =
\left(
\frac{1}{\omega_0}
\right) \,
\left(
\frac{1}{2} \,
t \,
\sin ( \omega_0 t )
\right)
=
\frac{1}{2 \omega_0} \,
t
\,
\sin ( \omega_0 t ).
\end{equation*}
Note the $t$ in front of the sine.  The solution, therefore, grows without
bound as $t$ gets large, meaning we get resonance.

Similarly,
we can solve any constant coefficient equation with an arbitrary forcing
function $f(t)$ as a definite integral using convolution.
A definite integral, rather than a closed form solution, is usually enough
for most practical purposes.  It is
not hard to numerically evaluate a definite integral.

\subsection{Volterra integral equation}

A common integral equation\index{integral equation}
is the \emph{\myindex{Volterra integral equation}}%
\footnote{Named for the Italian mathematician
\href{https://en.wikipedia.org/wiki/Vito_Volterra}{Vito Volterra}
(1860--1940).}
\begin{equation*}
x(t) = f(t) + \int_0^t g(t-\tau) x(\tau) ~ d\tau ,
\end{equation*}
where $f(t)$ and $g(t)$ are known functions and $x(t)$ is an unknown we
wish to solve for.
To find $x(t)$,
we apply the Laplace transform to the equation to obtain 
\begin{equation*}
X(s) = F(s) + G(s) X(s) ,
\end{equation*}
where $X(s)$, $F(s)$, and $G(s)$ are the Laplace transforms of $x(t)$, $f(t)$, and
$g(t)$ respectively.  We find
\begin{equation*}
X(s) = \frac{F(s)}{1-G(s)} .
\end{equation*}
To find $x(t)$ we now need to find the 
inverse Laplace transform of $X(s)$.

\begin{example}
Solve
\begin{equation*}
x(t) =  e^{-t} + \int_0^t \sinh(t-\tau) x(\tau) ~ d\tau .
\end{equation*}
\end{example}

\begin{exampleSol}
We apply Laplace transform to obtain
\begin{equation*}
X(s) = \frac{1}{s+1} + \frac{1}{s^2-1} X(s) ,
\end{equation*}
or
\begin{equation*}
X(s) = \frac{\frac{1}{s+1}}{1- \frac{1}{s^2-1}}
=
\frac{s-1}{s^2 - 2}
=
\frac{s}{s^2 - 2}
-
\frac{1}{s^2 - 2} .
\end{equation*}
It is not hard to apply \tablevref{lt:table} to find
\begin{equation*}
x(t) = \cosh \bigl( \sqrt{2} \, t \bigr) -
\frac{1}{\sqrt{2}} \sinh \bigl( \sqrt{2}\, t \bigr).
\end{equation*}
\end{exampleSol}

\subsection{Exercises}

\begin{exercise}
Let $f(t) = t^2$ for $t \geq 0$, and $g(t) = u(t-1)$.  Compute
$f * g$.
\end{exercise}

\begin{exercise}
Let $f(t) = t$ for $t \geq 0$, and $g(t) = \sin t $ for $t \geq 0$.  Compute
$f * g$.
\end{exercise}

\begin{exercise}\ansMark%
Let $f(t) = \cos t$ for $t \geq 0$, and $g(t) = e^{-t}$.  Compute
$f * g$.
\end{exercise}
\exsol{%
$\frac{1}{2}(\cos t + \sin t - e^{-t})$
}

\begin{exercise}
Find the solution to
\begin{equation*}
m x'' + c x' + k x = f(t) , \quad x(0) = 0, \quad x'(0) = 0 ,
\end{equation*}
for an arbitrary function $f(t)$, where $m > 0$, $c > 0$, $k > 0$,
and $c^2 - 4km > 0$ (system is overdamped).
Write the solution as a definite integral.
\end{exercise}

\begin{exercise}
Find the solution to
\begin{equation*}
m x'' + c x' + k x = f(t) , \quad x(0) = 0, \quad x'(0) = 0 ,
\end{equation*}
for an arbitrary function $f(t)$, where $m > 0$, $c > 0$, $k > 0$,
and $c^2 - 4km < 0$ (system is underdamped).
Write the solution as a definite integral.
\end{exercise}

\begin{exercise}
Find the solution to
\begin{equation*}
m x'' + c x' + k x = f(t) , \quad x(0) = 0, \quad x'(0) = 0 ,
\end{equation*}
for an arbitrary function $f(t)$, where $m > 0$, $c > 0$, $k > 0$,
and $c^2 = 4km$ (system is critically damped).
Write the solution as a definite integral.
\end{exercise}

\begin{exercise}
Solve
\begin{equation*}
x(t) =  e^{-t} + \int_0^t \cos(t-\tau) x(\tau) ~ d\tau .
\end{equation*}
\end{exercise}

\begin{exercise}\ansMark%
Solve $x''+x = \sin t$, $x(0) = 0$, $x'(0)=0$ using convolution.
\end{exercise}
\exsol{%
$\frac{1}{2}(\sin t - t \cos t)$
}

\begin{exercise}
Solve
\begin{equation*}
x(t) =  \cos t + \int_0^t \cos(t-\tau) x(\tau) ~ d\tau .
\end{equation*}
\end{exercise}

\begin{exercise}\ansMark%
Solve $x'''+x' = f(t)$, $x(0) = 0$, $x'(0)=0$, $x''(0)=0$ using convolution.
Write the result as a definite integral.
\end{exercise}
\exsol{%
$\int_0^t f(\tau) \bigl( 1 - \cos (t-\tau)\bigr)~ d\tau$
}

\begin{exercise}
Compute ${\mathcal{L}}^{-1} \left\{ \frac{s}{{(s^2+4)}^2} \right\}$ using
convolution.
\end{exercise}

\begin{exercise}\ansMark%
Compute ${\mathcal{L}}^{-1} \left\{ \frac{5}{s^4+s^2} \right\}$ using
convolution.
\end{exercise}
\exsol{%
$5t-5\sin t$
}

\begin{exercise}
Write down the solution to
$x''-2x=e^{-t^2}$, $x(0)=0$, $x'(0)=0$ as a
definite integral.  Hint: Do not try to compute the
Laplace transform of $e^{-t^2}$.
\end{exercise}

\setcounter{exercise}{100}




