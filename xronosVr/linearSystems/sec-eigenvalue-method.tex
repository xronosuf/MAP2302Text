\documentclass{ximera}
%\auor{Matthew Charnley and Jason Nowell}
\usepackage[margin=1.5cm]{geometry}
\usepackage{indentfirst}
\usepackage{sagetex}
\usepackage{lipsum}
\usepackage{amsmath}
\usepackage{mathrsfs}


%%% Random packages added without verifying what they are really doing - just to get initial compile to work.
\usepackage{tcolorbox}
\usepackage{hypcap}
\usepackage{booktabs}%% To get \toprule,\midrule,\bottomrule etc.
\usepackage{nicefrac}
\usepackage{caption}
\usepackage{units}

% This is my modified wrapfig that doesn't use intextsep
\usepackage{mywrapfig}
\usepackage{import}



%%% End to random added packages.


\graphicspath{
    {./figures/}
    {./../figures/}
    {./../../figures/}
}
\renewcommand{\log}{\ln}%%%%
\DeclareMathOperator{\arcsec}{arcsec}
%% New commands


%%%%%%%%%%%%%%%%%%%%
% New Conditionals %
%%%%%%%%%%%%%%%%%%%%


% referencing
\makeatletter
    \DeclareRobustCommand{\myvref}[2]{%
      \leavevmode%
      \begingroup
        \let\T@pageref\@pagerefstar
        \hyperref[{#2}]{%
	  #1~\ref*{#2}%
        }%
        \vpageref[\unskip]{#2}%
      \endgroup
    }%

    \DeclareRobustCommand{\myref}[2]{%
      \leavevmode%
      \begingroup
        \let\T@pageref\@pagerefstar
        \hyperref[{#2}]{%
	  #1~\ref*{#2}%
        }%
      \endgroup
    }%
\makeatother

\newcommand{\figurevref}[1]{\myvref{Figure}{#1}}
\newcommand{\figureref}[1]{\myref{Figure}{#1}}
\newcommand{\tablevref}[1]{\myvref{Table}{#1}}
\newcommand{\tableref}[1]{\myref{Table}{#1}}
\newcommand{\chapterref}[1]{\myref{chapter}{#1}}
\newcommand{\Chapterref}[1]{\myref{Chapter}{#1}}
\newcommand{\appendixref}[1]{\myref{appendix}{#1}}
\newcommand{\Appendixref}[1]{\myref{Appendix}{#1}}
\newcommand{\sectionref}[1]{\myref{\S}{#1}}
\newcommand{\subsectionref}[1]{\myref{subsection}{#1}}
\newcommand{\subsectionvref}[1]{\myvref{subsection}{#1}}
\newcommand{\exercisevref}[1]{\myvref{Exercise}{#1}}
\newcommand{\exerciseref}[1]{\myref{Exercise}{#1}}
\newcommand{\examplevref}[1]{\myvref{Example}{#1}}
\newcommand{\exampleref}[1]{\myref{Example}{#1}}
\newcommand{\thmvref}[1]{\myvref{Theorem}{#1}}
\newcommand{\thmref}[1]{\myref{Theorem}{#1}}


\renewcommand{\exampleref}[1]{ {\color{red} \bfseries Normally a reference to a previous example goes here.}}
\renewcommand{\figurevref}[1]{ {\color{red} \bfseries Normally a reference to a previous figure goes here.}}
\renewcommand{\tablevref}[1]{ {\color{red} \bfseries Normally a reference to a previous table goes here.}}
\renewcommand{\Appendixref}[1]{ {\color{red} \bfseries Normally a reference to an Appendix goes here.}}
\renewcommand{\exercisevref}[1]{ {\color{red} \bfseries Normally a reference to a previous exercise goes here.}}



\newcommand{\R}{\mathbb{R}}

%% Example Solution Env.
\def\beginSolclaim{\par\addvspace{\medskipamount}\noindent\hbox{\bf Solution:}\hspace{0.5em}\ignorespaces}
\def\endSolclaim{\par\addvspace{-1em}\hfill\rule{1em}{0.4pt}\hspace{-0.4pt}\rule{0.4pt}{1em}\par\addvspace{\medskipamount}}
\newenvironment{exampleSol}[1][]{\beginSolclaim}{\endSolclaim}

%% General figure formating from original book.
\newcommand{\mybeginframe}{%
\begin{tcolorbox}[colback=white,colframe=lightgray,left=5pt,right=5pt]%
}
\newcommand{\myendframe}{%
\end{tcolorbox}%
}

%%% Eventually return and fix this to make matlab code work correctly.
%% Define the matlab environment as another code environment
%\newenvironment{matlab}
%{% Begin Environment Code
%{ \centering \bfseries Matlab Code }
%\begin{code}
%}% End of Begin Environment Code
%{% Start of End Environment Code
%\end{code}
%}% End of End Environment Code


% this one should have a caption, first argument is the size
\newenvironment{mywrapfig}[2][]{
 \wrapfigure[#1]{r}{#2}
 \mybeginframe
 \centering
}{%
 \myendframe
 \endwrapfigure
}

% this one has no caption, first argument is size,
% the second argument is a larger size used for HTML (ignored by latex)
\newenvironment{mywrapfigsimp}[3][]{%
 \wrapfigure[#1]{r}{#2}%
 \centering%
}{%
 \endwrapfigure%
}
\newenvironment{myfig}
    {%
    \begin{figure}[h!t]
        \mybeginframe%
        \centering%
    }
    {%
        \myendframe
    \end{figure}%
    }


% graphics include
\newcommand{\diffyincludegraphics}[3]{\includegraphics[#1]{#3}}
\newcommand{\myincludegraphics}[3]{\includegraphics[#1]{#3}}
\newcommand{\inputpdft}[1]{\subimport*{../figures/}{#1.pdf_t}}


%% Not sure what these even do? They don't seem to actually work... fun!
%\newcommand{\mybxbg}[1]{\tcboxmath[colback=white,colframe=black,boxrule=0.5pt,top=1.5pt,bottom=1.5pt]{#1}}
%\newcommand{\mybxsm}[1]{\tcboxmath[colback=white,colframe=black,boxrule=0.5pt,left=0pt,right=0pt,top=0pt,bottom=0pt]{#1}}
\newcommand{\mybxsm}[1]{#1}
\newcommand{\mybxbg}[1]{#1}

%%% Something about tasks for practice/hw?
\usepackage{tasks}
\usepackage{footnote}
\makesavenoteenv{tasks}


%% For pdf only?
\newcommand{\diffypdfversion}[1]{#1}


%% Kill ``cite'' and go back later to fix it.
\renewcommand{\cite}[1]{}


%% Currently we can't really use index or its derivatives. So we are gonna kill them off.
\renewcommand{\index}[1]{}
\newcommand{\myindex}[1]{#1}






\title{Eigenvalue method}
\author{Matthew Charnley and Jason Nowell}


\outcome{Use the eigenvalue method to find straight-line solutions to constant-coefficient first order systems of ODE}
\outcome{Find general solutions to systems with real and distinct eigenvalues}
\outcome{Solve initial value problems from all of these cases once the general solution has been found.}


\begin{document}
\begin{abstract}
    We discuss Eigenvalue method
\end{abstract}
\maketitle

\label{eigenmethod:section}


In this section we will learn how to solve linear homogeneous constant coefficient systems of ODEs by the eigenvalue method. Suppose we have such a system
\begin{equation*}
    {\vec{x}}' = P\vec{x} ,
\end{equation*}
where $P$ is a constant square matrix. We wish to adapt the method for the single constant coefficient equation by trying the function $e^{\lambda t}$. However, $\vec{x}$ is a vector.  So we try $\vec{x} = \vec{v} e^{\lambda t}$, where $\vec{v}$ is an arbitrary constant vector.  We plug this $\vec{x}$ into the equation to get
\begin{equation*}
    \underbrace{\lambda \vec{v} e^{\lambda t}}_{{\vec{x}}'} = \underbrace{P\vec{v} e^{\lambda t}}_{P\vec{x}} .
\end{equation*}
We divide by $e^{\lambda t}$ and notice that we are looking for a scalar $\lambda$ and a vector $\vec{v}$ that satisfy the equation
\begin{equation*}
    \lambda \vec{v} = P\vec{v} .
\end{equation*}

This means that we are looking for an eigenvalue $\lambda$ with corresponding eigenvector $\vec{v}$ for the matrix $P$. When we can find these, we will get solutions to the original system of differential equations of the form 
\[ 
    \vec{x}(t) = \vec{v}e^{\lambda t}. 
\] 
We get the easiest route to solutions when the matrix $P$ has all real eigenvalues and the eigenvalues are all distinct, and can extend to deal with the complications that arise from complex and repeated eigenvalues.  

Another way to view these types of solutions are as ``straight-line solutions.'' A system of differential equations of the form 
\begin{equation*}
    {\vec{x}}' = P\vec{x} ,
\end{equation*}
is an autonomous system of differential equations, because there is no explicit dependence on $t$ on the right-hand side. When we solved autonomous equations in \sectionref{auteq:section}, we started by looking for equilibrium solutions and built up from there. In this particular case, we are looking for vectors $\vec{x}$ so that $P\vec{x} = 0$. As long as $P$ is invertible, the only vector that satisfies this is $\vec{x} = 0$. So, that's not super interesting, and doesn't really tell us too much about the solution to the problem.

\begin{mywrapfig}{3.25in}
    \capstart
    \myincludegraphics{width=2in}{width=3in}{eigen-straight-line}
    \caption{Position vector and possible direction vectors for straight line solutions.\label{posdir:source-eigfig}}
\end{mywrapfig}

The next more involved type of solution we could look for is a straight-line solution. The idea is that this solution will either move directly (in a straight-line) towards or away from the origin. In the first order autonomous equation case, all of our solutions did this; they either moved towards or away from these equilibrium solutions. This may not be the case for systems, but we can try to find them. If a solution is going to move directly towards or away from the origin, then the direction of change for the solution must be parallel to the position vector. In \figurevref{posdir:source-eigfig}, the vectors that point in the same or opposite direction of $\vec{x}$ will give rise to a straight-line solution, but vectors that do not point in this direction will give solutions that do not follow a straight-line through the origin. 

This criterion means that we need to have 
\begin{equation*}
    \vec{x}' = \lambda \vec{x}
\end{equation*}
for some constant $\lambda$. If this is the case, then we have 
\begin{equation*}
    P\vec{x} = \lambda \vec{x}
\end{equation*}
and this is the equation for eigenvalues and eigenvectors of $P$. We are back to the same type of solution that we found previously.

\subsection{The eigenvalue method with distinct real eigenvalues}

OK\@.  We have the system of equations
\begin{equation*}
    {\vec{x}}' = P\vec{x} .
\end{equation*}
We find the eigenvalues $\lambda_1$, $\lambda_2$, \ldots, $\lambda_n$ of the matrix $P$, and corresponding eigenvectors $\vec{v}_1$, $\vec{v}_2$, \ldots, $\vec{v}_n$. Now we notice that the functions $\vec{v}_1 e^{\lambda_1 t}$, $\vec{v}_2 e^{\lambda_2 t}$, \ldots, $\vec{v}_n e^{\lambda_n t}$ are solutions of the homogeneous system of equations and hence $\vec{x} = c_1 \vec{v}_1 e^{\lambda_1 t} + c_2 \vec{v}_2 e^{\lambda_2 t} + \cdots + c_n \vec{v}_n e^{\lambda_n t}$ is a solution by superposition.

\begin{theorem}{}
    Take ${\vec{x}}' = P\vec{x}$.  If $P$ is an $n \times n$ constant matrix that has $n$ distinct real eigenvalues $\lambda_1$, $\lambda_2$, \ldots, $\lambda_n$, then there exist $n$ linearly independent corresponding eigenvectors $\vec{v}_1$, $\vec{v}_2$, \ldots, $\vec{v}_n$, and the general solution to ${\vec{x}}' = P\vec{x}$ can be written as
    \begin{equation*}
        %\mybxbg{~~
        \vec{x} = c_1 \vec{v}_1 e^{\lambda_1 t} + c_2 \vec{v}_2 e^{\lambda_2 t} + \cdots + c_n \vec{v}_n e^{\lambda_n t} .
        %~~}
    \end{equation*}
\end{theorem}

The corresponding fundamental matrix solution is
\begin{equation*}
    X(t) = \bigl[\, \vec{v}_1 e^{\lambda_1 t} \quad \vec{v}_2 e^{\lambda_2 t} \quad \cdots \quad \vec{v}_n e^{\lambda_n t} \,\bigr].
\end{equation*}
That is, $X(t)$ is the matrix whose $j^{\text{th}}$ column is  $\vec{v}_j e^{\lambda_j t}$.

\begin{example}
    Consider the system
    \begin{equation*}
        {\vec{x}}' =
        \begin{bmatrix}
            2 & 1 & 1 \\
            1 & 2 & 0 \\
            0 & 0 & 2
        \end{bmatrix}
        \vec{x} .
    \end{equation*}
    Find the general solution.
\end{example}

\begin{exampleSol}
    Earlier, we found the eigenvalues are $1,2,3$.  We found the eigenvector $\left[ \begin{smallmatrix} 1 \\ 1 \\ 0 \end{smallmatrix} \right]$ for the eigenvalue 3.  Similarly we find the eigenvector  $\left[ \begin{smallmatrix} 1 \\ -1 \\ 0 \end{smallmatrix} \right]$ for the eigenvalue 1, and  $\left[ \begin{smallmatrix} 0 \\ 1 \\ -1 \end{smallmatrix} \right]$ for the eigenvalue 2 (exercise: check). Hence our general solution is
    \begin{equation*}
        \vec{x} = c_1
        \begin{bmatrix}
            1 \\ 
            -1 \\ 
            0
        \end{bmatrix}
        e^t + c_2
        \begin{bmatrix}
            0 \\ 
            1 \\ 
            -1
        \end{bmatrix}
        e^{2t} + c_3
        \begin{bmatrix}
            1 \\ 
            1 \\ 
            0
        \end{bmatrix}
        e^{3t} =
        \begin{bmatrix}
            c_1 e^t+c_3 e^{3t} \\ 
            -c_1 e^t + c_2 e^{2t} + c_3 e^{3t} \\ 
            - c_2 e^{2t}
        \end{bmatrix} .
    \end{equation*}
    In terms of a fundamental matrix solution,
    \begin{equation*}
        \vec{x} = X(t)\, \vec{c} =
        \begin{bmatrix}
            e^t & 0 & e^{3t} \\
            -e^t & e^{2t} & e^{3t} \\
            0 & -e^{2t} & 0
        \end{bmatrix}
        \begin{bmatrix}
            c_1 \\ 
            c_2 \\ 
            c_3
        \end{bmatrix} .
    \end{equation*}
\end{exampleSol}

\begin{exercise}
    Check that this $\vec{x}$ really solves the system.
\end{exercise}

Overall, the process for finding the solution for real and distinct eigenvalues is to first find the eigenvalues and eigenvectors of the matrix $P$. Once we have these, we get $n$ linearly independent solutions of the form $\vec{x}_i(t) = \vec{v}_ie^{\lambda_i t}$, so that the general solution is of the form
\[ 
    \vec{x}(t) = c_1\vec{v}_1e^{\lambda_1 t} + c_2\vec{v}_2e^{\lambda_2 t} + \cdots + c_n\vec{v}_ne^{\lambda_n t}. 
\] 
Then, if we need to solve for an initial condition, we figure out the coefficients $c_1$, $c_2$, ..., $c_n$ to satisfy this condition.

Note: If we write a single homogeneous linear constant coefficient $n^{\text{th}}$ order equation as a first order system (as we did in \sectionref{sec:introtosys}), then the eigenvalue equation

\begin{equation*}
    \det(P - \lambda I) = 0
\end{equation*}
is essentially the same as the characteristic equation we got in \sectionref{solinear:section} and \sectionref{sec:hol}. See the exercises for details about this.

\begin{example}
    Solve the initial value problem
    \begin{equation*}
        \vec{x}' = \begin{bmatrix} 0 & 4 \\ -3 & -7 \end{bmatrix}\vec{x} \qquad \vec{x}(0) = \begin{bmatrix} 1 \\ 1 \end{bmatrix}.
    \end{equation*}
\end{example}

\begin{exampleSol}
    Since we are in the case of a constant-coefficient linear system, we start by looking for the eigenvalues and eigenvectors of the coefficient matrix $P$. To do this, we compute
    \begin{equation*}
        \det(P - \lambda I) = (0-\lambda)(-7-\lambda) - (4)(-3) = \lambda^2 + 7\lambda + 12.
    \end{equation*}
    This polynomial factors as $(\lambda + 3)(\lambda + 4)$, and so the two eigenvalues are $\lambda_1 = -3$ and $\lambda_2 = -4$. 
    
    Next, we need to find the corresponding eigenvectors. For $\lambda = -3$, we get the matrix equation
    \begin{equation*}
        (P + 3I)\vec{v} = \begin{bmatrix} 3 & 4 \\ -3 & -4 \end{bmatrix} \vec{v} = \vec{0}. 
    \end{equation*}
    The two equations that you get here are redundant, which is $3v_1 + 4v_2 = 0$. One way to satisfy this is $v_1 = 4$, $v_2 = -3$, so that the eigenvector is $\left[\begin{smallmatrix} 4 \\ -3 \end{smallmatrix} \right]$. 
    
    For $\lambda = -4$, the matrix becomes
    \begin{equation*}
        (P + 4I)\vec{v} = \begin{bmatrix} 4 & 4 \\ -3 & -3 \end{bmatrix}\vec{v} = 0
    \end{equation*}
    so the eigenvector here is $\left[ \begin{smallmatrix} 1 \\ -1 \end{smallmatrix} \right]$. Therefore, the general solution to this differential equation, by superposition, is
    \begin{equation*}
        \vec{x}(t) = c_1 \begin{bmatrix} 4 \\ -3 \end{bmatrix}e^{-3t} + c_2 \begin{bmatrix} 1 \\ -1 \end{bmatrix}e^{-4t}.
    \end{equation*}
    
    Finally, we have to solve the initial value problem using the initial conditions. If we plug in $t=0$, we get the equation
    \begin{equation*}
        \vec{x}(0) = c_1 \begin{bmatrix} 4 \\ -3 \end{bmatrix} + c_2 \begin{bmatrix} 1 \\ -1 \end{bmatrix} = \begin{bmatrix} 1 \\ 1 \end{bmatrix}.
    \end{equation*}
    This results in needing to solve the system of equations
    \begin{equation*}
        4c_1 + c_2 = 1 \qquad -3c_1 - c_2 = 1.
    \end{equation*}
    These can be solved in any way, including row reduction. We will start by adding the two equations together, which gives $c_1 = 2$, and then the first equation implies that $c_2 = -7$. Therefore, the solution to the initial value problem is
    \begin{equation*}
        \vec{x}(t) = 2 \begin{bmatrix} 4 \\ -3 \end{bmatrix}e^{-3t}  - 7 \begin{bmatrix} 1 \\ -1 \end{bmatrix}e^{-4t} = \begin{bmatrix} 8e^{-3t} - 7e^{-4t} \\ -6e^{-3t} + 7e^{-4t} \end{bmatrix}.
    \end{equation*}
\end{exampleSol}

\subsection{Phase Portraits}

Now that we have these solutions, we want to get an idea for what they look like in the plane. We spent a lot of time in first order equations looking at direction fields, as well as phase lines for autonomous equations. We want to develop the same type of intuition for two-component systems in the plane, because much intuition can be obtained by studying this simple case. Suppose we use coordinates $(x,y)$ for the plane as usual, and suppose $P = \left[ \begin{smallmatrix} a & b \\ c & d \end{smallmatrix} \right]$ is a $2 \times 2$ matrix.  Consider the system

\begin{equation} \label{pln:eq}
    \begin{bmatrix} 
        x \\ 
        y 
    \end{bmatrix} 
    ' = P 
    \begin{bmatrix} 
        x \\ 
        y 
    \end{bmatrix} 
    \qquad \text{or} \qquad
    \begin{bmatrix} 
        x \\ 
        y 
    \end{bmatrix} ' =
    \begin{bmatrix} 
        a & b \\ 
        c & d 
    \end{bmatrix} 
    \begin{bmatrix} 
        x \\ 
        y 
    \end{bmatrix}.
\end{equation}
The system is autonomous (compare this section to \sectionref{auteq:section}) and so we can draw a vector field (see the end of \sectionref{sec:introtosys}). We will be able to visually tell what the vector field looks like and how the solutions behave, once we find the eigenvalues and eigenvectors of the matrix $P$. The goal is to be able to sketch what the different trajectories of the solutions look like for a variety of initial conditions, as well as classify the general type of picture that results depending on the matrix $P$. 
% For this section, we assume that $P$ has two eigenvalues and two corresponding eigenvectors.

\emph{Case 1.}  Suppose that the eigenvalues of $P$ are real and positive. We find two corresponding eigenvectors and plot them in the plane.  For example, take the matrix $\left[ \begin{smallmatrix} 1 & 1 \\ 0 & 2 \end{smallmatrix} \right]$. The eigenvalues are 1 and 2 and corresponding eigenvectors are $\left[ \begin{smallmatrix} 1 \\ 0 \end{smallmatrix} \right]$ and $\left[ \begin{smallmatrix} 1 \\ 1 \end{smallmatrix} \right]$.  See \figurevref{pln:source-eigfig}.

\begin{mywrapfig}{3.25in}
    \capstart
    \diffyincludegraphics{width=3in}{width=4.5in}{pln-source-eig}
    \caption{Eigenvectors of $P$.\label{pln:source-eigfig}}
\end{mywrapfig}

Suppose the point $(x,y)$ is on the line determined by an eigenvector $\vec{v}$ for an eigenvalue $\lambda$. That is, $\left[ \begin{smallmatrix} x \\ y \end{smallmatrix} \right] = \alpha \vec{v}$ for some scalar $\alpha$. Then 
\begin{equation*}
    \begin{bmatrix} 
        x \\ 
        y 
    \end{bmatrix} '
    = P 
    \begin{bmatrix} 
        x \\ 
        y 
    \end{bmatrix}
    = P ( \alpha \vec{v} ) =  \alpha ( P \vec{v} ) = \alpha \lambda \vec{v} .
\end{equation*}
The derivative is a multiple of $\vec{v}$ and hence points along the line determined by $\vec{v}$.  As $\lambda > 0$, the derivative points in the direction of $\vec{v}$ when $\alpha$ is positive and in the opposite direction when $\alpha$ is negative.  Let us draw the lines determined by the eigenvectors, and let us draw arrows on the lines to indicate the directions. See \figurevref{pln:source-eig-arrfig}.

We fill in the rest of the arrows for the vector field and we also draw a few solutions.  See \figurevref{pln:source-fullfig}. The picture looks like a source with arrows coming out from the origin. Hence we call this type of picture a \emph{\myindex{source}} or sometimes an \emph{\myindex{unstable node}}. Notice the two eigenvectors are drawn on the entire vector field figure with arrows, and the straight-line solutions follow them.  

\begin{myfig}
    \parbox[t]{3.0in}{
    \capstart
    \diffyincludegraphics{width=3in}{width=4.5in}{pln-source-eig-arr}
    \caption{Eigenvectors of $P$ with directions.\label{pln:source-eig-arrfig}}}
    \quad \parbox[t]{3.0in}{
    \capstart
    \diffyincludegraphics{width=3in}{width=4.5in}{pln-source-full}
    \caption{Example source vector field with eigenvectors and solutions.\label{pln:source-fullfig}}}
\end{myfig}

\emph{Case 2.} Suppose both eigenvalues are negative.  For example, take the negation of the matrix in case 1, $\left[ \begin{smallmatrix} -1 & -1 \\ 0 & -2 \end{smallmatrix} \right]$. The eigenvalues are $-1$ and $-2$ and corresponding eigenvectors are the same, $\left[ \begin{smallmatrix} 1 \\ 0 \end{smallmatrix} \right]$ and $\left[ \begin{smallmatrix} 1 \\ 1 \end{smallmatrix} \right]$.  The calculation and the picture are almost the same.  The only difference is that the eigenvalues are negative and hence all arrows are reversed.  We get the picture in \figurevref{pln:sink-fullfig}.  We call this kind of picture a \emph{\myindex{sink}} or a \emph{\myindex{asymptotically stable node}}.

\begin{myfig}
    \parbox[t]{3.0in}{
    \capstart
    \diffyincludegraphics{width=3in}{width=4.5in}{pln-sink-full}
    \caption{Example sink vector field with eigenvectors and solutions.\label{pln:sink-fullfig}}}
    \quad
    \parbox[t]{3.0in}{
    \capstart
    \diffyincludegraphics{width=3in}{width=4.5in}{pln-saddle-full}
    \caption{Example saddle vector field with eigenvectors and solutions.\label{pln:saddle-fullfig}}}
\end{myfig}

\emph{Case 3.} Suppose one eigenvalue is positive and one is negative. For example the matrix $\left[ \begin{smallmatrix} 1 & 1 \\ 0 & -2 \end{smallmatrix} \right]$. The eigenvalues are 1 and $-2$ and corresponding eigenvectors are $\left[ \begin{smallmatrix} 1 \\ 0 \end{smallmatrix} \right]$ and $\left[ \begin{smallmatrix} 1 \\ -3 \end{smallmatrix} \right]$.  We reverse the arrows on one line (corresponding to the negative eigenvalue) and we obtain the picture in \figurevref{pln:saddle-fullfig}.  We call this picture a \emph{\myindex{saddle point}}.



\end{document}












