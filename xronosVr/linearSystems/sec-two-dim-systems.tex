\documentclass{ximera}
%\auor{Matthew Charnley and Jason Nowell}
\usepackage[margin=1.5cm]{geometry}
\usepackage{indentfirst}
\usepackage{sagetex}
\usepackage{lipsum}
\usepackage{amsmath}
\usepackage{mathrsfs}


%%% Random packages added without verifying what they are really doing - just to get initial compile to work.
\usepackage{tcolorbox}
\usepackage{hypcap}
\usepackage{booktabs}%% To get \toprule,\midrule,\bottomrule etc.
\usepackage{nicefrac}
\usepackage{caption}
\usepackage{units}

% This is my modified wrapfig that doesn't use intextsep
\usepackage{mywrapfig}
\usepackage{import}



%%% End to random added packages.


\graphicspath{
    {./figures/}
    {./../figures/}
    {./../../figures/}
}
\renewcommand{\log}{\ln}%%%%
\DeclareMathOperator{\arcsec}{arcsec}
%% New commands


%%%%%%%%%%%%%%%%%%%%
% New Conditionals %
%%%%%%%%%%%%%%%%%%%%


% referencing
\makeatletter
    \DeclareRobustCommand{\myvref}[2]{%
      \leavevmode%
      \begingroup
        \let\T@pageref\@pagerefstar
        \hyperref[{#2}]{%
	  #1~\ref*{#2}%
        }%
        \vpageref[\unskip]{#2}%
      \endgroup
    }%

    \DeclareRobustCommand{\myref}[2]{%
      \leavevmode%
      \begingroup
        \let\T@pageref\@pagerefstar
        \hyperref[{#2}]{%
	  #1~\ref*{#2}%
        }%
      \endgroup
    }%
\makeatother

\newcommand{\figurevref}[1]{\myvref{Figure}{#1}}
\newcommand{\figureref}[1]{\myref{Figure}{#1}}
\newcommand{\tablevref}[1]{\myvref{Table}{#1}}
\newcommand{\tableref}[1]{\myref{Table}{#1}}
\newcommand{\chapterref}[1]{\myref{chapter}{#1}}
\newcommand{\Chapterref}[1]{\myref{Chapter}{#1}}
\newcommand{\appendixref}[1]{\myref{appendix}{#1}}
\newcommand{\Appendixref}[1]{\myref{Appendix}{#1}}
\newcommand{\sectionref}[1]{\myref{\S}{#1}}
\newcommand{\subsectionref}[1]{\myref{subsection}{#1}}
\newcommand{\subsectionvref}[1]{\myvref{subsection}{#1}}
\newcommand{\exercisevref}[1]{\myvref{Exercise}{#1}}
\newcommand{\exerciseref}[1]{\myref{Exercise}{#1}}
\newcommand{\examplevref}[1]{\myvref{Example}{#1}}
\newcommand{\exampleref}[1]{\myref{Example}{#1}}
\newcommand{\thmvref}[1]{\myvref{Theorem}{#1}}
\newcommand{\thmref}[1]{\myref{Theorem}{#1}}


\renewcommand{\exampleref}[1]{ {\color{red} \bfseries Normally a reference to a previous example goes here.}}
\renewcommand{\figurevref}[1]{ {\color{red} \bfseries Normally a reference to a previous figure goes here.}}
\renewcommand{\tablevref}[1]{ {\color{red} \bfseries Normally a reference to a previous table goes here.}}
\renewcommand{\Appendixref}[1]{ {\color{red} \bfseries Normally a reference to an Appendix goes here.}}
\renewcommand{\exercisevref}[1]{ {\color{red} \bfseries Normally a reference to a previous exercise goes here.}}



\newcommand{\R}{\mathbb{R}}

%% Example Solution Env.
\def\beginSolclaim{\par\addvspace{\medskipamount}\noindent\hbox{\bf Solution:}\hspace{0.5em}\ignorespaces}
\def\endSolclaim{\par\addvspace{-1em}\hfill\rule{1em}{0.4pt}\hspace{-0.4pt}\rule{0.4pt}{1em}\par\addvspace{\medskipamount}}
\newenvironment{exampleSol}[1][]{\beginSolclaim}{\endSolclaim}

%% General figure formating from original book.
\newcommand{\mybeginframe}{%
\begin{tcolorbox}[colback=white,colframe=lightgray,left=5pt,right=5pt]%
}
\newcommand{\myendframe}{%
\end{tcolorbox}%
}

%%% Eventually return and fix this to make matlab code work correctly.
%% Define the matlab environment as another code environment
%\newenvironment{matlab}
%{% Begin Environment Code
%{ \centering \bfseries Matlab Code }
%\begin{code}
%}% End of Begin Environment Code
%{% Start of End Environment Code
%\end{code}
%}% End of End Environment Code


% this one should have a caption, first argument is the size
\newenvironment{mywrapfig}[2][]{
 \wrapfigure[#1]{r}{#2}
 \mybeginframe
 \centering
}{%
 \myendframe
 \endwrapfigure
}

% this one has no caption, first argument is size,
% the second argument is a larger size used for HTML (ignored by latex)
\newenvironment{mywrapfigsimp}[3][]{%
 \wrapfigure[#1]{r}{#2}%
 \centering%
}{%
 \endwrapfigure%
}
\newenvironment{myfig}
    {%
    \begin{figure}[h!t]
        \mybeginframe%
        \centering%
    }
    {%
        \myendframe
    \end{figure}%
    }


% graphics include
\newcommand{\diffyincludegraphics}[3]{\includegraphics[#1]{#3}}
\newcommand{\myincludegraphics}[3]{\includegraphics[#1]{#3}}
\newcommand{\inputpdft}[1]{\subimport*{../figures/}{#1.pdf_t}}


%% Not sure what these even do? They don't seem to actually work... fun!
%\newcommand{\mybxbg}[1]{\tcboxmath[colback=white,colframe=black,boxrule=0.5pt,top=1.5pt,bottom=1.5pt]{#1}}
%\newcommand{\mybxsm}[1]{\tcboxmath[colback=white,colframe=black,boxrule=0.5pt,left=0pt,right=0pt,top=0pt,bottom=0pt]{#1}}
\newcommand{\mybxsm}[1]{#1}
\newcommand{\mybxbg}[1]{#1}

%%% Something about tasks for practice/hw?
\usepackage{tasks}
\usepackage{footnote}
\makesavenoteenv{tasks}


%% For pdf only?
\newcommand{\diffypdfversion}[1]{#1}


%% Kill ``cite'' and go back later to fix it.
\renewcommand{\cite}[1]{}


%% Currently we can't really use index or its derivatives. So we are gonna kill them off.
\renewcommand{\index}[1]{}
\newcommand{\myindex}[1]{#1}






\title{Two-dimensional systems and their vector fields}
\author{Matthew Charnley and Jason Nowell}


\outcome{Visualize and sketch the behavior of a two dimensional system based on the eigenvalues and eigenvectors.}


\begin{document}
\begin{abstract}
    We discuss Two-dimensional systems and their vector fields
\end{abstract}
\maketitle

\label{sec:twodimaut}


In the last three sections, we looked at the different options for two-component constant-coefficient systems. We want to determine a nice way to put all of this together. We summarize the behavior of linear homogeneous two-dimensional systems given by a nonsingular matrix in \tableref{pln:behtab}. Systems where one of the eigenvalues is zero (the matrix is singular) come up in practice from time to time, see \examplevref{sintro:closedbrine-example}, and the pictures are somewhat different (simpler in a way).  See the exercises.

\begin{table}[h!t]
    \mybeginframe
    \capstart
    \begin{center}
        \begin{tabular}{@{}ll@{}}
            \toprule
            Eigenvalues & Behavior \\
            \midrule
            real and both positive & source / unstable node \\
            real and both negative & sink / asymptotically stable node \\
            real and opposite signs & saddle \\
            purely imaginary & center point / ellipses \\
            complex with positive real part & spiral source \\
            complex with negative real part & spiral sink \\
            repeated with two eigenvectors & proper node  (asympt. stable or unstable) \\
            repeated with one eigenvector & improper node (asympt. stable or unstable) \\
            \bottomrule
        \end{tabular}
    \end{center}
    \caption{Summary of behavior of linear homogeneous two-dimensional systems.\label{pln:behtab}}
    \myendframe
\end{table}

The sketches of all of these different behaviors and phase portraits can be found in their respective sections. Make sure that you understand the terminology, general behavior, and sketches for each of these different cases. 

\subsection{Trace-Determinant Analysis}

One other way to interpret and analyze this information is using the trace and determinant of the matrix. Recall from \sectionref{sec:kernel} that the \myindex{trace} of a matrix is the sum of the diagonal entries of the matrix and the \myindex{determinant} of the matrix is computed from the entries and is a way to determine invertibility of the matrix. If we take a generic $2 \times 2$ matrix and find the characteristic polynomial, we get that for
\begin{equation*}
    A = \begin{bmatrix} a & b \\ c & d \end{bmatrix},
\end{equation*}
the characteristic polynomial is 
\begin{equation*}
    \det(A - \lambda I) = (a - \lambda)(d - \lambda) - (b)(c) = \lambda^2 - (a+d)\lambda + (ad-bc).
\end{equation*}
Since the trace of the matrix is $a+d$ and the determinant is $ad-bc$, we can rewrite this polynomial as
\begin{equation*}
    \lambda^2 - T\lambda + D = 0,
\end{equation*}
which also means that we can characterize the eigenvalues of the matrix in terms of the trace and determinant. We get that the eigenvalues are
\begin{equation}
    \lambda = \frac{T \pm \sqrt{T^2 - 4D}}{2}.
    \label{eq:eigenTD}
\end{equation}

There are a few important facts we can learn from this equation. 
\begin{enumerate}
    \item A lot depends on the value of $T^2 - 4D$. If $T^2 - 4D > 0,$ then we will have two real distinct eigenvalues. If $T^2 - 4D = 0,$ then there is a single repeated eigenvalue, and if $T^2 - 4D < 0$, we have complex eigenvalues.
    \item If $D < 0$, then $T^2 - 4D > T^2,$ which means that $\sqrt{T^2 - 4D} > |T|$. If we put this into \eqref{eq:eigenTD}, this will mean that the term that is after the $\pm$ will be larger than $T$ in absolute value. Therefore, the two eigenvalues will be real and have opposite signs.
    \item If $D \geq 0$, then the sign of the eigenvalues, or the sign of the real part in the complex case, is dictated by the sign of $T$. If $D \geq 0$, then $T^2 - 4D \leq T^2$, so that the part under the square root in \eqref{eq:eigenTD} is always smaller in absolute value than $T$. Thus, both the plus and minus version will have values that are the same sign as $T$. If the expression is complex, then the real part is exactly $T/2$, which is the same sign as $T$. 
\end{enumerate}

All of this means we can make a new table characterizing the eigenvalues and how they are connected to the trace and determinant. 

\begin{table}[h!t]
    \mybeginframe
    \capstart
    \begin{center}
        \begin{tabular}{@{}ll@{}}
            \toprule
            Eigenvalues & Trace and Determinant Classification \\
            \midrule
            real and both positive & $T > 0$, $D > 0$, $T^2 - 4D > 0$  \\
            real and both negative &  $T < 0$, $D > 0$, $T^2 - 4D > 0$\\
            real and opposite signs & $D < 0$ \\
            purely imaginary &  $T = 0$, $D > 0$\\
            complex with positive real part & $T > 0$, $T^2 - 4D < 0$  \\
            complex with negative real part &  $T < 0$, $T^2 - 4D < 0$ \\
            repeated & $T^2 - 4D = 0$ \\
            \bottomrule
        \end{tabular}
    \end{center}
    \caption{Summary of behavior of linear homogeneous two-dimensional systems.\label{pln:behtabTD}}
    \myendframe
\end{table}

Since these are all based on the relation between $T$ and $D$, we can also combine all of this into a figure to summarize the details. In \figurevref{fig:TDPlaneAnalysis}, $T$ is on the horizontal axis and $D$ is the vertical axis. The graph drawn is $D = \nicefrac{T^2}{4}$, which is the important criteria that shows up in the table. 

\begin{myfig}
    \capstart
    \myincludegraphics{width=5.5in}{width=5in}{eigTraceDet}
    \caption{Trace-Determinant plane for analysis of two-component linear systems. \label{fig:TDPlaneAnalysis}}
\end{myfig}

\figurevref{fig:TDPlaneAnalysis} can be used to determine the behavior of a two-component system without actually needing to solve the differential equation. The point is that the signs and type of the eigenvalues determine the structure of the solution, and we can determine the important qualities of these using just the trace and determinant of a matrix.

\begin{example}
    Use Trace-Determinant analysis to determine the overall behavior of the system
    \begin{equation*}
        {\vec{x}}' = \begin{bmatrix} 1 & 4 \\ -2 & 3 \end{bmatrix}\vec{x}.
    \end{equation*}
\end{example}

\begin{exampleSol}
    From the matrix, we can see that the trace is $1 + 3  = 4$ and the determinant is $(1)(3) - (4)(-2) = 11$. We see that $D > 0$ with $T^2 = 16$ and $4D = 44 > 16$. Therefore, we have $4D > T^2$, so we are above the curve on the graph, and so have a spiral. Since $T > 0$, this will be a spiral source. 
    
    \textbf{Note:} If you wanted to get a general solution or sketch a phase portrait for this differential equation, you would need to actually solve it out for that; you can not get enough information just from this image to sketch a proper phase portrait.
\end{exampleSol}

\begin{exercise}
    Compute the eigenvalues for the system above, find the general solution, and verify that this is a spiral source. The numbers here will not work out great, so having the quick analysis that it is a spiral source is nice. 
\end{exercise} 
 

\end{document}
