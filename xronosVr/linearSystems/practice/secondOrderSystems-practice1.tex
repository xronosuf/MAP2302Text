\documentclass{ximera}

\title{Practice for Second Order Systems}

%\auor{Matthew Charnley and Jason Nowell}
\usepackage[margin=1.5cm]{geometry}
\usepackage{indentfirst}
\usepackage{sagetex}
\usepackage{lipsum}
\usepackage{amsmath}
\usepackage{mathrsfs}
\usepackage{tikz}
\usetikzlibrary{matrix}

%%% Random packages added without verifying what they are really doing - just to get initial compile to work.
\usepackage{tcolorbox}
\usepackage{hypcap}
\usepackage{booktabs}%% To get \toprule,\midrule,\bottomrule etc.
\usepackage{caption}
\usepackage{units}
\usepackage{multicol}
\usepackage{hhline}


% This is my modified wrapfig that doesn't use intextsep
\usepackage{mywrapfig}
\usepackage{import}



%%% End to random added packages.


\graphicspath{
    {./}
    {./figures/}
    {./../figures/}
    {./../../figures/}
}
\renewcommand{\log}{\ln}%%%%
\DeclareMathOperator{\arcsec}{arcsec}
%% New commands


%%%%%%%%%%%%%%%%%%%%
% New Conditionals %
%%%%%%%%%%%%%%%%%%%%


% referencing
\makeatletter
    \DeclareRobustCommand{\myvref}[2]{%
      \leavevmode%
      \begingroup
        \let\T@pageref\@pagerefstar
        \hyperref[{#2}]{%
	  #1~\ref*{#2}%
        }%
        \vpageref[\unskip]{#2}%
      \endgroup
    }%

    \DeclareRobustCommand{\myref}[2]{%
      \leavevmode%
      \begingroup
        \let\T@pageref\@pagerefstar
        \hyperref[{#2}]{%
	  #1~\ref*{#2}%
        }%
      \endgroup
    }%
\makeatother

\newcommand{\figurevref}[1]{\myvref{Figure}{#1}}
\newcommand{\figureref}[1]{\myref{Figure}{#1}}
\newcommand{\tablevref}[1]{\myvref{Table}{#1}}
\newcommand{\tableref}[1]{\myref{Table}{#1}}
\newcommand{\chapterref}[1]{\myref{chapter}{#1}}
\newcommand{\Chapterref}[1]{\myref{Chapter}{#1}}
\newcommand{\appendixref}[1]{\myref{appendix}{#1}}
\newcommand{\Appendixref}[1]{\myref{Appendix}{#1}}
\newcommand{\sectionref}[1]{\myref{\S}{#1}}
\newcommand{\subsectionref}[1]{\myref{subsection}{#1}}
\newcommand{\subsectionvref}[1]{\myvref{subsection}{#1}}
\newcommand{\exercisevref}[1]{\myvref{Exercise}{#1}}
\newcommand{\exerciseref}[1]{\myref{Exercise}{#1}}
\newcommand{\examplevref}[1]{\myvref{Example}{#1}}
\newcommand{\exampleref}[1]{\myref{Example}{#1}}
\newcommand{\thmvref}[1]{\myvref{Theorem}{#1}}
\newcommand{\thmref}[1]{\myref{Theorem}{#1}}


\renewcommand{\exampleref}[1]{ {\color{red} \bfseries Normally a reference to a previous example goes here.}}
\renewcommand{\examplevref}[1]{ {\color{red} \bfseries Normally a reference to a previous example goes here.}}
\renewcommand{\figurevref}[1]{ {\color{red} \bfseries Normally a reference to a previous figure goes here.}}
\renewcommand{\tablevref}[1]{ {\color{red} \bfseries Normally a reference to a previous table goes here.}}
\renewcommand{\Appendixref}[1]{ {\color{red} \bfseries Normally a reference to an Appendix goes here.}}
\renewcommand{\exercisevref}[1]{ {\color{red} \bfseries Normally a reference to a previous exercise goes here.}}
\renewcommand{\thmvref}[1]{ {\color{red} \bfseries Normally a reference to a previous theorem goes here.}}
\renewcommand{\subsectionvref}[1]{ {\color{red} \bfseries Normally a reference to a previous subsection goes here.}}



\newcommand{\R}{\mathbb{R}}
\newcommand{\C}{\mathbb{C}}

%% Example Solution Env.
\def\beginSolclaim{\par\addvspace{\medskipamount}\noindent\hbox{\bf Solution:}\hspace{0.5em}\ignorespaces}
\def\endSolclaim{\par\addvspace{-1em}\hfill\rule{1em}{0.4pt}\hspace{-0.4pt}\rule{0.4pt}{1em}\par\addvspace{\medskipamount}}
\newenvironment{exampleSol}[1][]{\beginSolclaim}{\endSolclaim}

%% General figure formating from original book.
\newcommand{\mybeginframe}{%
\begin{tcolorbox}[colback=white,colframe=lightgray,left=5pt,right=5pt]%
}
\newcommand{\myendframe}{%
\end{tcolorbox}%
}

%%% Eventually return and fix this to make matlab code work correctly.
%% Define the matlab environment as another code environment
%\NewEnviron{matlab}{ {\centering\bfseries MATLAB Code} \\ \noexpand{\BODY} }
%\let\beginmatlab\begincode
%\let\endmatlab\endcode
%\newenvironment{matlab}{% Begin Environment Code
%\begin{minipage}{\linewidth}
%\begin{verbatim}
%}% End of Begin Environment Code
%{% Start of End Environment Code
%\end{verbatim}
%\end{minipage}
%}% End of End Environment Code


% this one should have a caption, first argument is the size
\newenvironment{mywrapfig}[2][]{
 \wrapfigure[#1]{r}{#2}
 \mybeginframe
 \centering
}{%
 \myendframe
 \endwrapfigure
}

% this one has no caption, first argument is size,
% the second argument is a larger size used for HTML (ignored by latex)
\newenvironment{mywrapfigsimp}[3][]{%
 \wrapfigure[#1]{r}{#2}%
 \centering%
}{%
 \endwrapfigure%
}
\newenvironment{myfig}
    {%
    \begin{figure}[h!t]
        \mybeginframe%
        \centering%
    }
    {%
        \myendframe
    \end{figure}%
    }


% graphics include
\newcommand{\diffyincludegraphics}[3]{\includegraphics[#1]{#3}}
\newcommand{\myincludegraphics}[3]{\includegraphics[#1]{#3}}
\newcommand{\inputpdft}[1]{\subimport*{../figures/}{#1.pdf_t}}


%% Not sure what these even do? They don't seem to actually work... fun!
%\newcommand{\mybxbg}[1]{\tcboxmath[colback=white,colframe=black,boxrule=0.5pt,top=1.5pt,bottom=1.5pt]{#1}}
%\newcommand{\mybxsm}[1]{\tcboxmath[colback=white,colframe=black,boxrule=0.5pt,left=0pt,right=0pt,top=0pt,bottom=0pt]{#1}}
\newcommand{\mybxsm}[1]{#1}
\newcommand{\mybxbg}[1]{#1}

%%% Something about tasks for practice/hw?
\usepackage{tasks}
\usepackage{footnote}
\makesavenoteenv{tasks}


%% For pdf only?
\newcommand{\diffypdfversion}[1]{#1}


%% Kill ``cite'' and go back later to fix it.
\renewcommand{\cite}[1]{}


%% Currently we can't really use index or its derivatives. So we are gonna kill them off.
\renewcommand{\index}[1]{}
\newcommand{\myindex}[1]{#1}







\begin{document}
\begin{abstract}
Why?
\end{abstract}
\maketitle

\begin{exercise}
    Find a particular solution to
    \begin{equation*}
        {\vec{x}}'' =
        \begin{bmatrix}
            -3 & 1 \\
            2 & -2
        \end{bmatrix}
        \vec{x}  + 
        \begin{bmatrix}
            0 \\ 
            2
        \end{bmatrix}
        \cos (2 t) .
    \end{equation*}
    [Use $A$, $B$, $C$ and $D$ for arbitrary constants]\\
    $\vec{x}(t) = \left[\begin{smallmatrix} \answer{1} \\ -1 \end{smallmatrix}\right]\left(\answer{A\cos(2t) + B\sin(2t)}\right) + \left[\begin{smallmatrix} \answer{1} \\ 2 \end{smallmatrix}\right](\answer{C}\cos(t) + D\answer{\sin(t)}) + \left[\begin{smallmatrix} \answer{-\frac{1}{6}} \\ 1/6 \end{smallmatrix}\right]t\sin(2t) + \left[\begin{smallmatrix} \answer{-\answer{2}{9}} \\ -\frac{4}{9} \end{smallmatrix}\right]\cos(2t)$
\end{exercise}
%\comboSol
%{%
%$\vec{x}(t) = \left[\begin{smallmatrix} 1 \\ -1 \end{smallmatrix}\right]\left(C_1\cos(2t) + C_2\sin(2t)\right) + \left[\begin{smallmatrix} 1 \\ 2 \end{smallmatrix}\right](C_3\cos(t) + C_4\sin(t)) + \left[\begin{smallmatrix} -1/6 \\ 1/6 \end{smallmatrix}\right]t\sin(2t) + \left[\begin{smallmatrix} -2/9 \\ -4/9 \end{smallmatrix}\right]\cos(2t)$
%}

\begin{exercise}%
    Find the general solution to
    $\left[ \begin{smallmatrix}
        1 & 0 & 0\\
        0 & 2 & 0\\
        0 & 0 & 3
    \end{smallmatrix}\right]
    \vec{x}\,'' =
    \left[ \begin{smallmatrix}
        -3 & 0 & 0 \\
        2 & -4 & 0 \\
        0 & 6 & -3
    \end{smallmatrix}\right]
    \vec{x} + 
    \left[ \begin{smallmatrix}
        \cos(2t) \\ 
        0 \\ 
        0
    \end{smallmatrix}\right]$.
\end{exercise}
%\exsol{%
%$\vec{x}
%=
%\left[ \begin{smallmatrix}
%1 \\ -1 \\ 1
%\end{smallmatrix}\right]
%\bigl( a_1 \cos (\sqrt{3}\, t)  + b_1 \sin (\sqrt{3}\, t) \bigr)
%+
%\left[ \begin{smallmatrix}
%0 \\ 1 \\ -2
%\end{smallmatrix}\right]
%\bigl( a_2 \cos (\sqrt{2}\, t)  + b_2 \sin (\sqrt{2}\, t) \bigr)
%+
%$
%\\
%$
%\left[ \begin{smallmatrix}
%0 \\ 0 \\ 1
%\end{smallmatrix}\right]
%\bigl( a_3 \cos (t)  + b_3 \sin (t) \bigr)
%+
%\left[ \begin{smallmatrix}
%-1 \\ \frac{1}{2} \\ -\frac{1}{3}
%\end{smallmatrix}\right]
%\cos (2t)$
%}

\begin{exercise}%[challenging]
    Let us take the example in figure \ref{sosa:twocartswallfig} with the same parameters as before: $m_1 = 2$, $k_1 = 4$, and $k_2 = 2$, except for $m_2$, which is unknown. Suppose that there is a force $\cos (5 t)$ acting on the first mass. Find an $m_2$ such that there exists a particular solution where the first mass does not move. $m_2 = \answer{\frac{2}{25}}$
    
    Note: This idea is called \emph{dynamic damping}. In practice there will be a small amount of damping and so any transient solution will disappear and after long enough time, the first mass will always come to a stop.
\end{exercise}
%\comboSol
%{%
%$m= 2/25$
%}

\begin{exercise}
    Let us take the example \ref{sosa:railcarexample}, but that at time of impact, car 2 is moving to the left at the speed of \unitfrac[3]{m}{s}.
    \begin{itemize}
        \item Find the behavior of the system after linkup.
        \item Will the second car hit the wall, or will it be moving away from the wall as time goes on? \wordChoice{\choice[correct]{It will hit the wall.}\choice{It will not hit the wall.}}
        \item At what speed would the first car have to be traveling for the system to essentially stay in place after linkup? $v_w = \answer{\frac{3}{2}}$m/s to the \wordChoice{\choice[correct]{right}\choice{left}}.
    \end{itemize}
\end{exercise}
%\comboSol
%{%
%a)~ $\vec{x}(t) = \left[\begin{smallmatrix} t + \frac{2}{\sqrt{3}}\sin(\sqrt{3}t) \\ t - \frac{4}{\sqrt{3}}\sin(\sqrt{3}t) \end{smallmatrix}\right]$ \quad b)~It will hit the wall. \quad c)~$v_2 = 1.5$ m/s to the right
%}

\begin{exercise}
    Let us take the example in figure \ref{sosa:twocartswallfig} with parameters $m_1 = m_2 = 1$, $k_1 = k_2 = 1$.  Does there exist a set of initial conditions for which the first cart moves but the second cart does not? \wordChoice{\choice{Yes.}\choice[correct]{No.}}%If so, find those conditions.  If not, argue why not.
\end{exercise}
%\comboSol
%{%
%No
%}

\begin{exercise}%
    Suppose there are three carts of equal mass $m$ and connected by two springs of constant $k$ (and no connections to walls).  Set up the system and find its general solution.
\end{exercise}
%\exsol{%
%$\left[ \begin{smallmatrix}
%m & 0 & 0\\
%0 & m & 0\\
%0 & 0 & m
%\end{smallmatrix}\right]
%\vec{x}\,''
%=
%\left[ \begin{smallmatrix}
%-k & k & 0 \\
%k & -2k & k \\
%0 & k & -k
%\end{smallmatrix}\right]
%\vec{x}$.
%Solution:
%$\vec{x} =
%\left[ \begin{smallmatrix}
%1 \\ -2 \\ 1
%\end{smallmatrix}\right]
%\bigl( a_1 \cos (\sqrt{\frac{3k}{m}}\, t)  + b_1 \sin
%(\sqrt{\frac{3k}{m}}\, t) \bigr)
%\allowbreak
%+
%\left[ \begin{smallmatrix}
%1 \\ 0 \\ -1
%\end{smallmatrix}\right]
%\bigl( a_2 \cos (\sqrt{\frac{k}{m}}\, t)  + b_2 \sin
%(\sqrt{\frac{k}{m}}\, t) \bigr)
%+
%\left[ \begin{smallmatrix}
%1 \\ 1 \\ 1
%\end{smallmatrix}\right]
%\bigl( a_3 t  + b_3 \bigr).$
%}

\begin{exercise}%
    Suppose a cart of mass \unit[2]{kg} is attached by a spring of constant $k=1$ to a cart of mass \unit[3]{kg}, which is attached to the wall by a spring also of constant $k=1$. Suppose that the initial position of the first cart is 1 meter in the positive direction from the rest position, and the second mass starts at the rest position.  The masses are not moving and are let go.  Find the position of the second mass as a function of time.
\end{exercise}
%\exsol{%
%$x_2 =
%( \frac{2}{5} )
%\cos (\sqrt{\frac{1}{6}}\, t)
%-
%( \frac{2}{5} )
%\cos (t)$
%%sol
%%$\vec{x} =
%%\left[ \begin{smallmatrix}
%%3 \\ 2
%%\end{smallmatrix}\right]
%%\bigl( a_1 \cos (\sqrt{\frac{1}{6}}\, t)  + b_1 \sin
%%(\sqrt{\frac{1}{6}}\, t) \bigr)
%%+
%%\left[ \begin{smallmatrix}
%%1 \\ -1
%%\end{smallmatrix}\right]
%%\bigl( a_2 \cos (t)  + b_2 \sin (t) \bigr)$.
%}

\begin{exercise}
    Find the general solution to $x_1'' = -6x_1+ 3x_2 + \cos (t)$,  $x_2'' = 2x_1 -7x_2 + 3\cos (t)$,
    \begin{itemize}
        \item using eigenvector decomposition,
        \item using undetermined coefficients.
    \end{itemize}
    $\vec{x}(t) = \left[\begin{smallmatrix} \answer{1} \\ -1 \end{smallmatrix}\right](C_1\answer{\cos(3t)} + C_2\answer{\sin(3t)}) + \left[\begin{smallmatrix} 3 \\ \answer{2} \end{smallmatrix}\right](C_3\cos(2t) + C_4\answer{\sin(2t)}) + \left[\begin{smallmatrix} \answer{\frac{8}{13}} \\ \frac{17}{26} \end{smallmatrix}\right]\cos(t)$
\end{exercise}
%\comboSol
%{%
%$\vec{x}(t) = \left[\begin{smallmatrix} 1 \\ -1 \end{smallmatrix}\right](C_1\cos(3t) + C_2\sin(3t)) + \left[\begin{smallmatrix} 3 \\ 2 \end{smallmatrix}\right](C_3\cos(2t) + C_4\sin(2t)) + \left[\begin{smallmatrix} 8/13 \\ 17/26 \end{smallmatrix}\right]\cos(t)$
%}

\begin{exercise}
    Find the general solution to $x_1'' = -6x_1+ 3x_2 + \cos (2t)$, $x_2'' = 2x_1 -7x_2 + 3\cos (2t)$,
    \begin{itemize}
        \item using eigenvector decomposition,
        \item using undetermined coefficients.
    \end{itemize}
    $\vec{x}(t) = \left[\begin{smallmatrix} \answer{1} \\ -1 \end{smallmatrix}\right](C_1\answer{\cos(3t)} + C_2\sin(3t)) + \left[\begin{smallmatrix} 3 \\ \answer{2} \end{smallmatrix}\right](C_3\cos(2t) + C_4\answer{\sin(2t)}) + \left[\begin{smallmatrix} \answer{-\frac{7}{25}\cos(2t)} + \frac{3}{5}t\sin(2t) \\ \answer{\frac{7}{25}} + \frac{2}{5}\answer{t\sin(2t)} \end{smallmatrix}\right]$
\end{exercise}
%\comboSol
%{%
%$\vec{x}(t) = \left[\begin{smallmatrix} 1 \\ -1 \end{smallmatrix}\right](C_1\cos(3t) + C_2\sin(3t)) + \left[\begin{smallmatrix} 3 \\ 2 \end{smallmatrix}\right](C_3\cos(2t) + C_4\sin(2t)) + \left[\begin{smallmatrix} -\frac{7}{25}\cos(2t) + \frac{3}{5}t\sin(2t) \\ \frac{7}{25} + \frac{2}{5}t\sin(2t) \end{smallmatrix}\right]$
%}

\begin{exercise}%
    Solve $x_1'' = -3x_1 + x_2 + t$, $x_2'' = 9x_1 + 5x_2 + \cos(t)$ with initial conditions $x_1(0) = 0$, $x_2(0) = 0$, $x_1'(0) = 0$, $x_2'(0) = 0$, using eigenvector decomposition.
\end{exercise}
%\exsol{%
%%a) 
%%eig:6,-4
%%[1,9], [1,-1]
%%
%%xi(0)
%%
%%xi1 = c_1 e^(sqrt(6) t)+c_2 e^(-sqrt(6) t)-t\/60-(cos(t))\/70
%%xi2 = c_2 sin(2 t)+c_1 cos(2 t)+(9 t)\/40-(cos(t))\/30
%%
%$\vec{x} = 
%\left[ \begin{smallmatrix}
%1 \\ 9
%\end{smallmatrix}\right]
%\left(\left(\frac{1}{140} + \frac{1}{120\sqrt{6}}\right) e^{\sqrt{6} t}+
% \left(\frac{1}{140} + \frac{1}{120\sqrt{6}}\right)
%e^{-\sqrt{6} t}-\frac{t}{60}-\frac{\cos(t)}{70} \right)
%$
%\\
%$
%+
%\left[ \begin{smallmatrix}
%1 \\ -1
%\end{smallmatrix}\right]
%\left(\frac{-9}{80} \sin(2 t)+ \frac{1}{30} \cos(2 t)+\frac{9 t}{40}-\frac{\cos(t)}{30}\right)$
%%
%%$\vec{x} (0) = 
%%\left[ \begin{smallmatrix}
%%1 \\ 9
%%\end{smallmatrix}\right]
%%(c_1 +c_2 -1/70)
%%+
%%\left[ \begin{smallmatrix}
%%1 \\ -1
%%\end{smallmatrix}\right]
%%(d_1-1/30)$
%%
%%$0 = (c_1+c_2-1/70) + (d_1-1/30)$
%%$0 = 9(c_1+c_2-1/70) - (d_1-1/30)$
%%$0 = c_1+c_2-1/70$
%%$0 = (d_1-1/30)$
%%
%%$d_1 = 1/30$
%%
%%$\vec{x}' = 
%%\left[ \begin{smallmatrix}
%%1 \\ 9
%%\end{smallmatrix}\right]
%%(c_1 sqrt(6) e^(sqrt(6) t)- sqrt(6)c_2 e^(-sqrt(6) t)-1/60+(sin(t))\/70)
%%+
%%\left[ \begin{smallmatrix}
%%1 \\ -1
%%\end{smallmatrix}\right]
%%(2d_2 cos(2 t)-2d_1 sin(2 t)+9/40+(sin(t))/30)$
%%
%%$\vec{x}'(0) = 
%%\left[ \begin{smallmatrix}
%%1 \\ 9
%%\end{smallmatrix}\right]
%%(c_1 sqrt(6) - sqrt(6)c_2 -1/60)
%%+
%%\left[ \begin{smallmatrix}
%%1 \\ -1
%%\end{smallmatrix}\right]
%%(2d_2 +9/40)$
%%
%%$0 = (c_1 sqrt(6) - c_2 sqrt(6) - 1/60) + (2d_2 + 9/40)$
%%$0 = 9(c_1 sqrt(6) - c_2 sqrt(6) - 1/60) - (2d_2 + 9/40)$
%%
%%$0 = c_1 sqrt(6) - c_2 sqrt(6) - 1/60$
%%$d_2 = -9/80$
%%
%%$0 = c_1 sqrt(6) - c_2 sqrt(6) - 1/60$
%%$0 = c_1+c_2-1/70$
%%
%%$c_1 + c_2 = 1/70$
%%$c_1 - c_2 = 1/(60sqrt(6))$
%%
%%$c_1 = 1/140 + 1/(120sqrt(6))$
%%$c_2 = 1/140 - 1/(120sqrt(6))$
%%
%%The general solution is
%%(particular solutions should agree with one of these):
%%$x(t) = \frac{1}{2} C_1 (e^t+e^{-t})+\frac{1}{2} C_2 (e^t-e^{-t})+te^t$
%%\quad
%%$y(t) = \frac{1}{2} C_1 (e^t-e^{-t})+\frac{1}{2} C_2 (e^t+e^{-t})+te^t$
%}



\end{document}