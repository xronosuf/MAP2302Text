\documentclass{ximera}

\title{Practice for Matrix Manipulations}

%\auor{Matthew Charnley and Jason Nowell}
\usepackage[margin=1.5cm]{geometry}
\usepackage{indentfirst}
\usepackage{sagetex}
\usepackage{lipsum}
\usepackage{amsmath}
\usepackage{mathrsfs}


%%% Random packages added without verifying what they are really doing - just to get initial compile to work.
\usepackage{tcolorbox}
\usepackage{hypcap}
\usepackage{booktabs}%% To get \toprule,\midrule,\bottomrule etc.
\usepackage{nicefrac}
\usepackage{caption}
\usepackage{units}

% This is my modified wrapfig that doesn't use intextsep
\usepackage{mywrapfig}
\usepackage{import}



%%% End to random added packages.


\graphicspath{
    {./figures/}
    {./../figures/}
    {./../../figures/}
}
\renewcommand{\log}{\ln}%%%%
\DeclareMathOperator{\arcsec}{arcsec}
%% New commands


%%%%%%%%%%%%%%%%%%%%
% New Conditionals %
%%%%%%%%%%%%%%%%%%%%


% referencing
\makeatletter
    \DeclareRobustCommand{\myvref}[2]{%
      \leavevmode%
      \begingroup
        \let\T@pageref\@pagerefstar
        \hyperref[{#2}]{%
	  #1~\ref*{#2}%
        }%
        \vpageref[\unskip]{#2}%
      \endgroup
    }%

    \DeclareRobustCommand{\myref}[2]{%
      \leavevmode%
      \begingroup
        \let\T@pageref\@pagerefstar
        \hyperref[{#2}]{%
	  #1~\ref*{#2}%
        }%
      \endgroup
    }%
\makeatother

\newcommand{\figurevref}[1]{\myvref{Figure}{#1}}
\newcommand{\figureref}[1]{\myref{Figure}{#1}}
\newcommand{\tablevref}[1]{\myvref{Table}{#1}}
\newcommand{\tableref}[1]{\myref{Table}{#1}}
\newcommand{\chapterref}[1]{\myref{chapter}{#1}}
\newcommand{\Chapterref}[1]{\myref{Chapter}{#1}}
\newcommand{\appendixref}[1]{\myref{appendix}{#1}}
\newcommand{\Appendixref}[1]{\myref{Appendix}{#1}}
\newcommand{\sectionref}[1]{\myref{\S}{#1}}
\newcommand{\subsectionref}[1]{\myref{subsection}{#1}}
\newcommand{\subsectionvref}[1]{\myvref{subsection}{#1}}
\newcommand{\exercisevref}[1]{\myvref{Exercise}{#1}}
\newcommand{\exerciseref}[1]{\myref{Exercise}{#1}}
\newcommand{\examplevref}[1]{\myvref{Example}{#1}}
\newcommand{\exampleref}[1]{\myref{Example}{#1}}
\newcommand{\thmvref}[1]{\myvref{Theorem}{#1}}
\newcommand{\thmref}[1]{\myref{Theorem}{#1}}


\renewcommand{\exampleref}[1]{ {\color{red} \bfseries Normally a reference to a previous example goes here.}}
\renewcommand{\figurevref}[1]{ {\color{red} \bfseries Normally a reference to a previous figure goes here.}}
\renewcommand{\tablevref}[1]{ {\color{red} \bfseries Normally a reference to a previous table goes here.}}
\renewcommand{\Appendixref}[1]{ {\color{red} \bfseries Normally a reference to an Appendix goes here.}}
\renewcommand{\exercisevref}[1]{ {\color{red} \bfseries Normally a reference to a previous exercise goes here.}}



\newcommand{\R}{\mathbb{R}}

%% Example Solution Env.
\def\beginSolclaim{\par\addvspace{\medskipamount}\noindent\hbox{\bf Solution:}\hspace{0.5em}\ignorespaces}
\def\endSolclaim{\par\addvspace{-1em}\hfill\rule{1em}{0.4pt}\hspace{-0.4pt}\rule{0.4pt}{1em}\par\addvspace{\medskipamount}}
\newenvironment{exampleSol}[1][]{\beginSolclaim}{\endSolclaim}

%% General figure formating from original book.
\newcommand{\mybeginframe}{%
\begin{tcolorbox}[colback=white,colframe=lightgray,left=5pt,right=5pt]%
}
\newcommand{\myendframe}{%
\end{tcolorbox}%
}

%%% Eventually return and fix this to make matlab code work correctly.
%% Define the matlab environment as another code environment
%\newenvironment{matlab}
%{% Begin Environment Code
%{ \centering \bfseries Matlab Code }
%\begin{code}
%}% End of Begin Environment Code
%{% Start of End Environment Code
%\end{code}
%}% End of End Environment Code


% this one should have a caption, first argument is the size
\newenvironment{mywrapfig}[2][]{
 \wrapfigure[#1]{r}{#2}
 \mybeginframe
 \centering
}{%
 \myendframe
 \endwrapfigure
}

% this one has no caption, first argument is size,
% the second argument is a larger size used for HTML (ignored by latex)
\newenvironment{mywrapfigsimp}[3][]{%
 \wrapfigure[#1]{r}{#2}%
 \centering%
}{%
 \endwrapfigure%
}
\newenvironment{myfig}
    {%
    \begin{figure}[h!t]
        \mybeginframe%
        \centering%
    }
    {%
        \myendframe
    \end{figure}%
    }


% graphics include
\newcommand{\diffyincludegraphics}[3]{\includegraphics[#1]{#3}}
\newcommand{\myincludegraphics}[3]{\includegraphics[#1]{#3}}
\newcommand{\inputpdft}[1]{\subimport*{../figures/}{#1.pdf_t}}


%% Not sure what these even do? They don't seem to actually work... fun!
%\newcommand{\mybxbg}[1]{\tcboxmath[colback=white,colframe=black,boxrule=0.5pt,top=1.5pt,bottom=1.5pt]{#1}}
%\newcommand{\mybxsm}[1]{\tcboxmath[colback=white,colframe=black,boxrule=0.5pt,left=0pt,right=0pt,top=0pt,bottom=0pt]{#1}}
\newcommand{\mybxsm}[1]{#1}
\newcommand{\mybxbg}[1]{#1}

%%% Something about tasks for practice/hw?
\usepackage{tasks}
\usepackage{footnote}
\makesavenoteenv{tasks}


%% For pdf only?
\newcommand{\diffypdfversion}[1]{#1}


%% Kill ``cite'' and go back later to fix it.
\renewcommand{\cite}[1]{}


%% Currently we can't really use index or its derivatives. So we are gonna kill them off.
\renewcommand{\index}[1]{}
\newcommand{\myindex}[1]{#1}







\begin{document}
\begin{abstract}
Why?
\end{abstract}
\maketitle


\begin{exercise}
    Let $A$ and $B$ be the matrices below.
    \[ 
        A = 
        \begin{bmatrix} 
            1 & 4 & -1 \\ 
            2 & 0 & 3 \\
            1 & -2 & 3 
        \end{bmatrix} 
        \qquad B = 
        \begin{bmatrix} 
            0 & 2 & 3 \\ 
            1 & -4 & -2 \\ 
            2 & -5 & 1 
        \end{bmatrix} 
    \]
    Compute: 
    \begin{itemize}
        \item $A + 3B$, 
            \[
                A+3B = \left[\begin{smallmatrix} \answer{1} & \answer{10} & \answer{8} \\ \answer{5} & \answer{-12} & \answer{-3} \\ \answer{7} & \answer{-17} & \answer{6} \end{smallmatrix}\right]
            \]
        \item $AB$, 
            \[
                AB = \left[\begin{smallmatrix}  \answer{2} & \answer{-9} & \answer{-6} \\ \answer{6} & \answer{-11} & \answer{9} \\ \answer{4} & \answer{-5} & \answer{10} \end{smallmatrix}\right]
            \]
        \item $BA$.
            \[
                BA = \left[\begin{smallmatrix} \answer{7} & \answer{-6} & \answer{15} \\ \answer{-9} & \answer{8} & \answer{-19} \\ \answer{-7} & \answer{6} & \answer{14} \end{smallmatrix}\right]
            \] 
    \end{itemize}
\end{exercise}
%\comboSol
%{%
%$A+3B = \left[\begin{smallmatrix} 1 & 10 & 8 \\ 5 & -12 & -3 \\ 7 & -17 & 6 \end{smallmatrix}\right]$, $AB = \left[\begin{smallmatrix}  2 & -9 & -6 \\ 6 & -11 & 9 \\ 4 & -5 & 10 \end{smallmatrix}\right]$, $BA = \left[\begin{smallmatrix} 7 & -6 & 15 \\ -9 & 8 & -19 \\ -7 & 6 & 14 \end{smallmatrix}\right]$
%}

\begin{exercise}
    Solve
    $\left[ 
        \begin{smallmatrix}
            1 & 2 \\
            3 & 4 
        \end{smallmatrix} 
    \right] 
    \vec{x} =
    \left[ 
        \begin{smallmatrix}
            5 \\
            6
        \end{smallmatrix} 
    \right]$ by using matrix inverse.
    \[
        \left[\begin{smallmatrix} \answer{-4} \\ \answer{\frac{9}{2}} \end{smallmatrix}\right]
    \]
\end{exercise}
%\comboSol
%{%
%$\left[\begin{smallmatrix} -4 \\ 9/2 \end{smallmatrix}\right]$
%}

\begin{exercise}
    Compute determinant of
    $\left[ 
        \begin{smallmatrix}
            9 & -2 & -6 \\
            -8 & 3 & 6 \\
            10 & -2 & -6
        \end{smallmatrix} 
    \right]$.\\
    $\answer{6}$
\end{exercise}
%\comboSol
%{%
%6
%}

\begin{exercise}%
    Compute determinant of
    $\left[ 
        \begin{smallmatrix}
            1 & 1 & 1 \\
            2 & 3 & -5 \\
            1 & -1 & 0
        \end{smallmatrix}
    \right]$\\
    $\answer{-15}$
\end{exercise}
%\exsol{%
%$-15$
%}

\begin{exercise}
    \begin{hint}
        Hint: Expand along the proper row or column to make the calculations simpler.\\
    \end{hint}
    Compute determinant of
    $\left[ 
        \begin{smallmatrix}
            1 & 2 & 3 & 1 \\
            4 & 0 & 5 & 0 \\
            6 & 0 & 7 & 0 \\
            8 & 0 & 10 & 1
        \end{smallmatrix} 
    \right]$.\\
    $\answer{4}$
\end{exercise}
%\comboSol
%{%
%4
%}

\begin{exercise}
    Compute inverse of
    $\left[ 
        \begin{smallmatrix}
            1 & 2 & 3 \\
            1 & 1 & 1 \\
            0 & 1 & 0
        \end{smallmatrix} 
    \right]$.\\
    $\left[\begin{smallmatrix} \answer{-\frac{1}{2}} & \answer{\frac{3}{2}} & \answer{-\frac{1}{2}} \\ \answer{0} & \answer{0} & \answer{1} \\ \answer{\frac{1}{2}} & \answer{-\frac{1}{2}} & \answer{-\frac{1}{2}} \end{smallmatrix}\right]$
\end{exercise}
%\comboSol
%{%
%$\left[\begin{smallmatrix} -1/2 & 3/2 & -1/2 \\ 0 & 0 & 1 \\ 1/2 & -1/2 & -1/2 \end{smallmatrix}\right]$
%}

\begin{exercise}
    For how many $h$ is
    $\left[ 
        \begin{smallmatrix}
            1 & 2 & 3 \\
            4 & 5 & 6 \\
            7 & 8 & h
        \end{smallmatrix} 
    \right]$
    not invertible? 
    \begin{multipleChoice}
        \choice{None - it is invertible for any $h$.}
        \choice[correct]{One value of $h$.}
        \choice{Infinitely many values of $h$.}
    \end{multipleChoice}
    \begin{problem}
        Which value of $h$ makes the matrix not invertible? $h = \answer{9}$ 
    \end{problem}
\end{exercise}
%\comboSol
%{%
%$h=9$
%}


\begin{exercise}%
    Find $t$ such that
    $\left[ 
        \begin{smallmatrix}
            1 & t \\
            -1 & 2
        \end{smallmatrix}
    \right]$
    is not invertible. $t = \answer{-2}$
\end{exercise}
%\exsol{%
%$-2$
%}

\begin{exercise}
    For which $h$ is
    $\left[ 
        \begin{smallmatrix}
            h & 1 & 1 \\
            0 & h & 0 \\
            1 & 1 & h
        \end{smallmatrix} 
    \right]$
    not invertible? $h = \answer{0}$, $\pm \answer{1}$
\end{exercise}
%\comboSol
%{%
%$h = 0, \pm 1$
%}

\begin{exercise}%
    Solve the system of equations
    \begin{equation*}
        \begin{split}
             4x_1 - 2x_2 + 4x_3 &= -8 \\ 
             x_1 - 3x_3 &= 12\\
            -4x_1 + 4x_2 + 4x_3 &= -8 
        \end{split}
    \end{equation*}
    or determine that no solution exists.\\
    $x_1 = \answer{3}$, $x_2 = \answer{4}$, $x_3 = \answer{-3}$.
\end{exercise}
%\exsol{%
%$x_1 = 3$, $x_2 = 4$, $x_3 = -3$.
%}%

\begin{exercise}%
    Solve the system of equations
    \begin{equation*}
        \begin{split}
             -x_1 - 4x_2 + 2x_3 &= 11 \\ 
             3x_1 -3x_2 + x_3 &= 13\\
            -5x_1 - 5x_2 + 3x_3 &= 9 
        \end{split}
    \end{equation*}
    or determine that no solution exists.
    \begin{multipleChoice}
        \choice{There is no solution.}
        \choice{There is a single solution.}
        \choice[correct]{There are infinitely many solutions.}
    \end{multipleChoice}
    \begin{problem}
        Since there are infinitely many solutions, let $x_3 = t$ for some real number $t$. Then: $x_1 = \answer{\frac{19}{15} + \frac{2}{15}t}$, $x_2 = \answer{\frac{7}{15}t - \frac{46}{15}}$
    \end{problem}
\end{exercise}
%\exsol{%
%Infinitely many solutions of the form $x_1 = \frac{19}{15} + \frac{2}{15}t$, $x_2 = \frac{7}{15}t - \frac{46}{15}$, $x_3 = t$ for any real number $t$. 
%}%

\begin{exercise}%
    Solve the system of equations
    \begin{equation*}
        \begin{split}
            x_1 + 3x_2 - 3x_3 &= 1 \\ 
             -3x_1 - 4x_2  +4x_3 &= -3\\
            4x_1 + 7x_2 - 7x_3 &= 7 
        \end{split}
    \end{equation*}
    or determine that no solution exists.
    \begin{multipleChoice}
        \choice[correct]{There is no solution.}
        \choice{There is a single solution.}
        \choice{There are infinitely many solutions.}
    \end{multipleChoice}
\end{exercise}
%\exsol{%
%No solution.
%}%

\begin{exercise}%
    Solve the system of equations
    \begin{equation*}
        \begin{split}
             x_1  + 3x_2  - x_3 &= 5 \\ 
             2x_1 +x_2 &= -3\\
            -3x_1 - 4x_2 + 2x_3 &= -6 
        \end{split}
    \end{equation*}
    or determine that no solution exists.
    \begin{multipleChoice}
        \choice{There is no solution.}
        \choice[correct]{There is a single solution.}
        \choice{There are infinitely many solutions.}
    \end{multipleChoice}
    \begin{problem}
        $x_1 = \answer{-2}$, $x_2 = \answer{1}$, $x_3 = \answer{-4}$.
    \end{problem}
\end{exercise}
%\exsol{%
%$x_1 = -2$, $x_2 = 1$, $x_3 = -4$.
%}%

\begin{exercise}
    Solve
    $\left[ 
        \begin{smallmatrix}
            9 & -2 & -6 \\
            -8 & 3 & 6 \\
            10 & -2 & -6
        \end{smallmatrix} 
    \right] 
    \vec{x} =
    \left[ 
        \begin{smallmatrix}
            1 \\
            2 \\
            3
        \end{smallmatrix} 
    \right]$.\\
    $\vec{x} = \left[\begin{smallmatrix} \answer{2} \\ \answer{1} \\ \answer{\frac{5}{2}} \end{smallmatrix}\right]$
    \begin{multipleChoice}
        \choice{There is no solution.}
        \choice[correct]{There is a single solution.}
        \choice{There are infinitely many solutions.}
    \end{multipleChoice}
    \begin{problem}
        $\vec{x} = \left[\begin{smallmatrix} \answer{2} \\ \answer{1} \\ \answer{\frac{5}{2}} \end{smallmatrix}\right]$
    \end{problem}
\end{exercise}
%\comboSol
%{%
%$\vec{x} = \left[\begin{smallmatrix} 2 \\ 1 \\ \frac{5}{2} \end{smallmatrix}\right]$
%}

\begin{exercise}
    Solve
    $\left[ 
        \begin{smallmatrix}
            5 & 3 & 7 \\
            8 & 4 & 4 \\
            6 & 3 & 3
        \end{smallmatrix} 
    \right] 
    \vec{x} =
    \left[ 
        \begin{smallmatrix}
            2 \\
            0 \\
            0
        \end{smallmatrix} 
    \right]$.
    \begin{multipleChoice}
        \choice{There is no solution.}
        \choice{There is a single solution.}
        \choice[correct]{There are infinitely many solutions.}
    \end{multipleChoice}
    \begin{problem}
        Since there are infinitely many solutions, let $t$ be any real number; then the solutions are of the form:\\
        $\vec{x} = \left[\begin{smallmatrix} \answer{-2+4t} \\ \answer{4-9t} \\ t \end{smallmatrix}\right]$ for any real $t$
    \end{problem}
\end{exercise}
%\comboSol
%{%
%Infinitely many solutions $\vec{x} = \left[\begin{smallmatrix} -2+4t \\ 4-9t \\ t \end{smallmatrix}\right]$ for any real $t$
%}

\begin{exercise}%
    Solve
    $\left[ 
        \begin{smallmatrix}
            1 & 1 \\
            1 & -1
        \end{smallmatrix}
    \right] 
    \vec{x} = 
    \left[ 
        \begin{smallmatrix}
            10 \\ 
            20
        \end{smallmatrix}
    \right]$.
    \begin{multipleChoice}
        \choice{There is no solution.}
        \choice[correct]{There is a single solution.}
        \choice{There are infinitely many solutions.}
    \end{multipleChoice}
    \begin{problem}
        $\vec{x} = \left[ \begin{smallmatrix} \answer{15} \\ \answer{-5} \end{smallmatrix}\right]$
    \end{problem}
\end{exercise}
%\exsol{%
%$\vec{x} =
%\left[ \begin{smallmatrix}
%15 \\ -5
%\end{smallmatrix}\right]$
%}


\begin{exercise}
    Solve
    $\left[ 
        \begin{smallmatrix}
            3 & 2 & 3 & 0 \\
            3 & 3 & 3 & 3 \\
            0 & 2 & 4 & 2 \\
            2 & 3 & 4 & 3 
        \end{smallmatrix} 
    \right] 
    \vec{x} =
    \left[ 
        \begin{smallmatrix}
            2 \\
            0 \\
            4 \\
            1
        \end{smallmatrix} 
    \right]$.
    \begin{multipleChoice}
        \choice[correct]{There is no solution.}
        \choice{There is a single solution.}
        \choice{There are infinitely many solutions.}
    \end{multipleChoice}
\end{exercise}
%\comboSol
%{%
%No solution
%}

\begin{exercise}
    Find 3 nonzero $2 \times 2$ matrices $A$, $B$, and $C$ such that $AB = AC$ but $B \not= C$.
\end{exercise}
%\comboSol
%{%
%Many examples. (Hint: What do you know has to be true about $A$?) $A = \left[\begin{smallmatrix}  1 & 1 \\ 1 & 1 \end{smallmatrix}\right]$, $B = \left[\begin{smallmatrix}  2 & 4\\ 1 & -2 \end{smallmatrix}\right]$, $C = \left[\begin{smallmatrix} -1 & 1 \\ 4 & 1 \end{smallmatrix}\right]$
%}

\begin{exercise}%
    Suppose $a, b, c$ are nonzero numbers. Let
    $M=\left[ 
        \begin{smallmatrix}
            a & 0 \\
            0 & b
        \end{smallmatrix}
    \right]$,
    $N=\left[ 
        \begin{smallmatrix}
            a & 0 & 0 \\
            0 & b & 0 \\
            0 & 0 & c
        \end{smallmatrix}
    \right]$.
    \begin{itemize}
        \item Compute $M^{-1}$.\\
        $\left[ \begin{smallmatrix} \answer{\frac{1}{a}} & \answer{0} \\ \answer{0} & \answer{\frac{1}{b}} \end{smallmatrix}\right]$
        \item Compute $N^{-1}$.\\
        $\left[ \begin{smallmatrix} \answer{\frac{1}{a}} & \answer{0} & \answer{0} \\ \answer{0} & \answer{\frac{1}{b}} & \answer{0} \\ \answer{0} & \answer{0} & \answer{\frac{1}{c}} \end{smallmatrix}\right]$
    \end{itemize}
\end{exercise}
%\exsol{%
%a) $\left[ \begin{smallmatrix}
%\frac{1}{a} & 0 \\
%0 & \frac{1}{b}
%\end{smallmatrix}\right]$
%\quad
%b)
%$\left[ \begin{smallmatrix}
%\frac{1}{a} & 0 & 0 \\
%0 & \frac{1}{b} & 0 \\
%0 & 0 & \frac{1}{c}
%\end{smallmatrix}\right]$
%}

\begin{exercise}[easy]
    Let $A$ be a $3 \times 3$ matrix with an eigenvalue of 3 and a corresponding eigenvector $\vec{v} = \left[ \begin{smallmatrix} 1 \\ -1 \\ 3 \end{smallmatrix} \right]$. Find $A \vec{v}$.\\
    $\left[\begin{smallmatrix} \answer{3} & \answer{-3} & \answer{9} \end{smallmatrix}\right]$
\end{exercise}
%\comboSol
%{%
%$\left[\begin{smallmatrix} 3 & -3 & 9 \end{smallmatrix}\right]$
%}

\begin{exercise}%
    Find the eigenvalues and eigenvectors for the matrix
    \[ 
        \begin{bmatrix} 
            0 & -2 \\ 
            1 & 3
        \end{bmatrix}. 
    \]
    $\lambda_1 = \answer{1}$, $\vec{v}_1 = \left[\begin{smallmatrix} \answer{-2} \\ \answer{1}\end{smallmatrix}\right]$, $\lambda_2 = \answer{2}$, $\vec{v}_2 = \left[\begin{smallmatrix} \answer{1} \\ \answer{-1} \end{smallmatrix}\right]$. 
\end{exercise}
%\exsol{%
%$\lambda_1 = 1$, $\vec{v}_1 = \left[\begin{smallmatrix} -2 \\ 1\end{smallmatrix}\right]$, $\lambda_2 = 2$, $\vec{v}_2 = \left[\begin{smallmatrix} 1 \\ -1 \end{smallmatrix}\right]$. 
%}

\begin{exercise}%
    Find the eigenvalues and eigenvectors for the matrix
    \[ 
        \begin{bmatrix} 
            -8 & -5 \\ 
            8 & 4
        \end{bmatrix}. 
    \]
    $\lambda_1 = \answer{-2 + 2i}$, $\vec{v}_1 = \left[\begin{smallmatrix} \answer{-3+i} \\ \answer{4}\end{smallmatrix}\right]$, $\lambda_2 = \answer{-2-2i}$, $\vec{v}_2 = \left[\begin{smallmatrix} \answer{-3-i} \\ \answer{4} \end{smallmatrix}\right]$. 
\end{exercise}
%\exsol{%
%$\lambda_1 = -2 + 2i$, $\vec{v}_1 = \left[\begin{smallmatrix} -3+i \\ 4\end{smallmatrix}\right]$, $\lambda_2 = -2-2i$, $\vec{v}_2 = \left[\begin{smallmatrix} -3-i \\ 4 \end{smallmatrix}\right]$. 
%}

\begin{exercise}%
    Find the eigenvalues and eigenvectors for the matrix
    \[ 
        \begin{bmatrix} 
            7 & -3 & 7 \\ 
            9 & -5 & 7 \\ 
            0 & 0 & -3
        \end{bmatrix}. 
    \]
    $\lambda_1 = \answer{4}$, $\vec{v}_1 = \left[\begin{smallmatrix} \answer{1} \\ \answer{1} \\ \answer{0}\end{smallmatrix}\right]$, $\lambda_2 = \answer{-2}$, $\vec{v}_2 = \left[\begin{smallmatrix} \answer{1} \\ \answer{3} \\ \answer{0} \end{smallmatrix}\right]$, $\lambda_3 = \answer{-3}$, $\vec{v}_3 = \left[\begin{smallmatrix} \answer{-1} \\ \answer{-1} \\ \answer{1} \end{smallmatrix}\right]$. 
\end{exercise}
%\exsol{%
%$\lambda_1 = 4$, $\vec{v}_1 = \left[\begin{smallmatrix} 1 \\ 1 \\ 0\end{smallmatrix}\right]$, $\lambda_2 = -2$, $\vec{v}_2 = \left[\begin{smallmatrix} 1 \\ 3 \\ 0 \end{smallmatrix}\right]$, $\lambda_3 = -3$, $\vec{v}_3 = \left[\begin{smallmatrix} -1 \\ -1 \\ 1 \end{smallmatrix}\right]$. 
%}


\end{document}