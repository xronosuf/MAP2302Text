\documentclass{ximera}
%\auor{Matthew Charnley and Jason Nowell}
\usepackage[margin=1.5cm]{geometry}
\usepackage{indentfirst}
\usepackage{sagetex}
\usepackage{lipsum}
\usepackage{amsmath}
\usepackage{mathrsfs}


%%% Random packages added without verifying what they are really doing - just to get initial compile to work.
\usepackage{tcolorbox}
\usepackage{hypcap}
\usepackage{booktabs}%% To get \toprule,\midrule,\bottomrule etc.
\usepackage{nicefrac}
\usepackage{caption}
\usepackage{units}

% This is my modified wrapfig that doesn't use intextsep
\usepackage{mywrapfig}
\usepackage{import}



%%% End to random added packages.


\graphicspath{
    {./figures/}
    {./../figures/}
    {./../../figures/}
}
\renewcommand{\log}{\ln}%%%%
\DeclareMathOperator{\arcsec}{arcsec}
%% New commands


%%%%%%%%%%%%%%%%%%%%
% New Conditionals %
%%%%%%%%%%%%%%%%%%%%


% referencing
\makeatletter
    \DeclareRobustCommand{\myvref}[2]{%
      \leavevmode%
      \begingroup
        \let\T@pageref\@pagerefstar
        \hyperref[{#2}]{%
	  #1~\ref*{#2}%
        }%
        \vpageref[\unskip]{#2}%
      \endgroup
    }%

    \DeclareRobustCommand{\myref}[2]{%
      \leavevmode%
      \begingroup
        \let\T@pageref\@pagerefstar
        \hyperref[{#2}]{%
	  #1~\ref*{#2}%
        }%
      \endgroup
    }%
\makeatother

\newcommand{\figurevref}[1]{\myvref{Figure}{#1}}
\newcommand{\figureref}[1]{\myref{Figure}{#1}}
\newcommand{\tablevref}[1]{\myvref{Table}{#1}}
\newcommand{\tableref}[1]{\myref{Table}{#1}}
\newcommand{\chapterref}[1]{\myref{chapter}{#1}}
\newcommand{\Chapterref}[1]{\myref{Chapter}{#1}}
\newcommand{\appendixref}[1]{\myref{appendix}{#1}}
\newcommand{\Appendixref}[1]{\myref{Appendix}{#1}}
\newcommand{\sectionref}[1]{\myref{\S}{#1}}
\newcommand{\subsectionref}[1]{\myref{subsection}{#1}}
\newcommand{\subsectionvref}[1]{\myvref{subsection}{#1}}
\newcommand{\exercisevref}[1]{\myvref{Exercise}{#1}}
\newcommand{\exerciseref}[1]{\myref{Exercise}{#1}}
\newcommand{\examplevref}[1]{\myvref{Example}{#1}}
\newcommand{\exampleref}[1]{\myref{Example}{#1}}
\newcommand{\thmvref}[1]{\myvref{Theorem}{#1}}
\newcommand{\thmref}[1]{\myref{Theorem}{#1}}


\renewcommand{\exampleref}[1]{ {\color{red} \bfseries Normally a reference to a previous example goes here.}}
\renewcommand{\figurevref}[1]{ {\color{red} \bfseries Normally a reference to a previous figure goes here.}}
\renewcommand{\tablevref}[1]{ {\color{red} \bfseries Normally a reference to a previous table goes here.}}
\renewcommand{\Appendixref}[1]{ {\color{red} \bfseries Normally a reference to an Appendix goes here.}}
\renewcommand{\exercisevref}[1]{ {\color{red} \bfseries Normally a reference to a previous exercise goes here.}}



\newcommand{\R}{\mathbb{R}}

%% Example Solution Env.
\def\beginSolclaim{\par\addvspace{\medskipamount}\noindent\hbox{\bf Solution:}\hspace{0.5em}\ignorespaces}
\def\endSolclaim{\par\addvspace{-1em}\hfill\rule{1em}{0.4pt}\hspace{-0.4pt}\rule{0.4pt}{1em}\par\addvspace{\medskipamount}}
\newenvironment{exampleSol}[1][]{\beginSolclaim}{\endSolclaim}

%% General figure formating from original book.
\newcommand{\mybeginframe}{%
\begin{tcolorbox}[colback=white,colframe=lightgray,left=5pt,right=5pt]%
}
\newcommand{\myendframe}{%
\end{tcolorbox}%
}

%%% Eventually return and fix this to make matlab code work correctly.
%% Define the matlab environment as another code environment
%\newenvironment{matlab}
%{% Begin Environment Code
%{ \centering \bfseries Matlab Code }
%\begin{code}
%}% End of Begin Environment Code
%{% Start of End Environment Code
%\end{code}
%}% End of End Environment Code


% this one should have a caption, first argument is the size
\newenvironment{mywrapfig}[2][]{
 \wrapfigure[#1]{r}{#2}
 \mybeginframe
 \centering
}{%
 \myendframe
 \endwrapfigure
}

% this one has no caption, first argument is size,
% the second argument is a larger size used for HTML (ignored by latex)
\newenvironment{mywrapfigsimp}[3][]{%
 \wrapfigure[#1]{r}{#2}%
 \centering%
}{%
 \endwrapfigure%
}
\newenvironment{myfig}
    {%
    \begin{figure}[h!t]
        \mybeginframe%
        \centering%
    }
    {%
        \myendframe
    \end{figure}%
    }


% graphics include
\newcommand{\diffyincludegraphics}[3]{\includegraphics[#1]{#3}}
\newcommand{\myincludegraphics}[3]{\includegraphics[#1]{#3}}
\newcommand{\inputpdft}[1]{\subimport*{../figures/}{#1.pdf_t}}


%% Not sure what these even do? They don't seem to actually work... fun!
%\newcommand{\mybxbg}[1]{\tcboxmath[colback=white,colframe=black,boxrule=0.5pt,top=1.5pt,bottom=1.5pt]{#1}}
%\newcommand{\mybxsm}[1]{\tcboxmath[colback=white,colframe=black,boxrule=0.5pt,left=0pt,right=0pt,top=0pt,bottom=0pt]{#1}}
\newcommand{\mybxsm}[1]{#1}
\newcommand{\mybxbg}[1]{#1}

%%% Something about tasks for practice/hw?
\usepackage{tasks}
\usepackage{footnote}
\makesavenoteenv{tasks}


%% For pdf only?
\newcommand{\diffypdfversion}[1]{#1}


%% Kill ``cite'' and go back later to fix it.
\renewcommand{\cite}[1]{}


%% Currently we can't really use index or its derivatives. So we are gonna kill them off.
\renewcommand{\index}[1]{}
\newcommand{\myindex}[1]{#1}






\title{Matrix exponentials}
\author{Matthew Charnley and Jason Nowell}


\outcome{Compute the exponential of a matrix}
\outcome{Use the matrix exponential to solve linear systems of differential equations.}


\begin{document}
\begin{abstract}
    We discuss Matrix exponentials
\end{abstract}
\maketitle

\label{sec:matexp}

\subsection{Definition}

There is another way of finding a fundamental matrix solution of a system.  Consider the constant coefficient equation
\begin{equation*}
    {\vec{x}}' = P \vec{x} .
\end{equation*}
If this would be just one equation (when $P$ is a number or a $1 \times 1$ matrix), then the solution would be
\begin{equation*}
    \vec{x} = e^{Pt} .
\end{equation*}
That doesn't make sense if $P$ is a larger matrix, but essentially the same computation that led to the above works for matrices when we define $e^{Pt}$ properly.  First let us write down the Taylor series for $e^{at}$ for some number $a$:
\begin{equation*}
    e^{at} = 1 + at + \frac{{(at)}^2}{2} + \frac{{(at)}^3}{6} + \frac{{(at)}^4}{24} + \cdots = \sum_{k=0}^\infty \frac{{(at)}^k}{k!} .
\end{equation*}
Recall $k! = 1 \cdot 2 \cdot 3 \cdots k$ is the factorial, and $0! = 1$. We differentiate this series term by term
\begin{equation*}
    \frac{d}{dt} \left(e^{at} \right) =  0 + a + a^2 t + \frac{a^3t^2}{2} + \frac{a^4t^3}{6} + \cdots = a \left( 1 + a t + \frac{{(at)}^2}{2} + \frac{{(at)}^3}{6} + \cdots \right) = a e^{at}.
\end{equation*}
Maybe we can try the same trick with matrices.
\begin{definition}
    For an $n \times n$ matrix $A$ we define the \emph{\myindex{matrix exponential}\index{exponential of a matrix}} as 
    \begin{equation*}
        % \mybxbg{~~
        e^A \overset{\text{def}}{=} I + A + \frac{1}{2} A^2 + \frac{1}{6} A^3 + \cdots + \frac{1}{k!} A^k + \cdots
        % ~~}
    \end{equation*}
\end{definition}
Let us not worry about convergence.  The series really does always converge. We usually write $Pt$ as $tP$ by convention when $P$ is a matrix. With this small change and by the exact same calculation as above we have that
\begin{equation*}
    \frac{d}{dt} \left(e^{tP} \right) = P e^{tP} .
\end{equation*}
Now $P$ and hence $e^{tP}$ is an $n \times n$ matrix.  What we are looking for is a vector.  In the $1 \times 1$ case we would at this point multiply by an arbitrary constant to get the general solution.  In the matrix case we multiply by a column vector $\vec{c}$.

\begin{theorem}{}
    Let $P$ be an $n \times n$ matrix.  Then the general solution to ${\vec{x}}' = P \vec{x}$ is
    \begin{equation*}
        \vec{x} = e^{tP} \vec{c} ,
    \end{equation*}
    where $\vec{c}$ is an arbitrary constant vector.  In fact, $\vec{x}(0) = \vec{c}$.
\end{theorem}

Let us check:
\begin{equation*}
    \frac{d}{dt} \vec{x} = \frac{d}{dt} \left( e^{tP} \vec{c}\, \right) = P e^{tP} \vec{c} = P \vec{x}.
\end{equation*}

Hence $e^{tP}$ is a \myindex{fundamental matrix solution} of the homogeneous system. So if we can compute the matrix exponential, we have another method of solving constant coefficient homogeneous systems.  It also makes it easy to solve for initial conditions.  To solve  ${\vec{x}}' = A \vec{x}$, $\vec{x}(0) = \vec{b}$, we take the solution
\begin{equation*}
    \vec{x} = e^{tA} \vec{b} .
\end{equation*}
This equation follows because $e^{0A} = I$, so $\vec{x} (0) = e^{0A} \vec{b} = \vec{b}$.

We mention a drawback of matrix exponentials. In general $e^{A+B} \not= e^A e^B$.  The trouble is that matrices do not commute, that is, in general $AB \not= BA$. If you try to prove $e^{A+B} \not= e^A e^B$ using the Taylor series, you will see why the lack of commutativity becomes a problem. However, it is still true that if $AB = BA$, that is, if $A$ and $B$ commute, then $e^{A+B} = e^Ae^B$.  We will find this fact useful.  Let us restate this as a theorem to make a point.

\begin{theorem}{}
    If $AB = BA$, then $e^{A+B} = e^Ae^B$.  Otherwise, $e^{A+B} \not= e^Ae^B$ in general.
\end{theorem}

\subsection{Simple cases}

In some instances it may work to just plug into the series definition. Suppose the matrix is diagonal\index{diagonal matrix!matrix exponential of}. For example,
$D = \left[ \begin{smallmatrix} a & 0 \\ 0 & b \end{smallmatrix} \right]$.
Then 
\begin{equation*}
    D^k = \begin{bmatrix} a^k & 0 \\ 0 & b^k \end{bmatrix} ,
\end{equation*}
and
\begin{equation*}
    \begin{split}
        e^D & = I + D + \frac{1}{2} D^2 + \frac{1}{6} D^3 + \cdots \\
        &=
        \begin{bmatrix} 
            1 & 0 \\ 
            0 & 1 
        \end{bmatrix} +
        \begin{bmatrix} 
            a & 0 \\ 
            0 & b 
        \end{bmatrix}
         + \frac{1}{2}
        \begin{bmatrix} 
            a^2 & 0 \\ 
            0 & b^2 
        \end{bmatrix}
         + \frac{1}{6}
        \begin{bmatrix} 
            a^3 & 0 \\ 
            0 & b^3 
        \end{bmatrix} 
        + \cdots =
        \begin{bmatrix} 
            e^a & 0 \\ 
            0 & e^b 
        \end{bmatrix} .
    \end{split}
\end{equation*}
So by this rationale
\begin{equation*}
    e^I = 
    \begin{bmatrix} 
        e & 0\\ 
        0 & e 
    \end{bmatrix}
    \qquad \text{and} \qquad e^{aI} = 
    \begin{bmatrix} 
        e^a & 0\\ 
        0 & e^a 
    \end{bmatrix}.
\end{equation*}

This makes exponentials of certain other matrices easy to compute. For example, the matrix 
$A = \left[ \begin{smallmatrix} 5 & 4 \\ -1 & 1 \end{smallmatrix} \right]$
can be written as $3I + B$ where
$B = \left[ \begin{smallmatrix} 2 & 4 \\ -1 & -2 \end{smallmatrix} \right]$.
Notice that $B^2 = \left[ \begin{smallmatrix} 0 & 0 \\ 0 & 0 \end{smallmatrix} \right]$.  So
$B^k = 0$ for all $k \geq 2$.  Therefore, $e^B = I + B$.  Suppose we actually want to compute $e^{tA}$.  The matrices $3tI$ and $tB$ commute (exercise: check this) and 
$e^{tB} = I + tB$, since ${(tB)}^2 = t^2 B^2 = 0$. We write
\begin{multline*}
    e^{tA} = e^{3tI + tB} = e^{3tI} e^{tB} = 
    \begin{bmatrix} 
        e^{3t} & 0 \\ 
        0 & e^{3t} 
    \end{bmatrix}
    \left( I + tB \right) = \\
    =
    \begin{bmatrix} 
        e^{3t} & 0 \\ 
        0 & e^{3t} 
    \end{bmatrix}
    \begin{bmatrix} 
        1+2t & 4t \\ 
        -t & 1-2t 
    \end{bmatrix}
    =
    \begin{bmatrix} 
        (1+2t)\,e^{3t} & 4te^{3t} \\ 
        -te^{3t} & (1-2t)\,e^{3t} 
    \end{bmatrix} .
\end{multline*}
We found a fundamental matrix solution for the system ${\vec{x}}' = A \vec{x}$.  Note that this matrix has a repeated eigenvalue with a defect; there is only one eigenvector for the eigenvalue 3.  So we found a perhaps easier way to handle this case.  In fact, if a matrix $A$ is $2 \times 2$ and has an eigenvalue $\lambda$ of multiplicity 2, then either $A = \lambda I$, or $A = \lambda I + B$ where $B^2 = 0$.  This is a good exercise.

\begin{exercise}%[challenging] No longer challenging with the hint?
    Suppose that $A$ is $2 \times 2$ and $\lambda$ is the only eigenvalue.  Show that ${(A - \lambda I)}^2 = 0$, and therefore that we can write $A = \lambda I + B$, where $B^2 = 0$ (and possibly $B=0$). Hint: First write down what does it mean for the eigenvalue to be of multiplicity 2. You will get an equation for the entries.  Now compute the square of $B$.
\end{exercise}

Matrices $B$ such that $B^k = 0$ for some $k$ are called \emph{\myindex{nilpotent}}.  Computation of the matrix exponential for nilpotent matrices is easy by just writing down the first $k$ terms of the Taylor series.

\subsection{General matrices}

In general, the exponential is not as easy to compute as above.  We usually cannot write a matrix as a sum of commuting matrices where the exponential is simple for each one.  But fear not, it is still not too difficult provided we can find enough eigenvectors.  First we need the following interesting result about matrix exponentials.  For two square matrices $A$ and $B$, with $B$ invertible, we have
\begin{equation*}
    e^{BAB^{-1}} = B e^A B^{-1} .
\end{equation*}
This can be seen by writing down the Taylor series.  First 
\begin{equation*}
    {(BAB^{-1})}^2 = BAB^{-1} BAB^{-1} = BAIAB^{-1} = BA^2B^{-1} .
\end{equation*}
And by the same reasoning ${(BAB^{-1})}^k = B A^k B^{-1}$.  Now write the Taylor series for $e^{BAB^{-1}}$:
\begin{equation*}
    \begin{split}
        e^{BAB^{-1}} & = I + {BAB^{-1}} + \frac{1}{2} {(BAB^{-1})}^2 + \frac{1}{6} {(BAB^{-1})}^3 + \cdots \\
        & = BB^{-1} + {BAB^{-1}} + \frac{1}{2} BA^2B^{-1} + \frac{1}{6} BA^3B^{-1} + \cdots \\
        & = B \bigl( I + A + \frac{1}{2} A^2 + \frac{1}{6} A^3 + \cdots \bigr) B^{-1} \\
        & = B e^A B^{-1} . 
    \end{split}
\end{equation*}

Given a square matrix $A$, we can usually write $A = E D E^{-1}$, where $D$ is diagonal and $E$ invertible. This procedure is called \emph{\myindex{diagonalization}}. If we can do that, the computation  of the 
exponential becomes easy as $e^D$ is just taking the exponential of the entries on the diagonal.  Adding $t$ into the mix, we can then compute the exponential
\begin{equation*}
    e^{tA} = E e^{tD} E^{-1} .
\end{equation*}

To diagonalize $A$ we need $n$ linearly independent eigenvectors of $A$. Otherwise, this method of computing the exponential does not work and we need to be trickier, but we will not get into such details. Let $E$ be the matrix with the eigenvectors as columns.  Let $\lambda_1$, $\lambda_2$, \ldots, $\lambda_n$ be the eigenvalues and let $\vec{v}_1$, $\vec{v}_2$, \ldots, $\vec{v}_n$ be the eigenvectors, then $E = [\, \vec{v}_1 \quad \vec{v}_2 \quad \cdots \quad \vec{v}_n \,]$. 
Make a diagonal matrix $D$ with the eigenvalues on the diagonal:
\begin{equation*}
    D =
    \begin{bmatrix}
        \lambda_1 & 0 & \cdots & 0 \\
        0 & \lambda_2 & \cdots & 0 \\
        \vdots & \vdots & \ddots & \vdots \\
        0 & 0 & \cdots & \lambda_n
    \end{bmatrix} .
\end{equation*}
We compute
\begin{equation*}
    \begin{split}
        AE & = A [\, \vec{v}_1 \quad \vec{v}_2 \quad \cdots \quad \vec{v}_n \,] \\
        & = [\, A\vec{v}_1 \quad A\vec{v}_2 \quad \cdots \quad A\vec{v}_n \,] \\
        & = [\, \lambda_1 \vec{v}_1 \quad \lambda_2 \vec{v}_2 \quad \cdots \quad \lambda_n \vec{v}_n \,] \\
        & = [\, \vec{v}_1 \quad \vec{v}_2 \quad \cdots \quad \vec{v}_n \,] D \\
        & = ED .
    \end{split}
\end{equation*}
The columns of $E$ are linearly independent as these are linearly independent eigenvectors of $A$.  Hence $E$ is invertible. Since $AE = ED$, we multiply on the right by $E^{-1}$ and we get
\begin{equation*}
    A = E D E^{-1}.
\end{equation*}
This means that $e^A = E e^D E^{-1}$.  Multiplying the matrix by $t$ we obtain
\begin{equation} \label{matexp:diagfundsol}
    %\mybxbg{~~
    e^{tA} = Ee^{tD}E^{-1} = E
    \begin{bmatrix}
        e^{\lambda_1 t} & 0 & \cdots & 0 \\
        0 & e^{\lambda_2 t} & \cdots & 0 \\
        \vdots & \vdots & \ddots & \vdots \\
        0 & 0 & \cdots & e^{\lambda_n t}
    \end{bmatrix} 
    E^{-1} .
    %~~}
\end{equation}
The formula \eqref{matexp:diagfundsol}, therefore, gives the formula for computing a fundamental matrix solution $e^{tA}$ for the system ${\vec{x}}' = A \vec{x}$, in the case where we have $n$ linearly independent eigenvectors.

This computation still works when the eigenvalues and eigenvectors are complex, though then you have to compute with complex numbers.  It is clear from the definition that if $A$ is real, then $e^{tA}$ is real.  So you will only need complex numbers in the computation and not for the result.  You may need to apply \hyperref[eulersformula]{Euler's formula} to simplify the result.  If simplified properly, the final matrix will not have any complex numbers in it.

\begin{example}
    Compute a fundamental matrix solution using the matrix exponential for the system
    \begin{equation*}
        \begin{bmatrix}
            x \\ 
            y
        \end{bmatrix} '
        =
        \begin{bmatrix}
            1 & 2 \\
            2 & 1
        \end{bmatrix}
        \begin{bmatrix}
            x \\ 
            y
        \end{bmatrix} .
    \end{equation*}
    Then compute the particular solution for the initial conditions $x(0) = 4$ and $y(0) = 2$.
    
    Let $A$ be the coefficient matrix 
    $\left[ 
        \begin{smallmatrix}
            1 & 2 \\
            2 & 1
        \end{smallmatrix} 
    \right]$.
    We first compute (exercise) that the eigenvalues are 3 and $-1$ and corresponding eigenvectors are
    $\left[ \begin{smallmatrix} 1 \\ 1 \end{smallmatrix} \right]$ and
    $\left[ \begin{smallmatrix} 1 \\ -1 \end{smallmatrix} \right]$.
    Hence the diagonalization of $A$ is
    \begin{equation*}
        \underbrace{
        \begin{bmatrix}
            1 & 2 \\
            2 & 1
        \end{bmatrix}
        }_{A} =
        \underbrace{
        \begin{bmatrix}
            1 & 1 \\
            1 & -1
        \end{bmatrix}
        }_{E}
        \underbrace{
        \begin{bmatrix}
            3 & 0 \\
            0 & -1
        \end{bmatrix}
        }_{D}
        \underbrace{
        \begin{bmatrix}
            1 & 1 \\
            1 & -1
        \end{bmatrix}^{-1}
        }_{E^{-1}} .
    \end{equation*}
    We write
    \begin{equation*}
        \begin{split}
            e^{t A} = E e^{tD} E^{-1}
            & =
            \begin{bmatrix}
                1 & 1 \\
                1 & -1
            \end{bmatrix}
            \begin{bmatrix}
                e^{3t} & 0 \\
                0 & e^{-t}
            \end{bmatrix}
            \begin{bmatrix}
                1 & 1 \\
                1 & -1
            \end{bmatrix}^{-1}
            \\
            & =
            \begin{bmatrix}
                1 & 1 \\
                1 & -1
            \end{bmatrix}
            \begin{bmatrix}
                e^{3t} & 0 \\
                0 & e^{-t}
            \end{bmatrix}
            \frac{-1}{2}
            \begin{bmatrix}
                -1 & -1 \\
                -1 & 1
            \end{bmatrix}
            \\
            & = \frac{-1}{2}
            \begin{bmatrix}
                e^{3t} & e^{-t} \\
                e^{3t} & -e^{-t}
            \end{bmatrix}
            \begin{bmatrix}
                -1 & -1 \\
                -1 & 1
            \end{bmatrix}
            \\
            & = \frac{-1}{2}
            \begin{bmatrix}
                -e^{3t}-e^{-t} & -e^{3t}+e^{-t} \\
                -e^{3t}+e^{-t} & -e^{3t}-e^{-t}
            \end{bmatrix}
             =
            \begin{bmatrix}
                \frac{e^{3t}+e^{-t}}{2} & \frac{e^{3t}-e^{-t}}{2} \\
                \frac{e^{3t}-e^{-t}}{2} & \frac{e^{3t}+e^{-t}}{2}
            \end{bmatrix} .
        \end{split}
    \end{equation*}
    
    The initial conditions are $x(0) = 4$ and $y(0) = 2$.  Hence, by the property that $e^{0A} = I$ we find that the particular solution we are looking for is $e^{tA} \vec{b}$ where $\vec{b}$ is 
    $\left[
        \begin{smallmatrix} 
            4 \\ 
            2 
        \end{smallmatrix} 
    \right]$.
    Then the particular solution we are looking for is
    \begin{equation*}
        \begin{bmatrix}
            x \\ 
            y
        \end{bmatrix}
        =
        \begin{bmatrix}
            \frac{e^{3t}+e^{-t}}{2} & \frac{e^{3t}-e^{-t}}{2} \\
            \frac{e^{3t}-e^{-t}}{2} & \frac{e^{3t}+e^{-t}}{2}
        \end{bmatrix}
        \begin{bmatrix}
            4 \\ 
            2
        \end{bmatrix}
        =
        \begin{bmatrix}
            2e^{3t}+2e^{-t} + e^{3t}-e^{-t} \\
            2e^{3t}-2e^{-t} + e^{3t}+e^{-t}
        \end{bmatrix}
        =
        \begin{bmatrix}
            3e^{3t}+e^{-t} \\
            3e^{3t}-e^{-t}
        \end{bmatrix} .
    \end{equation*}
\end{example}

\subsection{Fundamental matrix solutions}

We note that if you can compute a fundamental matrix solution in a different way, you can use this to find the matrix exponential $e^{tA}$. A fundamental matrix solution of a system of ODEs is not unique.  The exponential is the fundamental matrix solution with the property that for $t=0$ we get the identity matrix.  So we must find the right fundamental matrix solution.  Let $X$ be any fundamental matrix solution to ${\vec{x}}' = A \vec{x}$.  Then we claim
\begin{equation*}
    e^{tA} = X(t) \left[ X(0) \right]^{-1} .
\end{equation*}
Clearly, if we plug $t=0$ into $X(t) \left[ X(0) \right]^{-1}$ we get the identity. We can multiply a fundamental matrix solution on the right by any constant invertible matrix and we still get a fundamental matrix solution. All we are doing is changing what are the arbitrary constants in the general solution $\vec{x}(t) = X(t)\, \vec{c}$.

\subsection{Approximations}

If you think about it, the computation of any fundamental matrix solution $X$ using the eigenvalue method is just as difficult as the computation of $e^{tA}$. So perhaps we did not gain much by this new tool.  However, the Taylor series expansion actually gives us a way to approximate solutions, which the eigenvalue method did not.

The simplest thing we can do is to just compute the series up to a certain number of terms.  There 
are better ways to approximate the exponential%
\footnote{C.\ Moler and C.F.\ Van Loan, \emph{Nineteen Dubious Ways to Compute the Exponential of a Matrix, Twenty-Five Years Later}, SIAM Review 45 (1), 2003, 3--49}.
In many cases however, few terms of the Taylor series give a reasonable approximation for the exponential and may suffice for the application.  For example, let us compute the first 4 terms of the series for the matrix 
$A = 
\left[ 
    \begin{smallmatrix}
        1 & 2 \\
        2 & 1
    \end{smallmatrix} 
\right]$.
\begin{multline*}
    e^{tA} \approx I + tA + \frac{t^2}{2}A^2 + \frac{t^3}{6}A^3 = I + t
    \begin{bmatrix}
        1 & 2 \\
        \noalign{\smallskip}
        2 & 1
    \end{bmatrix}
    + t^2
    \begin{bmatrix}
        \frac{5}{2} & 2 \\
        \noalign{\smallskip}
        2 & \frac{5}{2}
    \end{bmatrix}
    + t^3
    \begin{bmatrix}
        \frac{13}{6} & \frac{7}{3} \\
        \noalign{\smallskip}
        \frac{7}{3} & \frac{13}{6}
    \end{bmatrix}
    = \\
    =
    \begin{bmatrix}
        1 + t + \frac{5}{2}\, t^2 + \frac{13}{6}\, t^3 & 2\,t + 2\, t^2   + \frac{7}{3}\, t^3 \\
        \noalign{\smallskip} 2\,t + 2\, t^2   + \frac{7}{3}\, t^3 & 1 + t + \frac{5}{2}\, t^2 + \frac{13}{6}\, t^3
    \end{bmatrix} .
\end{multline*}
Just like the scalar version of the Taylor series approximation, the approximation will be better for small $t$ and worse for larger $t$.  For larger $t$, we will generally have to compute more terms. Let us see how we stack up against the real solution with $t=0.1$.  The approximate solution is approximately (rounded to 8 decimal places)

\begin{equation*}
    e^{0.1\,A} \approx I + 0.1\,A + \frac{0.1^2}{2}A^2 + \frac{0.1^3}{6}A^3 =
    \begin{bmatrix}
        1.12716667 & 0.22233333 \\
        0.22233333 & 1.12716667 \\
    \end{bmatrix} .
\end{equation*}
And plugging $t=0.1$ into the real solution (rounded to 8 decimal places) we get
\begin{equation*}
    e^{0.1\,A} = 
    \begin{bmatrix}
        1.12734811 & 0.22251069 \\
        0.22251069 & 1.12734811
    \end{bmatrix} .
\end{equation*}
Not bad at all!  Although if we take the same approximation for $t=1$ we get
\begin{equation*}
    I + A + \frac{1}{2}A^2 + \frac{1}{6}A^3 =
    \begin{bmatrix}
        6.66666667 & 6.33333333 \\
        6.33333333 & 6.66666667
    \end{bmatrix} ,
\end{equation*}
while the real value is (again rounded to 8 decimal places)
\begin{equation*}
    e^{A} =
    \begin{bmatrix}
        10.22670818           & \phantom{0}9.85882874 \\
        \phantom{0}9.85882874 & 10.22670818
    \end{bmatrix} .
\end{equation*}
So the approximation is not very good once we get up to $t=1$.  To get a good approximation at $t=1$ (say up to 2 decimal places) we would need to go up to the ${11}^{\text{th}}$ power (exercise).

\subsection{Non-Homogeneous Systems}
\subsubsection{Integrating factor}

Now that we have matrix exponentials, we can try to use them to help us solve non-homogeneous systems of differential equations. First, let's recall what we did for first order equations. If we have an equation of the form 
\begin{equation*}
    x'(t) + p x(t) = f(t)
\end{equation*}
where will assume that $p$ is constant (even though it doesn't have to be). We would go about solving this problem by multiplying both sides of the equation by $e^{pt}$, writing the left-hand side as a product rule, integrating both sides, and solving. 

With matrix exponentials, we can do exactly the same thing with first order systems. Let us focus on the nonhomogeneous first order equation
\begin{equation*}
    {\vec{x}}'(t) = A\vec{x}(t) + \vec{f}(t) ,
\end{equation*}
where $A$ is a constant matrix.  The method we look at here is the \emph{integrating factor method\index{integrating factor method!for systems}}. For simplicity we rewrite the equation as
\begin{equation*}
    {\vec{x}}'(t) + P \vec{x}(t) = \vec{f}(t) ,
\end{equation*}
where $P = -A$. We multiply both sides of the equation by $e^{tP}$ (being mindful that we are dealing with matrices that may not commute) to obtain
\begin{equation*}
    e^{tP}{\vec{x}}'(t) + e^{tP}P\vec{x}(t) = e^{tP}\vec{f}(t) .
\end{equation*}
We notice that $P e^{tP} = e^{tP} P$.  This fact follows by writing down the series definition of $e^{tP}$:
\begin{equation*}
    \begin{split}
        P e^{tP} & = P \left( I + tP + \frac{1}{2} {(tP)}^2 + \cdots \right) = P + tP^2 + \frac{1}{2} t^2P^3 + \cdots = \\
        & = \left( I + tP + \frac{1}{2} {(tP)}^2 + \cdots \right) P = e^{tP} P . 
    \end{split}
\end{equation*}
So $\frac{d}{dt} \left( e^{tP} \right) = P e^{tP} = e^{tP} P$.  The product rule says
\begin{equation*}
    \frac{d}{dt} \Bigl( e^{tP} \vec{x}(t) \Bigr) = e^{tP}{\vec{x}}'(t) + e^{tP}P\vec{x}(t),
\end{equation*}
and so
\begin{equation*}
    \frac{d}{dt} \Bigl( e^{tP} \vec{x}(t) \Bigr) = e^{tP}\vec{f}(t) .
\end{equation*}
We can now integrate.  That is, we integrate each component of the vector separately
\begin{equation*}
    e^{tP} \vec{x}(t) = \int e^{tP}\vec{f}(t) \, dt + \vec{c} .
\end{equation*}
In \exerciseref{matexp:expinvex}, you will compute and verify that ${(e^{tP})}^{-1} = e^{-tP}$. Therefore, we obtain
\begin{equation*}
    \vec{x}(t) = e^{-tP} \int e^{tP}\vec{f}(t) \, dt + e^{-tP} \vec{c} .
\end{equation*}

Perhaps it is better understood as a definite integral.  In this case it will be easy to also solve for the initial conditions. Consider the equation with initial conditions
\begin{equation*}
    {\vec{x}}'(t) + P\vec{x}(t) = \vec{f}(t) , \qquad \vec{x}(0) = \vec{b} .
\end{equation*}
The solution can then be written as
\begin{equation} \label{nhsys:intfacsoleq}
    %\mybxbg{~~
    \vec{x}(t) = e^{-tP} \int_0^t e^{sP}\vec{f}(s) \, ds + e^{-tP} \vec{b} .
    %~~}
\end{equation}
Again, the integration means that each component of the vector  $e^{sP}\vec{f}(s)$ is integrated separately. It is not hard to see that \eqref{nhsys:intfacsoleq} really does satisfy the initial condition $\vec{x}(0) = \vec{b}$.
\begin{equation*}
    \vec{x}(0) = e^{-0P} \int_0^0 e^{sP}\vec{f}(s) \, ds + e^{-0P} \vec{b} = I \vec{b} = \vec{b} .
\end{equation*}

\begin{example}
    Suppose that we have the system
    \begin{align*}
        x_1' + 5x_1 - 3x_2 &= e^t , \\
        x_2' + 3x_1 - x_2 &= 0 ,
    \end{align*}
    with initial conditions $x_1(0) = 1, x_2(0) = 0$.
\end{example}

\begin{exampleSol}
    Let us write the system as
    \begin{equation*}
        {\vec{x}}' +
        \begin{bmatrix} 
            5 & -3 \\ 
            3 & -1 
        \end{bmatrix}
        \vec{x} =
        \begin{bmatrix} 
            e^t \\ 
            0 
        \end{bmatrix} ,
        \qquad \vec{x}(0) = 
        \begin{bmatrix} 
            1 \\ 
            0 
        \end{bmatrix} .
    \end{equation*}
    The matrix
    $P = \left[
        \begin{smallmatrix} 
            5 & -3 \\ 
            3 & -1 
        \end{smallmatrix} 
    \right]$
    has a doubled eigenvalue 2 with defect 1, and we leave it as an exercise to double check we computed $e^{tP}$ correctly.  Once we have $e^{tP}$, we find $e^{-tP}$, simply by negating $t$.
    \begin{equation*}
        e^{tP} = 
        \begin{bmatrix}
            (1+3t)\,e^{2t} & -3te^{2t} \\
            3te^{2t} & (1-3t)\,e^{2t}
        \end{bmatrix}
        , \qquad e^{-tP} = 
        \begin{bmatrix}
            (1-3t)\,e^{-2t} & 3te^{-2t} \\
            -3te^{-2t} & (1+3t)\,e^{-2t}
        \end{bmatrix}.
    \end{equation*}
    Instead of computing the whole formula at once, let us do it in stages. First
    \begin{equation*}
        \begin{split}
            \int_0^t e^{sP}\vec{f}(s) \, ds & = \int_0^t
            \begin{bmatrix}
                (1+3s)\,e^{2s} & -3se^{2s} \\
                3se^{2s} & (1-3s)\,e^{2s}
            \end{bmatrix}
            \begin{bmatrix}
                e^{s} \\ 
                0 
            \end{bmatrix} \, ds \\
            & = \int_0^t
            \begin{bmatrix}
                (1+3s)\,e^{3s} \\
                3se^{3s}
            \end{bmatrix}
            \, ds \\
            &=
            \begin{bmatrix}
                \int_0^t (1+3s)\,e^{3s} \,ds \\
                \int_0^t 3se^{3s} \,ds
            \end{bmatrix}
            \\
            & =
            \begin{bmatrix}
                t e^{3t} \\
                \frac{(3t-1) \,e^{3t} + 1}{3}
            \end{bmatrix} %.
            \qquad \qquad \text{(used integration by parts).}
        \end{split}
    \end{equation*}
    Then
    \begin{equation*}
        \begin{split}
            \vec{x}(t)
            & = e^{-tP} \int_0^t e^{sP}\vec{f}(s) \, ds + e^{-tP} \vec{b} \\
            & =
            \begin{bmatrix}
                (1-3t)\,e^{-2t} & 3te^{-2t} \\
                -3te^{-2t} & (1+3t)\,e^{-2t}
            \end{bmatrix}
            \begin{bmatrix}
                t e^{3t} \\
                \frac{(3t-1) \,e^{3t} + 1}{3}
            \end{bmatrix}
            +
            \begin{bmatrix}
                (1-3t)\,e^{-2t} & 3te^{-2t} \\
                -3te^{-2t} & (1+3t)\,e^{-2t}
            \end{bmatrix}
            \begin{bmatrix} 
                1 \\ 
                0 
            \end{bmatrix} \\
            & =
            \begin{bmatrix}
                te^{-2t} \\
                -\frac{e^t}{3}+\left( \frac{1}{3} + t \right) \, e^{-2t}
            \end{bmatrix}
            +
            \begin{bmatrix}
                (1-3t)\,e^{-2t} \\
                -3te^{-2t}
            \end{bmatrix} \\
            & =
            \begin{bmatrix}
                (1-2t)\,e^{-2t} \\
                -\frac{e^t}{3}+\left( \frac{1}{3} -2 t \right) \, e^{-2t}
            \end{bmatrix} .
        \end{split}
    \end{equation*}
    Phew!
    
    Let us check that this really works.
    \begin{equation*}
        x_1' + 5 x_1 - 3x_2 = (4te^{-2t} - 4 e^{-2t}) + 5 (1-2t)\,e^{-2t} +e^t-( 1 -6 t ) \, e^{-2t} = e^t .
    \end{equation*}
    Similarly (exercise) $x_2' + 3 x_1 - x_2 = 0$.   The initial conditions are also satisfied (exercise).
\end{exampleSol}

For systems, the integrating factor method only works if $P$ does not depend on $t$, that is, $P$ is constant.  The problem is that in general
\begin{equation*}
    \frac{d}{dt} \left[ e^{\int P(t)\,dt} \right] \not= P(t) \, e^{\int P(t)\,dt} ,
\end{equation*}
because matrix multiplication is not commutative.



\end{document}

