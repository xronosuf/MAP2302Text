\documentclass{ximera}
%\auor{Matthew Charnley and Jason Nowell}
\usepackage[margin=1.5cm]{geometry}
\usepackage{indentfirst}
\usepackage{sagetex}
\usepackage{lipsum}
\usepackage{amsmath}
\usepackage{mathrsfs}


%%% Random packages added without verifying what they are really doing - just to get initial compile to work.
\usepackage{tcolorbox}
\usepackage{hypcap}
\usepackage{booktabs}%% To get \toprule,\midrule,\bottomrule etc.
\usepackage{nicefrac}
\usepackage{caption}
\usepackage{units}

% This is my modified wrapfig that doesn't use intextsep
\usepackage{mywrapfig}
\usepackage{import}



%%% End to random added packages.


\graphicspath{
    {./figures/}
    {./../figures/}
    {./../../figures/}
}
\renewcommand{\log}{\ln}%%%%
\DeclareMathOperator{\arcsec}{arcsec}
%% New commands


%%%%%%%%%%%%%%%%%%%%
% New Conditionals %
%%%%%%%%%%%%%%%%%%%%


% referencing
\makeatletter
    \DeclareRobustCommand{\myvref}[2]{%
      \leavevmode%
      \begingroup
        \let\T@pageref\@pagerefstar
        \hyperref[{#2}]{%
	  #1~\ref*{#2}%
        }%
        \vpageref[\unskip]{#2}%
      \endgroup
    }%

    \DeclareRobustCommand{\myref}[2]{%
      \leavevmode%
      \begingroup
        \let\T@pageref\@pagerefstar
        \hyperref[{#2}]{%
	  #1~\ref*{#2}%
        }%
      \endgroup
    }%
\makeatother

\newcommand{\figurevref}[1]{\myvref{Figure}{#1}}
\newcommand{\figureref}[1]{\myref{Figure}{#1}}
\newcommand{\tablevref}[1]{\myvref{Table}{#1}}
\newcommand{\tableref}[1]{\myref{Table}{#1}}
\newcommand{\chapterref}[1]{\myref{chapter}{#1}}
\newcommand{\Chapterref}[1]{\myref{Chapter}{#1}}
\newcommand{\appendixref}[1]{\myref{appendix}{#1}}
\newcommand{\Appendixref}[1]{\myref{Appendix}{#1}}
\newcommand{\sectionref}[1]{\myref{\S}{#1}}
\newcommand{\subsectionref}[1]{\myref{subsection}{#1}}
\newcommand{\subsectionvref}[1]{\myvref{subsection}{#1}}
\newcommand{\exercisevref}[1]{\myvref{Exercise}{#1}}
\newcommand{\exerciseref}[1]{\myref{Exercise}{#1}}
\newcommand{\examplevref}[1]{\myvref{Example}{#1}}
\newcommand{\exampleref}[1]{\myref{Example}{#1}}
\newcommand{\thmvref}[1]{\myvref{Theorem}{#1}}
\newcommand{\thmref}[1]{\myref{Theorem}{#1}}


\renewcommand{\exampleref}[1]{ {\color{red} \bfseries Normally a reference to a previous example goes here.}}
\renewcommand{\figurevref}[1]{ {\color{red} \bfseries Normally a reference to a previous figure goes here.}}
\renewcommand{\tablevref}[1]{ {\color{red} \bfseries Normally a reference to a previous table goes here.}}
\renewcommand{\Appendixref}[1]{ {\color{red} \bfseries Normally a reference to an Appendix goes here.}}
\renewcommand{\exercisevref}[1]{ {\color{red} \bfseries Normally a reference to a previous exercise goes here.}}



\newcommand{\R}{\mathbb{R}}

%% Example Solution Env.
\def\beginSolclaim{\par\addvspace{\medskipamount}\noindent\hbox{\bf Solution:}\hspace{0.5em}\ignorespaces}
\def\endSolclaim{\par\addvspace{-1em}\hfill\rule{1em}{0.4pt}\hspace{-0.4pt}\rule{0.4pt}{1em}\par\addvspace{\medskipamount}}
\newenvironment{exampleSol}[1][]{\beginSolclaim}{\endSolclaim}

%% General figure formating from original book.
\newcommand{\mybeginframe}{%
\begin{tcolorbox}[colback=white,colframe=lightgray,left=5pt,right=5pt]%
}
\newcommand{\myendframe}{%
\end{tcolorbox}%
}

%%% Eventually return and fix this to make matlab code work correctly.
%% Define the matlab environment as another code environment
%\newenvironment{matlab}
%{% Begin Environment Code
%{ \centering \bfseries Matlab Code }
%\begin{code}
%}% End of Begin Environment Code
%{% Start of End Environment Code
%\end{code}
%}% End of End Environment Code


% this one should have a caption, first argument is the size
\newenvironment{mywrapfig}[2][]{
 \wrapfigure[#1]{r}{#2}
 \mybeginframe
 \centering
}{%
 \myendframe
 \endwrapfigure
}

% this one has no caption, first argument is size,
% the second argument is a larger size used for HTML (ignored by latex)
\newenvironment{mywrapfigsimp}[3][]{%
 \wrapfigure[#1]{r}{#2}%
 \centering%
}{%
 \endwrapfigure%
}
\newenvironment{myfig}
    {%
    \begin{figure}[h!t]
        \mybeginframe%
        \centering%
    }
    {%
        \myendframe
    \end{figure}%
    }


% graphics include
\newcommand{\diffyincludegraphics}[3]{\includegraphics[#1]{#3}}
\newcommand{\myincludegraphics}[3]{\includegraphics[#1]{#3}}
\newcommand{\inputpdft}[1]{\subimport*{../figures/}{#1.pdf_t}}


%% Not sure what these even do? They don't seem to actually work... fun!
%\newcommand{\mybxbg}[1]{\tcboxmath[colback=white,colframe=black,boxrule=0.5pt,top=1.5pt,bottom=1.5pt]{#1}}
%\newcommand{\mybxsm}[1]{\tcboxmath[colback=white,colframe=black,boxrule=0.5pt,left=0pt,right=0pt,top=0pt,bottom=0pt]{#1}}
\newcommand{\mybxsm}[1]{#1}
\newcommand{\mybxbg}[1]{#1}

%%% Something about tasks for practice/hw?
\usepackage{tasks}
\usepackage{footnote}
\makesavenoteenv{tasks}


%% For pdf only?
\newcommand{\diffypdfversion}[1]{#1}


%% Kill ``cite'' and go back later to fix it.
\renewcommand{\cite}[1]{}


%% Currently we can't really use index or its derivatives. So we are gonna kill them off.
\renewcommand{\index}[1]{}
\newcommand{\myindex}[1]{#1}






\title{Steady state temperature and the Laplacian}
\author{Matthew Charnley and Jason Nowell}


\outcome{Relate the heat equation independent of time to the Laplace equation}
\outcome{Use separation of variables to solve the Laplace equation on rectangular regions.}


\begin{document}
\begin{abstract}
    We discuss Steady state temperature and the Laplacian
\end{abstract}
\maketitle


\label{dirich:section}


% \sectionnotes{Verbatim from Lebl}

% \sectionnotes{1 lecture\EPref{, \S9.7 in \cite{EP}}\BDref{,
% \S10.8 in \cite{BD}}}

Consider an insulated wire, a plate, or a 3-dimensional object. We apply certain fixed temperatures on the ends of the wire, the edges of the plate, or on all sides of the 3-dimensional object.  We wish to find out what is the \emph{\myindex{steady state temperature}} distribution.  That is, we wish to know what will be the temperature after long enough period of time.

We are really looking for a solution to the heat equation that is not dependent on time.  Let us first solve the problem in one space variable.  We are looking for a function $u$ that satisfies
\begin{equation*}
    u_t = k u_{xx} ,
\end{equation*}
but such that $u_t = 0$ for all $x$ and $t$.  Hence, we are looking for a function of $x$ alone that satisfies $u_{xx} = 0$.  It is easy to solve this equation by integration and we see that $u = Ax+B$ for some constants $A$ and $B$.

Consider an insulated wire where we apply constant temperature $T_1$ at one end (say where $x=0$) and $T_2$ on the other end (at $x=L$ where $L$ is the length of the wire).  Our steady state solution is
\begin{equation*}
    u(x) = \frac{T_2-T_1}{L} x + T_1 .
\end{equation*}
This solution agrees with our common sense intuition with how the heat should be distributed in the wire.  So in one dimension, the steady state solutions are basically just straight lines.

Things are more complicated in two or more space dimensions.  Let us restrict to two space dimensions for simplicity.  The heat equation in two space variables is
\begin{equation} \label{dirich:heateq}
    u_t = k(u_{xx} + u_{yy}) ,
\end{equation}
or more commonly written as $u_t = k \Delta u$ or $u_t = k \nabla^2 u$.  Here the $\Delta$ and $\nabla^2$ symbols mean $\frac{\partial^2}{\partial x^2} + \frac{\partial^2}{\partial y^2}$.  We will use $\Delta$ from now on.  The reason for using such a notation is that you can define $\Delta$ to be the right thing for any number of space dimensions and then the heat equation is always $u_t = k \Delta u$.  The operator $\Delta$ is called the \emph{\myindex{Laplacian}}.

OK\@, now that we have notation out of the way, let us see what does an equation for the steady state solution look like.  We are looking for a solution to \eqref{dirich:heateq} that does not depend on $t$, or in other words $u_t = 0$.  Hence we are looking for a function $u(x,y)$ such that
\begin{equation*}
    %\mybxbg{~~
    \Delta u = u_{xx} + u_{yy} = 0 .
    %~~}
\end{equation*}
This equation is called the \emph{\myindex{Laplace equation}}%
\footnote{Named after the French mathematician \href{https://en.wikipedia.org/wiki/Laplace}{Pierre-Simon, marquis de Laplace} (1749--1827).}, 
and is an example of an elliptic equation. Solutions to the Laplace equation are called \emph{harmonic functions\index{harmonic function}} and have many nice properties and applications far beyond the steady state heat problem. One of these main applications is in electrostatics, as the electric potential $V$ in a region also solves the Laplace equation
\[ 
    \Delta V = 0. 
\]

Harmonic functions in two variables are no longer just linear (plane graphs).  For example, you can check that the functions $x^2-y^2$ and $xy$ are harmonic.  However, note that if $u_{xx}$ is positive, $u$ is concave up in the $x$ direction, then $u_{yy}$ must be negative and $u$ must be concave down in the $y$ direction.  A harmonic function can never have any ``hilltop'' or ``valley'' on the graph.  This observation is consistent with our intuitive idea of steady state heat distribution; the hottest or coldest spot will not be inside.

Commonly the Laplace equation is part of a so-called \emph{\myindex{Dirichlet problem}}%
\footnote{Named after the German mathematician \href{https://en.wikipedia.org/wiki/Dirichlet}{Johann Peter Gustav Lejeune Dirichlet} (1805--1859).}.
That is, we have a region in the $xy$-plane and we specify certain values along the boundaries of the region.  We then try to find a solution $u$ to the Laplace equation defined on this region such that $u$ agrees with the values we specified on the boundary.

In this section we consider a rectangular region.  For simplicity we specify boundary values to be zero at 3 of the four edges and only specify an arbitrary function at one edge.  As we still have the principle of superposition, we can use this simpler solution to derive the general solution for arbitrary boundary values by solving 4 different problems, one for each edge, and adding those solutions together. This setup is left as an exercise.

We wish to solve the following problem.  Let $h$ and $w$ be the height and width of our rectangle, with one corner at the origin and lying in the first quadrant.

% FIXME: numbering does not work now since there are no hard coded numbers
% here!
%mbx <mdn>
%mbx   <mrow xml:id="dirich_eq1" number="%MBXEQNNUMBER%">
%mbx     &amp; \Delta u = 0 ,
%mbx   </mrow>
%mbx   <mrow xml:id="dirich_eq2" number="%MBXEQNNUMBER%">
%mbx     &amp; u(0,y) = 0 \quad \text{for } 0 &lt; y &lt; h,
%mbx   </mrow>
%mbx   <mrow xml:id="dirich_eq3" number="%MBXEQNNUMBER%">
%mbx     &amp; u(x,h) = 0 \quad \text{for } 0 &lt; x &lt; w,
%mbx   </mrow>
%mbx   <mrow xml:id="dirich_eq4" number="%MBXEQNNUMBER%">
%mbx     &amp; u(w,y) = 0 \quad \text{for } 0 &lt; y &lt; h,
%mbx   </mrow>
%mbx   <mrow xml:id="dirich_eq5" number="%MBXEQNNUMBER%">
%mbx     &amp; u(x,0) = f(x) \quad \text{for } 0 &lt; x &lt; w.
%mbx   </mrow>
%mbx </mdn>

\begin{center}
    %mbxSTARTIGNORE
    \begin{minipage}[b]{2.8in}
        \vspace{\fill}
        \begin{align}
            & \Delta u = 0 , & &  \label{dirich:eq1} \\
            & u(0,y) = 0 & & \text{for }  0 < y < h,\label{dirich:eq2} \\
            & u(x,h) = 0 & & \text{for }  0 < x < w,\label{dirich:eq3} \\
            & u(w,y) = 0 & & \text{for }  0 < y < h,\label{dirich:eq4} \\
            & u(x,0) = f(x) & & \text{for }  0 < x < w.\label{dirich:eq5}
        \end{align}
        \vspace{\fill}
    \end{minipage}
    \qquad
    %mbxENDIGNORE
    \input{figures/dirichsetup.pdf_t}
    %mbxSTARTIGNORE
    \qquad
    %mbxENDIGNORE
\end{center}

The method we apply is separation of variables.  Again, we will come up with enough building-block solutions satisfying all the homogeneous boundary conditions (all conditions except \eqref{dirich:eq5}).  We notice that superposition still works for the equation and all the homogeneous conditions. Therefore, we can use the Fourier series for $f(x)$ to solve the problem as before.

We try $u(x,y) = X(x)Y(y)$.  We plug $u$ into the equation to get
\begin{equation*}
    X''Y + XY'' = 0 .
\end{equation*}
We put the $X$s on one side and the $Y$s on the other to get
\begin{equation*}
    - \frac{X''}{X} = \frac{Y''}{Y} .
\end{equation*}
The left-hand side only depends on $x$ and the right-hand side only depends on $y$.  Therefore, there is some constant $\lambda$ such that $\lambda = \frac{-X''}{X} = \frac{Y''}{Y}$. And we get two equations
\begin{align*}
    & X'' + \lambda X = 0 , \\
    & Y'' - \lambda Y = 0 .
\end{align*}
Furthermore, the homogeneous boundary conditions imply that $X(0) = X(w) = 0$ and $Y(h) = 0$.  Taking the equation for $X$ we have already seen that we have a nontrivial solution if and only if $\lambda = \lambda_n = \frac{n^2 \pi^2}{w^2}$ and the solution is a multiple of
\begin{equation*}
    X_n(x) = \sin \left( \frac{n \pi}{w} x \right) .
\end{equation*}
For these given $\lambda_n$, the general solution for $Y$ (one for each $n$) is
\begin{equation} \label{dirich:Yngensol}
    Y_n(y) = A_n \cosh \left( \frac{n \pi}{w} y \right) + B_n \sinh \left( \frac{n \pi}{w} y \right) .
\end{equation}
We only have one condition on $Y_n$ and hence we can pick one of $A_n$ or $B_n$ to be something convenient. It will be useful to have $Y_n(0) = 1$, so we let $A_n=1$. Setting $Y_n(h) = 0$ and solving for $B_n$ we get that
\begin{equation*}
    B_n = \frac{- \cosh \left( \frac{n \pi h }{w} \right)}%
    {\sinh \left( \frac{n \pi h }{w} \right)} .
\end{equation*}
After we plug the $A_n$ and $B_n$ we into \eqref{dirich:Yngensol} and simplify by using the identity 
$\sinh(\alpha-\beta) = \sinh(\alpha) \cosh(\beta) - \cosh(\alpha) \sinh(\beta)$, 
we find
\begin{equation*}
    Y_n(y) = \frac{\sinh \left( \frac{n \pi (h-y) }{w} \right)}% 
    {\sinh \left( \frac{n \pi h }{w} \right)} .
\end{equation*}
We define $u_n(x,y) = X_n(x)Y_n(y)$. And note that $u_n$ satisfies \eqref{dirich:eq1}--\eqref{dirich:eq4}.


Observe that
\begin{equation*}
    u_n(x,0) = X_n(x)Y_n(0) = \sin \left( \frac{n \pi}{w} x \right) .
\end{equation*}
Suppose
\begin{equation*}
    f(x) =
    %\frac{a_0}{2} +
    \sum_{n=1}^\infty
    %a_n \cos \left( \frac{n \pi x }{w} \right)
    %+
    b_n \sin \left( \frac{n \pi x }{w} \right) .
\end{equation*}
Then we get a solution of \eqref{dirich:eq1}--\eqref{dirich:eq5} of the following form.
\begin{equation*}
    %\mybxbg{
    %~~
    u(x,y) = \sum_{n=1}^\infty
    b_n u_n(x,y) = \sum_{n=1}^\infty
    b_n \sin \left( \frac{n \pi}{w} x \right)\left( \frac{\sinh \left( \frac{n \pi (h-y) }{w} \right)}%
    {\sinh \left( \frac{n \pi h }{w} \right)} \right).
    %~~}
\end{equation*}
As $u_n$ satisfies \eqref{dirich:eq1}--\eqref{dirich:eq4} and any linear combination (finite or infinite) of $u_n$ also satisfies  \eqref{dirich:eq1}--\eqref{dirich:eq4}, then $u$ satisfies \eqref{dirich:eq1}--\eqref{dirich:eq4}. By plugging in $y=0$, we see $u$ satisfies \eqref{dirich:eq5} as well.

\begin{example}
    Take $w=h=\pi$ and let $f(x) = \pi$.  Let us compute the sine series for the function $\pi$ (same as the series for the square wave). For $0 < x < \pi$, we have
    \begin{equation*}
        f(x) = \sum_{\substack{n=1 \\ n \text{ odd}}}^\infty \frac{4}{n} \sin (n x) .
    \end{equation*}
    Therefore the solution $u(x,y)$, see \figurevref{dirichsquareplot:fig}, to the corresponding Dirichlet problem is given as
    \begin{equation*}
        u(x,y) = \sum_{\substack{n=1 \\ n \text{ odd}}}^\infty \frac{4}{n} \sin (n x) \left( \frac{\sinh \bigl( n (\pi-y) \bigr) }{\sinh (n \pi)} \right).
    \end{equation*}
    
    \begin{myfig}
        \capstart
        \diffyincludegraphics{width=5in}{width=7.5in}{dirichsquareplot}
        \caption{Steady state temperature of a square plate, three sides held at zero and one side held at $\pi$.\label{dirichsquareplot:fig}}
    \end{myfig}
\end{example}

This scenario corresponds to the steady state temperature on a square plate of width $\pi$ with 3 sides held at 0 degrees and one side held at $\pi$ degrees. If we have arbitrary initial data on all sides, then we solve four problems, each using one piece of nonhomogeneous data.  Then we use the principle of superposition to add up all four solutions to have a solution to the original problem.

A different way to visualize solutions of the Laplace equation is to take a wire and bend it so that it corresponds to the graph of the temperature above the boundary of your region.  Cut a rubber sheet in the shape of your region---a square in our case---and stretch it fixing the edges of the sheet to the wire. The rubber sheet is a good approximation of the graph of the solution to the Laplace equation with the given boundary data.


\end{document}