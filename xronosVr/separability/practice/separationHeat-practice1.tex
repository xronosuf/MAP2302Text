\documentclass{ximera}

\title{Practice for Heat Equation}

%\auor{Matthew Charnley and Jason Nowell}
\usepackage[margin=1.5cm]{geometry}
\usepackage{indentfirst}
\usepackage{sagetex}
\usepackage{lipsum}
\usepackage{amsmath}
\usepackage{mathrsfs}
\usepackage{tikz}
\usetikzlibrary{matrix}

%%% Random packages added without verifying what they are really doing - just to get initial compile to work.
\usepackage{tcolorbox}
\usepackage{hypcap}
\usepackage{booktabs}%% To get \toprule,\midrule,\bottomrule etc.
\usepackage{caption}
\usepackage{units}
\usepackage{multicol}
\usepackage{hhline}


% This is my modified wrapfig that doesn't use intextsep
\usepackage{mywrapfig}
\usepackage{import}



%%% End to random added packages.


\graphicspath{
    {./}
    {./figures/}
    {./../figures/}
    {./../../figures/}
}
\renewcommand{\log}{\ln}%%%%
\DeclareMathOperator{\arcsec}{arcsec}
%% New commands


%%%%%%%%%%%%%%%%%%%%
% New Conditionals %
%%%%%%%%%%%%%%%%%%%%


% referencing
\makeatletter
    \DeclareRobustCommand{\myvref}[2]{%
      \leavevmode%
      \begingroup
        \let\T@pageref\@pagerefstar
        \hyperref[{#2}]{%
	  #1~\ref*{#2}%
        }%
        \vpageref[\unskip]{#2}%
      \endgroup
    }%

    \DeclareRobustCommand{\myref}[2]{%
      \leavevmode%
      \begingroup
        \let\T@pageref\@pagerefstar
        \hyperref[{#2}]{%
	  #1~\ref*{#2}%
        }%
      \endgroup
    }%
\makeatother

\newcommand{\figurevref}[1]{\myvref{Figure}{#1}}
\newcommand{\figureref}[1]{\myref{Figure}{#1}}
\newcommand{\tablevref}[1]{\myvref{Table}{#1}}
\newcommand{\tableref}[1]{\myref{Table}{#1}}
\newcommand{\chapterref}[1]{\myref{chapter}{#1}}
\newcommand{\Chapterref}[1]{\myref{Chapter}{#1}}
\newcommand{\appendixref}[1]{\myref{appendix}{#1}}
\newcommand{\Appendixref}[1]{\myref{Appendix}{#1}}
\newcommand{\sectionref}[1]{\myref{\S}{#1}}
\newcommand{\subsectionref}[1]{\myref{subsection}{#1}}
\newcommand{\subsectionvref}[1]{\myvref{subsection}{#1}}
\newcommand{\exercisevref}[1]{\myvref{Exercise}{#1}}
\newcommand{\exerciseref}[1]{\myref{Exercise}{#1}}
\newcommand{\examplevref}[1]{\myvref{Example}{#1}}
\newcommand{\exampleref}[1]{\myref{Example}{#1}}
\newcommand{\thmvref}[1]{\myvref{Theorem}{#1}}
\newcommand{\thmref}[1]{\myref{Theorem}{#1}}


\renewcommand{\exampleref}[1]{ {\color{red} \bfseries Normally a reference to a previous example goes here.}}
\renewcommand{\examplevref}[1]{ {\color{red} \bfseries Normally a reference to a previous example goes here.}}
\renewcommand{\figurevref}[1]{ {\color{red} \bfseries Normally a reference to a previous figure goes here.}}
\renewcommand{\tablevref}[1]{ {\color{red} \bfseries Normally a reference to a previous table goes here.}}
\renewcommand{\Appendixref}[1]{ {\color{red} \bfseries Normally a reference to an Appendix goes here.}}
\renewcommand{\exercisevref}[1]{ {\color{red} \bfseries Normally a reference to a previous exercise goes here.}}
\renewcommand{\thmvref}[1]{ {\color{red} \bfseries Normally a reference to a previous theorem goes here.}}
\renewcommand{\subsectionvref}[1]{ {\color{red} \bfseries Normally a reference to a previous subsection goes here.}}



\newcommand{\R}{\mathbb{R}}
\newcommand{\C}{\mathbb{C}}

%% Example Solution Env.
\def\beginSolclaim{\par\addvspace{\medskipamount}\noindent\hbox{\bf Solution:}\hspace{0.5em}\ignorespaces}
\def\endSolclaim{\par\addvspace{-1em}\hfill\rule{1em}{0.4pt}\hspace{-0.4pt}\rule{0.4pt}{1em}\par\addvspace{\medskipamount}}
\newenvironment{exampleSol}[1][]{\beginSolclaim}{\endSolclaim}

%% General figure formating from original book.
\newcommand{\mybeginframe}{%
\begin{tcolorbox}[colback=white,colframe=lightgray,left=5pt,right=5pt]%
}
\newcommand{\myendframe}{%
\end{tcolorbox}%
}

%%% Eventually return and fix this to make matlab code work correctly.
%% Define the matlab environment as another code environment
%\NewEnviron{matlab}{ {\centering\bfseries MATLAB Code} \\ \noexpand{\BODY} }
%\let\beginmatlab\begincode
%\let\endmatlab\endcode
%\newenvironment{matlab}{% Begin Environment Code
%\begin{minipage}{\linewidth}
%\begin{verbatim}
%}% End of Begin Environment Code
%{% Start of End Environment Code
%\end{verbatim}
%\end{minipage}
%}% End of End Environment Code


% this one should have a caption, first argument is the size
\newenvironment{mywrapfig}[2][]{
 \wrapfigure[#1]{r}{#2}
 \mybeginframe
 \centering
}{%
 \myendframe
 \endwrapfigure
}

% this one has no caption, first argument is size,
% the second argument is a larger size used for HTML (ignored by latex)
\newenvironment{mywrapfigsimp}[3][]{%
 \wrapfigure[#1]{r}{#2}%
 \centering%
}{%
 \endwrapfigure%
}
\newenvironment{myfig}
    {%
    \begin{figure}[h!t]
        \mybeginframe%
        \centering%
    }
    {%
        \myendframe
    \end{figure}%
    }


% graphics include
\newcommand{\diffyincludegraphics}[3]{\includegraphics[#1]{#3}}
\newcommand{\myincludegraphics}[3]{\includegraphics[#1]{#3}}
\newcommand{\inputpdft}[1]{\subimport*{../figures/}{#1.pdf_t}}


%% Not sure what these even do? They don't seem to actually work... fun!
%\newcommand{\mybxbg}[1]{\tcboxmath[colback=white,colframe=black,boxrule=0.5pt,top=1.5pt,bottom=1.5pt]{#1}}
%\newcommand{\mybxsm}[1]{\tcboxmath[colback=white,colframe=black,boxrule=0.5pt,left=0pt,right=0pt,top=0pt,bottom=0pt]{#1}}
\newcommand{\mybxsm}[1]{#1}
\newcommand{\mybxbg}[1]{#1}

%%% Something about tasks for practice/hw?
\usepackage{tasks}
\usepackage{footnote}
\makesavenoteenv{tasks}


%% For pdf only?
\newcommand{\diffypdfversion}[1]{#1}


%% Kill ``cite'' and go back later to fix it.
\renewcommand{\cite}[1]{}


%% Currently we can't really use index or its derivatives. So we are gonna kill them off.
\renewcommand{\index}[1]{}
\newcommand{\myindex}[1]{#1}







\begin{document}
\begin{abstract}
Why?
\end{abstract}
\maketitle

\begin{exercise}
    Consider a wire of length 2, with $k=0.001$ and an initial temperature distribution $u(x,0) = 50 x$.  Both ends are embedded in ice (temperature 0).  Find the solution as a series.
\end{exercise}

\begin{exercise}
    Find a series solution of
    \begin{align*}
        & u_t =  u_{xx} , \\
        & u(0,t) = u(1,t) = 0 , \\
        & u(x,0) = 100 \qquad \text{for } \; 0 < x < 1 .
    \end{align*}
\end{exercise}

\begin{exercise}\%
    Find a series solution of
    \begin{align*}
        & u_t =  3 u_{xx} , \\
        & u(0,t) = u(\pi,t) = 0 , \\
        & u(x,0) = 5\sin (x) + 2\sin (5x) \qquad \text{for } \; 0 < x < \pi .
    \end{align*}
\end{exercise}
%\exsol{%
%$u(x,t) = 
%5
%\sin (x)
%\, e^{- 3 t}
%+
%2
%\sin (5x)
%\, e^{-75 t}$
%}

\begin{exercise}
    Find a series solution of
    \begin{align*}
        & u_t =  u_{xx} , \\
        & u_x(0,t) = u_x(\pi,t) = 0 , \\
        & u(x,0) = 3\cos (x) + \cos (3x) \qquad \text{for } \; 0 < x < \pi .
    \end{align*}
\end{exercise}

\begin{exercise}\%
    Find a series solution of
    \begin{align*}
        & u_t =  0.1 u_{xx} , \\
        & u_x(0,t) = u_x(\pi,t) = 0 , \\
        & u(x,0) = 1 + 2\cos (x) \qquad \text{for } \; 0 < x < \pi .
    \end{align*}
\end{exercise}
%\exsol{%
%$u(x,t) = 
%1 + 
%2
%\cos (x)
%\, e^{-0.1 t}$
%}

\begin{exercise} \label{heat:cosexr}
    Find a series solution of
    \begin{align*}
        & u_t = \frac{1}{3} u_{xx} , \\
        & u_x(0,t) = u_x(\pi,t) = 0 , \\
        & u(x,0) = \frac{10x}{\pi} \qquad \text{for } \; 0 < x < \pi .
    \end{align*}
\end{exercise}

\begin{exercise} \label{heat:oneto100exr}
    Find a series solution of
    \begin{align*}
        & u_t =  u_{xx} , \\
        & u(0,t) = 0 , \quad u(1,t) = 100 , \\
        & u(x,0) = \sin (\pi x) \qquad \text{for } \; 0 < x < 1 .
    \end{align*}
    Hint: Use the fact that $u(x,t) = 100 x$ is a solution satisfying $u_t = u_{xx}$, $u(0,t) = 0$, $u(1,t) = 100$.  Then use superposition.
\end{exercise}

\begin{exercise}
    Find the \emph{\myindex{steady state temperature}} solution as a function of $x$ alone, by letting $t \to \infty$ in the solution from exercises \ref{heat:cosexr} and \ref{heat:oneto100exr}. Verify that it satisfies the equation $u_{xx} = 0$.
\end{exercise}

\begin{exercise}%[challenging]
    Suppose that one end of the wire is insulated (say at $x=0$) and the other end is kept at zero temperature.  That is, find a series solution of
    \begin{align*}
        & u_t = k u_{xx} , \\
        & u_x(0,t) = u(L,t) = 0 , \\
        & u(x,0) = f(x) \qquad \text{for } \; 0 < x < L .
    \end{align*}
    Express any coefficients in the series by integrals of $f(x)$.
\end{exercise}

\begin{exercise}%[challenging]
    Suppose that the wire is circular and insulated, so there are no ends. You can think of this as simply connecting the two ends and making sure the solution matches up at the ends. That is, find a series solution of
    \begin{align*}
        & u_t = k u_{xx} , \\
        & u(0,t) = u(L,t) , \qquad u_x(0,t) = u_x(L,t) , \\
        & u(x,0) = f(x) \qquad \text{for } \; 0 < x < L .
    \end{align*}
    Express any coefficients in the series by integrals of $f(x)$.
\end{exercise}

\begin{exercise}
    Consider a wire insulated on both ends, $L=1$, $k=1$, and $u(x,0) = \cos^2(\pi x)$.
    \begin{tasks}
        \task Find the solution $u(x,t)$.  Hint: a trig identity.
        \task Find the average temperature.
        \task Initially the temperature variation is 1 (maximum minus the minimum). Find the time when the variation is $\nicefrac{1}{2}$.
    \end{tasks}
\end{exercise}

\begin{exercise}\%
    Suppose that the temperature on the wire is fixed at $0$ at the ends, $L=1$, $k=1$, and $u(x,0) = 100\sin(2 \pi x)$.
    \begin{tasks}
        \task What is the temperature at $x = \nicefrac{1}{2}$ at any time.
        \task What is the maximum and the minimum temperature on the wire at $t=0$.
        \task At what time is the maximum temperature on the wire exactly one half of the initial maximum at $t=0$.
    \end{tasks}
\end{exercise}



\end{document}