\documentclass{ximera}

\title{Practice for One Dim Wave}

%\auor{Matthew Charnley and Jason Nowell}
\usepackage[margin=1.5cm]{geometry}
\usepackage{indentfirst}
\usepackage{sagetex}
\usepackage{lipsum}
\usepackage{amsmath}
\usepackage{mathrsfs}


%%% Random packages added without verifying what they are really doing - just to get initial compile to work.
\usepackage{tcolorbox}
\usepackage{hypcap}
\usepackage{booktabs}%% To get \toprule,\midrule,\bottomrule etc.
\usepackage{nicefrac}
\usepackage{caption}
\usepackage{units}

% This is my modified wrapfig that doesn't use intextsep
\usepackage{mywrapfig}
\usepackage{import}



%%% End to random added packages.


\graphicspath{
    {./figures/}
    {./../figures/}
    {./../../figures/}
}
\renewcommand{\log}{\ln}%%%%
\DeclareMathOperator{\arcsec}{arcsec}
%% New commands


%%%%%%%%%%%%%%%%%%%%
% New Conditionals %
%%%%%%%%%%%%%%%%%%%%


% referencing
\makeatletter
    \DeclareRobustCommand{\myvref}[2]{%
      \leavevmode%
      \begingroup
        \let\T@pageref\@pagerefstar
        \hyperref[{#2}]{%
	  #1~\ref*{#2}%
        }%
        \vpageref[\unskip]{#2}%
      \endgroup
    }%

    \DeclareRobustCommand{\myref}[2]{%
      \leavevmode%
      \begingroup
        \let\T@pageref\@pagerefstar
        \hyperref[{#2}]{%
	  #1~\ref*{#2}%
        }%
      \endgroup
    }%
\makeatother

\newcommand{\figurevref}[1]{\myvref{Figure}{#1}}
\newcommand{\figureref}[1]{\myref{Figure}{#1}}
\newcommand{\tablevref}[1]{\myvref{Table}{#1}}
\newcommand{\tableref}[1]{\myref{Table}{#1}}
\newcommand{\chapterref}[1]{\myref{chapter}{#1}}
\newcommand{\Chapterref}[1]{\myref{Chapter}{#1}}
\newcommand{\appendixref}[1]{\myref{appendix}{#1}}
\newcommand{\Appendixref}[1]{\myref{Appendix}{#1}}
\newcommand{\sectionref}[1]{\myref{\S}{#1}}
\newcommand{\subsectionref}[1]{\myref{subsection}{#1}}
\newcommand{\subsectionvref}[1]{\myvref{subsection}{#1}}
\newcommand{\exercisevref}[1]{\myvref{Exercise}{#1}}
\newcommand{\exerciseref}[1]{\myref{Exercise}{#1}}
\newcommand{\examplevref}[1]{\myvref{Example}{#1}}
\newcommand{\exampleref}[1]{\myref{Example}{#1}}
\newcommand{\thmvref}[1]{\myvref{Theorem}{#1}}
\newcommand{\thmref}[1]{\myref{Theorem}{#1}}


\renewcommand{\exampleref}[1]{ {\color{red} \bfseries Normally a reference to a previous example goes here.}}
\renewcommand{\figurevref}[1]{ {\color{red} \bfseries Normally a reference to a previous figure goes here.}}
\renewcommand{\tablevref}[1]{ {\color{red} \bfseries Normally a reference to a previous table goes here.}}
\renewcommand{\Appendixref}[1]{ {\color{red} \bfseries Normally a reference to an Appendix goes here.}}
\renewcommand{\exercisevref}[1]{ {\color{red} \bfseries Normally a reference to a previous exercise goes here.}}



\newcommand{\R}{\mathbb{R}}

%% Example Solution Env.
\def\beginSolclaim{\par\addvspace{\medskipamount}\noindent\hbox{\bf Solution:}\hspace{0.5em}\ignorespaces}
\def\endSolclaim{\par\addvspace{-1em}\hfill\rule{1em}{0.4pt}\hspace{-0.4pt}\rule{0.4pt}{1em}\par\addvspace{\medskipamount}}
\newenvironment{exampleSol}[1][]{\beginSolclaim}{\endSolclaim}

%% General figure formating from original book.
\newcommand{\mybeginframe}{%
\begin{tcolorbox}[colback=white,colframe=lightgray,left=5pt,right=5pt]%
}
\newcommand{\myendframe}{%
\end{tcolorbox}%
}

%%% Eventually return and fix this to make matlab code work correctly.
%% Define the matlab environment as another code environment
%\newenvironment{matlab}
%{% Begin Environment Code
%{ \centering \bfseries Matlab Code }
%\begin{code}
%}% End of Begin Environment Code
%{% Start of End Environment Code
%\end{code}
%}% End of End Environment Code


% this one should have a caption, first argument is the size
\newenvironment{mywrapfig}[2][]{
 \wrapfigure[#1]{r}{#2}
 \mybeginframe
 \centering
}{%
 \myendframe
 \endwrapfigure
}

% this one has no caption, first argument is size,
% the second argument is a larger size used for HTML (ignored by latex)
\newenvironment{mywrapfigsimp}[3][]{%
 \wrapfigure[#1]{r}{#2}%
 \centering%
}{%
 \endwrapfigure%
}
\newenvironment{myfig}
    {%
    \begin{figure}[h!t]
        \mybeginframe%
        \centering%
    }
    {%
        \myendframe
    \end{figure}%
    }


% graphics include
\newcommand{\diffyincludegraphics}[3]{\includegraphics[#1]{#3}}
\newcommand{\myincludegraphics}[3]{\includegraphics[#1]{#3}}
\newcommand{\inputpdft}[1]{\subimport*{../figures/}{#1.pdf_t}}


%% Not sure what these even do? They don't seem to actually work... fun!
%\newcommand{\mybxbg}[1]{\tcboxmath[colback=white,colframe=black,boxrule=0.5pt,top=1.5pt,bottom=1.5pt]{#1}}
%\newcommand{\mybxsm}[1]{\tcboxmath[colback=white,colframe=black,boxrule=0.5pt,left=0pt,right=0pt,top=0pt,bottom=0pt]{#1}}
\newcommand{\mybxsm}[1]{#1}
\newcommand{\mybxbg}[1]{#1}

%%% Something about tasks for practice/hw?
\usepackage{tasks}
\usepackage{footnote}
\makesavenoteenv{tasks}


%% For pdf only?
\newcommand{\diffypdfversion}[1]{#1}


%% Kill ``cite'' and go back later to fix it.
\renewcommand{\cite}[1]{}


%% Currently we can't really use index or its derivatives. So we are gonna kill them off.
\renewcommand{\index}[1]{}
\newcommand{\myindex}[1]{#1}







\begin{document}
\begin{abstract}
Why?
\end{abstract}
\maketitle

\begin{exercise}
    Solve
    \begin{equation*}
        \begin{array}{ll}
            y_{tt} = 9 y_{xx} , &  \\
            y(0,t) = y(1,t) = 0 , &  \\
            y(x,0) = \sin (3\pi x) + \frac{1}{4} \sin (6 \pi x) & \qquad \text{for } \; 0 < x < 1 , \\
            y_t(x,0) = 0 & \qquad \text{for } \; 0 < x < 1 .
        \end{array}
    \end{equation*}
\end{exercise}

\begin{exercise}\%
    Solve
    \begin{equation*}
        \begin{array}{ll}
            y_{tt} = y_{xx} , &  \\
            y(0,t) = y(\pi,t) = 0 , &  \\
            y(x,0) = \sin(x) & \qquad \text{for } \; 0 < x < \pi , \\
            y_t(x,0) = \sin(x) & \qquad \text{for } \; 0 < x < \pi .
        \end{array}
    \end{equation*}
\end{exercise}
%\exsol{%
%$
%y(x,t)
%=
%\sin(x)
%\bigl(\sin(t) + \cos(t)\bigr)
%$
%}

\begin{exercise}
    Solve
    \begin{equation*}
    \begin{array}{ll}
        y_{tt} = 4 y_{xx} , &  \\
        y(0,t) = y(1,t) = 0 , &  \\
        y(x,0) = \sin (3\pi x) + \frac{1}{4} \sin (6 \pi x) & \qquad \text{for } \; 0 < x < 1 , \\
        y_t(x,0) = \sin (9 \pi x) & \qquad \text{for } \; 0 < x < 1 .
    \end{array}
    \end{equation*}
\end{exercise}

\begin{exercise}\%
    Solve
    \begin{equation*}
    \begin{array}{ll}
        y_{tt} = 25 y_{xx} , &  \\
        y(0,t) = y(2,t) = 0 , &  \\
        y(x,0) = 0 & \qquad \text{for } \; 0 < x < 2 , \\
        y_t(x,0) = \sin(\pi t) + 0.1 \sin(2\pi t) & \qquad \text{for } \; 0 < x < 2 .
    \end{array}
    \end{equation*}
\end{exercise}
%\exsol{%
%$y(x,t)
%=
%\frac{1}{5 \pi}
%\sin (\pi x)
%\sin (5 \pi t)
%+
%\frac{1}{100 \pi}
%\sin(2\pi x)
%\sin(10\pi t)$
%}
%%$
%%y(x,t)
%%=
%%\sum\limits_{n=1}^\infty
%%\sin \left( \frac{n \pi}{L} x \right)
%%\left[
%%b_n
%%\frac{L}{n \pi a}
%%\sin \left( \frac{n \pi a}{L} t \right) 
%%+
%%c_n
%%\cos \left( \frac{n \pi a}{L} t \right) 
%%\right] .

\begin{exercise}\%
    Solve
    \begin{equation*}
    \begin{array}{ll}
        y_{tt} = 2 y_{xx} , &  \\
        y(0,t) = y(\pi,t) = 0 , &  \\
        y(x,0) = x & \qquad \text{for } \; 0 < x < \pi , \\
        y_t(x,0) = 0 & \qquad \text{for } \; 0 < x < \pi .
    \end{array}
    \end{equation*}
\end{exercise}
%\exsol{%
%$
%y(x,t)
%=
%\sum\limits_{n=1}^\infty
%\frac{2{(-1)}^{n+1}}{n}
%\sin(nx)
%\cos( n \sqrt{2}\,t ) 
%$
%%$
%%y(x,t)
%%=
%%\sum\limits_{n=1}^\infty
%%\sin \left( \frac{n \pi}{L} x \right)
%%\left[
%%b_n
%%\frac{L}{n \pi a}
%%\sin \left( \frac{n \pi a}{L} t \right) 
%%+
%%c_n
%%\cos \left( \frac{n \pi a}{L} t \right) 
%%$
%}

\begin{exercise}
    Derive the solution for a general plucked string of length $L$ and any constant $a$ (in the equation $y_{tt} = a^2 y_{xx}$), where we raise the string some distance $b$ at the midpoint and let go.
\end{exercise}


\begin{exercise}
    Imagine that a stringed musical instrument falls on the floor.  Suppose that the length of the string is 1 and $a=1$.  When the musical instrument hits the ground the string was in rest position and hence $y(x,0) = 0$.  However, the string was moving at some velocity at impact ($t=0$), say $y_t(x,0) = -1$.  Find the solution $y(x,t)$ for the shape of the string at time $t$.
\end{exercise}


\begin{exercise}\%
    Let's see what happens when $a=0$.  Find a solution to $y_{tt} = 0$, $y(0,t) = y(\pi,t) = 0$, $y(x,0) = \sin(2x)$, $y_t(x,0) = \sin(x)$.
\end{exercise}
%\exsol{%
%$y(x,t) = \sin(2x)+t\sin(x)$
%}

\begin{exercise}[challenging]
    Suppose that you have a vibrating string and that there is air resistance proportional to the velocity.  That is, you have
    \begin{equation*}
    \begin{array}{ll}
        y_{tt} = a^2 y_{xx} - k y_t , &  \\
        y(0,t) = y(1,t) = 0 , &  \\
        y(x,0) = f(x) & \qquad \text{for } \; 0 < x < 1 , \\
        y_t(x,0) = 0 & \qquad \text{for } \; 0 < x < 1 .
    \end{array}
    \end{equation*}
    Suppose that $0 < k < 2 \pi a$. Derive a series solution to the problem.  Any coefficients in the series should be expressed as integrals of $f(x)$.
\end{exercise}

\begin{exercise}
    Suppose you touch the guitar string exactly in the middle to ensure another condition $u(\nicefrac{L}{2},t) = 0$ for all time. Which multiples of the fundamental frequency $\frac{\pi a}{L}$ show up in the solution?
\end{exercise}

%\setcounter{exercise}{100}


\end{document}