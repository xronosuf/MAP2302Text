\documentclass{ximera}

\title{Practice for the Dirichlet Disc}

%\auor{Matthew Charnley and Jason Nowell}
\usepackage[margin=1.5cm]{geometry}
\usepackage{indentfirst}
\usepackage{sagetex}
\usepackage{lipsum}
\usepackage{amsmath}
\usepackage{mathrsfs}
\usepackage{tikz}
\usetikzlibrary{matrix}

%%% Random packages added without verifying what they are really doing - just to get initial compile to work.
\usepackage{tcolorbox}
\usepackage{hypcap}
\usepackage{booktabs}%% To get \toprule,\midrule,\bottomrule etc.
\usepackage{caption}
\usepackage{units}
\usepackage{multicol}
\usepackage{hhline}


% This is my modified wrapfig that doesn't use intextsep
\usepackage{mywrapfig}
\usepackage{import}



%%% End to random added packages.


\graphicspath{
    {./}
    {./figures/}
    {./../figures/}
    {./../../figures/}
}
\renewcommand{\log}{\ln}%%%%
\DeclareMathOperator{\arcsec}{arcsec}
%% New commands


%%%%%%%%%%%%%%%%%%%%
% New Conditionals %
%%%%%%%%%%%%%%%%%%%%


% referencing
\makeatletter
    \DeclareRobustCommand{\myvref}[2]{%
      \leavevmode%
      \begingroup
        \let\T@pageref\@pagerefstar
        \hyperref[{#2}]{%
	  #1~\ref*{#2}%
        }%
        \vpageref[\unskip]{#2}%
      \endgroup
    }%

    \DeclareRobustCommand{\myref}[2]{%
      \leavevmode%
      \begingroup
        \let\T@pageref\@pagerefstar
        \hyperref[{#2}]{%
	  #1~\ref*{#2}%
        }%
      \endgroup
    }%
\makeatother

\newcommand{\figurevref}[1]{\myvref{Figure}{#1}}
\newcommand{\figureref}[1]{\myref{Figure}{#1}}
\newcommand{\tablevref}[1]{\myvref{Table}{#1}}
\newcommand{\tableref}[1]{\myref{Table}{#1}}
\newcommand{\chapterref}[1]{\myref{chapter}{#1}}
\newcommand{\Chapterref}[1]{\myref{Chapter}{#1}}
\newcommand{\appendixref}[1]{\myref{appendix}{#1}}
\newcommand{\Appendixref}[1]{\myref{Appendix}{#1}}
\newcommand{\sectionref}[1]{\myref{\S}{#1}}
\newcommand{\subsectionref}[1]{\myref{subsection}{#1}}
\newcommand{\subsectionvref}[1]{\myvref{subsection}{#1}}
\newcommand{\exercisevref}[1]{\myvref{Exercise}{#1}}
\newcommand{\exerciseref}[1]{\myref{Exercise}{#1}}
\newcommand{\examplevref}[1]{\myvref{Example}{#1}}
\newcommand{\exampleref}[1]{\myref{Example}{#1}}
\newcommand{\thmvref}[1]{\myvref{Theorem}{#1}}
\newcommand{\thmref}[1]{\myref{Theorem}{#1}}


\renewcommand{\exampleref}[1]{ {\color{red} \bfseries Normally a reference to a previous example goes here.}}
\renewcommand{\examplevref}[1]{ {\color{red} \bfseries Normally a reference to a previous example goes here.}}
\renewcommand{\figurevref}[1]{ {\color{red} \bfseries Normally a reference to a previous figure goes here.}}
\renewcommand{\tablevref}[1]{ {\color{red} \bfseries Normally a reference to a previous table goes here.}}
\renewcommand{\Appendixref}[1]{ {\color{red} \bfseries Normally a reference to an Appendix goes here.}}
\renewcommand{\exercisevref}[1]{ {\color{red} \bfseries Normally a reference to a previous exercise goes here.}}
\renewcommand{\thmvref}[1]{ {\color{red} \bfseries Normally a reference to a previous theorem goes here.}}
\renewcommand{\subsectionvref}[1]{ {\color{red} \bfseries Normally a reference to a previous subsection goes here.}}



\newcommand{\R}{\mathbb{R}}
\newcommand{\C}{\mathbb{C}}

%% Example Solution Env.
\def\beginSolclaim{\par\addvspace{\medskipamount}\noindent\hbox{\bf Solution:}\hspace{0.5em}\ignorespaces}
\def\endSolclaim{\par\addvspace{-1em}\hfill\rule{1em}{0.4pt}\hspace{-0.4pt}\rule{0.4pt}{1em}\par\addvspace{\medskipamount}}
\newenvironment{exampleSol}[1][]{\beginSolclaim}{\endSolclaim}

%% General figure formating from original book.
\newcommand{\mybeginframe}{%
\begin{tcolorbox}[colback=white,colframe=lightgray,left=5pt,right=5pt]%
}
\newcommand{\myendframe}{%
\end{tcolorbox}%
}

%%% Eventually return and fix this to make matlab code work correctly.
%% Define the matlab environment as another code environment
%\NewEnviron{matlab}{ {\centering\bfseries MATLAB Code} \\ \noexpand{\BODY} }
%\let\beginmatlab\begincode
%\let\endmatlab\endcode
%\newenvironment{matlab}{% Begin Environment Code
%\begin{minipage}{\linewidth}
%\begin{verbatim}
%}% End of Begin Environment Code
%{% Start of End Environment Code
%\end{verbatim}
%\end{minipage}
%}% End of End Environment Code


% this one should have a caption, first argument is the size
\newenvironment{mywrapfig}[2][]{
 \wrapfigure[#1]{r}{#2}
 \mybeginframe
 \centering
}{%
 \myendframe
 \endwrapfigure
}

% this one has no caption, first argument is size,
% the second argument is a larger size used for HTML (ignored by latex)
\newenvironment{mywrapfigsimp}[3][]{%
 \wrapfigure[#1]{r}{#2}%
 \centering%
}{%
 \endwrapfigure%
}
\newenvironment{myfig}
    {%
    \begin{figure}[h!t]
        \mybeginframe%
        \centering%
    }
    {%
        \myendframe
    \end{figure}%
    }


% graphics include
\newcommand{\diffyincludegraphics}[3]{\includegraphics[#1]{#3}}
\newcommand{\myincludegraphics}[3]{\includegraphics[#1]{#3}}
\newcommand{\inputpdft}[1]{\subimport*{../figures/}{#1.pdf_t}}


%% Not sure what these even do? They don't seem to actually work... fun!
%\newcommand{\mybxbg}[1]{\tcboxmath[colback=white,colframe=black,boxrule=0.5pt,top=1.5pt,bottom=1.5pt]{#1}}
%\newcommand{\mybxsm}[1]{\tcboxmath[colback=white,colframe=black,boxrule=0.5pt,left=0pt,right=0pt,top=0pt,bottom=0pt]{#1}}
\newcommand{\mybxsm}[1]{#1}
\newcommand{\mybxbg}[1]{#1}

%%% Something about tasks for practice/hw?
\usepackage{tasks}
\usepackage{footnote}
\makesavenoteenv{tasks}


%% For pdf only?
\newcommand{\diffypdfversion}[1]{#1}


%% Kill ``cite'' and go back later to fix it.
\renewcommand{\cite}[1]{}


%% Currently we can't really use index or its derivatives. So we are gonna kill them off.
\renewcommand{\index}[1]{}
\newcommand{\myindex}[1]{#1}







\begin{document}
\begin{abstract}
Why?
\end{abstract}
\maketitle


\begin{exercise}
    Using series solve $\Delta u = 0$, $u(1,\theta) = \lvert \theta \rvert$, for $-\pi < \theta \leq \pi$.
\end{exercise}

\begin{exercise}%
    Using series solve $\Delta u = 0$, $u(1,\theta) = 1+ \sum\limits_{n=1}^\infty \frac{1}{n^2}\sin(n\theta)$.
\end{exercise}
%\exsol{%
%$u = 1+ \sum\limits_{n=1}^\infty \frac{1}{n^2}r^n\sin(n\theta)$
%}

\begin{exercise}%
    Using the series solution find the solution to $\Delta u = 0$, $u(1,\theta) = 1- \cos(\theta)$.  Express the solution in Cartesian coordinates (that is, using $x$ and $y$).
\end{exercise}
%\exsol{%
%$u = 1-x$
%}

\begin{exercise}
    Using series solve $\Delta u = 0$, $u(1,\theta) = g(\theta)$ for the following data.  Hint: trig identities.
    \begin{tasks}
        \task $g(\theta) = \nicefrac{1}{2} + 3\sin(\theta) + \cos(3\theta)$
        \task $g(\theta) = 3\cos(3\theta) + 3\sin(3\theta) + \sin(9\theta)$
        \task $g(\theta) = 2 \cos(\theta+1)$
        \task $g(\theta) = \sin^2(\theta)$
    \end{tasks}
\end{exercise}

\begin{exercise}
    Using the Poisson kernel, give the solution to $\Delta u = 0$, where $u(1,\theta)$ is zero for $\theta$ outside the interval $[-\nicefrac{\pi}{4},\nicefrac{\pi}{4}]$ and  $u(1,\theta)$ is 1 for $\theta$ on the interval $[-\nicefrac{\pi}{4},\nicefrac{\pi}{4}]$.
\end{exercise}

\begin{exercise}
    \begin{tasks}
        \task Draw a graph for the Poisson kernel as a function of $\alpha$ when $r=\nicefrac{1}{2}$ and $\theta = 0$. 
        \task Describe what happens to the graph when you make $r$ bigger (as it approaches 1).
        \task Knowing that the solution $u(r,\theta)$ is the weighted average of $g(\theta)$ with Poisson kernel as the weight, explain what your answer to part b) means.
    \end{tasks}
\end{exercise}

\begin{exercise} \label{exercise:dirichproblemxy}
    Let $g(\theta)$ be the function $xy = \cos \theta \sin \theta$ on the boundary.  Use the series solution to find a solution to the Dirichlet problem $\Delta u = 0$, $u(1,\theta) = g(\theta)$.  Now convert the solution to Cartesian coordinates $x$ and $y$.  Is this solution surprising?  Hint: use your trig identities.
\end{exercise}

\begin{exercise}%
    \begin{tasks}
        \task Try and guess a solution to $\Delta u = -1$, $u(1,\theta) = 0$. Hint: try a solution that only depends on $r$.  Also first, don't worry eabout the boundary condition.
        \task Now solve $\Delta u = -1$, $u(1,\theta) = \sin(2\theta)$ using superposition.
    \end{tasks}
\end{exercise}
%\exsol{%
%a) $u = \frac{-1}{4} r^2 + \frac{1}{4}$
%b) $u = \frac{-1}{4} r^2 + \frac{1}{4} + r^2 \sin(2\theta)$
%}

\begin{exercise}
    Carry out the computation we needed in the separation of variables and solve $r^2 R'' + r R' - n^2 R = 0$, for $n=0,1,2,3,\ldots$.
\end{exercise}

\begin{exercise}%[challenging]
    Derive the series solution to the Dirichlet problem if the region is a circle of radius $\rho$ rather than 1. That is, solve $\Delta u = 0$, $u(\rho,\theta) = g(\theta)$. 
\end{exercise}

\begin{exercise}%[challenging]
    \begin{tasks}
        \task Find the solution for $\Delta u = 0$, $u(1,\theta) = x^2y^3 + 5 x^2$.  Write the answer in Cartesian coordinates.
        \task Now solve $\Delta u = 0$, $u(1,\theta) = x^k y^\ell$. Write the solution in Cartesian coordinates.
        \task Suppose you have a polynomial $P(x,y) = \sum_{j=0}^m \sum_{k=0}^n c_{j,k} x^j y^k$, solve $\Delta u = 0$, $u(1,\theta) = P(x,y)$ (that is, write down the formula for the answer).  Write the answer in Cartesian coordinates.
    \end{tasks}
        Notice the answer is again a polynomial in $x$ and $y$. See also \exerciseref{exercise:dirichproblemxy}.
\end{exercise}

\begin{exercise}%[challenging]%
    Derive the Poisson kernel solution if the region is a circle of radius $\rho$ rather than 1.  That is, solve $\Delta u = 0$, $u(\rho,\theta) = g(\theta)$.
\end{exercise}
%\exsol{%
%$\displaystyle
%u(r,\theta) = 
%\frac{1}{2\pi} \int_{-\pi}^{\pi}
%\frac{\rho^2 -r^2}{\rho - 2r\rho\cos(\theta-\alpha) +r^2} g(\alpha) ~ d\alpha$
%}
%
%\setcounter{exercise}{100}



\end{document}