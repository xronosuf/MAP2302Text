\documentclass{ximera}

\title{Practice for Inner Products}

%\auor{Matthew Charnley and Jason Nowell}
\usepackage[margin=1.5cm]{geometry}
\usepackage{indentfirst}
\usepackage{sagetex}
\usepackage{lipsum}
\usepackage{amsmath}
\usepackage{mathrsfs}


%%% Random packages added without verifying what they are really doing - just to get initial compile to work.
\usepackage{tcolorbox}
\usepackage{hypcap}
\usepackage{booktabs}%% To get \toprule,\midrule,\bottomrule etc.
\usepackage{nicefrac}
\usepackage{caption}
\usepackage{units}

% This is my modified wrapfig that doesn't use intextsep
\usepackage{mywrapfig}
\usepackage{import}



%%% End to random added packages.


\graphicspath{
    {./figures/}
    {./../figures/}
    {./../../figures/}
}
\renewcommand{\log}{\ln}%%%%
\DeclareMathOperator{\arcsec}{arcsec}
%% New commands


%%%%%%%%%%%%%%%%%%%%
% New Conditionals %
%%%%%%%%%%%%%%%%%%%%


% referencing
\makeatletter
    \DeclareRobustCommand{\myvref}[2]{%
      \leavevmode%
      \begingroup
        \let\T@pageref\@pagerefstar
        \hyperref[{#2}]{%
	  #1~\ref*{#2}%
        }%
        \vpageref[\unskip]{#2}%
      \endgroup
    }%

    \DeclareRobustCommand{\myref}[2]{%
      \leavevmode%
      \begingroup
        \let\T@pageref\@pagerefstar
        \hyperref[{#2}]{%
	  #1~\ref*{#2}%
        }%
      \endgroup
    }%
\makeatother

\newcommand{\figurevref}[1]{\myvref{Figure}{#1}}
\newcommand{\figureref}[1]{\myref{Figure}{#1}}
\newcommand{\tablevref}[1]{\myvref{Table}{#1}}
\newcommand{\tableref}[1]{\myref{Table}{#1}}
\newcommand{\chapterref}[1]{\myref{chapter}{#1}}
\newcommand{\Chapterref}[1]{\myref{Chapter}{#1}}
\newcommand{\appendixref}[1]{\myref{appendix}{#1}}
\newcommand{\Appendixref}[1]{\myref{Appendix}{#1}}
\newcommand{\sectionref}[1]{\myref{\S}{#1}}
\newcommand{\subsectionref}[1]{\myref{subsection}{#1}}
\newcommand{\subsectionvref}[1]{\myvref{subsection}{#1}}
\newcommand{\exercisevref}[1]{\myvref{Exercise}{#1}}
\newcommand{\exerciseref}[1]{\myref{Exercise}{#1}}
\newcommand{\examplevref}[1]{\myvref{Example}{#1}}
\newcommand{\exampleref}[1]{\myref{Example}{#1}}
\newcommand{\thmvref}[1]{\myvref{Theorem}{#1}}
\newcommand{\thmref}[1]{\myref{Theorem}{#1}}


\renewcommand{\exampleref}[1]{ {\color{red} \bfseries Normally a reference to a previous example goes here.}}
\renewcommand{\figurevref}[1]{ {\color{red} \bfseries Normally a reference to a previous figure goes here.}}
\renewcommand{\tablevref}[1]{ {\color{red} \bfseries Normally a reference to a previous table goes here.}}
\renewcommand{\Appendixref}[1]{ {\color{red} \bfseries Normally a reference to an Appendix goes here.}}
\renewcommand{\exercisevref}[1]{ {\color{red} \bfseries Normally a reference to a previous exercise goes here.}}



\newcommand{\R}{\mathbb{R}}

%% Example Solution Env.
\def\beginSolclaim{\par\addvspace{\medskipamount}\noindent\hbox{\bf Solution:}\hspace{0.5em}\ignorespaces}
\def\endSolclaim{\par\addvspace{-1em}\hfill\rule{1em}{0.4pt}\hspace{-0.4pt}\rule{0.4pt}{1em}\par\addvspace{\medskipamount}}
\newenvironment{exampleSol}[1][]{\beginSolclaim}{\endSolclaim}

%% General figure formating from original book.
\newcommand{\mybeginframe}{%
\begin{tcolorbox}[colback=white,colframe=lightgray,left=5pt,right=5pt]%
}
\newcommand{\myendframe}{%
\end{tcolorbox}%
}

%%% Eventually return and fix this to make matlab code work correctly.
%% Define the matlab environment as another code environment
%\newenvironment{matlab}
%{% Begin Environment Code
%{ \centering \bfseries Matlab Code }
%\begin{code}
%}% End of Begin Environment Code
%{% Start of End Environment Code
%\end{code}
%}% End of End Environment Code


% this one should have a caption, first argument is the size
\newenvironment{mywrapfig}[2][]{
 \wrapfigure[#1]{r}{#2}
 \mybeginframe
 \centering
}{%
 \myendframe
 \endwrapfigure
}

% this one has no caption, first argument is size,
% the second argument is a larger size used for HTML (ignored by latex)
\newenvironment{mywrapfigsimp}[3][]{%
 \wrapfigure[#1]{r}{#2}%
 \centering%
}{%
 \endwrapfigure%
}
\newenvironment{myfig}
    {%
    \begin{figure}[h!t]
        \mybeginframe%
        \centering%
    }
    {%
        \myendframe
    \end{figure}%
    }


% graphics include
\newcommand{\diffyincludegraphics}[3]{\includegraphics[#1]{#3}}
\newcommand{\myincludegraphics}[3]{\includegraphics[#1]{#3}}
\newcommand{\inputpdft}[1]{\subimport*{../figures/}{#1.pdf_t}}


%% Not sure what these even do? They don't seem to actually work... fun!
%\newcommand{\mybxbg}[1]{\tcboxmath[colback=white,colframe=black,boxrule=0.5pt,top=1.5pt,bottom=1.5pt]{#1}}
%\newcommand{\mybxsm}[1]{\tcboxmath[colback=white,colframe=black,boxrule=0.5pt,left=0pt,right=0pt,top=0pt,bottom=0pt]{#1}}
\newcommand{\mybxsm}[1]{#1}
\newcommand{\mybxbg}[1]{#1}

%%% Something about tasks for practice/hw?
\usepackage{tasks}
\usepackage{footnote}
\makesavenoteenv{tasks}


%% For pdf only?
\newcommand{\diffypdfversion}[1]{#1}


%% Kill ``cite'' and go back later to fix it.
\renewcommand{\cite}[1]{}


%% Currently we can't really use index or its derivatives. So we are gonna kill them off.
\renewcommand{\index}[1]{}
\newcommand{\myindex}[1]{#1}







\begin{document}
\begin{abstract}
Why?
\end{abstract}
\maketitle



\begin{exercise}
    Find the $s$ that makes the following vectors orthogonal: $(1,2,3)$, $(1,1,s)$.
\end{exercise}

\begin{exercise}%
    Find the $s$ that makes the following vectors orthogonal: $(1,1,1)$, $(1,s,1)$.
\end{exercise}
%\exsol{%
%$s=-2$
%}


\begin{exercise}
    Find the angle $\theta$ between $(1,3,1)$, $(2,1,-1)$.
\end{exercise}

\begin{exercise}
    Find the angle $\theta$ between $(1,2,3)$, $(1,1,1)$.
\end{exercise}
%\exsol{%
%$\theta \approx 0.3876$
%}


\begin{exercise}
    Given that $\langle \vec{v} , \vec{w} \rangle = 3$ and $\langle \vec{v} , \vec{u} \rangle = -1$ compute
    \begin{itemize}
        \item $\langle \vec{u} , 2 \vec{v} \rangle$
        \item $\langle \vec{v} , 2 \vec{w} + 3 \vec{u} \rangle$
        \item $\langle \vec{w} + 3 \vec{u}, \vec{v} \rangle$
    \end{itemize}
\end{exercise}

\begin{exercise}
    Given that $\langle \vec{v} , \vec{w} \rangle = 1$ and $\langle \vec{v} , \vec{u} \rangle = -1$  and $\lVert \vec{v} \rVert = 3$  and
    \begin{itemize}
        \item $\langle 3 \vec{u} , 5 \vec{v} \rangle$
        \item $\langle \vec{v} , 2 \vec{w} + 3 \vec{u} \rangle$
        \item $\langle \vec{w} + 3 \vec{v}, \vec{v} \rangle$
    \end{itemize}
\end{exercise}
%\exsol{%
%a)~-15 \quad
%b)~-1 \quad
%c)~28
%}


\begin{exercise}
    Suppose $\vec{v} = (1,1,-1)$.  Find
    \begin{itemize}
        \item $\operatorname{proj}_{\vec{v}}\bigl( (1,0,0) \bigr)$
        \item $\operatorname{proj}_{\vec{v}}\bigl( (1,2,3) \bigr)$
        \item $\operatorname{proj}_{\vec{v}}\bigl( (1,-1,0) \bigr)$
    \end{itemize}
\end{exercise}

\begin{exercise}
    Suppose $\vec{v} = (1,0,-1)$.  Find
    \begin{itemize}
        \item $\operatorname{proj}_{\vec{v}}\bigl( (0,2,1) \bigr)$
        \item $\operatorname{proj}_{\vec{v}}\bigl( (1,0,1) \bigr)$
        \item $\operatorname{proj}_{\vec{v}}\bigl( (4,-1,0) \bigr)$
    \end{itemize}
\end{exercise}
%\exsol{%
%a)~$(\frac{-1}{2},0,\frac{1}{2})$ \quad
%b)~$(0,0,0)$ \quad
%c)~$(2,0,-2)$
%}

\begin{exercise}
    Consider the vectors $(1,2,3)$, $(-3,0,1)$, $(1,-5,3)$.
    \begin{itemize}
        \item Check that the vectors are linearly independent and so form a basis.
        \item Check that the vectors are mutually orthogonal, and are therefore an orthogonal basis.
        \item Represent $(1,1,1)$ as a linear combination of this basis.
        \item Make the basis orthonormal.
    \end{itemize}
\end{exercise}


\begin{exercise}
    The vectors $(1,1,-1)$, $(2,-1,1)$, $(1,-5,3)$ form an orthonormal basis. Represent the following vectors in terms of this basis.
    \begin{itemize}
        \item $(1,-8,4)$
        \item $(5,-7,5)$
        \item $(0,-6,2)$
    \end{itemize}
\end{exercise}
%\exsol{%
%a)~$(1,1,-1)-(2,-1,1)+2(1,-5,3)$ \quad
%b)~$2(2,-1,1)+(1,-5,3)$ \quad
%c)~$2(1,1,-1)-2(2,-1,1)+2(1,-5,3)$
%}

\begin{exercise}
    Let $S$ be the subspace spanned by $(1,3,-1)$, $(1,1,1)$.  Find an orthogonal basis of $S$ by the Gram-Schmidt process.
\end{exercise}

\begin{exercise}%
    Let $S$ be the subspace spanned by $(2,-1,1)$, $(2,2,2)$.  Find an orthogonal basis of $S$ by the Gram-Schmidt process.
\end{exercise}
%\exsol{%
%$(2,-1,1)$, $(\frac{2}{3},\frac{8}{3},\frac{4}{3})$
%}

\begin{exercise}
    Starting with $(1,2,3)$, $(1,1,1)$, $(2,2,0)$, follow the Gram-Schmidt process to find an orthogonal basis of ${\mathbb{R}}^3$.
\end{exercise}

\begin{exercise}
    Starting with $(1,1,-1)$, $(2,3,-1)$, $(1,-1,1)$, follow the Gram-Schmidt process to find an orthogonal basis of ${\mathbb{R}}^3$.
\end{exercise}
%\exsol{%
%$(1,1,-1)$, $(0,1,1)$, $(\frac{4}{3},\frac{-2}{3},\frac{2}{3})$
%}

\begin{exercise}
    Find an orthogonal basis of ${\mathbb{R}}^3$ such that $(3,1,-2)$ is one of the vectors.  Hint: First find two extra vectors to make a linearly independent set.
\end{exercise}

\begin{exercise}
    Using cosines and sines of $\theta$, find a unit vector $\vec{u}$ in ${\mathbb{R}}^2$ that makes angle $\theta$ with $\vec{\imath} = (1,0)$.  What is $\langle \vec{\imath} , \vec{u} \rangle$?
\end{exercise}

%\setcounter{exercise}{100}


\end{document}