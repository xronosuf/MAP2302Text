\documentclass{ximera}

\title{Practice for the Power Series}

%\auor{Matthew Charnley and Jason Nowell}
\usepackage[margin=1.5cm]{geometry}
\usepackage{indentfirst}
\usepackage{sagetex}
\usepackage{lipsum}
\usepackage{amsmath}
\usepackage{mathrsfs}
\usepackage{tikz}
\usetikzlibrary{matrix}

%%% Random packages added without verifying what they are really doing - just to get initial compile to work.
\usepackage{tcolorbox}
\usepackage{hypcap}
\usepackage{booktabs}%% To get \toprule,\midrule,\bottomrule etc.
\usepackage{caption}
\usepackage{units}
\usepackage{multicol}
\usepackage{hhline}


% This is my modified wrapfig that doesn't use intextsep
\usepackage{mywrapfig}
\usepackage{import}



%%% End to random added packages.


\graphicspath{
    {./}
    {./figures/}
    {./../figures/}
    {./../../figures/}
}
\renewcommand{\log}{\ln}%%%%
\DeclareMathOperator{\arcsec}{arcsec}
%% New commands


%%%%%%%%%%%%%%%%%%%%
% New Conditionals %
%%%%%%%%%%%%%%%%%%%%


% referencing
\makeatletter
    \DeclareRobustCommand{\myvref}[2]{%
      \leavevmode%
      \begingroup
        \let\T@pageref\@pagerefstar
        \hyperref[{#2}]{%
	  #1~\ref*{#2}%
        }%
        \vpageref[\unskip]{#2}%
      \endgroup
    }%

    \DeclareRobustCommand{\myref}[2]{%
      \leavevmode%
      \begingroup
        \let\T@pageref\@pagerefstar
        \hyperref[{#2}]{%
	  #1~\ref*{#2}%
        }%
      \endgroup
    }%
\makeatother

\newcommand{\figurevref}[1]{\myvref{Figure}{#1}}
\newcommand{\figureref}[1]{\myref{Figure}{#1}}
\newcommand{\tablevref}[1]{\myvref{Table}{#1}}
\newcommand{\tableref}[1]{\myref{Table}{#1}}
\newcommand{\chapterref}[1]{\myref{chapter}{#1}}
\newcommand{\Chapterref}[1]{\myref{Chapter}{#1}}
\newcommand{\appendixref}[1]{\myref{appendix}{#1}}
\newcommand{\Appendixref}[1]{\myref{Appendix}{#1}}
\newcommand{\sectionref}[1]{\myref{\S}{#1}}
\newcommand{\subsectionref}[1]{\myref{subsection}{#1}}
\newcommand{\subsectionvref}[1]{\myvref{subsection}{#1}}
\newcommand{\exercisevref}[1]{\myvref{Exercise}{#1}}
\newcommand{\exerciseref}[1]{\myref{Exercise}{#1}}
\newcommand{\examplevref}[1]{\myvref{Example}{#1}}
\newcommand{\exampleref}[1]{\myref{Example}{#1}}
\newcommand{\thmvref}[1]{\myvref{Theorem}{#1}}
\newcommand{\thmref}[1]{\myref{Theorem}{#1}}


\renewcommand{\exampleref}[1]{ {\color{red} \bfseries Normally a reference to a previous example goes here.}}
\renewcommand{\examplevref}[1]{ {\color{red} \bfseries Normally a reference to a previous example goes here.}}
\renewcommand{\figurevref}[1]{ {\color{red} \bfseries Normally a reference to a previous figure goes here.}}
\renewcommand{\tablevref}[1]{ {\color{red} \bfseries Normally a reference to a previous table goes here.}}
\renewcommand{\Appendixref}[1]{ {\color{red} \bfseries Normally a reference to an Appendix goes here.}}
\renewcommand{\exercisevref}[1]{ {\color{red} \bfseries Normally a reference to a previous exercise goes here.}}
\renewcommand{\thmvref}[1]{ {\color{red} \bfseries Normally a reference to a previous theorem goes here.}}
\renewcommand{\subsectionvref}[1]{ {\color{red} \bfseries Normally a reference to a previous subsection goes here.}}



\newcommand{\R}{\mathbb{R}}
\newcommand{\C}{\mathbb{C}}

%% Example Solution Env.
\def\beginSolclaim{\par\addvspace{\medskipamount}\noindent\hbox{\bf Solution:}\hspace{0.5em}\ignorespaces}
\def\endSolclaim{\par\addvspace{-1em}\hfill\rule{1em}{0.4pt}\hspace{-0.4pt}\rule{0.4pt}{1em}\par\addvspace{\medskipamount}}
\newenvironment{exampleSol}[1][]{\beginSolclaim}{\endSolclaim}

%% General figure formating from original book.
\newcommand{\mybeginframe}{%
\begin{tcolorbox}[colback=white,colframe=lightgray,left=5pt,right=5pt]%
}
\newcommand{\myendframe}{%
\end{tcolorbox}%
}

%%% Eventually return and fix this to make matlab code work correctly.
%% Define the matlab environment as another code environment
%\NewEnviron{matlab}{ {\centering\bfseries MATLAB Code} \\ \noexpand{\BODY} }
%\let\beginmatlab\begincode
%\let\endmatlab\endcode
%\newenvironment{matlab}{% Begin Environment Code
%\begin{minipage}{\linewidth}
%\begin{verbatim}
%}% End of Begin Environment Code
%{% Start of End Environment Code
%\end{verbatim}
%\end{minipage}
%}% End of End Environment Code


% this one should have a caption, first argument is the size
\newenvironment{mywrapfig}[2][]{
 \wrapfigure[#1]{r}{#2}
 \mybeginframe
 \centering
}{%
 \myendframe
 \endwrapfigure
}

% this one has no caption, first argument is size,
% the second argument is a larger size used for HTML (ignored by latex)
\newenvironment{mywrapfigsimp}[3][]{%
 \wrapfigure[#1]{r}{#2}%
 \centering%
}{%
 \endwrapfigure%
}
\newenvironment{myfig}
    {%
    \begin{figure}[h!t]
        \mybeginframe%
        \centering%
    }
    {%
        \myendframe
    \end{figure}%
    }


% graphics include
\newcommand{\diffyincludegraphics}[3]{\includegraphics[#1]{#3}}
\newcommand{\myincludegraphics}[3]{\includegraphics[#1]{#3}}
\newcommand{\inputpdft}[1]{\subimport*{../figures/}{#1.pdf_t}}


%% Not sure what these even do? They don't seem to actually work... fun!
%\newcommand{\mybxbg}[1]{\tcboxmath[colback=white,colframe=black,boxrule=0.5pt,top=1.5pt,bottom=1.5pt]{#1}}
%\newcommand{\mybxsm}[1]{\tcboxmath[colback=white,colframe=black,boxrule=0.5pt,left=0pt,right=0pt,top=0pt,bottom=0pt]{#1}}
\newcommand{\mybxsm}[1]{#1}
\newcommand{\mybxbg}[1]{#1}

%%% Something about tasks for practice/hw?
\usepackage{tasks}
\usepackage{footnote}
\makesavenoteenv{tasks}


%% For pdf only?
\newcommand{\diffypdfversion}[1]{#1}


%% Kill ``cite'' and go back later to fix it.
\renewcommand{\cite}[1]{}


%% Currently we can't really use index or its derivatives. So we are gonna kill them off.
\renewcommand{\index}[1]{}
\newcommand{\myindex}[1]{#1}







\begin{document}
\begin{abstract}
Why?
\end{abstract}
\maketitle



\begin{exercise}
    Is the power series $\displaystyle \sum_{k=0}^\infty e^k x^k$ convergent? If so, what is the radius of convergence?
\end{exercise}

\begin{exercise}
    Is the power series $\displaystyle \sum_{k=0}^\infty k x^k$ convergent? If so, what is the radius of convergence?
\end{exercise}

\begin{exercise}
    Is the power series $\displaystyle \sum_{n=1}^\infty {(0.1)}^n x^n$ convergent? If so, what is the radius of convergence?
\end{exercise}
%\exsol{%
%Yes.  Radius of convergence is $10$.
%}


\begin{exercise}
    Is the power series $\displaystyle \sum_{k=0}^\infty k! x^k$ convergent? If so, what is the radius of convergence?
\end{exercise}

\begin{exercise}
    Is the power series $\displaystyle \sum_{k=0}^\infty \frac{1}{(2k)!} {(x-10)}^k$ convergent?  If so, what is the radius of convergence?
\end{exercise}

\begin{exercise}%[challenging]%
    Is the power series $\displaystyle \sum_{n=1}^\infty \frac{n!}{n^n} x^n$ convergent? If so, what is the radius of convergence?
\end{exercise}
%\exsol{%
%Yes.  Radius of convergence is $e$.
%}

\begin{exercise}
    Determine the Taylor series for $\sin x$ around the point $x_0 = \pi$.
\end{exercise}

\begin{exercise}
    Determine the Taylor series for $\ln x$ around the point $x_0 = 1$, and find the radius of convergence.
\end{exercise}

\begin{exercise}%[challenging]%
    Find the Taylor series for $x^7 e^x$ around $x_0 = 0$.
\end{exercise}
%\exsol{%
%$\sum\limits_{n=7}^\infty
%\frac{1}{(n-7)!} x^n$
%}

\begin{exercise}
    Determine the Taylor series and its radius of convergence of $\dfrac{1}{1+x}$ around $x_0 = 0$.
\end{exercise}

\begin{exercise}
    Determine the Taylor series and its radius of convergence of $\dfrac{x}{4-x^2}$ around $x_0 = 0$.  Hint: You will not be able to use the ratio test.
\end{exercise}

\begin{exercise}
    Expand $x^5+5x+1$ as a power series around $x_0 = 5$.
\end{exercise}

\begin{exercise}%
    Using the geometric series, expand $\frac{1}{1-x}$ around $x_0=2$. For what $x$ does the series converge?
\end{exercise}
%\exsol{%
%$\frac{1}{1-x} = -\frac{1}{1-(2-x)}$ so
%$\frac{1}{1-x} =
%\sum\limits_{n=0}^\infty {(-1)}^{n+1} {(x-2)}^n$,
%which converges for $1 < x < 3$.
%}

\begin{exercise}
    Suppose that the ratio test applies to a series $\displaystyle \sum_{k=0}^\infty a_k x^k$.  Show, using the ratio test, that the radius of convergence of the differentiated series is the same as that of the original series.
\end{exercise}

\begin{exercise}
    Suppose that $f$ is an analytic function such that $f^{(n)}(0) = n$.  Find $f(1)$.
\end{exercise}

\begin{exercise}%[challenging]%
    Imagine $f$ and $g$ are analytic functions such that $f^{(k)}(0) = g^{(k)}(0)$ for all large enough $k$.  What can you say about $f(x)-g(x)$?
\end{exercise}
%\exsol{%
%$f(x)-g(x)$ is a polynomial.  Hint: Use Taylor series.
%}
%
%\setcounter{exercise}{100}



\end{document}