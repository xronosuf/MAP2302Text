\documentclass{ximera}

\title{Practice for Series Solutions}

%\auor{Matthew Charnley and Jason Nowell}
\usepackage[margin=1.5cm]{geometry}
\usepackage{indentfirst}
\usepackage{sagetex}
\usepackage{lipsum}
\usepackage{amsmath}
\usepackage{mathrsfs}


%%% Random packages added without verifying what they are really doing - just to get initial compile to work.
\usepackage{tcolorbox}
\usepackage{hypcap}
\usepackage{booktabs}%% To get \toprule,\midrule,\bottomrule etc.
\usepackage{nicefrac}
\usepackage{caption}
\usepackage{units}

% This is my modified wrapfig that doesn't use intextsep
\usepackage{mywrapfig}
\usepackage{import}



%%% End to random added packages.


\graphicspath{
    {./figures/}
    {./../figures/}
    {./../../figures/}
}
\renewcommand{\log}{\ln}%%%%
\DeclareMathOperator{\arcsec}{arcsec}
%% New commands


%%%%%%%%%%%%%%%%%%%%
% New Conditionals %
%%%%%%%%%%%%%%%%%%%%


% referencing
\makeatletter
    \DeclareRobustCommand{\myvref}[2]{%
      \leavevmode%
      \begingroup
        \let\T@pageref\@pagerefstar
        \hyperref[{#2}]{%
	  #1~\ref*{#2}%
        }%
        \vpageref[\unskip]{#2}%
      \endgroup
    }%

    \DeclareRobustCommand{\myref}[2]{%
      \leavevmode%
      \begingroup
        \let\T@pageref\@pagerefstar
        \hyperref[{#2}]{%
	  #1~\ref*{#2}%
        }%
      \endgroup
    }%
\makeatother

\newcommand{\figurevref}[1]{\myvref{Figure}{#1}}
\newcommand{\figureref}[1]{\myref{Figure}{#1}}
\newcommand{\tablevref}[1]{\myvref{Table}{#1}}
\newcommand{\tableref}[1]{\myref{Table}{#1}}
\newcommand{\chapterref}[1]{\myref{chapter}{#1}}
\newcommand{\Chapterref}[1]{\myref{Chapter}{#1}}
\newcommand{\appendixref}[1]{\myref{appendix}{#1}}
\newcommand{\Appendixref}[1]{\myref{Appendix}{#1}}
\newcommand{\sectionref}[1]{\myref{\S}{#1}}
\newcommand{\subsectionref}[1]{\myref{subsection}{#1}}
\newcommand{\subsectionvref}[1]{\myvref{subsection}{#1}}
\newcommand{\exercisevref}[1]{\myvref{Exercise}{#1}}
\newcommand{\exerciseref}[1]{\myref{Exercise}{#1}}
\newcommand{\examplevref}[1]{\myvref{Example}{#1}}
\newcommand{\exampleref}[1]{\myref{Example}{#1}}
\newcommand{\thmvref}[1]{\myvref{Theorem}{#1}}
\newcommand{\thmref}[1]{\myref{Theorem}{#1}}


\renewcommand{\exampleref}[1]{ {\color{red} \bfseries Normally a reference to a previous example goes here.}}
\renewcommand{\figurevref}[1]{ {\color{red} \bfseries Normally a reference to a previous figure goes here.}}
\renewcommand{\tablevref}[1]{ {\color{red} \bfseries Normally a reference to a previous table goes here.}}
\renewcommand{\Appendixref}[1]{ {\color{red} \bfseries Normally a reference to an Appendix goes here.}}
\renewcommand{\exercisevref}[1]{ {\color{red} \bfseries Normally a reference to a previous exercise goes here.}}



\newcommand{\R}{\mathbb{R}}

%% Example Solution Env.
\def\beginSolclaim{\par\addvspace{\medskipamount}\noindent\hbox{\bf Solution:}\hspace{0.5em}\ignorespaces}
\def\endSolclaim{\par\addvspace{-1em}\hfill\rule{1em}{0.4pt}\hspace{-0.4pt}\rule{0.4pt}{1em}\par\addvspace{\medskipamount}}
\newenvironment{exampleSol}[1][]{\beginSolclaim}{\endSolclaim}

%% General figure formating from original book.
\newcommand{\mybeginframe}{%
\begin{tcolorbox}[colback=white,colframe=lightgray,left=5pt,right=5pt]%
}
\newcommand{\myendframe}{%
\end{tcolorbox}%
}

%%% Eventually return and fix this to make matlab code work correctly.
%% Define the matlab environment as another code environment
%\newenvironment{matlab}
%{% Begin Environment Code
%{ \centering \bfseries Matlab Code }
%\begin{code}
%}% End of Begin Environment Code
%{% Start of End Environment Code
%\end{code}
%}% End of End Environment Code


% this one should have a caption, first argument is the size
\newenvironment{mywrapfig}[2][]{
 \wrapfigure[#1]{r}{#2}
 \mybeginframe
 \centering
}{%
 \myendframe
 \endwrapfigure
}

% this one has no caption, first argument is size,
% the second argument is a larger size used for HTML (ignored by latex)
\newenvironment{mywrapfigsimp}[3][]{%
 \wrapfigure[#1]{r}{#2}%
 \centering%
}{%
 \endwrapfigure%
}
\newenvironment{myfig}
    {%
    \begin{figure}[h!t]
        \mybeginframe%
        \centering%
    }
    {%
        \myendframe
    \end{figure}%
    }


% graphics include
\newcommand{\diffyincludegraphics}[3]{\includegraphics[#1]{#3}}
\newcommand{\myincludegraphics}[3]{\includegraphics[#1]{#3}}
\newcommand{\inputpdft}[1]{\subimport*{../figures/}{#1.pdf_t}}


%% Not sure what these even do? They don't seem to actually work... fun!
%\newcommand{\mybxbg}[1]{\tcboxmath[colback=white,colframe=black,boxrule=0.5pt,top=1.5pt,bottom=1.5pt]{#1}}
%\newcommand{\mybxsm}[1]{\tcboxmath[colback=white,colframe=black,boxrule=0.5pt,left=0pt,right=0pt,top=0pt,bottom=0pt]{#1}}
\newcommand{\mybxsm}[1]{#1}
\newcommand{\mybxbg}[1]{#1}

%%% Something about tasks for practice/hw?
\usepackage{tasks}
\usepackage{footnote}
\makesavenoteenv{tasks}


%% For pdf only?
\newcommand{\diffypdfversion}[1]{#1}


%% Kill ``cite'' and go back later to fix it.
\renewcommand{\cite}[1]{}


%% Currently we can't really use index or its derivatives. So we are gonna kill them off.
\renewcommand{\index}[1]{}
\newcommand{\myindex}[1]{#1}







\begin{document}
\begin{abstract}
Why?
\end{abstract}
\maketitle



In the following exercises, when asked to solve an equation using power series methods, you should find the first few terms of the series, and if possible find a general formula for the $k^{\text{th}}$ coefficient.

\begin{exercise}
    Use power series methods to solve $y''+y = 0$ at the point $x_0 = 1$.
\end{exercise}

\begin{exercise}
    Use power series methods to solve $y''+4xy = 0$ at the point $x_0 = 0$.
\end{exercise}

\begin{exercise}%
    Use power series methods to solve $y'' + 2 x^3 y = 0$ at the point $x_0 = 0$.
\end{exercise}
%\exsol{%
%%\begin{equation*}
%%\begin{split}
%%0 = y''+2 x^3 y &= 
%%\Biggl( \sum_{k=2}^\infty k\,(k-1) \, a_k x^{k-2}  \Biggr)
%%+
%%2 x^3
%%\Biggl( \sum_{k=0}^\infty a_k x^k \Biggr)
%%\\
%%&=
%%\Biggl( \sum_{k=2}^\infty k\,(k-1) \, a_k x^{k-2}  \Biggr)
%%+
%%\Biggl( \sum_{k=0}^\infty 2 a_k x^{k+3} \Biggr) .
%%\\
%%&=
%%\Biggl( \sum_{k=0}^\infty (k+2)\,(k+1) \, a_{k+2} x^k  \Biggr)
%%+
%%\Biggl( \sum_{k=3}^\infty 2 a_{k-3} x^k \Biggr) .
%%\\
%%&=
%%2 a_2 +
%%6 a_3 x +
%%12 a_4 x^2 +
%%\Biggl( \sum_{k=3}^\infty (k+2)\,(k+1) \, a_{k+2} x^k  \Biggr)
%%+
%%\Biggl( \sum_{k=3}^\infty 2 a_{k-3} x^k \Biggr) .
%%\end{split}
%%\end{equation*}
%$a_2 = 0$, $a_3 = 0$, $a_4 = 0$, recurrence relation (for $k \geq 5$): $a_k
%= \frac{- 2 a_{k-5}}{k(k-1)}$,
%so:\\
%$y(x) = a_0 + a_1 x -\frac{a_0}{10} x^5 - \frac{a_1}{15} x^6
%+ \frac{a_0}{450} x^{10} + \frac{a_1}{825} x^{11}
%- \frac{a_0}{47250} x^{15} - \frac{a_1}{99000} x^{16}
%+
%\cdots$
%}


\begin{exercise}
    Use power series methods to solve $y''-xy = 0$ at the point $x_0 = 1$.
\end{exercise}

\begin{exercise}
    Use power series methods to solve $y''+x^2y = 0$ at the point $x_0 = 0$.
\end{exercise}

\begin{exercise} 
    The methods work for other orders than second order.  Try the methods of this section to solve the first order system $y'-xy = 0$ at the point $x_0 = 0$.
\end{exercise}

\begin{exercise}%
    Attempt to solve $x^2 y'' - y = 0$ at $x_0 = 0$ using the power series method of this section ($x_0$ is a singular point). Can you find at least one solution?  Can you find more than one solution?
\end{exercise}
%\exsol{%
%%\begin{equation*}
%%\begin{split}
%%0 = x^2 y''-y &= 
%%x^2 \Biggl( \sum_{k=2}^\infty k\,(k-1) \, a_k x^{k-2}  \Biggr)
%%-
%%\Biggl( \sum_{k=0}^\infty a_k x^k \Biggr)
%%\\
%%&=
%%\Biggl( \sum_{k=2}^\infty k\,(k-1) \, a_k x^k  \Biggr)
%%-
%%a_0
%%-
%%a_1 x
%%-
%%\Biggl( \sum_{k=2}^\infty a_k x^k \Biggr) .
%%\end{split}
%%\end{equation*}
%%so $a_0 = 0$, $a_1 = 0$, $k(k-1) a_k = a_k$
%Applying the method of this section directly we obtain $a_k = 0$ for
%all $k$ and so $y(x) = 0$ is the only solution we find.
%}

\begin{exercise}%[\myindex{Chebyshev's equation of order $p$}]
    \begin{itemize}
        \item Solve $(1-x^2)y''-xy' + p^2y = 0$ using power series methods at $x_0=0$.
        \item For what $p$ is there a polynomial solution?
    \end{itemize}
\end{exercise}

\begin{exercise}
    Find a polynomial solution to $(x^2+1) y''-2xy'+2y = 0$ using power series methods.
\end{exercise}

\begin{exercise}
    \begin{itemize}
        \item Use power series methods to solve $(1-x)y''+y = 0$ at the point $x_0 = 0$.
        \item Use the solution to part a) to find a solution for $xy''+y=0$ around the point $x_0=1$.
    \end{itemize}
\end{exercise}


\begin{exercise}%[challenging]%
    Power series methods also work for nonhomogeneous equations.
    \begin{itemize}
        \item Use power series methods to solve $y'' - x y = \frac{1}{1-x}$ at the point $x_0 = 0$. Hint: Recall the geometric series.
        \item Now solve for the initial condition $y(0)=0$, $y'(0) = 0$.
    \end{itemize}
\end{exercise}
%\exsol{%
%%\begin{equation*}
%%\begin{split}
%%\frac{1}{1-x} =
%%\sum_{k=0}^\infty x^k
%%=
%%y''-x y &= 
%%\Biggl( \sum_{k=2}^\infty k\,(k-1) \, a_k x^{k-2}  \Biggr)
%%-
%%x
%%\Biggl( \sum_{k=0}^\infty a_k x^k \Biggr)
%%\\
%%&=
%%\Biggl( \sum_{k=2}^\infty k\,(k-1) \, a_k x^{k-2}  \Biggr)
%%-
%%\Biggl( \sum_{k=0}^\infty a_k x^{k+1} \Biggr) .
%%\\
%%&=
%%\Biggl( \sum_{k=0}^\infty (k+2)\,(k+1) \, a_{k+2} x^k  \Biggr)
%%-
%%\Biggl( \sum_{k=1}^\infty a_{k-1} x^k \Biggr) .
%%\\
%%&=
%%2 a_2 + 
%%\Biggl( \sum_{k=1}^\infty (k+2)\,(k+1) \, a_{k+2} x^k  \Biggr)
%%-
%%\Biggl( \sum_{k=1}^\infty a_{k-1} x^k \Biggr) .
%%\end{split}
%%\end{equation*}
%a) $a_2 = \frac{1}{2}$, and for $k \geq 1$ we have
%$a_k = \frac{a_{k-3} + 1}{k(k-1)}$, so \\
%$y(x) = a_0 + a_1 x + \frac{1}{2} x^2
%+ \frac{a_0 + 1}{6} x^3
%+ \frac{a_1 + 1}{12} x^4
%+ \frac{3}{40} x^5
%+ \frac{a_0 + 2}{30} x^6
%+ \frac{a_1 + 2}{42} x^7
%+ \frac{5}{112} x^8
%+ \frac{a_0 + 3}{72} x^9
%+ \frac{a_1 + 3}{90} x^{10} +
%\cdots$
%\\
%b)
%$y(x) = \frac{1}{2} x^2
%+ \frac{1}{6} x^3
%+ \frac{1}{12} x^4
%+ \frac{3}{40} x^5
%+ \frac{1}{15} x^6
%+ \frac{1}{21} x^7
%+ \frac{5}{112} x^8
%+ \frac{1}{24} x^9
%+ \frac{1}{30} x^{10} +
%\cdots$
%}

%\setcounter{exercise}{100}



\end{document}