\documentclass{ximera}
%\auor{Matthew Charnley and Jason Nowell}
\usepackage[margin=1.5cm]{geometry}
\usepackage{indentfirst}
\usepackage{sagetex}
\usepackage{lipsum}
\usepackage{amsmath}
\usepackage{mathrsfs}


%%% Random packages added without verifying what they are really doing - just to get initial compile to work.
\usepackage{tcolorbox}
\usepackage{hypcap}
\usepackage{booktabs}%% To get \toprule,\midrule,\bottomrule etc.
\usepackage{nicefrac}
\usepackage{caption}
\usepackage{units}

% This is my modified wrapfig that doesn't use intextsep
\usepackage{mywrapfig}
\usepackage{import}



%%% End to random added packages.


\graphicspath{
    {./figures/}
    {./../figures/}
    {./../../figures/}
}
\renewcommand{\log}{\ln}%%%%
\DeclareMathOperator{\arcsec}{arcsec}
%% New commands


%%%%%%%%%%%%%%%%%%%%
% New Conditionals %
%%%%%%%%%%%%%%%%%%%%


% referencing
\makeatletter
    \DeclareRobustCommand{\myvref}[2]{%
      \leavevmode%
      \begingroup
        \let\T@pageref\@pagerefstar
        \hyperref[{#2}]{%
	  #1~\ref*{#2}%
        }%
        \vpageref[\unskip]{#2}%
      \endgroup
    }%

    \DeclareRobustCommand{\myref}[2]{%
      \leavevmode%
      \begingroup
        \let\T@pageref\@pagerefstar
        \hyperref[{#2}]{%
	  #1~\ref*{#2}%
        }%
      \endgroup
    }%
\makeatother

\newcommand{\figurevref}[1]{\myvref{Figure}{#1}}
\newcommand{\figureref}[1]{\myref{Figure}{#1}}
\newcommand{\tablevref}[1]{\myvref{Table}{#1}}
\newcommand{\tableref}[1]{\myref{Table}{#1}}
\newcommand{\chapterref}[1]{\myref{chapter}{#1}}
\newcommand{\Chapterref}[1]{\myref{Chapter}{#1}}
\newcommand{\appendixref}[1]{\myref{appendix}{#1}}
\newcommand{\Appendixref}[1]{\myref{Appendix}{#1}}
\newcommand{\sectionref}[1]{\myref{\S}{#1}}
\newcommand{\subsectionref}[1]{\myref{subsection}{#1}}
\newcommand{\subsectionvref}[1]{\myvref{subsection}{#1}}
\newcommand{\exercisevref}[1]{\myvref{Exercise}{#1}}
\newcommand{\exerciseref}[1]{\myref{Exercise}{#1}}
\newcommand{\examplevref}[1]{\myvref{Example}{#1}}
\newcommand{\exampleref}[1]{\myref{Example}{#1}}
\newcommand{\thmvref}[1]{\myvref{Theorem}{#1}}
\newcommand{\thmref}[1]{\myref{Theorem}{#1}}


\renewcommand{\exampleref}[1]{ {\color{red} \bfseries Normally a reference to a previous example goes here.}}
\renewcommand{\figurevref}[1]{ {\color{red} \bfseries Normally a reference to a previous figure goes here.}}
\renewcommand{\tablevref}[1]{ {\color{red} \bfseries Normally a reference to a previous table goes here.}}
\renewcommand{\Appendixref}[1]{ {\color{red} \bfseries Normally a reference to an Appendix goes here.}}
\renewcommand{\exercisevref}[1]{ {\color{red} \bfseries Normally a reference to a previous exercise goes here.}}



\newcommand{\R}{\mathbb{R}}

%% Example Solution Env.
\def\beginSolclaim{\par\addvspace{\medskipamount}\noindent\hbox{\bf Solution:}\hspace{0.5em}\ignorespaces}
\def\endSolclaim{\par\addvspace{-1em}\hfill\rule{1em}{0.4pt}\hspace{-0.4pt}\rule{0.4pt}{1em}\par\addvspace{\medskipamount}}
\newenvironment{exampleSol}[1][]{\beginSolclaim}{\endSolclaim}

%% General figure formating from original book.
\newcommand{\mybeginframe}{%
\begin{tcolorbox}[colback=white,colframe=lightgray,left=5pt,right=5pt]%
}
\newcommand{\myendframe}{%
\end{tcolorbox}%
}

%%% Eventually return and fix this to make matlab code work correctly.
%% Define the matlab environment as another code environment
%\newenvironment{matlab}
%{% Begin Environment Code
%{ \centering \bfseries Matlab Code }
%\begin{code}
%}% End of Begin Environment Code
%{% Start of End Environment Code
%\end{code}
%}% End of End Environment Code


% this one should have a caption, first argument is the size
\newenvironment{mywrapfig}[2][]{
 \wrapfigure[#1]{r}{#2}
 \mybeginframe
 \centering
}{%
 \myendframe
 \endwrapfigure
}

% this one has no caption, first argument is size,
% the second argument is a larger size used for HTML (ignored by latex)
\newenvironment{mywrapfigsimp}[3][]{%
 \wrapfigure[#1]{r}{#2}%
 \centering%
}{%
 \endwrapfigure%
}
\newenvironment{myfig}
    {%
    \begin{figure}[h!t]
        \mybeginframe%
        \centering%
    }
    {%
        \myendframe
    \end{figure}%
    }


% graphics include
\newcommand{\diffyincludegraphics}[3]{\includegraphics[#1]{#3}}
\newcommand{\myincludegraphics}[3]{\includegraphics[#1]{#3}}
\newcommand{\inputpdft}[1]{\subimport*{../figures/}{#1.pdf_t}}


%% Not sure what these even do? They don't seem to actually work... fun!
%\newcommand{\mybxbg}[1]{\tcboxmath[colback=white,colframe=black,boxrule=0.5pt,top=1.5pt,bottom=1.5pt]{#1}}
%\newcommand{\mybxsm}[1]{\tcboxmath[colback=white,colframe=black,boxrule=0.5pt,left=0pt,right=0pt,top=0pt,bottom=0pt]{#1}}
\newcommand{\mybxsm}[1]{#1}
\newcommand{\mybxbg}[1]{#1}

%%% Something about tasks for practice/hw?
\usepackage{tasks}
\usepackage{footnote}
\makesavenoteenv{tasks}


%% For pdf only?
\newcommand{\diffypdfversion}[1]{#1}


%% Kill ``cite'' and go back later to fix it.
\renewcommand{\cite}[1]{}


%% Currently we can't really use index or its derivatives. So we are gonna kill them off.
\renewcommand{\index}[1]{}
\newcommand{\myindex}[1]{#1}






\title{Series solutions of linear second order ODEs}
\author{Matthew Charnley and Jason Nowell}


\outcome{Use power series methods to solve second order linear ODEs near ordinary points}
\outcome{Write a recurrence relation for the coefficients in a power series solution to an ODE.}


\begin{document}
\begin{abstract}
    We discuss Series solutions of linear second order ODEs
\end{abstract}
\maketitle

\label{seriessols:section}


% \sectionnotes{Verbatim from Lebl}

% \sectionnotes{1 or 1.5 lecture\EPref{, \S8.2 in \cite{EP}}\BDref{,
% \S5.2 and \S5.3 in \cite{BD}}}

Suppose we have a linear second order homogeneous ODE of the form
\begin{equation*}
    p(x) y'' + q(x) y' + r(x) y = 0 .
\end{equation*}
Suppose that $p(x)$, $q(x)$, and $r(x)$ are polynomials.  We will  try a solution of the form
\begin{equation*}
    y = \sum_{k=0}^\infty a_k {(x-x_0)}^k
\end{equation*}
and solve for the $a_k$ to try to obtain a solution defined in some interval around $x_0$.

The point $x_0$ is called an \emph{\myindex{ordinary point}} if $p(x_0) \not= 0$.  That is, the functions
\begin{equation*}
    \frac{q(x)}{p(x)} \qquad \text{and} \qquad \frac{r(x)}{p(x)}
\end{equation*}
are defined for $x$ near $x_0$.  If $p(x_0) = 0$, then we say $x_0$ is a \emph{\myindex{singular point}}.  Handling singular points is harder than ordinary points and so we now focus only on ordinary points.

\begin{example}
    Let us start with a very simple example
    \begin{equation*}
        y'' - y = 0 .
    \end{equation*}
\end{example}

\begin{exampleSol}
    Let us try a power series solution near $x_0 = 0$, which is an ordinary point.  Every point is an ordinary point in fact, as the equation is constant coefficient.  We already know we should obtain exponentials or the hyperbolic sine and cosine, but let us pretend we do not know this.
    
    We try
    \begin{equation*}
        y = \sum_{k=0}^\infty a_k x^k .
    \end{equation*}
    If we differentiate, the $k=0$ term is a constant and hence disappears. We therefore get
    \begin{equation*}
        y' = \sum_{k=1}^\infty k a_k x^{k-1} .
    \end{equation*}
    We differentiate yet again to obtain (now the $k=1$ term disappears)
    \begin{equation*}
        y'' = \sum_{k=2}^\infty k(k-1) a_k x^{k-2} .
    \end{equation*}
    We reindex the series (replace $k$ with $k+2$) to obtain
    \begin{equation*}
        y'' = \sum_{k=0}^\infty (k+2)\,(k+1) \, a_{k+2} x^k .
    \end{equation*}
    Now we plug $y$ and $y''$ into the differential equation
    \begin{equation*}
        \begin{split}
            0 = y''-y 
            & = \Biggl( \sum_{k=0}^\infty (k+2)\,(k+1) \, a_{k+2} x^k  \Biggr) - \Biggl( \sum_{k=0}^\infty a_k x^k \Biggr) \\
            & = \sum_{k=0}^\infty \,\Bigl( (k+2)\,(k+1) \, a_{k+2} x^k  - a_k x^k \Bigr) \\
            & = \sum_{k=0}^\infty \,\bigl( (k+2)\,(k+1) \,a_{k+2} - a_k \bigr) \, x^k  .
        \end{split}
    \end{equation*}
    As $y'' - y$ is supposed to be equal to 0, we know that the coefficients of the resulting series must be equal to 0.  Therefore,
    \begin{equation*}
        (k+2)\,(k+1) \,a_{k+2} - a_k = 0 ,
        \qquad \text{or} \qquad
        a_{k+2} = \frac{a_k}{(k+2)(k+1)} .
    \end{equation*}
    The equation above is called a \emph{\myindex{recurrence relation}} for the coefficients of the power series. It did not matter what $a_0$ or $a_1$ was.  They can be arbitrary. But once we pick $a_0$ and $a_1$, then all other coefficients are determined by the recurrence relation.
    
    Let us see what the coefficients must be.  First, $a_0$ and $a_1$ are arbitrary.  Then,
    \begin{equation*}
        a_2 = \frac{a_0}{2}, \quad
        a_3 = \frac{a_1}{(3)(2)}, \quad
        a_4 = \frac{a_2}{(4)(3)} = \frac{a_0}{(4)(3)(2)}, \quad
        a_5 = \frac{a_3}{(5)(4)} = \frac{a_1}{(5)(4)(3)(2)}, \quad \ldots
    \end{equation*}
    So for even $k$, that is $k=2n$, we have
    \begin{equation*}
        a_k = a_{2n} = \frac{a_0}{(2n)!} ,
    \end{equation*}
    and for odd $k$, that is $k=2n+1$, we have
    \begin{equation*}
        a_k = a_{2n+1} = \frac{a_1}{(2n+1)!} .
    \end{equation*}
    Let us write down the series
    \begin{equation*}
        y = \sum_{k=0}^\infty a_k x^k = \sum_{n=0}^\infty \left( \frac{a_0}{(2n)!} \,x^{2n} + \frac{a_1}{(2n+1)!} \,x^{2n+1} \right)
        = a_0 \sum_{n=0}^\infty \frac{1}{(2n)!} \,x^{2n} + a_1 \sum_{n=0}^\infty \frac{1}{(2n+1)!} \,x^{2n+1} .
    \end{equation*}
    We recognize the two series as the hyperbolic sine and cosine. Therefore,
    \begin{equation*}
        y = a_0 \cosh x + a_1 \sinh x .
    \end{equation*}
\end{exampleSol}

Of course, in general we will not be able to recognize the series that appears, since usually there will not be any elementary function that matches it.  In that case we will be content with the series.

\begin{example}
    Let us do a more complex example.  Consider \emph{\myindex{Airy's equation}}%
    \footnote{
        Named after the English mathematician \href{http://en.wikipedia.org/wiki/George_Biddell_Airy}{Sir George Biddell Airy} (1801--1892).
        }:
    \begin{equation*}
        y'' - xy = 0 ,
    \end{equation*}
    near the point $x_0 = 0$.  Note that $x_0 = 0$ is an ordinary point.
\end{example}

\begin{exampleSol}
    We try
    \begin{equation*}
        y = \sum_{k=0}^\infty a_k x^k .
    \end{equation*}
    We differentiate twice (as above) to obtain
    \begin{equation*}
        y'' = \sum_{k=2}^\infty k\,(k-1) \, a_k x^{k-2} .
    \end{equation*}
    We plug $y$ into the equation
    \begin{equation*}
        \begin{split}
            0 = y''-xy &=  \Biggl( \sum_{k=2}^\infty k\,(k-1) \, a_k x^{k-2}  \Biggr) - x \Biggl( \sum_{k=0}^\infty a_k x^k \Biggr) \\
            &= \Biggl( \sum_{k=2}^\infty k\,(k-1) \, a_k x^{k-2}  \Biggr) - \Biggl( \sum_{k=0}^\infty a_k x^{k+1} \Biggr) .
        \end{split}
    \end{equation*}
    We reindex to make things easier to sum
    \begin{equation*}
    \begin{split}
        0 = y''-xy
        &= \Biggl( 2 a_2 + \sum_{k=1}^\infty (k+2)\,(k+1) \, a_{k+2} x^k  \Biggr) - \Biggl( \sum_{k=1}^\infty a_{k-1} x^k \Biggr) \\
        &= 2 a_2 +  \sum_{k=1}^\infty \Bigl( (k+2)\,(k+1) \, a_{k+2} - a_{k-1} \Bigr) \, x^k .
    \end{split}
    \end{equation*}
    Again $y''-xy$ is supposed to be 0, so $a_2 = 0$, and
    \begin{equation*}
        (k+2)\,(k+1) \,a_{k+2} - a_{k-1} = 0 ,
        \qquad \text{or} \qquad
        a_{k+2} = \frac{a_{k-1}}{(k+2)(k+1)} .
    \end{equation*}
    We jump in steps of three.  First, since $a_2 = 0$ we must have , $a_5 = 0$, $a_8 = 0$, $a_{11}=0$, etc. In general, $a_{3n+2} = 0$.
    
    The constants $a_0$ and $a_1$ are arbitrary and we obtain
    \begin{equation*}
        a_3 = \frac{a_0}{(3)(2)}, \quad a_4 = \frac{a_1}{(4)(3)}, \quad a_6 = \frac{a_3}{(6)(5)} = \frac{a_0}{(6)(5)(3)(2)}, \quad a_7 = \frac{a_4}{(7)(6)} = \frac{a_1}{(7)(6)(4)(3)}, \quad \ldots
    \end{equation*}
    For $a_k$ where $k$ is a multiple of $3$, that is $k=3n$ we notice that
    \begin{equation*}
        a_{3n} = \frac{a_0}{(2)(3)(5)(6) \cdots (3n-1)(3n)} .
    \end{equation*}
    For $a_k$ where $k = 3n+1$, we notice
    \begin{equation*}
        a_{3n+1} = \frac{a_1}{(3)(4)(6)(7) \cdots (3n)(3n+1)} .
    \end{equation*}
    In other words, if we write down the series for $y$, it has two parts
    \begin{equation*}
        \begin{split}
            y &=
            \left( a_0 + \frac{a_0}{6} x^3 + \frac{a_0}{180} x^6 + \cdots + \frac{a_0}{(2)(3)(5)(6) \cdots (3n-1)(3n)} x^{3n} + \cdots \right) \\
            &\phantom{=} +
            \left( a_1 x + \frac{a_1}{12} x^4 + \frac{a_1}{504} x^7 + \cdots + \frac{a_1}{(3)(4)(6)(7) \cdots (3n)(3n+1)} x^{3n+1} + \cdots \right) \\
            & = a_0
            \left( 1 + \frac{1}{6} x^3 + \frac{1}{180} x^6 + \cdots + \frac{1}{(2)(3)(5)(6) \cdots (3n-1)(3n)} x^{3n} + \cdots \right) \\
            &\phantom{=} + a_1
            \left( x + \frac{1}{12} x^4 + \frac{1}{504} x^7 + \cdots + \frac{1}{(3)(4)(6)(7) \cdots (3n)(3n+1)} x^{3n+1} + \cdots \right) .
        \end{split}
    \end{equation*}
    We define
    \begin{align*}
        y_1(x) &= 1 + \frac{1}{6} x^3 + \frac{1}{180} x^6 + \cdots + \frac{1}{(2)(3)(5)(6) \cdots (3n-1)(3n)} x^{3n} + \cdots, \\
        y_2(x) &=  x + \frac{1}{12} x^4 + \frac{1}{504} x^7 + \cdots + \frac{1}{(3)(4)(6)(7) \cdots (3n)(3n+1)} x^{3n+1} + \cdots ,
    \end{align*}
    and write the general solution to the equation as $y(x)= a_0 y_1(x) + a_1 y_2(x)$.  If we plug in $x=0$ into the power series for $y_1$ and $y_2$, we find $y_1(0) = 1$ and $y_2(0) = 0$.  Similarly, $y_1'(0) = 0$ and $y_2'(0) = 1$.  Therefore $y = a_0 y_1 + a_1 y_2$ is a solution that satisfies the initial conditions $y(0) = a_0$ and $y'(0) = a_1$.
    
    \begin{myfig}
        \capstart
        \diffyincludegraphics{width=3in}{width=4.5in}{ps-airy}
        \caption{The two solutions $y_1$ and $y_2$ to Airy's equation.\label{ps:airyfig}}
    \end{myfig}
\end{exampleSol}
The functions $y_1$ and $y_2$ cannot be written in terms of the elementary functions that you know.  See \figurevref{ps:airyfig} for the plot of the solutions $y_1$ and $y_2$.  These functions have many interesting properties.  For example, they are oscillatory for negative $x$ (like solutions to $y''+y=0$) and for positive $x$ they grow without bound (like solutions to $y''-y=0$).


Sometimes a solution may turn out to be a polynomial.

\begin{example}
    Find a solution to the so-called \emph{\myindex{Hermite's equation of order $n$}}%
    \footnote{Named after the French mathematician \href{http://en.wikipedia.org/wiki/Hermite}{Charles Hermite} (1822--1901).}:
    \begin{equation*}
        y'' -2xy' + 2n y = 0 .
    \end{equation*}
\end{example}

\begin{exampleSol}
Let us find a solution around the point $x_0 = 0$. We try
\begin{equation*}
    y = \sum_{k=0}^\infty a_k x^k .
\end{equation*}
We differentiate (as above) to obtain
\begin{align*}
    y' &= \sum_{k=1}^\infty k a_k x^{k-1} ,\\
    y'' &= \sum_{k=2}^\infty k\,(k-1) \, a_k x^{k-2} .
\end{align*}

Now we plug into the equation
\begin{equation*}
    \begin{split}
        0 = y''-2xy'&+2ny \\
        &=\Biggl( \sum_{k=2}^\infty k(k-1) a_k x^{k-2}  \Biggr)
        - 2x \Biggl( \sum_{k=1}^\infty k a_k x^{k-1} \Biggr)
        + 2n \Biggl( \sum_{k=0}^\infty a_k x^k \Biggr)\\
        &=\Biggl( \sum_{k=2}^\infty k(k-1) a_k x^{k-2}  \Biggr) 
        - \Biggl( \sum_{k=1}^\infty 2k a_k x^k \Biggr) 
        + \Biggl( \sum_{k=0}^\infty 2n a_k x^k \Biggr) \\
        &=\Biggl(2a_2 + \sum_{k=1}^\infty (k+2)(k+1) a_{k+2} x^k  \Biggr) 
        - \Biggl( \sum_{k=1}^\infty 2k a_k x^k \Biggr) 
        + \Biggl( 2na_0 + \sum_{k=1}^\infty 2n a_k x^k \Biggr) \\
        &= 2a_2+2na_0+ \sum_{k=1}^\infty \bigl( (k+2)(k+1)  a_{k+2} - 2ka_k + 2n a_k \bigr) x^k .
    \end{split}
\end{equation*}
As $y''-2xy'+2ny = 0$ we have
\begin{equation*}
    (k+2)(k+1)  a_{k+2} + ( - 2k+ 2n) a_k = 0 ,
    \qquad \text{or} \qquad
    a_{k+2} = \frac{(2k-2n)}{(k+2)(k+1)} a_k .
\end{equation*}
This recurrence relation actually includes $a_2 = -na_0$ (which comes about from $2a_2+2na_0 = 0$). Again $a_0$ and $a_1$ are arbitrary.
\begin{align*}
    & a_2 = \frac{-2n}{(2)(1)}a_0, \qquad a_3 = \frac{2(1-n)}{(3)(2)} a_1, \displaybreak[0]\\
    & a_4 = \frac{2(2-n)}{(4)(3)} a_2 = \frac{2^2(2-n)(-n)}{(4)(3)(2)(1)} a_0 , \displaybreak[0]\\
    & a_5 = \frac{2(3-n)}{(5)(4)} a_3 = \frac{2^2(3-n)(1-n)}{(5)(4)(3)(2)} a_1 , \quad \ldots
\end{align*}
Let us separate the even and odd coefficients. We find that 
\begin{align*}
    a_{2m} &=\frac{2^m(-n)(2-n)\cdots(2m-2-n)}{(2m)!} , \\
    a_{2m+1} &=\frac{2^m(1-n)(3-n)\cdots(2m-1-n)}{(2m+1)!} .
\end{align*}

Let us write down the two series, one with the even powers and one with the odd.
\begin{align*}
    y_1(x) & =  1+\frac{2(-n)}{2!} x^2 + \frac{2^2(-n)(2-n)}{4!} x^4 +  \frac{2^3(-n)(2-n)(4-n)}{6!} x^6 + \cdots , \\
    y_2(x) & =  x+\frac{2(1-n)}{3!} x^3 + \frac{2^2(1-n)(3-n)}{5!} x^5 +  \frac{2^3(1-n)(3-n)(5-n)}{7!} x^7 + \cdots .
\end{align*}
We then write
\begin{equation*}
    y(x) = a_0 y_1(x) + a_1 y_2(x) .
\end{equation*}

We remark that if $n$ is a positive even integer, then $y_1(x)$ is a polynomial as all the coefficients in the series beyond a certain degree are zero.  If $n$ is a positive odd integer, then $y_2(x)$ is a polynomial.  For example, if $n=4$, then
\begin{equation*}
    y_1(x) = 1 + \frac{2(-4)}{2!} x^2 + \frac{2^2(-4)(2-4)}{4!} x^4 = 1 - 4x^2 + \frac{4}{3} x^4 .
\end{equation*}
\end{exampleSol}



\end{document}
