\documentclass{ximera}

\title{Practice for Solving Exact ODEs}

%\auor{Matthew Charnley and Jason Nowell}
\usepackage[margin=1.5cm]{geometry}
\usepackage{indentfirst}
\usepackage{sagetex}
\usepackage{lipsum}
\usepackage{amsmath}
\usepackage{mathrsfs}


%%% Random packages added without verifying what they are really doing - just to get initial compile to work.
\usepackage{tcolorbox}
\usepackage{hypcap}
\usepackage{booktabs}%% To get \toprule,\midrule,\bottomrule etc.
\usepackage{nicefrac}
\usepackage{caption}
\usepackage{units}

% This is my modified wrapfig that doesn't use intextsep
\usepackage{mywrapfig}
\usepackage{import}



%%% End to random added packages.


\graphicspath{
    {./figures/}
    {./../figures/}
    {./../../figures/}
}
\renewcommand{\log}{\ln}%%%%
\DeclareMathOperator{\arcsec}{arcsec}
%% New commands


%%%%%%%%%%%%%%%%%%%%
% New Conditionals %
%%%%%%%%%%%%%%%%%%%%


% referencing
\makeatletter
    \DeclareRobustCommand{\myvref}[2]{%
      \leavevmode%
      \begingroup
        \let\T@pageref\@pagerefstar
        \hyperref[{#2}]{%
	  #1~\ref*{#2}%
        }%
        \vpageref[\unskip]{#2}%
      \endgroup
    }%

    \DeclareRobustCommand{\myref}[2]{%
      \leavevmode%
      \begingroup
        \let\T@pageref\@pagerefstar
        \hyperref[{#2}]{%
	  #1~\ref*{#2}%
        }%
      \endgroup
    }%
\makeatother

\newcommand{\figurevref}[1]{\myvref{Figure}{#1}}
\newcommand{\figureref}[1]{\myref{Figure}{#1}}
\newcommand{\tablevref}[1]{\myvref{Table}{#1}}
\newcommand{\tableref}[1]{\myref{Table}{#1}}
\newcommand{\chapterref}[1]{\myref{chapter}{#1}}
\newcommand{\Chapterref}[1]{\myref{Chapter}{#1}}
\newcommand{\appendixref}[1]{\myref{appendix}{#1}}
\newcommand{\Appendixref}[1]{\myref{Appendix}{#1}}
\newcommand{\sectionref}[1]{\myref{\S}{#1}}
\newcommand{\subsectionref}[1]{\myref{subsection}{#1}}
\newcommand{\subsectionvref}[1]{\myvref{subsection}{#1}}
\newcommand{\exercisevref}[1]{\myvref{Exercise}{#1}}
\newcommand{\exerciseref}[1]{\myref{Exercise}{#1}}
\newcommand{\examplevref}[1]{\myvref{Example}{#1}}
\newcommand{\exampleref}[1]{\myref{Example}{#1}}
\newcommand{\thmvref}[1]{\myvref{Theorem}{#1}}
\newcommand{\thmref}[1]{\myref{Theorem}{#1}}


\renewcommand{\exampleref}[1]{ {\color{red} \bfseries Normally a reference to a previous example goes here.}}
\renewcommand{\figurevref}[1]{ {\color{red} \bfseries Normally a reference to a previous figure goes here.}}
\renewcommand{\tablevref}[1]{ {\color{red} \bfseries Normally a reference to a previous table goes here.}}
\renewcommand{\Appendixref}[1]{ {\color{red} \bfseries Normally a reference to an Appendix goes here.}}
\renewcommand{\exercisevref}[1]{ {\color{red} \bfseries Normally a reference to a previous exercise goes here.}}



\newcommand{\R}{\mathbb{R}}

%% Example Solution Env.
\def\beginSolclaim{\par\addvspace{\medskipamount}\noindent\hbox{\bf Solution:}\hspace{0.5em}\ignorespaces}
\def\endSolclaim{\par\addvspace{-1em}\hfill\rule{1em}{0.4pt}\hspace{-0.4pt}\rule{0.4pt}{1em}\par\addvspace{\medskipamount}}
\newenvironment{exampleSol}[1][]{\beginSolclaim}{\endSolclaim}

%% General figure formating from original book.
\newcommand{\mybeginframe}{%
\begin{tcolorbox}[colback=white,colframe=lightgray,left=5pt,right=5pt]%
}
\newcommand{\myendframe}{%
\end{tcolorbox}%
}

%%% Eventually return and fix this to make matlab code work correctly.
%% Define the matlab environment as another code environment
%\newenvironment{matlab}
%{% Begin Environment Code
%{ \centering \bfseries Matlab Code }
%\begin{code}
%}% End of Begin Environment Code
%{% Start of End Environment Code
%\end{code}
%}% End of End Environment Code


% this one should have a caption, first argument is the size
\newenvironment{mywrapfig}[2][]{
 \wrapfigure[#1]{r}{#2}
 \mybeginframe
 \centering
}{%
 \myendframe
 \endwrapfigure
}

% this one has no caption, first argument is size,
% the second argument is a larger size used for HTML (ignored by latex)
\newenvironment{mywrapfigsimp}[3][]{%
 \wrapfigure[#1]{r}{#2}%
 \centering%
}{%
 \endwrapfigure%
}
\newenvironment{myfig}
    {%
    \begin{figure}[h!t]
        \mybeginframe%
        \centering%
    }
    {%
        \myendframe
    \end{figure}%
    }


% graphics include
\newcommand{\diffyincludegraphics}[3]{\includegraphics[#1]{#3}}
\newcommand{\myincludegraphics}[3]{\includegraphics[#1]{#3}}
\newcommand{\inputpdft}[1]{\subimport*{../figures/}{#1.pdf_t}}


%% Not sure what these even do? They don't seem to actually work... fun!
%\newcommand{\mybxbg}[1]{\tcboxmath[colback=white,colframe=black,boxrule=0.5pt,top=1.5pt,bottom=1.5pt]{#1}}
%\newcommand{\mybxsm}[1]{\tcboxmath[colback=white,colframe=black,boxrule=0.5pt,left=0pt,right=0pt,top=0pt,bottom=0pt]{#1}}
\newcommand{\mybxsm}[1]{#1}
\newcommand{\mybxbg}[1]{#1}

%%% Something about tasks for practice/hw?
\usepackage{tasks}
\usepackage{footnote}
\makesavenoteenv{tasks}


%% For pdf only?
\newcommand{\diffypdfversion}[1]{#1}


%% Kill ``cite'' and go back later to fix it.
\renewcommand{\cite}[1]{}


%% Currently we can't really use index or its derivatives. So we are gonna kill them off.
\renewcommand{\index}[1]{}
\newcommand{\myindex}[1]{#1}







\begin{document}
\begin{abstract}
    Why?
\end{abstract}
\maketitle

\begin{exercise}
    Solve the following exact equations, implicit general solutions will suffice:
    \begin{itemize}
        \item $(2 xy + x^2) dx + (x^2+y^2+1) dy = 0$, $\answer{x^2y + \frac{x^3}{3} + \frac{y^3}{3} + y} = C$
        \item $x^5 + y^5 \frac{dy}{dx} = 0$, $\answer{\frac{x^6}{6} + \frac{y^6}{6}} = C$
        \item $e^x+y^3 + 3xy^2 \frac{dy}{dx} = 0$, $\answer{e^x + xy^3} = C$
        \item $(x+y)\cos(x)+\sin(x) + \sin(x)y' = 0$, $\answer{(x+y)\sin(x) + \cos(x)} = C$
    \end{itemize}
\end{exercise}
%\comboSol
%{%
%a)~ $x^2y + \frac{x^3}{3} + \frac{y^3}{3} + y = C$ \quad b)~ $\frac{x^6}{6} + \frac{y^6}{6} = C$ \quad
%c)~ $e^x + xy^3 = C$ \quad d)~ $(x+y)\sin(x) + \cos(x) = C$
%}

\begin{exercise}%
    Solve the following exact equations, implicit general solutions will suffice:
    \begin{itemize}
        \item $\cos(x)+ye^{xy} + xe^{xy} y' = 0$, $\answer{e^{xy}+\sin(x)} = C$
        \item $(2x+y)dx + (x-4y) dy = 0$, $\answer{x^2+xy-2y^2} = C$
        \item $e^x + e^y \frac{dy}{dx} = 0$, $\answer{e^x+e^y} = C$
        \item $(3x^2+3y)dx + (3y^2+3x)dy = 0$, $\answer{x^3 + 3xy+ y^3} = C$
    \end{itemize}
\end{exercise}
%\exsol{%
%a) $e^{xy}+\sin(x)=C$ \quad
%b) $x^2+xy-2y^2=C$ \quad
%c) $e^x+e^y=C$ \quad
%d) $x^3 + 3xy+ y^3 = C$
%}

\begin{exercise}
    Solve the differential equation $(2ye^{2xy} - 2x) + (2xe^{2xy} + \cos(y))y' = 0$
    \[
        C = \answer{e^{2xy} - x^2 + \sin(y)}
    \]
\end{exercise}
%\comboSol
%{%
%$e^{2xy} - x^2 + \sin(y) = C$
%}

\begin{exercise}
    Solve the differential equation $(-y\sin(xy) - 2xe^{x^2}) + (-x\sin(xy) + 1)y' = 0$
    \[
        C = \answer{\cos(xy) - e^{x^2} + y}
    \]
\end{exercise}
%\comboSol
%{%
%$\cos(xy) - e^{x^2} + y = C$
%}

\begin{exercise}
    
    Solve the differential equation $(2x + 3y\sin(xy)) + (3x\sin(xy) - e^y)y' = 0$ with $y(2) = 0$.
    
    [Note: Make sure to enter the full equation, e.g. 0 = x + 4]
    \[
        \answer{x^2 - 3\cos(xy) - e^y = 0}
    \]
    
\end{exercise}
%\comboSol
%{%
%$x^2 - 3\cos(xy) - e^y = 0$
%}

\begin{exercise}
    Solve the differential equation $x + yy' = 0$ with $y(0) = 8$. Write this as an explicit function.
    [Note: Make sure to enter the full equation, e.g. 0 = x + 4]
    \[
        \answer{y = \sqrt{64-x^2}}
    \]
    \begin{problem}
        Determine the interval of $x$ values where the solution is valid:
        \wordChoice{\choice{(}\choice[correct]{[}}$\answer{-8},\answer{8}$\wordChoice{\choice{)}\choice[correct]{]}}
    \end{problem}
    
\end{exercise}
%\comboSol
%{%
%$y = \sqrt{64-x^2}$, valid on $[-8, 8]$
%}

\begin{exercise}
    Solve the differential equation $2x-2 + (8y+16)y' = 0$ with $y(2) = 0$. Write this as an explicit function.
    [Note: Make sure to enter the full equation, e.g. 0 = x + 4]
    \[
        \answer{y = \frac{-4 + \sqrt{17 - (x-1)^2}}{2}}
    \]
    \begin{problem}
        Determine the interval of $x$ values where the solution is valid:
        \wordChoice{\choice{(}\choice[correct]{[}}$\answer{1 - \sqrt{17}},\answer{1 + \sqrt{17}}$\wordChoice{\choice{)}\choice[correct]{]}}
    \end{problem}
\end{exercise}
%\comboSol
%{%
%$y = \frac{-4 + \sqrt{17 - (x-1)^2}}{2}$, valid on $[1 - \sqrt{17}, 1 +\sqrt{17}]$
%}

\begin{exercise}
    Find the integrating factor for the following equations making them into exact equations. You can either use the formulas in this section or guess what the integrating factor should be.
    \begin{itemize}
        \item $e^{xy} dx + \frac{y}{x} e^{xy} dy = 0$, $\answer{xe^{-xy}}$
        \item $\frac{e^x+y^3}{y^2} dx + 3x dy = 0$, $\answer{y^2}$
        \item $(4x^5 + 9x^4y^2 + 10\frac{y}{x})dx + (6x^5y + 3x^2y^2 - 5)dy = 0$, $\answer{x^{-2}}$
        \item $2\sin(y) dx + x\cos(y)dy = 0$, $\answer{x}$
    \end{itemize}
\end{exercise}
%\comboSol
%{%
%a)~ $xe^{-xy}$ \quad b)~$y^2$ \quad c)~$x^{-2}$ \quad d)~$x$
%}


\begin{exercise}%
    Find the integrating factor for the following equations making them into exact equations:
    \begin{itemize}
        \item $\frac{1}{y}dx + 3y dy = 0$, 
            Integrating Factor: $\answer{y}$,
            Exact Equation: $\answer{dx + 3y^2dy = 0}$
        \item $dx - e^{-x-y} dy = 0$, $\answer{xe^{-xy}}$, 
            Integrating Factor: $\answer{e^x}$,
            Exact Equation: $\answer{e^x dx - e^{-y}dy= 0}$
        \item $\bigl( \frac{\cos(x)}{y^2} + \frac{1}{y} \bigr) dx + \frac{x}{y^2} dy = 0$, $\answer{xe^{-xy}}$, 
            Integrating Factor: $\answer{y^2}$,
            Exact Equation: $\answer{(\cos(x)+y)dx + xdy = 0}$
        \item $\bigl( 2y + \frac{y^2}{x} \bigr) dx + ( 2y+x )dy = 0$, $\answer{xe^{-xy}}$, 
            Integrating Factor: $\answer{x}$,
            Exact Equation: $\answer{( 2xy+y^2 )dx + (x^2+2xy)dy = 0}$
    \end{itemize}
\end{exercise}

%\exsol{%
%a) Integrating factor is $y$, equation becomes $dx + 3y^2dy = 0$.
%\quad
%b) Integrating factor is $e^x$, equation becomes $e^x dx - e^{-y}dy
%= 0$.
%\quad
%c) Integrating factor is $y^2$, equation becomes $(\cos(x)+y)dx +
%xdy = 0$.
%\quad
%d) Integrating factor is $x$, equation becomes $( 2xy+y^2 )dx +
%(x^2+2xy)dy = 0$.
%}

\begin{exercise}
    Suppose you have an equation of the form: $f(x) + g(y) \frac{dy}{dx} = 0$.
    
    Is it exact? \wordChoice{\choice[correct]{Yes.}\choice{No.}}
    \begin{problem}
        Find the form of the potential function in terms of $f$ and $g$. 
        \[
            \int \answer{f(x)}dx + \int \answer{g(y)}dy = 0.
        \]
    \end{problem}
\end{exercise}
%\comboSol
%{%
%a)~Yes. \quad b)~ $\int f(x)\ dx + \int g(y)\ dy = C$
%}

\begin{exercise}
    Suppose that we have the equation $f(x) dx - dy = 0$.
    
    Is this equation exact? \wordChoice{\choice[correct]{Yes.}\choice{No.}}
    \begin{problem}
        Find the general solution using a definite integral. $y = \int \answer{f(x)}dx + C$
    \end{problem}
\end{exercise}
%\comboSol
%{%
%a)~ Yes. \quad b)~$y = \int f(x)\ dx + C$ 
%}

\begin{exercise}
    Find the potential function $F(x,y)$ of the exact equation $\frac{1+xy}{x}dx + \bigl(\frac{1}{y} + x \bigr) dy = 0$ in two different ways.
    \begin{itemize}
        \item Integrate $M$ in terms of $x$ and then differentiate in $y$ and set to $N$. $\answer{xy + \ln(x) + \ln(y) = C}$
        \item Integrate $N$ in terms of $y$ and then differentiate in $x$ and set to $M$. $\answer{xy + \ln(x) + \ln(y) = C}$
    \end{itemize}
\end{exercise}
%\comboSol
%{%
%$xy + \ln(x) + \ln(y) = C$
%}

\begin{exercise}
    A function $u(x,y)$ is said to be a \emph{harmonic function} if $u_{xx} + u_{yy} = 0$.
    
    Compute: the difference between $M_y = -u_{yy}$ and $N_x = u_{xx}$: $\answer{0}$.
    \begin{problem}
        Since the difference is zero, they are equal, thus $-u_y dx + u_x dy = 0$ is an exact equation.  So there exists (at least locally) the so-called \emph{harmonic conjugate} function $v(x,y)$ such that $v_x = -u_y$ and $v_y = u_x$.
        
        Verify that the following $u$ are harmonic and find the corresponding harmonic conjugates $v$:
        \begin{itemize}
            \item $u = 2xy$, corresponding harmonic conjugate: $\answer{y^2 - x^2}$.
            \item $u = e^x \cos y$, corresponding harmonic conjugate: $\answer{e^{x}\sin(y)}$.
            \item $u = x^3-3xy^2$, corresponding harmonic conjugate: $\answer{3x^2y - y^3}$.
        \end{itemize}
    \end{problem}
\end{exercise}
%\comboSol
%{%
%a)~ $M_y = -u_{yy}$ and $N_x = u_{xx}$, and the difference is zero, so they are equal. \\
%b)~ (i)~$y^2 - x^2$ \quad (ii)~$e^{x}\sin(y)$ \quad (iii)~ $3x^2y - y^3$
%}

\begin{exercise}\label{ex:separableExact}%
    Consider equations of the form $y' = f(x)g(y)$. Compute: $M_y = \answer{0}$, and $N_x = \answer{0}$.
    \begin{problem}
        Since $M_y = 0 = N_x$, we know that it is an exact function. We can thus use the substitution $y' = xy$ to rewrite it as an exact equation and solve. Start by using separation of variables to get:
        \[
            \answer{-x }dx + \answer{\frac{1}{y}} dy = 0
        \]
        \begin{problem}
            Then taking the integral we get:
            \[
                F(x,y) = \answer{-\frac{x^2}{2} + \ln \lvert y \rvert} + C
            \]
        \end{problem}
    \end{problem}
\end{exercise}
%\exsol{%
%a) The equation is $ - f(x) dx + \frac{1}{g(y)} dy$,
%and this is exact
%because $M = -f(x)$, $N = \frac{1}{g(y)}$, so $M_y = 0 = N_x$.
%\quad
%b) $-x dx + \frac{1}{y} dy = 0$, leads to
%potential function $F(x,y) = -\frac{x^2}{2} + \ln \lvert y \rvert$, solving
%$F(x,y) = C$ leads to the same solution as the example.
%}

%\setcounter{exercise}{100}



\end{document}