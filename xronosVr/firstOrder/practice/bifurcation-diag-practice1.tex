\documentclass{ximera}

\title{Practice for bifurcation diagrams}

%\auor{Matthew Charnley and Jason Nowell}
\usepackage[margin=1.5cm]{geometry}
\usepackage{indentfirst}
\usepackage{sagetex}
\usepackage{lipsum}
\usepackage{amsmath}
\usepackage{mathrsfs}


%%% Random packages added without verifying what they are really doing - just to get initial compile to work.
\usepackage{tcolorbox}
\usepackage{hypcap}
\usepackage{booktabs}%% To get \toprule,\midrule,\bottomrule etc.
\usepackage{nicefrac}
\usepackage{caption}
\usepackage{units}

% This is my modified wrapfig that doesn't use intextsep
\usepackage{mywrapfig}
\usepackage{import}



%%% End to random added packages.


\graphicspath{
    {./figures/}
    {./../figures/}
    {./../../figures/}
}
\renewcommand{\log}{\ln}%%%%
\DeclareMathOperator{\arcsec}{arcsec}
%% New commands


%%%%%%%%%%%%%%%%%%%%
% New Conditionals %
%%%%%%%%%%%%%%%%%%%%


% referencing
\makeatletter
    \DeclareRobustCommand{\myvref}[2]{%
      \leavevmode%
      \begingroup
        \let\T@pageref\@pagerefstar
        \hyperref[{#2}]{%
	  #1~\ref*{#2}%
        }%
        \vpageref[\unskip]{#2}%
      \endgroup
    }%

    \DeclareRobustCommand{\myref}[2]{%
      \leavevmode%
      \begingroup
        \let\T@pageref\@pagerefstar
        \hyperref[{#2}]{%
	  #1~\ref*{#2}%
        }%
      \endgroup
    }%
\makeatother

\newcommand{\figurevref}[1]{\myvref{Figure}{#1}}
\newcommand{\figureref}[1]{\myref{Figure}{#1}}
\newcommand{\tablevref}[1]{\myvref{Table}{#1}}
\newcommand{\tableref}[1]{\myref{Table}{#1}}
\newcommand{\chapterref}[1]{\myref{chapter}{#1}}
\newcommand{\Chapterref}[1]{\myref{Chapter}{#1}}
\newcommand{\appendixref}[1]{\myref{appendix}{#1}}
\newcommand{\Appendixref}[1]{\myref{Appendix}{#1}}
\newcommand{\sectionref}[1]{\myref{\S}{#1}}
\newcommand{\subsectionref}[1]{\myref{subsection}{#1}}
\newcommand{\subsectionvref}[1]{\myvref{subsection}{#1}}
\newcommand{\exercisevref}[1]{\myvref{Exercise}{#1}}
\newcommand{\exerciseref}[1]{\myref{Exercise}{#1}}
\newcommand{\examplevref}[1]{\myvref{Example}{#1}}
\newcommand{\exampleref}[1]{\myref{Example}{#1}}
\newcommand{\thmvref}[1]{\myvref{Theorem}{#1}}
\newcommand{\thmref}[1]{\myref{Theorem}{#1}}


\renewcommand{\exampleref}[1]{ {\color{red} \bfseries Normally a reference to a previous example goes here.}}
\renewcommand{\figurevref}[1]{ {\color{red} \bfseries Normally a reference to a previous figure goes here.}}
\renewcommand{\tablevref}[1]{ {\color{red} \bfseries Normally a reference to a previous table goes here.}}
\renewcommand{\Appendixref}[1]{ {\color{red} \bfseries Normally a reference to an Appendix goes here.}}
\renewcommand{\exercisevref}[1]{ {\color{red} \bfseries Normally a reference to a previous exercise goes here.}}



\newcommand{\R}{\mathbb{R}}

%% Example Solution Env.
\def\beginSolclaim{\par\addvspace{\medskipamount}\noindent\hbox{\bf Solution:}\hspace{0.5em}\ignorespaces}
\def\endSolclaim{\par\addvspace{-1em}\hfill\rule{1em}{0.4pt}\hspace{-0.4pt}\rule{0.4pt}{1em}\par\addvspace{\medskipamount}}
\newenvironment{exampleSol}[1][]{\beginSolclaim}{\endSolclaim}

%% General figure formating from original book.
\newcommand{\mybeginframe}{%
\begin{tcolorbox}[colback=white,colframe=lightgray,left=5pt,right=5pt]%
}
\newcommand{\myendframe}{%
\end{tcolorbox}%
}

%%% Eventually return and fix this to make matlab code work correctly.
%% Define the matlab environment as another code environment
%\newenvironment{matlab}
%{% Begin Environment Code
%{ \centering \bfseries Matlab Code }
%\begin{code}
%}% End of Begin Environment Code
%{% Start of End Environment Code
%\end{code}
%}% End of End Environment Code


% this one should have a caption, first argument is the size
\newenvironment{mywrapfig}[2][]{
 \wrapfigure[#1]{r}{#2}
 \mybeginframe
 \centering
}{%
 \myendframe
 \endwrapfigure
}

% this one has no caption, first argument is size,
% the second argument is a larger size used for HTML (ignored by latex)
\newenvironment{mywrapfigsimp}[3][]{%
 \wrapfigure[#1]{r}{#2}%
 \centering%
}{%
 \endwrapfigure%
}
\newenvironment{myfig}
    {%
    \begin{figure}[h!t]
        \mybeginframe%
        \centering%
    }
    {%
        \myendframe
    \end{figure}%
    }


% graphics include
\newcommand{\diffyincludegraphics}[3]{\includegraphics[#1]{#3}}
\newcommand{\myincludegraphics}[3]{\includegraphics[#1]{#3}}
\newcommand{\inputpdft}[1]{\subimport*{../figures/}{#1.pdf_t}}


%% Not sure what these even do? They don't seem to actually work... fun!
%\newcommand{\mybxbg}[1]{\tcboxmath[colback=white,colframe=black,boxrule=0.5pt,top=1.5pt,bottom=1.5pt]{#1}}
%\newcommand{\mybxsm}[1]{\tcboxmath[colback=white,colframe=black,boxrule=0.5pt,left=0pt,right=0pt,top=0pt,bottom=0pt]{#1}}
\newcommand{\mybxsm}[1]{#1}
\newcommand{\mybxbg}[1]{#1}

%%% Something about tasks for practice/hw?
\usepackage{tasks}
\usepackage{footnote}
\makesavenoteenv{tasks}


%% For pdf only?
\newcommand{\diffypdfversion}[1]{#1}


%% Kill ``cite'' and go back later to fix it.
\renewcommand{\cite}[1]{}


%% Currently we can't really use index or its derivatives. So we are gonna kill them off.
\renewcommand{\index}[1]{}
\newcommand{\myindex}[1]{#1}







\begin{document}
\begin{abstract}
    Why?
\end{abstract}
\maketitle

\begin{exercise}
    Start with the logistic equation $\frac{dx}{dt} = kx(M-x)$. Suppose we modify our harvesting. That is we will only harvest an amount proportional to current population. In other words, we harvest $hx$ per unit of time for some $h > 0$ (Similar to earlier example with $h$ replaced with $hx$).
    
    Construct the differential equation: $\frac{dx}{dt} = \answer{kx\left(\frac{Mk-h}{k} - x\right)}$
    \begin{problem}
        If $kM > h$, then what can you say about the equation?
        \begin{multipleChoice}
            \choice{The equation is no longer logistic.}
            \choice[correct]{The equation is still logistic.}
            \choice{It depends - so we can't answer without more information.}
        \end{multipleChoice}
        \begin{problem}
            What happens when $kM < h$?
            \begin{multipleChoice}
                \choice{The population always increases.}
                \choice{The population doesn't change.}
                \choice{The population fluctuates - increasing sometimes and decreasing sometimes.}
                \choice[correct]{The population always decreases.}
            \end{multipleChoice}
        \end{problem}
    \end{problem}
\end{exercise}

%\comboSol
%{%
%a)~ $\frac{dx}{dt} = kx\left(\frac{Mk-h}{k} - x\right)$ \quad b)~It changes the effective M \quad c)~Population always decreases.
%}

\begin{exercise}
    Assume that a population of fish in a lake satisfies $\frac{dx}{dt} = kx(M-x)$.  Also suppose that fish are continually added at $A$ fish per unit of time.
    
    Find the differential equation for $x$: $\frac{dx}{dt} = \answer{kx(M-x) + A}$
    \begin{problem}
        What is the new limiting population? $\answer{\frac{kM + \sqrt{{(kM)}^2 + 4Ak}}{2k}}$
    \end{problem}
\end{exercise}
%\exsol{%
%a) $\frac{dx}{dt} = kx(M-x)+A$
%\quad
%b) $\frac{kM + \sqrt{{(kM)}^2 + 4Ak}}{2k}$
%}

\begin{exercise}
    Consider the differential equation with parameter $\alpha$ given by $y' = y(y - \alpha + 1)$. 
    \begin{hint}
        Sketch a phase diagram for this differential equation with $\alpha = -3$, $\alpha = 1$, and $\alpha = 3$.
    \end{hint}
    \begin{hint}
        Draw a bifurcation diagram for this differential equation with parameter. 
    \end{hint}
     What is the bifurcation point for this equation? $\alpha = \answer{1}$
     
     %What changes when $\alpha$ passes over the bifurcation point?

\end{exercise}
%\comboSol
%{%
%c)~$\alpha = 1$. The solution at $y=0$ changes stability.
%}

\begin{exercise}
    Consider the differential equation with parameter $\alpha$ given by $y' = y^2(y^2 - \alpha)$. 
    \begin{hint}
        Sketch a phase diagram for this differential equation with $\alpha = -3$, $\alpha = 1$, and $\alpha = 3$.
    \end{hint}
    \begin{hint}
        Draw a bifurcation diagram for this differential equation with parameter. 
    \end{hint}
        What is the bifurcation point for this equation? $\alpha = \answer{0}$
        
        %What changes when $\alpha$ passes over the bifurcation point?

\end{exercise}
%\comboSol
%{%
%c)~ $\alpha = 0$. Two new equilibrium solutions are created. 
%}

\begin{exercise}
    Consider the differential equation with parameter $\alpha$ given by $y' = y(\alpha - y)$. 
    \begin{hint}
        Sketch a phase diagram for this differential equation with $\alpha = -3$, $\alpha = 1$, and $\alpha = 3$.
    \end{hint}
    \begin{hint}
        Draw a bifurcation diagram for this differential equation with parameter. 
    \end{hint}
        What is the bifurcation point for this equation? $\alpha = \answer{0}$
        
        %What changes when $\alpha$ passes over the bifurcation point?
\end{exercise}
%\comboSol
%{%
%c)~ $\alpha = 0$, $y=0$ and $y=\alpha$ both change stability.
%}

\end{document}

%\setcounter{exercise}{100}