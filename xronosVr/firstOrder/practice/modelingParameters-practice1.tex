\documentclass{ximera}

\title{Practice for Modeling Parameters}

%\auor{Matthew Charnley and Jason Nowell}
\usepackage[margin=1.5cm]{geometry}
\usepackage{indentfirst}
\usepackage{sagetex}
\usepackage{lipsum}
\usepackage{amsmath}
\usepackage{mathrsfs}


%%% Random packages added without verifying what they are really doing - just to get initial compile to work.
\usepackage{tcolorbox}
\usepackage{hypcap}
\usepackage{booktabs}%% To get \toprule,\midrule,\bottomrule etc.
\usepackage{nicefrac}
\usepackage{caption}
\usepackage{units}

% This is my modified wrapfig that doesn't use intextsep
\usepackage{mywrapfig}
\usepackage{import}



%%% End to random added packages.


\graphicspath{
    {./figures/}
    {./../figures/}
    {./../../figures/}
}
\renewcommand{\log}{\ln}%%%%
\DeclareMathOperator{\arcsec}{arcsec}
%% New commands


%%%%%%%%%%%%%%%%%%%%
% New Conditionals %
%%%%%%%%%%%%%%%%%%%%


% referencing
\makeatletter
    \DeclareRobustCommand{\myvref}[2]{%
      \leavevmode%
      \begingroup
        \let\T@pageref\@pagerefstar
        \hyperref[{#2}]{%
	  #1~\ref*{#2}%
        }%
        \vpageref[\unskip]{#2}%
      \endgroup
    }%

    \DeclareRobustCommand{\myref}[2]{%
      \leavevmode%
      \begingroup
        \let\T@pageref\@pagerefstar
        \hyperref[{#2}]{%
	  #1~\ref*{#2}%
        }%
      \endgroup
    }%
\makeatother

\newcommand{\figurevref}[1]{\myvref{Figure}{#1}}
\newcommand{\figureref}[1]{\myref{Figure}{#1}}
\newcommand{\tablevref}[1]{\myvref{Table}{#1}}
\newcommand{\tableref}[1]{\myref{Table}{#1}}
\newcommand{\chapterref}[1]{\myref{chapter}{#1}}
\newcommand{\Chapterref}[1]{\myref{Chapter}{#1}}
\newcommand{\appendixref}[1]{\myref{appendix}{#1}}
\newcommand{\Appendixref}[1]{\myref{Appendix}{#1}}
\newcommand{\sectionref}[1]{\myref{\S}{#1}}
\newcommand{\subsectionref}[1]{\myref{subsection}{#1}}
\newcommand{\subsectionvref}[1]{\myvref{subsection}{#1}}
\newcommand{\exercisevref}[1]{\myvref{Exercise}{#1}}
\newcommand{\exerciseref}[1]{\myref{Exercise}{#1}}
\newcommand{\examplevref}[1]{\myvref{Example}{#1}}
\newcommand{\exampleref}[1]{\myref{Example}{#1}}
\newcommand{\thmvref}[1]{\myvref{Theorem}{#1}}
\newcommand{\thmref}[1]{\myref{Theorem}{#1}}


\renewcommand{\exampleref}[1]{ {\color{red} \bfseries Normally a reference to a previous example goes here.}}
\renewcommand{\figurevref}[1]{ {\color{red} \bfseries Normally a reference to a previous figure goes here.}}
\renewcommand{\tablevref}[1]{ {\color{red} \bfseries Normally a reference to a previous table goes here.}}
\renewcommand{\Appendixref}[1]{ {\color{red} \bfseries Normally a reference to an Appendix goes here.}}
\renewcommand{\exercisevref}[1]{ {\color{red} \bfseries Normally a reference to a previous exercise goes here.}}



\newcommand{\R}{\mathbb{R}}

%% Example Solution Env.
\def\beginSolclaim{\par\addvspace{\medskipamount}\noindent\hbox{\bf Solution:}\hspace{0.5em}\ignorespaces}
\def\endSolclaim{\par\addvspace{-1em}\hfill\rule{1em}{0.4pt}\hspace{-0.4pt}\rule{0.4pt}{1em}\par\addvspace{\medskipamount}}
\newenvironment{exampleSol}[1][]{\beginSolclaim}{\endSolclaim}

%% General figure formating from original book.
\newcommand{\mybeginframe}{%
\begin{tcolorbox}[colback=white,colframe=lightgray,left=5pt,right=5pt]%
}
\newcommand{\myendframe}{%
\end{tcolorbox}%
}

%%% Eventually return and fix this to make matlab code work correctly.
%% Define the matlab environment as another code environment
%\newenvironment{matlab}
%{% Begin Environment Code
%{ \centering \bfseries Matlab Code }
%\begin{code}
%}% End of Begin Environment Code
%{% Start of End Environment Code
%\end{code}
%}% End of End Environment Code


% this one should have a caption, first argument is the size
\newenvironment{mywrapfig}[2][]{
 \wrapfigure[#1]{r}{#2}
 \mybeginframe
 \centering
}{%
 \myendframe
 \endwrapfigure
}

% this one has no caption, first argument is size,
% the second argument is a larger size used for HTML (ignored by latex)
\newenvironment{mywrapfigsimp}[3][]{%
 \wrapfigure[#1]{r}{#2}%
 \centering%
}{%
 \endwrapfigure%
}
\newenvironment{myfig}
    {%
    \begin{figure}[h!t]
        \mybeginframe%
        \centering%
    }
    {%
        \myendframe
    \end{figure}%
    }


% graphics include
\newcommand{\diffyincludegraphics}[3]{\includegraphics[#1]{#3}}
\newcommand{\myincludegraphics}[3]{\includegraphics[#1]{#3}}
\newcommand{\inputpdft}[1]{\subimport*{../figures/}{#1.pdf_t}}


%% Not sure what these even do? They don't seem to actually work... fun!
%\newcommand{\mybxbg}[1]{\tcboxmath[colback=white,colframe=black,boxrule=0.5pt,top=1.5pt,bottom=1.5pt]{#1}}
%\newcommand{\mybxsm}[1]{\tcboxmath[colback=white,colframe=black,boxrule=0.5pt,left=0pt,right=0pt,top=0pt,bottom=0pt]{#1}}
\newcommand{\mybxsm}[1]{#1}
\newcommand{\mybxbg}[1]{#1}

%%% Something about tasks for practice/hw?
\usepackage{tasks}
\usepackage{footnote}
\makesavenoteenv{tasks}


%% For pdf only?
\newcommand{\diffypdfversion}[1]{#1}


%% Kill ``cite'' and go back later to fix it.
\renewcommand{\cite}[1]{}


%% Currently we can't really use index or its derivatives. So we are gonna kill them off.
\renewcommand{\index}[1]{}
\newcommand{\myindex}[1]{#1}







\begin{document}
\begin{abstract}
    Why?
\end{abstract}
\maketitle


\begin{exercise}
    Bob is getting coffee from a restaurant and knows that the temperature of the coffee will follow Newton's Law of Cooling, which says that
    \begin{equation*}
        \frac{dT}{dt} = k(T_0 - T)
    \end{equation*}
    where $T_0$ is the ambient temperature and $k$ is a constant depending on the object and geometry. His car is held at a constant 20$^\circ$ C, and when he receives the coffee, he measures the temperature to be 90$^\circ$ C. Two minutes later, the temperature is 81$^\circ$C.
    \begin{tasks}
        \task Use these two points of data to determine the value of $k$ for this coffee.
        \task Bob only wants to drink his coffee once it reaches 65$^\circ$ C. How long does he have to wait for this to happen? 
        \task If the coffee is too cold for Bob's taste once it reaches 35$^\circ$ C, how long is the acceptable window for Bob to drink his coffee?
    \end{tasks} 
\end{exercise}
%\comboSol
%{%
%a)~ $\approx$ 0.06681 \quad b)~$\approx$ 6.42 min \quad c)~ $\approx$ 16 min.
%}

\begin{exercise}
    Assume that a falling object has a velocity (in meters per second) that obeys the differential equation
    \begin{equation*}
        \frac{dv}{dt} = 9.8 - \alpha v
    \end{equation*}
    where $\alpha$ represents the drag coefficient of the object.
    \begin{tasks}
        \task Solve this differential equation with initial condition $v(0) = 0$ to get a solution that depends on $\alpha$. 
        \task Assume that you drop an object from a height of 10 meters and it hits the ground after $3$ seconds. What is the value of $\alpha$ here? (Note: You solved for $v(t)$ in the previous part, and now you need to get to position. What does that require?)
        \task Assume that a second object hits the ground in 6 seconds. How does this change the value of $\alpha$?
    \end{tasks}
\end{exercise}
%\comboSol
%{%
%a)~ $v(t) = \frac{9.8}{\alpha}(1 - e^{-\alpha t})$ \quad b)~2.94 \quad c)~5.88
%}

\begin{exercise}
    A restaurant is trying to analyze the to-go coffee cups that it uses in order to best serve their customers. They assume that the coffee follows Newton's Law of Cooling and place it in a room with ambient temperature 22$^\circ$ C. They record the following data for the temperature of the coffee as a function of time.
    \begin{table}[h!!]
        \centering
        \begin{tabular}{|c|c|}\hline
             \textbf{t} (min)& \textbf{T} ($^\circ$ C)  \\ \hline
            0 & 80 \\
             0.5 & 74 \\
             1.1 & 67 \\
             1.7 & 61 \\
             2.3 & 56 \\ \hline
        \end{tabular}
    \end{table}
    \begin{tasks}
        \task Use code to determine the best-fit value of $k$ for this data.
        \task The restaurant determines that to avoid any potential legal issues, the coffee can be no warmer than 60 $^\circ$C when it is served. If the coffee comes out of the machine at 90 $^\circ C$, how long do they have to wait before they can serve the coffee?
    \end{tasks}
\end{exercise}
%\comboSol
%{%
%a)~ $k \approx 0.2317$ \quad b)~$\approx 2.51$ min
%}

\newpage

\begin{exercise}\label{ex:ModelingParamVel}%
    Assume that an object falling has a velocity that follows the differential equation 
    \begin{equation*}
        \frac{dv}{dt} = 9.8 - \alpha v^2
    \end{equation*}
    where the velocity is in \unitfrac{m}{s} and $\alpha$ represents the drag coefficient. During a physics experiment, a student measures data for the velocity of a falling object over time given in the table below.
    
    Use this data (and code) to estimate the value of $\alpha$ for this object. 
\end{exercise} 
%\exsol{%
%$\alpha = .1244$. The alpha value used before noise was added to the data is $0.124$, so very close, but not identically the same.%
%}

\begin{minipage}{0.49\textwidth}
\centering
    \begin{tabular}{|c|c|}\hline
         \textbf{t} (s)& \textbf{v} (m/s)  \\ \hline
        0 & 0 \\
    	0.1 &1.0 \\
    	0.2 &  1.9\\
    	0.4 & 3.6 \\
    	0.6 & 5.2 \\
    	0.9 & 6.8 \\
    	1.1 & 7.4 \\
    	1.3 & 7.9 \\
    	1.5 & 8.2 \\
    	1.8 & 8.5\\
    	2.1 & 8.8  \\ \hline
    \end{tabular}
    \captionof{table}{Data for \exerciseref{ex:ModelingParamVel}.}
\end{minipage}%
\begin{minipage}{0.49\textwidth}
    \centering
    \begin{tabular}{|c|c|}\hline
         \textbf{t} (d)& \textbf{P} (thousands)  \\ \hline
        0 & 50 \\
       7  & 60 \\
        14 & 70 \\
        28 & 97 \\
        37 & 117 \\
         50 & 148 \\
        78 & 220 \\
        100 & 268
         \\ \hline
    \end{tabular}
    \captionof{table}{Data for \exerciseref{ex:ModelingParamPop}.}
\end{minipage}

\begin{exercise}\label{ex:ModelingParamPop}
    Assume that a species of fish in a lake has a population that is modeled by the differential equation
    \begin{equation*}
        \frac{dP}{dt} = \frac{1}{100}rP(K - P) - \alpha 
    \end{equation*}
    where $r$, $K$, and $\alpha$ are parameters, $r$ representing the growth rate, $K$ the carrying capacity, and $\alpha$ the harvesting rate, and the population $P$ is in thousands., with $t$ given in years. From previous studies, you know that the best value of $r$ is $3.12$. After studying the population over a period of time, you get the data given above.
    \begin{tasks}
        \task Your friend tells you that in a previous study, he found that the value of $K$ for this particular lake is $450$. Use code to determine the best value of $\alpha$ for this situation. Note that the equation expects $t$ in years, but the data is given in days. Search for $\alpha$ in the range $(0, 400)$.
        \task That answer doesn't look great. Plot the solution with these parameters along with the data and compare them.
        \task The fit does not look great, so maybe your friend's value was not quite right. Run code to find best values for $K$ and $\alpha$ simultaneously. Use the range $(0, 400)$ for both $\alpha$ and $K$.
    \end{tasks}

\end{exercise}
%\comboSol
%{%
%a)~ $\alpha = 337.55$ \quad c)~ $\alpha = 6.53$, $K = 350.45$. (The base data was $K = 350$, $\alpha = 3.79$.)
%}



\end{document}