\documentclass{ximera}

\title{Practice for Modeling}

%\auor{Matthew Charnley and Jason Nowell}
\usepackage[margin=1.5cm]{geometry}
\usepackage{indentfirst}
\usepackage{sagetex}
\usepackage{lipsum}
\usepackage{amsmath}
\usepackage{mathrsfs}


%%% Random packages added without verifying what they are really doing - just to get initial compile to work.
\usepackage{tcolorbox}
\usepackage{hypcap}
\usepackage{booktabs}%% To get \toprule,\midrule,\bottomrule etc.
\usepackage{nicefrac}
\usepackage{caption}
\usepackage{units}

% This is my modified wrapfig that doesn't use intextsep
\usepackage{mywrapfig}
\usepackage{import}



%%% End to random added packages.


\graphicspath{
    {./figures/}
    {./../figures/}
    {./../../figures/}
}
\renewcommand{\log}{\ln}%%%%
\DeclareMathOperator{\arcsec}{arcsec}
%% New commands


%%%%%%%%%%%%%%%%%%%%
% New Conditionals %
%%%%%%%%%%%%%%%%%%%%


% referencing
\makeatletter
    \DeclareRobustCommand{\myvref}[2]{%
      \leavevmode%
      \begingroup
        \let\T@pageref\@pagerefstar
        \hyperref[{#2}]{%
	  #1~\ref*{#2}%
        }%
        \vpageref[\unskip]{#2}%
      \endgroup
    }%

    \DeclareRobustCommand{\myref}[2]{%
      \leavevmode%
      \begingroup
        \let\T@pageref\@pagerefstar
        \hyperref[{#2}]{%
	  #1~\ref*{#2}%
        }%
      \endgroup
    }%
\makeatother

\newcommand{\figurevref}[1]{\myvref{Figure}{#1}}
\newcommand{\figureref}[1]{\myref{Figure}{#1}}
\newcommand{\tablevref}[1]{\myvref{Table}{#1}}
\newcommand{\tableref}[1]{\myref{Table}{#1}}
\newcommand{\chapterref}[1]{\myref{chapter}{#1}}
\newcommand{\Chapterref}[1]{\myref{Chapter}{#1}}
\newcommand{\appendixref}[1]{\myref{appendix}{#1}}
\newcommand{\Appendixref}[1]{\myref{Appendix}{#1}}
\newcommand{\sectionref}[1]{\myref{\S}{#1}}
\newcommand{\subsectionref}[1]{\myref{subsection}{#1}}
\newcommand{\subsectionvref}[1]{\myvref{subsection}{#1}}
\newcommand{\exercisevref}[1]{\myvref{Exercise}{#1}}
\newcommand{\exerciseref}[1]{\myref{Exercise}{#1}}
\newcommand{\examplevref}[1]{\myvref{Example}{#1}}
\newcommand{\exampleref}[1]{\myref{Example}{#1}}
\newcommand{\thmvref}[1]{\myvref{Theorem}{#1}}
\newcommand{\thmref}[1]{\myref{Theorem}{#1}}


\renewcommand{\exampleref}[1]{ {\color{red} \bfseries Normally a reference to a previous example goes here.}}
\renewcommand{\figurevref}[1]{ {\color{red} \bfseries Normally a reference to a previous figure goes here.}}
\renewcommand{\tablevref}[1]{ {\color{red} \bfseries Normally a reference to a previous table goes here.}}
\renewcommand{\Appendixref}[1]{ {\color{red} \bfseries Normally a reference to an Appendix goes here.}}
\renewcommand{\exercisevref}[1]{ {\color{red} \bfseries Normally a reference to a previous exercise goes here.}}



\newcommand{\R}{\mathbb{R}}

%% Example Solution Env.
\def\beginSolclaim{\par\addvspace{\medskipamount}\noindent\hbox{\bf Solution:}\hspace{0.5em}\ignorespaces}
\def\endSolclaim{\par\addvspace{-1em}\hfill\rule{1em}{0.4pt}\hspace{-0.4pt}\rule{0.4pt}{1em}\par\addvspace{\medskipamount}}
\newenvironment{exampleSol}[1][]{\beginSolclaim}{\endSolclaim}

%% General figure formating from original book.
\newcommand{\mybeginframe}{%
\begin{tcolorbox}[colback=white,colframe=lightgray,left=5pt,right=5pt]%
}
\newcommand{\myendframe}{%
\end{tcolorbox}%
}

%%% Eventually return and fix this to make matlab code work correctly.
%% Define the matlab environment as another code environment
%\newenvironment{matlab}
%{% Begin Environment Code
%{ \centering \bfseries Matlab Code }
%\begin{code}
%}% End of Begin Environment Code
%{% Start of End Environment Code
%\end{code}
%}% End of End Environment Code


% this one should have a caption, first argument is the size
\newenvironment{mywrapfig}[2][]{
 \wrapfigure[#1]{r}{#2}
 \mybeginframe
 \centering
}{%
 \myendframe
 \endwrapfigure
}

% this one has no caption, first argument is size,
% the second argument is a larger size used for HTML (ignored by latex)
\newenvironment{mywrapfigsimp}[3][]{%
 \wrapfigure[#1]{r}{#2}%
 \centering%
}{%
 \endwrapfigure%
}
\newenvironment{myfig}
    {%
    \begin{figure}[h!t]
        \mybeginframe%
        \centering%
    }
    {%
        \myendframe
    \end{figure}%
    }


% graphics include
\newcommand{\diffyincludegraphics}[3]{\includegraphics[#1]{#3}}
\newcommand{\myincludegraphics}[3]{\includegraphics[#1]{#3}}
\newcommand{\inputpdft}[1]{\subimport*{../figures/}{#1.pdf_t}}


%% Not sure what these even do? They don't seem to actually work... fun!
%\newcommand{\mybxbg}[1]{\tcboxmath[colback=white,colframe=black,boxrule=0.5pt,top=1.5pt,bottom=1.5pt]{#1}}
%\newcommand{\mybxsm}[1]{\tcboxmath[colback=white,colframe=black,boxrule=0.5pt,left=0pt,right=0pt,top=0pt,bottom=0pt]{#1}}
\newcommand{\mybxsm}[1]{#1}
\newcommand{\mybxbg}[1]{#1}

%%% Something about tasks for practice/hw?
\usepackage{tasks}
\usepackage{footnote}
\makesavenoteenv{tasks}


%% For pdf only?
\newcommand{\diffypdfversion}[1]{#1}


%% Kill ``cite'' and go back later to fix it.
\renewcommand{\cite}[1]{}


%% Currently we can't really use index or its derivatives. So we are gonna kill them off.
\renewcommand{\index}[1]{}
\newcommand{\myindex}[1]{#1}







\begin{document}
\begin{abstract}
    Why?
\end{abstract}
\maketitle

\begin{exercise}
    Suppose there are two lakes located on a stream.  Clean water flows into the first lake, then the water from the first lake flows into the second lake, and then water from the second lake flows further downstream. The in and out flow from each lake is 500 liters per hour. The first lake contains 100 thousand liters of water and the second lake contains 200 thousand liters of water. A truck with 500 kg of toxic substance crashes into the first lake.  Assume that the water is being continually mixed perfectly by the stream.
    \begin{itemize}
        \item Find the concentration of toxic substance as a function of time in both lakes. $c_1(t) = \answer{\frac{1}{200}e^{-t/200}}$, $c_2(t) = \answer{\frac{1}{200}(e^{-t/400} - e^{-t/200})}$
        \item When will the concentration in the first lake be below 0.001 kg per liter? $\answer{321.89}$ hours. [Round your answer to two decimal places]
        \item When will the concentration in the second lake be maximal? $\answer{277.26}$ hours. [Round your answer to two decimal places]
    \end{itemize}
\end{exercise}
%\comboSol
%{%
%a)~$c_1(t) = \frac{1}{200}e^{-t/200}$\ $c_2(t) = \frac{1}{200}(e^{-t/400} - e^{-t/200})$ \quad
%b)~321.89 hours \quad c)~277.26 hours
%}

\begin{exercise}
    Newton's law of cooling states that $\frac{dx}{dt} = -k(x-A)$ where $x$ is the temperature, $t$ is time, $A$ is the ambient temperature, and $k > 0$ is a constant. Suppose that $A = A_0 \cos (\omega t)$ for some constants $A_0$ and $\omega$. That is, the ambient temperature oscillates (for example night and day temperatures).
    \begin{itemize}
        \item Find the general solution: $x(t) = \answer{kA_0 \left(\frac{\omega \sin(\omega t) + k\cos(\omega t)}{\omega^2 + k^2}\right) + Ce^{-kt}}$. [Use $C$ for an arbitrary constant if needed].
        \item In the long term, will the initial conditions make much of a difference to your equation? \wordChoice{\choice[correct]{Yes.}\choice{No.}}.
    \end{itemize}
\end{exercise}
%\comboSol
%{%
%a)~$x(t) = kA_0 \left(\frac{\omega \sin(\omega t) + k\cos(\omega t)}{\omega^2 + k^2}\right) + Ce^{-kt}$ \quad
%b)~No. Only in $C$.
%}

\begin{exercise}
    Initially 5 grams of salt are dissolved in 20 liters of water. Brine with concentration of 2 grams of salt per liter is added at a rate of 3 liters per minute.  The tank is mixed well and is drained at 3 liters per minute.  How long does the process have to continue until there are 20 grams of salt in the tank? [Round your answer to 2 decimals] $\answer{3.73}$ minutes.
\end{exercise}
%\comboSol
%{%
%$\approx 3.73$ min
%}

\begin{exercise}
    Initially a tank contains 10 liters of pure water. Brine of unknown (but constant) concentration of salt is flowing in at 1 liter per minute. The water is mixed well and drained at 1 liter per minute. In 20 minutes there are 15 grams of salt in the tank.  What is the concentration of salt in the incoming brine? [Round your answer to 3 decimals] $\answer{1.735}$ g/L.
\end{exercise}
%\comboSol
%{%
%$\approx 1.735$ g/L
%}

\begin{exercise}%
    Suppose a water tank is being pumped out at \unitfrac[3]{L}{min}.  The water tank starts at \unit[10]{L} of clean water. Water with toxic substance is flowing into the tank at \unitfrac[2]{L}{min}, with concentration \unitfrac[$20t$]{g}{L} at time $t$. When the tank is half empty, how many grams of toxic substance are in the tank (assuming perfect mixing)? [Round your answer to nearest whole number] $\answer{250}$ grams.
\end{exercise}
%\exsol{%
%$250$ grams
%}

\begin{exercise}
    A 300 gallon well-mixed water tank initially starts with 200 gallons of water and 15 lbs of salt. One stream with salt concentration one pound per gallon flows into the tank at a rate of 3 gallons per minute and water is removed from the well-mixed tank at a rate of 2 gallons per minute.
    
    Write an initial value problem for the volume of water in the tank at any time $t$. $\frac{dV}{dt} = \answer{1}$.
    
    \begin{problem}
        What is the solution to this DE? $V(0) = \answer{200}$, $V(t) = \answer{200 + t}$
    \end{problem}
    
    Set up an initial value problem for the amount of salt in the tank at any time $t$. $\frac{dQ}{dt} = \answer{3 - \frac{2Q}{200+t}}$, $Q(0) = \answer{15}$.
    
    \begin{problem}
        Is the solution to this initial value problem a valid representation of the physical model for all times $t > 0$?
        \begin{multipleChoice}
            \choice{Yes.}
            \choice[correct]{No.}
        \end{multipleChoice}
        \begin{problem}
            Indeed, the water will overflow. Determine the time when this happens: $t = \answer{100}$.
            \begin{problem}
                Solve the initial value problem.
                \[
                    Q(t) = \answer{(200 + t) - \frac{185(200)^2}{(200+t)^2}}
                \]
            \end{problem}
        \end{problem}
    \end{problem}
%    \begin{itemize}
%        \item Write and solve an initial value problem for the volume of water in the tank at any time $t$. $\frac{dV}{dt} = \answer{1}$.
%        \item Set up an initial value problem for the amount of salt in the tank at any time $t$. You do not need to solve it (yet), but should make sure to state it fully.
%        \item Is the solution to this initial value problem a valid representation of the physical model for all times $t > 0$? If so, use the information in the equation to determine the long-time behavior of the solution. If not, explain why, determine the time when the representation breaks down, and what happens at that point in time.
%        \item Solve the initial value problem above and compare this to your answer to the previous part.
%    \end{itemize}
\end{exercise}
%\comboSol
%{%
%a)~$\frac{dV}{dt} = 1$, $V(0) = 200$. $V(t) = 200+t$. \quad
%b)~$\frac{dQ}{dt} = 3 - \frac{2Q}{200+t}$, $Q(0) = 15$. \quad
%c)~No. Overflows at $t=100$. \quad
%d)~$Q(t) = (200+t) - \frac{185(200)^2}{(200+t)^2}$ This function is defined for all positive $t$.
%}

\begin{exercise}
    A 500 gallon well-mixed water tank initially starts with 300 gallons of water and 200 lbs of salt. One stream with salt concentration of \unitfrac[0.5]{lb}{gal} flows into the tank at a rate of \unitfrac[5]{gal}{min} and water is removed from the well-mixed tank at a rate of \unitfrac[7]{gal}{min}.
    
    Write an initial value problem for the volume of water in the tank at any time $t$.
    \[
        \frac{dV}{dt} = \answer{-2}, \quad V(0) = \answer{300}
    \]
    \begin{problem}
        Solve the initial value problem.
        \[
            V(t) = \answer{300-2t} 
        \]
        \begin{problem}
            Set up an initial value problem for the amount of salt in the tank at any time $t$
            \[
                \frac{dQ}{dt} = \answer{2.5 - \frac{7Q}{300-2t}}, \quad Q(0) = \answer{200}
            \]
            \begin{problem}
                Is the solution to this initial value problem a valid representation of the physical model for all times $t > 0$?
                \begin{multipleChoice}
                    \choice{Yes.}
                    \choice[correct]{No.}
                \end{multipleChoice}
                \begin{problem}
                    Indeed, the tank empties. Determine when this happens: $t = \answer{150}$.
                    \begin{problem}
                        Solve the IVP above: $Q(t) = \answer{0.5(300-2t) + \frac{50}{(300)^{7/2}}(300-2t)^{7/2}}$
                    \end{problem}
                \end{problem}
            \end{problem}
        \end{problem}
    \end{problem}
    
%    \begin{tasks}
%        \task Write and solve an initial value problem for the volume of water in the tank at any time $t$.
%        \task Set up an initial value problem for the amount of salt in the tank at any time $t$. You do not need to solve it (yet), but should make sure to state it fully.
%        \task Is the solution to this initial value problem a valid representation of the physical model for all times $t > 0$? If so, use the information in the equation to determine the long-time behavior of the solution. If not, explain why, determine the time when the representation breaks down, and what happens at that point in time.
%        \task Solve the initial value problem above and compare this to your answer to the previous part.
%    \end{tasks}
\end{exercise}
%\comboSol
%{%
%a)~$\frac{dV}{dt} = -2$, $V(0) = 300$. $V(t) = 300-2t$.\quad
%b)~$\frac{dQ}{dt} = 2.5 - \frac{7Q}{300-2t}$, $Q(0) = 200$ \quad
%c)~No, tank empties at $t = 150$. \quad
%d)~$Q(t) = 0.5(300-2t) + \frac{50}{(300)^{7/2}}(300-2t)^{7/2}$. $Q(t)$ is negative for $t > 150$. 
%}

\begin{exercise}
    A 200 gallon well-mixed water tank initially starts with 150 gallons of water and 50 lbs of salt. One stream with salt concentration of \unitfrac[0.2]{lb}{gal} flows into the tank at a rate of \unitfrac[4]{gal}{min} and water is removed from the well-mixed tank at a rate of \unitfrac[4]{gal}{min}.

    Write an initial value problem for the volume of water in the tank at any time $t$.
    \[
        \frac{dV}{dt} = \answer{0}, \quad V(0) = \answer{150}
    \]
    \begin{problem}
        Solve the initial value problem.
        \[
            V(t) = \answer{150} 
        \]
        \begin{problem}
            Set up an initial value problem for the amount of salt in the tank at any time $t$
            \[
                \frac{dQ}{dt} = \answer{0.8 - \frac{4}{150}Q}, \quad Q(0) = \answer{50}
            \]
            \begin{problem}
                Is the solution to this initial value problem a valid representation of the physical model for all times $t > 0$?
                \begin{multipleChoice}
                    \choice[correct]{Yes.}
                    \choice{No.}
                \end{multipleChoice}
                \begin{problem}
                    Indeed, the tank stabilizes. Determine what it stabilizes to: $\answer{30}$.
                    \begin{problem}
                        Solve the IVP above: $Q(t) = \answer{30 + 20e^{-2t/75}}$
                    \end{problem}
                \end{problem}
            \end{problem}
        \end{problem}
    \end{problem}
%    \begin{tasks}
%        \task Write and solve an initial value problem for the volume of water in the tank at any time $t$.
%        \task Set up an initial value problem for the amount of salt in the tank at any time $t$. You do not need to solve it (yet), but should make sure to state it fully.
%        \task Is the solution to this initial value problem a valid representation of the physical model for all times $t > 0$? If so, use the information in the equation to determine the long-time behavior of the solution. If not, explain why, determine the time when the representation breaks down, and what happens at that point in time.
%        \task Solve the initial value problem above and compare this to your answer to the previous part.
%    \end{tasks}
\end{exercise}
%\comboSol
%{%
%a)~ $\frac{dV}{dt} = 0$, $V(0) = 150$. $V(t) = 150$.\quad
%b)~ $\frac{dQ}{dt} = 0.8 - \frac{4}{150}Q$, $Q(0) = 50$.\quad
%c)~Yes. Solution tends towards 30.\quad
%d)~ $Q(t) = 30 + 20e^{-2t/75}$. Matches.
%}

\begin{exercise}%
    Suppose we have bacteria on a plate and suppose that we are slowly adding a toxic substance such that the rate of growth is slowing down.  That is, suppose that $\frac{dP}{dt} = (2 - 0.1t)P$.  If $P(0) = 1000$, find the population at $t=5$. [Round to 3 significant figures]$\answer{6310000}$.
\end{exercise}
%\exsol{%
%$P(5) = 1000 e^{2 \times 5 - 0.05 \times {5}^2} = 1000 e^{8.75} \approx
%6.31 \times {10}^6$
%}

\begin{exercise}%
    A cylindrical water tank has water flowing in at $I$ cubic meters per second. Let $A$ be the area of the cross section of the tank in meters. Suppose water is flowing from the bottom of the tank at a rate proportional to the height of the water level.  Set up the differential equation for $h$, the height of the water. Use $k$ as a constant with units $\unit{m^2}{s}$
    \[
        Ah' = \answer{I - kh}
    \]
\end{exercise}
%\exsol{%
%$Ah' = I - kh$, where $k$ is a constant with units $\unit{m^2}{s}$.
%}

\begin{exercise}
    An object in free fall has a velocity that increases at a rate of 32 $ft/s^2$. Due to drag, the velocity decreases at a rate of 0.1 times the velocity of the object squared, when written in feet per second. 
    
    Write a differential equation to model the velocity of this object over time.
    \[
        \frac{dv}{dt} = \answer{32-0.1v^2}
    \]
    \begin{problem}
        This equation is autonomous, so draw a phase diagram for this equation and classify all critical points.
        \begin{itemize}
            \item $v = \answer{\sqrt{320}}$ is asymptotically stable.
            \item $v = \answer{-\sqrt{320}}$ is asymptotically unstable.
        \end{itemize}
        \begin{problem}
            What will happen to the velocity if the object is dropped at $t=0$? What about if the object is thrown downwards at a rate of $10 ft/s$? It will:
            \begin{multipleChoice}
                \choice{change depending on the initial velocity the object is dropped.}
                \choice{always become unstable.}
                \choice{always become stable.}
            \end{multipleChoice}
            \begin{problem}
                It will always stabilize toward the value $\answer{\sqrt{320}}$.
            \end{problem}
        \end{problem}
    \end{problem}
\end{exercise}
%\comboSol
%{%
%a)~$\frac{dv}{dt} = 32-0.1v^2$ \quad b)~ $v = \sqrt{320}$ is asymptotically stable, $v = -\sqrt{320}$ is unstable. \quad c)~ Always tends towards $\sqrt{320}$
%}

\begin{exercise}
    The number of people in a town that support a given measure decays at a constant rate of $10$ people per day. However, the support for the measure can be increased by individuals discussing the issue. This results in an increase of the support at a rate of $ay(1000 - y)$ people per day, where $y$ is the number of people who support the measure, and $a$ is a constant depending on the way in which the issue is being discussed. Write a differential equation to model this situation.
    \[
        \frac{dy}{dt} = \answer{ay(1000-y) - 10}
    \]
    \begin{problem}
        Determine the amount of people who will support the measure long-term if $a$ is set to $10^{-4}$. [Round your answer to 1 decimal place] $\answer{887.3}$
    \end{problem}
\end{exercise}
%\comboSol
%{%
%$\frac{dy}{dt} = ay(1000-y) - 10$, $=100(5 + \sqrt{15}) \approx 887.3$.
%}

\begin{exercise}
    Newton's Law of Procrastination states that the rate at which one accomplishes a chore is {\it proportional to the amount of the chore not yet done}. Unbeknownst to Newton, this applies to robots too. A Roomba is attempting to vacuum a house measuring 1000 square feet. When none of the house is clean, the roomba can clean 200 square feet per hour. What makes this problem fun is that there is also a dog. It's whatever kind of dog you like, take your pick. The dog dirties the house at a constant rate of 50 square feet per hour.
    
    Assume that none of the house is clean at $t=0$. Write a DE for the number of square feet that are clean as a function of time
    \[
        frac{dy}{dt} = \answer{\frac{1}{5}(1000 - y) - 50}
    \]
    \begin{problem}
        Solve for that quantity. $y(t) = \answer{750(1-e^{-t/5})}$
        \begin{problem}
            How long will it take before the house is half clean? $t = \answer{5\ln(3)}$
            \begin{problem}
                Will it ever be entirely clean?
                \begin{multipleChoice}
                    \choice{Yes.}
                    \choice[correct]{No.}
                \end{multipleChoice}
            \end{problem}
        \end{problem}
    \end{problem}
\end{exercise}
%\comboSol
%{%
%a)~$\frac{dy}{dt} = \frac{1}{5}(1000 - y) - 50 = 150 - \frac{1}{5}y$. $y(t) = 750(1-e^{-t/5})$ \\
%b)~$t = 5\ln(3)$ for half clean. Never fully clean.
%}

\begin{exercise}
    A student has a loan for \$50000 with 5\% interest. The student makes \$300 payments on the loan each month. The rate here is an annual rate, compounded continuously, and the differential equation you write should be in years.
    
    With this setup, how long does it take the student to pay off the loan? [Round your answer to 3 decimals] $\answer{27.712}$ years.
    \begin{problem}
        How much money does the student pay over this period of time?  Approximately \$$\answer{85,500}$ [Use 3 significant figures]
        \begin{problem}
            What is the minimal amount the student should pay each month if they want to pay off the loan within 5 years? \$$\answer{942}$
        \end{problem}
    \end{problem}
\end{exercise}
%\comboSol
%{%
%a)~$27.712$ years or 285 months. Pays approximately \$85,500 \quad b)~ $\approx \$942$
%}

\begin{exercise}
    A factory pumps 6 tons of sludge per day into a nearby pond. The pond initially contains 100,000 gallons of water, and no sludge. Each day, 3,000 gallons of rain water falls into the pond, and 1,000 gallons per day leave the pond via a river. Assume, for no good reason, that the water leaving the pond has the same concentration of sludge as the pond as a whole. How much sludge will there be in the pond after 150 days? %1.10
    
    $\answer{700}$ gallons.
\end{exercise}
%\comboSol
%{%
%700 gallons
%}

\begin{exercise}
    In this exercise, we compare two different young people and their investment strategies. Both of these people are investing in an account with 7.5\% annual rate of return. Person 1 invests \$50 a month starting at age 20, and Person 2 invests \$100 per month starting at age 30. Write differential equations to model each of these account balances over time, and compute the amount of money in each account at age 50. Who has more money in the account? 
    \begin{multipleChoice}
        \choice[correct]{Person 1.}
        \choice{Person 2.}
    \end{multipleChoice}
    \begin{problem}
        Who has invested more money? 
        \begin{multipleChoice}
            \choice{Person 1.}
            \choice[correct]{Person 2.}
        \end{multipleChoice}
        \begin{problem}
            What would person 2 have to invest each month in order for the two balances to be equal at age 50? [Round to whole dollar] \$$\answer{122}$
        \end{problem}
    \end{problem}
\end{exercise}
%\comboSol
%{%
%Person 1 has more money ($\approx \$67,902$ compared to $\approx \$55,707$). Person 2 has invested more ($\$24,000$ compared to $\$18,000$). Person 2 needs to invest $\approx \$122$. 
%}

\begin{exercise}
    Radioactive decay follows similar rules to interest, where a certain portion of the material decays over time, resulting in an equation of the form 
    \[ 
        \frac{dy}{dt} = -ky 
    \] 
    for some constant $k$. The \emph{half-life} of a material is the amount of time that it takes for half of the material to have decayed away. Assume that the \emph{half-life} of a given substance is $T$ minutes. Find a formula for $k$, the coefficient in the decay equation, in terms of $T$.
    \[
        k = \answer{\frac{1}{T}\ln(2)}
    \]
\end{exercise}

%\comboSol
%{%
%$k = \frac{1}{T}\ln(2)$
%}



\end{document}