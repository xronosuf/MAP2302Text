\documentclass{ximera}

\title{Practice for Euler Method}

%\auor{Matthew Charnley and Jason Nowell}
\usepackage[margin=1.5cm]{geometry}
\usepackage{indentfirst}
\usepackage{sagetex}
\usepackage{lipsum}
\usepackage{amsmath}
\usepackage{mathrsfs}


%%% Random packages added without verifying what they are really doing - just to get initial compile to work.
\usepackage{tcolorbox}
\usepackage{hypcap}
\usepackage{booktabs}%% To get \toprule,\midrule,\bottomrule etc.
\usepackage{nicefrac}
\usepackage{caption}
\usepackage{units}

% This is my modified wrapfig that doesn't use intextsep
\usepackage{mywrapfig}
\usepackage{import}



%%% End to random added packages.


\graphicspath{
    {./figures/}
    {./../figures/}
    {./../../figures/}
}
\renewcommand{\log}{\ln}%%%%
\DeclareMathOperator{\arcsec}{arcsec}
%% New commands


%%%%%%%%%%%%%%%%%%%%
% New Conditionals %
%%%%%%%%%%%%%%%%%%%%


% referencing
\makeatletter
    \DeclareRobustCommand{\myvref}[2]{%
      \leavevmode%
      \begingroup
        \let\T@pageref\@pagerefstar
        \hyperref[{#2}]{%
	  #1~\ref*{#2}%
        }%
        \vpageref[\unskip]{#2}%
      \endgroup
    }%

    \DeclareRobustCommand{\myref}[2]{%
      \leavevmode%
      \begingroup
        \let\T@pageref\@pagerefstar
        \hyperref[{#2}]{%
	  #1~\ref*{#2}%
        }%
      \endgroup
    }%
\makeatother

\newcommand{\figurevref}[1]{\myvref{Figure}{#1}}
\newcommand{\figureref}[1]{\myref{Figure}{#1}}
\newcommand{\tablevref}[1]{\myvref{Table}{#1}}
\newcommand{\tableref}[1]{\myref{Table}{#1}}
\newcommand{\chapterref}[1]{\myref{chapter}{#1}}
\newcommand{\Chapterref}[1]{\myref{Chapter}{#1}}
\newcommand{\appendixref}[1]{\myref{appendix}{#1}}
\newcommand{\Appendixref}[1]{\myref{Appendix}{#1}}
\newcommand{\sectionref}[1]{\myref{\S}{#1}}
\newcommand{\subsectionref}[1]{\myref{subsection}{#1}}
\newcommand{\subsectionvref}[1]{\myvref{subsection}{#1}}
\newcommand{\exercisevref}[1]{\myvref{Exercise}{#1}}
\newcommand{\exerciseref}[1]{\myref{Exercise}{#1}}
\newcommand{\examplevref}[1]{\myvref{Example}{#1}}
\newcommand{\exampleref}[1]{\myref{Example}{#1}}
\newcommand{\thmvref}[1]{\myvref{Theorem}{#1}}
\newcommand{\thmref}[1]{\myref{Theorem}{#1}}


\renewcommand{\exampleref}[1]{ {\color{red} \bfseries Normally a reference to a previous example goes here.}}
\renewcommand{\figurevref}[1]{ {\color{red} \bfseries Normally a reference to a previous figure goes here.}}
\renewcommand{\tablevref}[1]{ {\color{red} \bfseries Normally a reference to a previous table goes here.}}
\renewcommand{\Appendixref}[1]{ {\color{red} \bfseries Normally a reference to an Appendix goes here.}}
\renewcommand{\exercisevref}[1]{ {\color{red} \bfseries Normally a reference to a previous exercise goes here.}}



\newcommand{\R}{\mathbb{R}}

%% Example Solution Env.
\def\beginSolclaim{\par\addvspace{\medskipamount}\noindent\hbox{\bf Solution:}\hspace{0.5em}\ignorespaces}
\def\endSolclaim{\par\addvspace{-1em}\hfill\rule{1em}{0.4pt}\hspace{-0.4pt}\rule{0.4pt}{1em}\par\addvspace{\medskipamount}}
\newenvironment{exampleSol}[1][]{\beginSolclaim}{\endSolclaim}

%% General figure formating from original book.
\newcommand{\mybeginframe}{%
\begin{tcolorbox}[colback=white,colframe=lightgray,left=5pt,right=5pt]%
}
\newcommand{\myendframe}{%
\end{tcolorbox}%
}

%%% Eventually return and fix this to make matlab code work correctly.
%% Define the matlab environment as another code environment
%\newenvironment{matlab}
%{% Begin Environment Code
%{ \centering \bfseries Matlab Code }
%\begin{code}
%}% End of Begin Environment Code
%{% Start of End Environment Code
%\end{code}
%}% End of End Environment Code


% this one should have a caption, first argument is the size
\newenvironment{mywrapfig}[2][]{
 \wrapfigure[#1]{r}{#2}
 \mybeginframe
 \centering
}{%
 \myendframe
 \endwrapfigure
}

% this one has no caption, first argument is size,
% the second argument is a larger size used for HTML (ignored by latex)
\newenvironment{mywrapfigsimp}[3][]{%
 \wrapfigure[#1]{r}{#2}%
 \centering%
}{%
 \endwrapfigure%
}
\newenvironment{myfig}
    {%
    \begin{figure}[h!t]
        \mybeginframe%
        \centering%
    }
    {%
        \myendframe
    \end{figure}%
    }


% graphics include
\newcommand{\diffyincludegraphics}[3]{\includegraphics[#1]{#3}}
\newcommand{\myincludegraphics}[3]{\includegraphics[#1]{#3}}
\newcommand{\inputpdft}[1]{\subimport*{../figures/}{#1.pdf_t}}


%% Not sure what these even do? They don't seem to actually work... fun!
%\newcommand{\mybxbg}[1]{\tcboxmath[colback=white,colframe=black,boxrule=0.5pt,top=1.5pt,bottom=1.5pt]{#1}}
%\newcommand{\mybxsm}[1]{\tcboxmath[colback=white,colframe=black,boxrule=0.5pt,left=0pt,right=0pt,top=0pt,bottom=0pt]{#1}}
\newcommand{\mybxsm}[1]{#1}
\newcommand{\mybxbg}[1]{#1}

%%% Something about tasks for practice/hw?
\usepackage{tasks}
\usepackage{footnote}
\makesavenoteenv{tasks}


%% For pdf only?
\newcommand{\diffypdfversion}[1]{#1}


%% Kill ``cite'' and go back later to fix it.
\renewcommand{\cite}[1]{}


%% Currently we can't really use index or its derivatives. So we are gonna kill them off.
\renewcommand{\index}[1]{}
\newcommand{\myindex}[1]{#1}







\begin{document}
\begin{abstract}
    Why?
\end{abstract}
\maketitle


\begin{exercise}
    Consider $\dfrac{dx}{dt} = {(2t-x)}^2$, $x(0)=2$.  Use Euler's method with step size $h=0.5$ to approximate $x(1)$. [Enter your answer exactly] $\answer{\frac{17}{2}}$.
\end{exercise}
%\comboSol
%{%
%17/2
%}

\begin{exercise}
    Consider the differential equation $\frac{dy}{dt} = t^2 - 3y + 1$ with $y(1) = 4$. Approximate the solution at $t=3$ using Euler's method with a step size of: 
    \begin{itemize}
        \item $h=1$, $\answer{17}$
        \item $h=0.5$, $\answer{\frac{85}{32}}$
    \end{itemize}
    \begin{problem}
        Compute the actual solution at $t=3$, [Enter the exact value, not a decimal approximation] $\answer{\frac{52}{27} + \frac{94}{27}e^{-6}}$
        \begin{feedback}[correct]
            Compare the actual solution to the approximations!
        \end{feedback}
    \end{problem}
\end{exercise}
%\comboSol
%{%
%$h=1$: 17, $h=0.5$: 85/32, Actual: $\frac{52}{27} + \frac{94}{27}e^{-6} \approx 1.9346$
%}

\begin{exercise}
    Consider the differential equation $\frac{dy}{dt} = 2ty + y^2$ with $y(0) = 1$. Approximate the solution at $t=2$ using Euler's method with a step size of $h=1$ and $h=0.5$.
    \begin{itemize}
        \item $h=1$, $\answer{10}$
        \item $h=0.5$, [Enter your answer to two decimal places] $\answer{108.56}$
    \end{itemize}
%    \begin{problem}
%        Compute the actual solution at $t=3$, [Enter the exact value, not a decimal approximation] $\answer{\frac{52}{27} + \frac{94}{27}e^{-6}}$
%        \begin{feedback}[correct]
%            Compare the actual solution to the approximations!
%        \end{feedback}
%    \end{problem}
\end{exercise}
%\comboSol
%{%
%$h=1$: 10, $h=0.5$: $y(2) \approx 108.56$
%}

\begin{exercise}
    Consider $\dfrac{dx}{dt} = t-x$, $x(0)=1$.
    
    Use Euler's method to approximate $x(1)$, with step sizes: 
    \begin{itemize}
        \item $h = 1$, $\answer{0}$
        \item $h = \frac{1}{2}$, $\answer{\frac{1}{2}}$
        \item $h = \frac{1}{4}$, $\answer{\frac{81}{128}}$
        \item $h = \frac{1}{8}$, [Enter your answer to four decimal places] $\answer{0.6872}$
    \end{itemize}
    
    \begin{problem}
        Solve the equation exactly. $x(t) = \answer{t-1+2e^{-t}}$
        \begin{problem}
            Describe what happens to the errors for each $h$ you used.  That is, find the factor by which the error changed each time you halved the interval.
            
            Error \wordChoice{\choice{increases}\choice{stays the same}\choice[correct]{decreases}} by factors of: 
            \begin{itemize}
                \item When going from $h=1$ to $h=\frac{1}{2}$: [Enter to two decimal places] $\answer{0.32}$.
                \item When going from $h=\frac{1}{2}$ to $h=\frac{1}{4}$: [Enter to four decimal places] $\answer{0.4365}$.
                \item When going from $h=\frac{1}{4}$ to $h=\frac{1}{8}$: [Enter to four decimal places] $\answer{0.4723}$.
            \end{itemize}
        \end{problem}
    \end{problem}
\end{exercise}
%\comboSol
%{%
%a)~ $h=1$, 0. $h=\frac{1}{2}$, 0.5. $h = \frac{1}{4}$,$ \frac{81}{128}$. $h = \frac{1}{8}$ $x(1) \approx .6872$. \\
%b)~$x(t) = t-1+2e^{-t}$\\
%c)~Error decreases by factors of 0.32, 0.4365, 0.4723.
%}

\begin{exercise}%
    Let $x' = \sin(xt)$, and $x(0)=1$. Approximate [up to four decimal digits] $x(1)$ using Euler's method with step sizes:
    \begin{itemize}
        \item $h = 1$, $\answer{1}$
        \item $h = \frac{1}{2}$, $\answer{1.2397}$
        \item $h = \frac{1}{4}$, $\answer{1.3829}$
    \end{itemize}
\end{exercise}
%\exsol{%
%Approximately: 1.0000, 1.2397, 1.3829
%}


\begin{exercise}
    Approximate the value of $e$ by looking at the initial value problem $y'=y$ with $y(0)=1$ and approximating $y(1)$ using Euler's method with a step size of $0.2$. [Enter your answer to four decimal places] $\answer{2.4883}$.
\end{exercise}
%\comboSol
%{%
%2.4883
%}

\begin{exercise}%
    Let $x' = 2t$, and $x(0)=0$.
    
    Approximate $x(4)$ using Euler's method with step sizes:
    \begin{itemize}
        \item $h = 4$, $\answer{0}$
        \item $h = 2$, $\answer{8}$
        \item $h = 1$, $\answer{12}$
    \end{itemize}
    \begin{problem}
        Solve exactly, $x(4) = \answer{16}$ 
        \begin{problem}
            compute the errors: 
            \begin{itemize}
                \item $h = 4$, $\answer{16}$
                \item $h = 2$, $\answer{8}$
                \item $h = 1$, $\answer{4}$
            \end{itemize}
            \begin{problem}
                Compute the factor by which the errors changed.
                \begin{itemize}
                    \item $h = 4$, $\answer{\frac{1}{2}}$
                    \item $h = 2$, $\answer{\frac{1}{2}}$
                    \item $h = 1$, $\answer{\frac{1}{2}}$
                \end{itemize}
            \end{problem}
        \end{problem}
    \end{problem}
\end{exercise}
%\exsol{%
%a) 0, 8, 12
%\quad
%b) $x(4) = 16$, so errors are: 16, 8, 4.
%\quad
%c) Factors are 0.5, 0.5, 0.5.
%}


\begin{exercise}%
    Let $x' = x e^{xt+1}$, and $x(0)=0$.
    
    Approximate $x(4)$ using Euler's method with step sizes:
    \begin{itemize}
        \item $h = 4$, $\answer{0}$
        \item $h = 2$, $\answer{0}$
        \item $h = 1$, $\answer{0}$
    \end{itemize}
    \begin{problem}
        Guess an exact solution based on the previous part: $x = \answer{0}$ 
        \begin{problem}
            Compute the errors.
        \begin{itemize}
            \item $h = 4$, $\answer{0}$
            \item $h = 2$, $\answer{0}$
            \item $h = 1$, $\answer{0}$
        \end{itemize}
        \end{problem}
    \end{problem}
\end{exercise}
%\exsol{%
%a) 0, 0, 0
%\quad
%b) $x=0$ is a solution so errors are: 0, 0, 0.
%}


\begin{exercise}
    Example of numerical instability: Take $y' = -5y$, $y(0) = 1$.  We know that the solution should decay to zero as $x$ grows. Using Euler's method, start with $h=1$ and compute $y_1, y_2, y_3, y_4$ to try to approximate $y(4)$.\\
    $y(4) = \answer{256}$
    
    \begin{problem}
        What happened? Now halve the interval and re-approximate $y(4)$: [Enter your answer to 4 decimal digits] $y(4) = \answer{25.6289}$
        \begin{feedback}[correct]
            Keep halving the interval and approximating $y(4)$ until the numbers you are getting start to stabilize (that is, until they start going towards zero). Note: You might want to use a calculator.
        \end{feedback}
    \end{problem}
\end{exercise}
%\comboSol
%{%
%For $h=1$, $y_4 = 256$. $h=0.5$, $y_4 = 25.6289$, $h=0.25$ goes to zero, but oscillates. $h=0.125$ just goes to zero.
%}

There is a simple way to improve Euler's method to make it a second order method by doing just one extra step. Consider $\frac{dy}{dx}=f(x,y)$, $y(x_0) = y_0$, and a step size $h$. What we do is to pretend we compute the next step as in Euler, that is, we start with $(x_i,y_i)$, we compute a slope $k_1 = f(x_i,y_i)$, and then look at the point $(x_i+h,y_i + k_1h)$. Instead of letting our new point be $(x_i+h,y_i + k_1h)$, we compute the slope at that point, call it $k_2$, and then take the average of $k_1$ and $k_2$, hoping that the average is going to be closer to the actual slope on the interval from $x_i$ to $x_i+h$.  And we are correct, if we halve the step, the error should go down by a factor of $2^2 = 4$. To summarize, the setup is the same as for regular Euler, except the computation of $y_{i+1}$ and $x_{i+1}$.
\begin{align*}
    & k_1 = f(x_i,y_i) , & & 
    x_{i+1} = x_i + h , \\
    & k_2 = f(x_i + h,y_i + k_1h) ,
    & & y_{i+1} = y_i + \frac{k_1+k_2}{2}\,h .
\end{align*}


\begin{exercise}%
    Consider $\dfrac{dy}{dx} = x+y$, $y(0)=1$.
    
    Use the improved Euler's method (see above) with step sizes $h=\frac{1}{4}$ and $h=\frac{1}{8}$ to approximate $y(1)$.
    \begin{itemize}
        \item $h = \frac{1}{4}$, [Enter your answer to four decimal places] $y(1) ~ \answer{3.3897}$
        \item $h = \frac{1}{8}$, [Enter your answer to four decimal places] $y(1) ~ \answer{3.4237}$
    \end{itemize}
    \begin{problem}
        Use Euler's method with $h=\frac{1}{4}$ and $h=\frac{1}{8}$.
        \begin{itemize}
            \item $h = \frac{1}{4}$, [Enter your answer to four decimal places] $y(1) ~ \answer{2.8828}$
            \item $h = \frac{1}{8}$, [Enter your answer to four decimal places] $y(1) ~ \answer{3.1316}$
        \end{itemize}
        \begin{problem}
            Solve exactly, find the exact value of $y(1)$. $y(1) = \answer{2e^x-x-1}$
            \begin{feedback}[correct]
                So the exact value is approximately: $y(1) = 3.4366$
            \end{feedback}
            \begin{problem}
                Compute the errors:
                \begin{itemize}
                    \item Improved Euler's Method: [Enter answers up to six decimal places]
                    \begin{itemize}
                        \item $h=\frac{1}{4}$, $\answer{0.046852}$
                        \item $h=\frac{1}{8}$, $\answer{0.012881}$
                    \end{itemize}
                    \item Standard Euler's Method: [Enter answers up to five decimal places]
                    \begin{itemize}
                        \item $h=\frac{1}{4}$, $\answer{0.55375}$
                        \item $h=\frac{1}{8}$, $\answer{0.30499}$
                    \end{itemize}
                    \begin{problem}
                        Compute the factors by which the errors changed: [Enter to two decimal places]
                        \begin{itemize}
                            \item Improved Euler's Method: $\answer{0.27}$
                            \item Standard Euler's Method: $\answer{0.55}$
                        \end{itemize}
                    \end{problem}
                \end{itemize}
            \end{problem}
        \end{problem}
    \end{problem}
\end{exercise}
%\exsol{%
%a) Improved Euler: $y(1) \approx 3.3897$ for $h=\frac{1}{4}$, 
%$y(1) \approx 3.4237$ for $h=\frac{1}{8}$, 
%\quad
%b) Standard Euler: $y(1) \approx 2.8828$ for $h=\frac{1}{4}$, 
%$y(1) \approx 3.1316$ for $h=\frac{1}{8}$, 
%\quad
%c) $y = 2e^x-x-1$, so $y(2)$ is approximately $3.4366$.
%\quad
%d) Approximate errors for improved Euler:
%$0.046852$ for $h=\frac{1}{4}$, 
%and
%$0.012881$ for $h=\frac{1}{8}$. 
%\quad
%For standard Euler:
%$0.55375$ for $h=\frac{1}{4}$, 
%and
%$0.30499$ for $h=\frac{1}{8}$. 
%\quad
%Factor is approximately $0.27$ for improved Euler, and
%$0.55$ for standard Euler.
%}


The simplest method used in practice is the \emph{Runge--Kutta method}. Consider $\frac{dy}{dx}=f(x,y)$, $y(x_0) = y_0$, and a step size $h$.  Everything is the same as in Euler's method, except the computation of $y_{i+1}$ and $x_{i+1}$.
\begin{align*}
    & k_1 = f(x_i,y_i) , & & \\
    & k_2 = f\bigl(x_i + \frac{h}{2},y_i + k_1 (\frac{h}{2})\bigr) ,
    & & 
    x_{i+1} = x_i + h , \\
    & k_3 = f\bigl(x_i + \frac{h}{2},y_i + k_2 (\frac{h}{2})\bigr) ,
    & &
    y_{i+1} = y_i + \frac{k_1 + 2k_2 + 2k_3 + k_4}{6}\,h ,  \\
    & k_4 = f(x_i + h,y_i + k_3 h) .
\end{align*}


\begin{exercise}
    Consider $\dfrac{dy}{dx} = yx^2$, $y(0)=1$.
    
    Use Runge--Kutta (see above) with step sizes $h=1$ and $h=\frac{1}{2}$ to approximate $y(1)$.
    
    \begin{problem}
        Use Euler's method with:
        \begin{itemize}
            \item $h=1$, [Enter up to 4 decimal digits] $\answer{1.3906}$
            \item $h=\frac{1}{2}$, [Enter up to 3 decimal digits] $\answer{1.125}$
        \end{itemize} 
        \begin{problem}
            Solve exactly, find the exact value of $y(1)$: [Enter up to 3 decimal digits] $\answer{1.3956}$%, and compare.
        \end{problem}
    \end{problem}
\end{exercise}
%\comboSol
%{%
%a)~ $h=1$ gives 1.3906. $h=0.5$ gives 1.3953.
%b)~ $h=1$ gives 1. $h=0.5$ gives 1.125.
%c)~ Exact: 1.3956. Runge--Kutta matches three decimal places at $h=0.5$. Euler needs to go to $h=0.005$ to get the first two decimals right.
%}
%
%\setcounter{exercise}{100}


\end{document}