\documentclass{ximera}

\title{Practice for Determinants}

%\auor{Matthew Charnley and Jason Nowell}
\usepackage[margin=1.5cm]{geometry}
\usepackage{indentfirst}
\usepackage{sagetex}
\usepackage{lipsum}
\usepackage{amsmath}
\usepackage{mathrsfs}
\usepackage{tikz}
\usetikzlibrary{matrix}

%%% Random packages added without verifying what they are really doing - just to get initial compile to work.
\usepackage{tcolorbox}
\usepackage{hypcap}
\usepackage{booktabs}%% To get \toprule,\midrule,\bottomrule etc.
\usepackage{caption}
\usepackage{units}
\usepackage{multicol}
\usepackage{hhline}


% This is my modified wrapfig that doesn't use intextsep
\usepackage{mywrapfig}
\usepackage{import}



%%% End to random added packages.


\graphicspath{
    {./}
    {./figures/}
    {./../figures/}
    {./../../figures/}
}
\renewcommand{\log}{\ln}%%%%
\DeclareMathOperator{\arcsec}{arcsec}
%% New commands


%%%%%%%%%%%%%%%%%%%%
% New Conditionals %
%%%%%%%%%%%%%%%%%%%%


% referencing
\makeatletter
    \DeclareRobustCommand{\myvref}[2]{%
      \leavevmode%
      \begingroup
        \let\T@pageref\@pagerefstar
        \hyperref[{#2}]{%
	  #1~\ref*{#2}%
        }%
        \vpageref[\unskip]{#2}%
      \endgroup
    }%

    \DeclareRobustCommand{\myref}[2]{%
      \leavevmode%
      \begingroup
        \let\T@pageref\@pagerefstar
        \hyperref[{#2}]{%
	  #1~\ref*{#2}%
        }%
      \endgroup
    }%
\makeatother

\newcommand{\figurevref}[1]{\myvref{Figure}{#1}}
\newcommand{\figureref}[1]{\myref{Figure}{#1}}
\newcommand{\tablevref}[1]{\myvref{Table}{#1}}
\newcommand{\tableref}[1]{\myref{Table}{#1}}
\newcommand{\chapterref}[1]{\myref{chapter}{#1}}
\newcommand{\Chapterref}[1]{\myref{Chapter}{#1}}
\newcommand{\appendixref}[1]{\myref{appendix}{#1}}
\newcommand{\Appendixref}[1]{\myref{Appendix}{#1}}
\newcommand{\sectionref}[1]{\myref{\S}{#1}}
\newcommand{\subsectionref}[1]{\myref{subsection}{#1}}
\newcommand{\subsectionvref}[1]{\myvref{subsection}{#1}}
\newcommand{\exercisevref}[1]{\myvref{Exercise}{#1}}
\newcommand{\exerciseref}[1]{\myref{Exercise}{#1}}
\newcommand{\examplevref}[1]{\myvref{Example}{#1}}
\newcommand{\exampleref}[1]{\myref{Example}{#1}}
\newcommand{\thmvref}[1]{\myvref{Theorem}{#1}}
\newcommand{\thmref}[1]{\myref{Theorem}{#1}}


\renewcommand{\exampleref}[1]{ {\color{red} \bfseries Normally a reference to a previous example goes here.}}
\renewcommand{\examplevref}[1]{ {\color{red} \bfseries Normally a reference to a previous example goes here.}}
\renewcommand{\figurevref}[1]{ {\color{red} \bfseries Normally a reference to a previous figure goes here.}}
\renewcommand{\tablevref}[1]{ {\color{red} \bfseries Normally a reference to a previous table goes here.}}
\renewcommand{\Appendixref}[1]{ {\color{red} \bfseries Normally a reference to an Appendix goes here.}}
\renewcommand{\exercisevref}[1]{ {\color{red} \bfseries Normally a reference to a previous exercise goes here.}}
\renewcommand{\thmvref}[1]{ {\color{red} \bfseries Normally a reference to a previous theorem goes here.}}
\renewcommand{\subsectionvref}[1]{ {\color{red} \bfseries Normally a reference to a previous subsection goes here.}}



\newcommand{\R}{\mathbb{R}}
\newcommand{\C}{\mathbb{C}}

%% Example Solution Env.
\def\beginSolclaim{\par\addvspace{\medskipamount}\noindent\hbox{\bf Solution:}\hspace{0.5em}\ignorespaces}
\def\endSolclaim{\par\addvspace{-1em}\hfill\rule{1em}{0.4pt}\hspace{-0.4pt}\rule{0.4pt}{1em}\par\addvspace{\medskipamount}}
\newenvironment{exampleSol}[1][]{\beginSolclaim}{\endSolclaim}

%% General figure formating from original book.
\newcommand{\mybeginframe}{%
\begin{tcolorbox}[colback=white,colframe=lightgray,left=5pt,right=5pt]%
}
\newcommand{\myendframe}{%
\end{tcolorbox}%
}

%%% Eventually return and fix this to make matlab code work correctly.
%% Define the matlab environment as another code environment
%\NewEnviron{matlab}{ {\centering\bfseries MATLAB Code} \\ \noexpand{\BODY} }
%\let\beginmatlab\begincode
%\let\endmatlab\endcode
%\newenvironment{matlab}{% Begin Environment Code
%\begin{minipage}{\linewidth}
%\begin{verbatim}
%}% End of Begin Environment Code
%{% Start of End Environment Code
%\end{verbatim}
%\end{minipage}
%}% End of End Environment Code


% this one should have a caption, first argument is the size
\newenvironment{mywrapfig}[2][]{
 \wrapfigure[#1]{r}{#2}
 \mybeginframe
 \centering
}{%
 \myendframe
 \endwrapfigure
}

% this one has no caption, first argument is size,
% the second argument is a larger size used for HTML (ignored by latex)
\newenvironment{mywrapfigsimp}[3][]{%
 \wrapfigure[#1]{r}{#2}%
 \centering%
}{%
 \endwrapfigure%
}
\newenvironment{myfig}
    {%
    \begin{figure}[h!t]
        \mybeginframe%
        \centering%
    }
    {%
        \myendframe
    \end{figure}%
    }


% graphics include
\newcommand{\diffyincludegraphics}[3]{\includegraphics[#1]{#3}}
\newcommand{\myincludegraphics}[3]{\includegraphics[#1]{#3}}
\newcommand{\inputpdft}[1]{\subimport*{../figures/}{#1.pdf_t}}


%% Not sure what these even do? They don't seem to actually work... fun!
%\newcommand{\mybxbg}[1]{\tcboxmath[colback=white,colframe=black,boxrule=0.5pt,top=1.5pt,bottom=1.5pt]{#1}}
%\newcommand{\mybxsm}[1]{\tcboxmath[colback=white,colframe=black,boxrule=0.5pt,left=0pt,right=0pt,top=0pt,bottom=0pt]{#1}}
\newcommand{\mybxsm}[1]{#1}
\newcommand{\mybxbg}[1]{#1}

%%% Something about tasks for practice/hw?
\usepackage{tasks}
\usepackage{footnote}
\makesavenoteenv{tasks}


%% For pdf only?
\newcommand{\diffypdfversion}[1]{#1}


%% Kill ``cite'' and go back later to fix it.
\renewcommand{\cite}[1]{}


%% Currently we can't really use index or its derivatives. So we are gonna kill them off.
\renewcommand{\index}[1]{}
\newcommand{\myindex}[1]{#1}







\begin{document}
\begin{abstract}
Why?
\end{abstract}
\maketitle



\begin{exercise}
    Compute the determinant of the following matrices:
    \begin{itemize}
        \item
        $\begin{bmatrix}
            3
        \end{bmatrix}$

        $\answer{3}$
        \item
        $\begin{bmatrix}
            1 & 3 \\
            2 & 1
        \end{bmatrix}$

        $\answer{-5}$
        \item
        $\begin{bmatrix}
            2 & 1 \\
            4 & 2
        \end{bmatrix}$

        $\answer{0}$
        \item
        $\begin{bmatrix}
            1 & 2 & 3 \\
            0 & 4 & 5 \\
            0 & 0 & 6
        \end{bmatrix}$

        $\answer{24}$
        \item
        $\begin{bmatrix}
            2 & 1 & 0 \\
            -2 & 7 & -3 \\
            0 & 2 & 0
        \end{bmatrix}$

        $\answer{12}$
        \item
        $\begin{bmatrix}
            2 & 1 & 3 \\
            8 & 6 & 3 \\
            7 & 9 & 7
        \end{bmatrix}$

        $\answer{85}$
        \item
        $\begin{bmatrix}
            0 & 2 & 5 & 7 \\
            0 & 0 & 2 & -3 \\
            3 & 4 & 5 & 7 \\
            0 & 0 & 2 & 4
        \end{bmatrix}$

        $\answer{84}$
        \item
        $\begin{bmatrix}
            0 &  1 &  2 &  0 \\
            1 &  1 & -1 & 2 \\
            1 &  1 &  2 & 1 \\
            2 & -1 & -2 & 3
        \end{bmatrix}$

        $\answer{-3}$
    \end{itemize}
\end{exercise}
%\comboSol
%{%
%a)~ 3 \quad b)~ -5 \quad c)~ 0 \quad d)~24 \quad e)~12 \quad f)~ 85 \quad g)~84 \quad h)~-3
%}

\begin{exercise}\%
    Compute the determinant of the following matrices:
    \begin{itemize}
        \item
        $\begin{bmatrix}
            -2
        \end{bmatrix}$

        $\answer{25}$
        \item
        $\begin{bmatrix}
            2 & -2 \\
            1 & 3
        \end{bmatrix}$

        $\answer{8}$
        \item
        $\begin{bmatrix}
            2 & 2 \\
            2 & 2
        \end{bmatrix}$

        $\answer{0}$
        \item
        $\begin{bmatrix}
            2 & 9 & -11 \\
            0 & -1 & 5 \\
            0 & 0 & 3
        \end{bmatrix}$

        $\answer{-6}$
        \item
        $\begin{bmatrix}
            2 & 1 & 0 \\
            -2 & 7 & 3 \\
            1 & 1 & 0
        \end{bmatrix}$

        $\answer{-3}$
        \item
        $\begin{bmatrix}
            5 & 1 & 3 \\
            4 & 1 & 1 \\
            4 & 5 & 1
        \end{bmatrix}$

        $\answer{28}$
        \item
        $\begin{bmatrix}
            3 & 2 & 5 & 7 \\
            0 & 0 & 2 & 0 \\
            0 & 4 & 5 & 0 \\
            2 & 1 & 2 & 4
        \end{bmatrix}$

        $\answer{16}$
        \item
        $\begin{bmatrix}
             0 &  2 &  1 &  0 \\
             1 &  2 & -3 &  4 \\
             5 &  6 & -7 &  8 \\
             1 &  2 &  3 & -2
        \end{bmatrix}$

        $\answer{-24}$
    \end{itemize}
\end{exercise}
%\exsol{%
%a)~$-2$
%\quad b)~$8$
%\quad c)~$0$
%\quad d)~$-6$
%\quad e)~$-3$
%\quad f)~$28$
%\quad g)~$16$
%\quad h)~$-24$
%}

\begin{exercise}
    For which $x$ are the following matrices singular (not invertible).
    \begin{itemize}
        \item
        $\begin{bmatrix}
            2 & 3 \\
            2 & x
        \end{bmatrix}$
        
        For $x = \answer{3}$
        \item
        $\begin{bmatrix}
            2 & x \\
            1 & 2
        \end{bmatrix}$
        
        For $x = \answer{4}$
        \item
        $\begin{bmatrix}
            x & 1 \\
            4 & x
        \end{bmatrix}$
        
        For $x = \pm\answer{2}$
        \item
        $\begin{bmatrix}
            x & 0 & 1 \\
            1 & 4 & 2 \\
            1 & 6 & 2
        \end{bmatrix}$
        
        For $x = \answer{\frac{1}{2}}$
    \end{itemize}
\end{exercise}
%\comboSol
%{%
%a)~ $x=3$ \quad b)~$x=4$ \quad c)~ $x=\pm2$ \quad d)~$x = \frac{1}{2}$
%}

\begin{exercise}\%
    For which $x$ are the following matrices singular (not invertible).
    \begin{itemize}
        \item
        $\begin{bmatrix}
            1 & 3 \\
            1 & x
        \end{bmatrix}$
        
        For $x = \answer{3}$
        \item
        $\begin{bmatrix}
            3 & x \\
            1 & 3
        \end{bmatrix}$
        
        For $x = \answer{9}$
        \item
        $\begin{bmatrix}
            x & 3 \\
            3 & x
        \end{bmatrix}$
        
        For $x = \pm\answer{3}$
        \item
        $\begin{bmatrix}
            x & 1 & 0 \\
            1 & 4 & 0 \\
            1 & 6 & 2
        \end{bmatrix}$
        
        For $x = \answer{\frac{1}{4}}$
    \end{itemize}
\end{exercise}
%\exsol{%
%a)~$3$
%\quad b)~$9$
%\quad c)~$3$, $-3$
%\quad d)~$\frac{1}{4}$
%}

\begin{exercise}\%
    Consider the matrix
    \begin{equation*}
        A = \begin{bmatrix}
        0 & -1 & 0 \\ -5 & -4 & -5 \\ 2 & 3 & 4
        \end{bmatrix}.
    \end{equation*}
    \begin{itemize}
        \item Compute the determinant of $A$ using cofactor expansion along row 1. $\answer{-10}$
        \item Compute the determinant of $A$ using cofactor expansion along column 2. $\answer{-10}$
        \item Compute the determinant using row reduction. $\answer{-10}$
    \end{itemize}
\end{exercise}
%\exsol{%
%-10
%}%

\begin{exercise}\%
    Consider the matrix
    \begin{equation*}
        A = \begin{bmatrix}
        -1 & 0 & -3 \\ 1 & 2 & 1 \\ 3 & 3 & 3
        \end{bmatrix}.
    \end{equation*}
    \begin{itemize}
        \item Compute the determinant of $A$ using cofactor expansion along row 1. $\answer{6}$
        \item Compute the determinant of $A$ using cofactor expansion along column 3. $\answer{6}$
        \item Compute the determinant using row reduction. $\answer{6}$
    \end{itemize}
\end{exercise}
%\exsol{%
%6
%}%


\begin{exercise}\%
    Consider the matrix
    \begin{equation*}
        A = \begin{bmatrix}
        -2 & 0 & 1 & 0 \\ 0 & -1 & 1 & -2 \\ -5 &3 & 1 & 3 \\ -3 & 4 & 1 & 3
        \end{bmatrix}.
    \end{equation*}
    \begin{itemize}
        \item Compute the determinant of $A$ using cofactor expansion along row 1. $\answer{6}$
        \item Compute the determinant of $A$ using cofactor expansion along column 4. $\answer{6}$
        \item Compute the determinant using row reduction. $\answer{6}$
    \end{itemize}
\end{exercise}
%\exsol{%
%6
%}%



\begin{exercise}
    Is the matrix $A$ below invertible?% How do you know?
    \[ A = \begin{bmatrix}4&0&3&1 \\ 2 &1&-2&0 \\ 0&0&1&-3 \\3 & 2 & 1 & -5 \end{bmatrix} \]
    \begin{multipleChoice}
        \choice[correct]{Yes.}
        \choice{No.}
    \end{multipleChoice}
\end{exercise}
%\comboSol
%{%
%Yes
%}

\begin{exercise}\%
    Compute the determinant of the matrix
    \[ A = \begin{bmatrix}
        5 & 4  & 3\\
        -4 &-3 &-4\\
        -5 &-5 & 4
    \end{bmatrix}
    \]
    using row reduction.  $\answer{-1}$
    \begin{problem}
        What does this say about the solutions to $A\vec{x} = 0$?
        \begin{multipleChoice}
            \choice{Nothing.}
            \choice{There are infinitely many solutions.}
            \choice{There is only one solution, but we don't know what it is without further effort.}
            \choice[correct]{There is only one solution, and it must be $\vec{x}=0$.}
        \end{multipleChoice}
    \end{problem}
\end{exercise}
%\exsol{%
%-1. The only solution is $\vec{x} = 0$. 
%}

\begin{exercise}\%
    Compute the determinant of the matrix
    \[ A = \begin{bmatrix}
        -5 & -3 & -5 & -1\\
        4 & 0 &-5 &  4\\
        0&-2 &-1 &-2\\
        -1& -5 &-4 &-4
    \end{bmatrix}
    \]
    using row reduction.  $\answer{2}$
    \begin{problem}
        What does this say about the columns of $A$?
        \begin{multipleChoice}
            \choice{Nothing.}
            \choice[correct]{The columns are linearly independent.}
            \choice{The columns are linearly dependent.}
        \end{multipleChoice}
    \end{problem}
\end{exercise}
%\exsol{%
%2. The colums are linearly independent.
%}

\begin{exercise}\%
    Compute the determinant of the matrix
    \[ A = \begin{bmatrix}
        4  & 1 & -3 &  0\\
        -1 &  4 &  2 & -2\\
        -1 & -3 & 3 & 2\\
        -5 & -4 & 1 & 1
    \end{bmatrix}
    \]
    using row reduction.  $\answer{8}$
    \begin{problem}
        What does this say about the solutions to $A\vec{x} = \left[ \begin{smallmatrix} 1 \\ 0 \\ -1 \\ 1 \end{smallmatrix} \right]$.
        \begin{multipleChoice}
            \choice{Nothing.}
            \choice{There are infinitely many solutions.}
            \choice[correct]{There is only one solution, but we don't know what it is without further effort.}
            \choice{There is only one solution, and it must be $\vec{x}=0$.}
        \end{multipleChoice}
    \end{problem}
\end{exercise}
%\exsol{%
%8. There is exactly one solution, found by row reduction or multiplying by $A^{-1}$. 
%}

\begin{exercise}
    Compute
    \begin{equation*}
    \det \left( \begin{bmatrix}
        2 & 1 & 2 & 3 \\
        0 & 8 & 6 & 5 \\
        0 & 0 & 3 & 9 \\
        0 & 0 & 0 & 1
    \end{bmatrix}^{-1}
    \right)
    \end{equation*}
    without computing the inverse. $\answer{\frac{1}{48}}$
\end{exercise}
%\comboSol
%{%
%$1/48$
%}

\begin{exercise}\%
    Compute
    \begin{equation*}
    \det \left( \begin{bmatrix}
        3 & 4 & 7 & 12 \\
        0 & -1 & 9 & -8 \\
        0 & 0 & -2 & 4 \\
        0 & 0 & 0 & 2
    \end{bmatrix}^{-1}
    \right)
    \end{equation*}
    without computing the inverse. $\answer{\frac{1}{12}}$
\end{exercise}
%\exsol{%
%$1/12$
%}

\begin{exercise}
    Suppose
    \begin{equation*}
        L = \begin{bmatrix}
            1 & 0 & 0 & 0 \\
            2 & 1 & 0 & 0 \\
            7 & \pi & 1 & 0 \\
            2^8 & 5 & -99 & 1
        \end{bmatrix}
        \qquad \text{and} \qquad
        U = \begin{bmatrix}
            5 & 9 & 1 & -\sin(1) \\
            0 & 1 & 88 & -1 \\
            0 & 0 & 1 & 3 \\
            0 & 0 & 0 & 1
        \end{bmatrix} .
    \end{equation*}
    Let $A = LU$.  Compute $\det(A)$ in a simple way, without computing what is $A$. Hint: First read off $\det(L)$ and $\det(U)$.\\
    $\answer{5}$
\end{exercise}
%\comboSol
%{%
%$5$
%}

\begin{exercise}
    Consider the linear mapping from ${\mathbb R}^2$ to ${\mathbb R}^2$ given by the  matrix
    $A = \left[ \begin{smallmatrix}
        1 & x \\
        2 & 1
    \end{smallmatrix} \right]$
    for some number $x$.  You wish to make $A$ such that it doubles the area of every geometric figure.  What are the possibilities for $x$ (there are two answers).\\
    $\answer{-\frac{1}{2}}$, $\answer{\frac{3}{2}}$.
\end{exercise}
%\comboSol
%{%
%$x = -\frac{1}{2},\ \frac{3}{2}$
%}

\begin{exercise}[challenging]\%
    Consider the matrix
    \begin{equation*}
        A(x) = \begin{bmatrix}
            1 & 2 \\ 
            1 & x
        \end{bmatrix}^{-1}
    \end{equation*}
    as a function of the variable $x$.
    \begin{itemize}
        \item Find all the $x$ so that $A(x)$ and the matrix inverse $A(x)^{-1}$ have only integer entries (no fractions). Note that there are two answers.\\
        $\answer{1}$, $\answer{3}$.
        \item Find all the $x$ so that the matrix inverse $A(x)^{-1}$ has only integer entries (no fractions). (Use $k$ to represent an unknown integer.)\\
        $\answer{2 + \frac{1}{k}}$.
    \end{itemize}
\end{exercise}
%\exsol{%
%a) ~$1$ and $3$ \quad b)~ $2 + \frac{1}{k}$ for any integer $k$
%}

\begin{exercise}
    Suppose $A$ and $S$ are $n \times n$ matrices, and $S$ is invertible. Suppose that $\det(A) = 3$.  Compute $\det(S^{-1}AS)$ and $\det(SAS^{-1})$. $\answer{3}$ %Justify your answer using the theorems in this section.
\end{exercise}
%\comboSol
%{%
%3
%}

\begin{exercise}
    Let $A$ be an $n \times n$ matrix such that $\det(A)=1$. Compute $\det(x A)$ given a number $x$.\\ Hint: First try computing $\det(xI)$, then note that $xA = (xI)A$. $\answer{x^n}$ 
\end{exercise}
%\comboSol
%{%
%$x^n$
%}
%
%\setcounter{exercise}{100}



\end{document}