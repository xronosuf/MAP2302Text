\documentclass{ximera}

\title{Practice for Elimination Method}

%\auor{Matthew Charnley and Jason Nowell}
\usepackage[margin=1.5cm]{geometry}
\usepackage{indentfirst}
\usepackage{sagetex}
\usepackage{lipsum}
\usepackage{amsmath}
\usepackage{mathrsfs}


%%% Random packages added without verifying what they are really doing - just to get initial compile to work.
\usepackage{tcolorbox}
\usepackage{hypcap}
\usepackage{booktabs}%% To get \toprule,\midrule,\bottomrule etc.
\usepackage{nicefrac}
\usepackage{caption}
\usepackage{units}

% This is my modified wrapfig that doesn't use intextsep
\usepackage{mywrapfig}
\usepackage{import}



%%% End to random added packages.


\graphicspath{
    {./figures/}
    {./../figures/}
    {./../../figures/}
}
\renewcommand{\log}{\ln}%%%%
\DeclareMathOperator{\arcsec}{arcsec}
%% New commands


%%%%%%%%%%%%%%%%%%%%
% New Conditionals %
%%%%%%%%%%%%%%%%%%%%


% referencing
\makeatletter
    \DeclareRobustCommand{\myvref}[2]{%
      \leavevmode%
      \begingroup
        \let\T@pageref\@pagerefstar
        \hyperref[{#2}]{%
	  #1~\ref*{#2}%
        }%
        \vpageref[\unskip]{#2}%
      \endgroup
    }%

    \DeclareRobustCommand{\myref}[2]{%
      \leavevmode%
      \begingroup
        \let\T@pageref\@pagerefstar
        \hyperref[{#2}]{%
	  #1~\ref*{#2}%
        }%
      \endgroup
    }%
\makeatother

\newcommand{\figurevref}[1]{\myvref{Figure}{#1}}
\newcommand{\figureref}[1]{\myref{Figure}{#1}}
\newcommand{\tablevref}[1]{\myvref{Table}{#1}}
\newcommand{\tableref}[1]{\myref{Table}{#1}}
\newcommand{\chapterref}[1]{\myref{chapter}{#1}}
\newcommand{\Chapterref}[1]{\myref{Chapter}{#1}}
\newcommand{\appendixref}[1]{\myref{appendix}{#1}}
\newcommand{\Appendixref}[1]{\myref{Appendix}{#1}}
\newcommand{\sectionref}[1]{\myref{\S}{#1}}
\newcommand{\subsectionref}[1]{\myref{subsection}{#1}}
\newcommand{\subsectionvref}[1]{\myvref{subsection}{#1}}
\newcommand{\exercisevref}[1]{\myvref{Exercise}{#1}}
\newcommand{\exerciseref}[1]{\myref{Exercise}{#1}}
\newcommand{\examplevref}[1]{\myvref{Example}{#1}}
\newcommand{\exampleref}[1]{\myref{Example}{#1}}
\newcommand{\thmvref}[1]{\myvref{Theorem}{#1}}
\newcommand{\thmref}[1]{\myref{Theorem}{#1}}


\renewcommand{\exampleref}[1]{ {\color{red} \bfseries Normally a reference to a previous example goes here.}}
\renewcommand{\figurevref}[1]{ {\color{red} \bfseries Normally a reference to a previous figure goes here.}}
\renewcommand{\tablevref}[1]{ {\color{red} \bfseries Normally a reference to a previous table goes here.}}
\renewcommand{\Appendixref}[1]{ {\color{red} \bfseries Normally a reference to an Appendix goes here.}}
\renewcommand{\exercisevref}[1]{ {\color{red} \bfseries Normally a reference to a previous exercise goes here.}}



\newcommand{\R}{\mathbb{R}}

%% Example Solution Env.
\def\beginSolclaim{\par\addvspace{\medskipamount}\noindent\hbox{\bf Solution:}\hspace{0.5em}\ignorespaces}
\def\endSolclaim{\par\addvspace{-1em}\hfill\rule{1em}{0.4pt}\hspace{-0.4pt}\rule{0.4pt}{1em}\par\addvspace{\medskipamount}}
\newenvironment{exampleSol}[1][]{\beginSolclaim}{\endSolclaim}

%% General figure formating from original book.
\newcommand{\mybeginframe}{%
\begin{tcolorbox}[colback=white,colframe=lightgray,left=5pt,right=5pt]%
}
\newcommand{\myendframe}{%
\end{tcolorbox}%
}

%%% Eventually return and fix this to make matlab code work correctly.
%% Define the matlab environment as another code environment
%\newenvironment{matlab}
%{% Begin Environment Code
%{ \centering \bfseries Matlab Code }
%\begin{code}
%}% End of Begin Environment Code
%{% Start of End Environment Code
%\end{code}
%}% End of End Environment Code


% this one should have a caption, first argument is the size
\newenvironment{mywrapfig}[2][]{
 \wrapfigure[#1]{r}{#2}
 \mybeginframe
 \centering
}{%
 \myendframe
 \endwrapfigure
}

% this one has no caption, first argument is size,
% the second argument is a larger size used for HTML (ignored by latex)
\newenvironment{mywrapfigsimp}[3][]{%
 \wrapfigure[#1]{r}{#2}%
 \centering%
}{%
 \endwrapfigure%
}
\newenvironment{myfig}
    {%
    \begin{figure}[h!t]
        \mybeginframe%
        \centering%
    }
    {%
        \myendframe
    \end{figure}%
    }


% graphics include
\newcommand{\diffyincludegraphics}[3]{\includegraphics[#1]{#3}}
\newcommand{\myincludegraphics}[3]{\includegraphics[#1]{#3}}
\newcommand{\inputpdft}[1]{\subimport*{../figures/}{#1.pdf_t}}


%% Not sure what these even do? They don't seem to actually work... fun!
%\newcommand{\mybxbg}[1]{\tcboxmath[colback=white,colframe=black,boxrule=0.5pt,top=1.5pt,bottom=1.5pt]{#1}}
%\newcommand{\mybxsm}[1]{\tcboxmath[colback=white,colframe=black,boxrule=0.5pt,left=0pt,right=0pt,top=0pt,bottom=0pt]{#1}}
\newcommand{\mybxsm}[1]{#1}
\newcommand{\mybxbg}[1]{#1}

%%% Something about tasks for practice/hw?
\usepackage{tasks}
\usepackage{footnote}
\makesavenoteenv{tasks}


%% For pdf only?
\newcommand{\diffypdfversion}[1]{#1}


%% Kill ``cite'' and go back later to fix it.
\renewcommand{\cite}[1]{}


%% Currently we can't really use index or its derivatives. So we are gonna kill them off.
\renewcommand{\index}[1]{}
\newcommand{\myindex}[1]{#1}







\begin{document}
\begin{abstract}
Why?
\end{abstract}
\maketitle


\begin{exercise}
    Compute the reduced row echelon form for the following matrices:
    \begin{itemize}
        \item
        $\begin{bmatrix}
            1 & 3 & 1 \\
            0 & 1 & 1
        \end{bmatrix}$
        \[
            \left[\begin{smallmatrix} \answer{1} & \answer{0} & \answer{-2} \\ \answer{0} & \answer{1} & \answer{1} \end{smallmatrix}\right]
        \]
        \item
        $\begin{bmatrix}
            3 & 3 \\
            6 & -3
        \end{bmatrix}$
        \[
            \left[\begin{smallmatrix} \answer{1} & \answer{0} \\ \answer{0} & \answer{1} \end{smallmatrix}\right]
        \]
        \item
        $\begin{bmatrix}
            3 & 6 \\
            -2 & -3
        \end{bmatrix}$
        \[
            \left[\begin{smallmatrix} \answer{1} & \answer{0} \\ \answer{0} & \answer{1} \end{smallmatrix}\right]
        \]
        \item
        $\begin{bmatrix}
            6 & 6 & 7 & 7 \\
            1 & 1 & 0 & 1
        \end{bmatrix}$
        \[
            \left[\begin{smallmatrix} \answer{1} & \answer{1} & \answer{0} & \answer{1} \\ \answer{0} & \answer{0} & \answer{1} & \answer{\frac{1}{7}} \end{smallmatrix}\right]
        \]
        \item
        $\begin{bmatrix}
            9 & 3 & 0 & 2 \\
            8 & 6 & 3 & 6 \\
            7 & 9 & 7 & 9
        \end{bmatrix}$
        \[
            \left[\begin{smallmatrix} \answer{1} & \answer{0} & \answer{0} & \answer{-\frac{1}{2}} \\ \answer{0} & \answer{1} & \answer{0} & \answer{\frac{13}{6}} \\ \answer{0} & \answer{0} & \answer{1} & \answer{-1}  \end{smallmatrix}\right]
        \]
        \item
        $\begin{bmatrix}
            2 & 1 & 3 & -3 \\
            6 & 0 & 0 & -1 \\
            -2 & 4 & 4 & 3
        \end{bmatrix}$
        \[
            \left[\begin{smallmatrix} \answer{1} & \answer{0} & \answer{0} & \answer{-\frac{1}{6}} \\ \answer{0} & \answer{1} & \answer{0} & \answer{\frac{7}{3}} \\ \answer{0} & \answer{0} & \answer{1} & \answer{-\frac{5}{3}} \end{smallmatrix}\right]
        \]
        \item
        $\begin{bmatrix}
            6 & 6 & 5 \\
            0 & -2 & 2 \\
            6 & 5 & 6
        \end{bmatrix}$
        \[
            \left[\begin{smallmatrix} \answer{1} & \answer{0} & \answer{\frac{11}{6}} \\ \answer{0} & \answer{1} & \answer{-1} \\ \answer{0} & \answer{0} & \answer{0} \end{smallmatrix}\right]
        \]
        \item
        $\begin{bmatrix}
            0 & 2 & 0 & -1 \\
            6 & 6 & -3 & 3 \\
            6 & 2 & -3 & 5
        \end{bmatrix}$
        \[
            \left[\begin{smallmatrix} \answer{1} & \answer{0} & \answer{-\frac{1}{2}} & \answer{1} \\ \answer{0} & \answer{1} & \answer{0} & \answer{-\frac{1}{2}} \\ \answer{0} & \answer{0} & \answer{0} & \answer{0} \end{smallmatrix}\right]
        \]
    \end{itemize}
\end{exercise}
%\comboSol
%{%
%a)~$\left[\begin{smallmatrix} 1 & 0 & -2 \\ 0 & 1 & 1 \end{smallmatrix}\right]$ \quad b)~$\left[\begin{smallmatrix} 1 & 0 \\ 0 & 1 \end{smallmatrix}\right]$ \quad c)~$\left[\begin{smallmatrix} 1 & 0 \\ 0 & 1 \end{smallmatrix}\right]$ \quad d)~$\left[\begin{smallmatrix} 1 & 1 & 0 & 1 \\ 0 & 0 & 1 & 1/7 \end{smallmatrix}\right]$ \\
%e)~$\left[\begin{smallmatrix} 1 & 0 & 0 & -1/2 \\ 0 & 1 & 0 & 13/6 \\ 0 & 0 & 1 & -1  \end{smallmatrix}\right]$ \quad f)~$\left[\begin{smallmatrix} 1 & 0 & 0 & -1/6 \\ 0 & 1 & 0 & 7/3 \\ 0 & 0 & 1 & -5/3 \end{smallmatrix}\right]$ \quad g)~$\left[\begin{smallmatrix} 1 & 0 & 11/6 \\ 0 & 1 & -1 \\ 0 & 0 & 0 \end{smallmatrix}\right]$ \quad h)~$\left[\begin{smallmatrix} 1 & 0 & -1/2 & 1 \\ 0 & 1 & 0 & -1/2 \\ 0 & 0 & 0 & 0 \end{smallmatrix}\right]$
%}

\begin{exercise}%
    Compute the reduced row echelon form for the following matrices:
    \begin{itemize}
        \item
        $\begin{bmatrix}
            1 & 0 & 1 \\
            0 & 1 & 0
        \end{bmatrix}$
        \[
            \left[\begin{smallmatrix} \answer{1} & \answer{0} & \answer{1} \\ \answer{0} & \answer{1} & \answer{0} \end{smallmatrix}\right]
        \]
        \item
        $\begin{bmatrix}
            1 & 2 \\
            3 & 4
        \end{bmatrix}$
        \[
            \left[\begin{smallmatrix} \answer{1} & \answer{0} \\ \answer{0} & \answer{1} \end{smallmatrix}\right]
        \]
        \item
        $\begin{bmatrix}
            1 & 1 \\
            -2 & -2
        \end{bmatrix}$
        \[
            \left[\begin{smallmatrix} \answer{1} & \answer{1} \\ \answer{0} & \answer{0} \end{smallmatrix}\right]
        \]
        \item
        $\begin{bmatrix}
            1 & -3 & 1 \\
            4 & 6 & -2 \\
            -2 & 6 & -2
        \end{bmatrix}$
        \[
            \left[\begin{smallmatrix} \answer{1} & \answer{0} & \answer{0} \\ \answer{0} & \answer{1} & \answer{-\frac{1}{3}} \\ \answer{0} & \answer{0} & \answer{0} \end{smallmatrix}\right]
        \]
        \item
        $\begin{bmatrix}
            2 & 2 & 5 & 2 \\
            1 & -2 & 4 & -1 \\
            0 & 3 & 1 & -2
        \end{bmatrix}$
        \[
            \left[\begin{smallmatrix} \answer{1} & \answer{0} & \answer{0} & \answer{\frac{77}{15}} \\ \answer{0} & \answer{1} & \answer{0} & \answer{-\frac{2}{15}} \\ \answer{0} & \answer{0} & \answer{1} & \answer{-\frac{8}{5}} \end{smallmatrix}\right]
        \]
        \item
        $\begin{bmatrix}
            -2 & 6 & 4 & 3 \\
            6 & 0 & -3 & 0 \\
            4 & 2 & -1 & 1
        \end{bmatrix}$
        \[
            \left[\begin{smallmatrix} \answer{1} & \answer{0} & \answer{-\frac{1}{2}} & \answer{0} \\ \answer{0} & \answer{1} & \answer{\frac{1}{2}} & \answer{\frac{1}{2}} \\ \answer{0} & \answer{0} & \answer{0} & \answer{0} \end{smallmatrix}\right]
        \]
        \item
        $\begin{bmatrix}
            0 & 0 & 0 & 0 \\
            0 & 0 & 0 & 0
        \end{bmatrix}$
        \[
            \left[\begin{smallmatrix} \answer{0} & \answer{0} & \answer{0} & \answer{0} \\ \answer{0} & \answer{0} & \answer{0} & \answer{0} \end{smallmatrix}\right]
        \]
        \item
        $\begin{bmatrix}
            1 & 2 & 3 & 3 \\
            1 & 2 & 3 & 5
        \end{bmatrix}$
        \[
            \left[\begin{smallmatrix} \answer{1} & \answer{2} & \answer{3} & \answer{0} \\ \answer{0} & \answer{0} & \answer{0} & \answer{1} \end{smallmatrix}\right]
        \]
    \end{itemize}
\end{exercise}
%\exsol{%
%a)~$\left[\begin{smallmatrix} 1 & 0 & 1 \\ 0 & 1 & 0 \end{smallmatrix}\right]$
%\quad b)~$\left[\begin{smallmatrix} 1 & 0 \\ 0 & 1 \end{smallmatrix}\right]$
%\quad c)~$\left[\begin{smallmatrix} 1 & 1 \\ 0 & 0 \end{smallmatrix}\right]$
%\quad d)~$\left[\begin{smallmatrix} 1 & 0 & 0 \\ 0 & 1 & -\frac{1}{3} \\ 0 & 0 & 0 \end{smallmatrix}\right]$
%\quad e)~$\left[\begin{smallmatrix} 1 & 0 & 0 & \frac{77}{15} \\ 0 & 1 & 0 & -\frac{2}{15} \\ 0 & 0 & 1 & -\frac{8}{5} \end{smallmatrix}\right]$
%\quad f)~$\left[\begin{smallmatrix} 1 & 0 & -\frac{1}{2} & 0 \\ 0 & 1 & \frac{1}{2} & \frac{1}{2} \\ 0 & 0 & 0 & 0 \end{smallmatrix}\right]$
%\quad g)~$\left[\begin{smallmatrix} 0 & 0 & 0 & 0 \\ 0 & 0 & 0 & 0 \end{smallmatrix}\right]$
%\quad h)~$\left[\begin{smallmatrix} 1 & 2 & 3 & 0 \\ 0 & 0 & 0 & 1 \end{smallmatrix}\right]$
%}

\begin{exercise}
    Solve (find all solutions), or show no solution exists
    \begin{itemize}
        \item
        $\begin{aligned}
         4x_1+3x_2 & = -2 \\
         -x_1+\phantom{3} x_2 & = 4
        \end{aligned}$
        
        $x_1 = \answer{-2}$, $x_2 = \answer{2}$ 
        \item
        $\begin{aligned}
          x_1+5x_2+3x_3 & = 7 \\
         8x_1+7x_2+8x_3 & = 8 \\
         4x_1+8x_2+6x_3 & = 4
        \end{aligned}$
        
        $x_1 = \answer{-4}$, $x_2 = \answer{-16}$. $x_3 = \answer{36}$
        \item
        $\begin{aligned}
         4x_1+8x_2+2x_3 & = 3 \\
         -x_1-2x_2+3x_3 & = 1 \\
         4x_1+8x_2 \phantom{{}+3x_3} & = 2
        \end{aligned}$
        
        $x_1 = \answer{\frac{1}{2} - 2t}$, $x_2 = \answer{t}$, $x_3 = \answer{\frac{1}{2}}$
        \item
        $\begin{aligned}
          x+2y+3z & = 4 \\
        2  x-\phantom{2} y+3z & = 1 \\
        3  x+\phantom{2} y+6z & = 6
        \end{aligned}$
        \begin{multipleChoice}
            \choice{There is a solution.}
            \choice[correct]{There is no solution.}
            \choice{There are infinitely many solutions.}
        \end{multipleChoice}
    \end{itemize}
\end{exercise}
%\comboSol
%{%
%a)~$x_1 = -2$, $x_2 = 2$ \quad b)~ $x_1 = -4$, $x_2 = -16$. $x_3 = 36$ \\
%c)~$x_1 = \frac{1}{2} - 2t$, $x_2 = t$, $x_3 = \frac{1}{2}$, any real $t$ \quad d)~ No solution
%}

\begin{exercise}%
    Solve (find all solutions), or show no solution exists
    \begin{itemize}
        \item
        $\begin{aligned}
            4x_1+3x_2 & = -1 \\
            5x_1+6x_2 & = 4
        \end{aligned}$
        \begin{multipleChoice}
            \choice[correct]{There is a solution.}
            \choice{There is no solution.}
            \choice{There are infinitely many solutions.}
        \end{multipleChoice}
        \begin{problem}
            $x_1 = \answer{-2}$, $x_2 = \answer{\frac{7}{3}}$
        \end{problem}
        \item
        $\begin{aligned}
             5x+6y+5z & = 7 \\
             6x+8y+6z & = -1 \\
             5x+2y+5z & = 2
        \end{aligned}$
        \begin{multipleChoice}
            \choice{There is a solution.}
            \choice[correct]{There is no solution.}
            \choice{There are infinitely many solutions.}
        \end{multipleChoice}
        \item
        $\begin{aligned}
            a+\phantom{5}b+\phantom{6}c & = -1 \\
            a+5b+6c & = -1 \\
            -2a+5b+6c & = 8
        \end{aligned}$
        \begin{multipleChoice}
            \choice[correct]{There is a solution.}
            \choice{There is no solution.}
            \choice{There are infinitely many solutions.}
        \end{multipleChoice}
        \begin{problem}
            $a = \answer{-3}$, $b = \answer{10}$, $c = \answer{-8}$
        \end{problem}
        \item
        $\begin{aligned}
            -2 x_1+2x_2+8x_3 & = 6 \\
            x_2+\phantom{8}x_3 & = 2 \\
            x_1+4x_2+\phantom{8}x_3 & = 7
        \end{aligned}$
        \begin{multipleChoice}
            \choice{There is a solution.}
            \choice{There is no solution.}
            \choice[correct]{There are infinitely many solutions.}
        \end{multipleChoice}
        \begin{problem}
            There is one free variable, so if we denote $x_3 = A$ then: $x_1 = \answer{ -1 + 3A}$, $x_2 = \answer{2 - A}$
        \end{problem}
    \end{itemize}
\end{exercise}
%\exsol{%
%a) $x_1=-2$, $x_2 = \frac{7}{3}$
%\quad
%b)~no solution
%\quad
%c)~$a = -3$, $b=10$, $c=-8$
%\quad
%d)~$x_3$ is free, $x_1 = -1+3x_3$, $x_2 = 2-x_3$
%}

\begin{exercise}%
    Solve the system of equations
    \begin{equation*}
        \begin{split}
            -4x_2 + x_3 + 2x_4 &= 16 \\
            2x_1 + 2x_2 - 4x_3 - 3x_4 &= 1 \\
            x_1 + x_2 + 2x_3 + 3x_4 &= 6 \\
            2x_1 - 2x_3 + 4x_4 &= 24
        \end{split}
    \end{equation*}
    \begin{multipleChoice}
        \choice[correct]{There is a solution.}
        \choice{There is no solution.}
        \choice{There are infinitely many solutions.}
    \end{multipleChoice}
    \begin{problem}
        $x_1 = \answer{4}$, $x_2 = \answer{-3}$, $x_3 = \answer{-2}$, $x_4 = \answer{3}$
    \end{problem}
\end{exercise}
%\exsol{%
%$x_1 = 4$, $x_2 = -3$, $x_3 = -2$, $x_4 = 3$
%}%

\begin{exercise}%
    Solve the system of equations
    \begin{equation*}
        \begin{split}
            3x_2 + 3x_3 + 2x_4 &= 4 \\
            4x_1 + 4x_2 + 2x_3 - 4x_4 &= -26 \\
            x_1 - 3x_2 -2x_3 + 2x_4 &= 1 \\
            3x_1 + 3x_2 + 3x_3  - x_4 &= -14
        \end{split}
    \end{equation*}
    or determine that no solution exists. 
    \begin{multipleChoice}
        \choice[correct]{There is a solution.}
        \choice{There is no solution.}
        \choice{There are infinitely many solutions.}
    \end{multipleChoice}
    \begin{problem}
        $x_1 = \answer{-4}$, $x_2 = \answer{-1}$, $x_3 = \answer{1}$, $x_4 = \answer{2}$
    \end{problem}
\end{exercise}
%\exsol{%
%$x_1 = -4$, $x_2 = -1$, $x_3 = 1$, $x_4 = 2$
%}%

\begin{exercise}%
    Solve the system of equations
    \begin{equation*}
        \begin{split}
            2x_1 + x_2 - x_3 + 4x_4 &= 11 \\
            x_1 + 4x_2  - 4x_3 - x_4 &= -7 \\
            -2x_1 - 3x_2 + 2x_3 + x_4 &= 11 \\
            3x_1 + x_3  + 4x_4 &= 3
        \end{split}
    \end{equation*}
    or determine that no solution exists. 
    \begin{multipleChoice}
        \choice{There is a solution.}
        \choice[correct]{There is no solution.}
        \choice{There are infinitely many solutions.}
    \end{multipleChoice}
    \begin{problem}
        $x_1 = \answer{-4}$, $x_2 = \answer{-1}$, $x_3 = \answer{1}$, $x_4 = \answer{2}$
    \end{problem}
\end{exercise}
%\exsol{%
%No solution exists.
%}%

\begin{exercise}%
    Solve the system of equations
    \begin{equation*}
        \begin{split}
            x_1 - x_3 - 4x_4 &= -3 \\
            x_1 + x_2 + x_4 &= 0\\
            x_1 + 3x_2 + 3x_3 - 4x_4 &= -28 \\
            6x_1 + 3x_2 - 4x_3 + 6x_4 &= 25
        \end{split}
    \end{equation*}
    or determine that no solution exists. 
    \begin{multipleChoice}
        \choice{There is a solution.}
        \choice{There is no solution.}
        \choice[correct]{There are infinitely many solutions.}
    \end{multipleChoice}
    \begin{problem}
        Since there are infinitely many solutions, let $x_4 = t$. Then: $x_1 = \answer{19t - 37}$, $x_2 = \answer{37 - 20t}$, $x_3 = \answer{15t - 34}$, $x_4 = t$
    \end{problem}
\end{exercise}
%\exsol{%
%Infinitely many solutions of the form $x_1 = 19t-37$, $x_2 = 37-20t$, $x_3 = 15t-34$, $x_4 = t$ for any real number $t$.
%}%

\begin{exercise}%
    Assume that you are solving a three component linear system of equations via row reduction of an augmented matrix and reach the matrix
    \begin{equation*}
        \left[ 
        \begin{array}{ccc|c}
            1 & 0 & 3 & 4 \\
            0 & 0 & 1 & 3 \\
            0 & 0 & 0 & 1
        \end{array}
        \right].
    \end{equation*}
    What does this mean about the solution to this system of equations?
    \begin{multipleChoice}
        \choice{There is a solution.}
        \choice[correct]{There is no solution.}
        \choice{There are infinitely many solutions.}
    \end{multipleChoice}
\end{exercise}
%\exsol{%
%There is no solution.
%}

\begin{exercise}%
    Assume that you are solving a three component linear system of equations via row reduction of an augmented matrix and reach the matrix
    \begin{equation*}
        \left[ 
        \begin{array}{ccc|c}
            1 & 1 & 3 & 6 \\
            0 & 1 & 2 & 4 \\
            0 & 0 & 0 & 0
        \end{array}
        \right].
    \end{equation*}
    What does this mean about the solution to this system of equations?
    \begin{multipleChoice}
        \choice{There is a solution.}
        \choice{There is no solution.}
        \choice[correct]{There are infinitely many solutions.}
    \end{multipleChoice}
    \begin{problem}
        Since there are infinitely many solutions, let $x_3 = t$. Then: $x_1 = \answer{2 - t}$, $x_2 = \answer{4 - 2t}$, $x_3 = t$
    \end{problem}
\end{exercise}
%\exsol{%
%There are infinitely many solutions of the form $x_1 = 2-t$, $x_2 = 4-2t$, $x_3 = t$ for any real number $t$.
%}

\begin{exercise}
    Assume that you are solving a four component linear system of equations via row reduction of an augmented matrix and reach the matrix
    \begin{equation*}
        \left[ 
        \begin{array}{cccc|c}
            1 & 2 & 3 & 5 & 1 \\
            0 & 2 & 1 & 4 & 2 \\
            0 & 1 & 0 & 3 & 0 \\
            0 & 3 & 2 & -1 & 1
        \end{array}
        \right].
    \end{equation*}
    %What is the next step in reducing this matrix? 
    Carry out the rest of this problem to solve the corresponding system of equations. 
    \begin{multipleChoice}
        \choice[correct]{There is a solution.}
        \choice{There is no solution.}
        \choice{There are infinitely many solutions.}
    \end{multipleChoice}
    \begin{problem}
        $x_1 = \answer{-\frac{15}{2}}$, $x_2 = \answer{-\frac{3}{2}}$, $x_3 = \answer{3}$, $x_4 = \answer{\frac{1}{2}}$
    \end{problem}
\end{exercise}
%\comboSol
%{%
%Swap rows 2 and 3, then cancel down. $x_1 = -\frac{15}{2}$, $x_2 = -\frac{3}{2}$, $x_3 = 3$, $x_4 = \frac{1}{2}$
%}

\begin{exercise}%
    Assume that someone else has provided you the solution to an augmented matrix reduction for solving a system of equations given below
    \begin{equation*}
        \begin{bmatrix}
            1 & 1 & 2 & 4 \\
            0 & 1 & 3 & 5 \\
            1 & 2 & 4 & -1
        \end{bmatrix}  \rightarrow  
        \begin{bmatrix} 
        1 & 1 & 2 & 4 \\ 
        0 & 1 & 3 & 5 \\ 
        0 & 1 & 2 & -5 
        \end{bmatrix} \rightarrow 
        \begin{bmatrix} 
        1 & 1 & 2 & 4 \\ 
        0 & 0 & 1 & 1 \\ 
        0 & 0 & 0 & -9 
        \end{bmatrix}.
    \end{equation*} 
    Is this work correct? 
    \begin{multipleChoice}
        \choice[correct]{Yes.}
        \choice{No.}
    \end{multipleChoice}
    \begin{problem}
        Correct the work to solve the system.\\
        $x_1 = \answer{9}$, $x_2 = \answer{-25}$, and $x_3 = \answer{10}$
    \end{problem}
\end{exercise}
%\exsol{%
%The work is not correct. It looks like the author used row 1 to try to cancel the second column from rows 2 and 3, which we can not do. (This work would say no solution) The correct method would be to use row 2 to cancel row 3, resulting in a solution $x_1 = 9$, $x_2 = -25$, and $x_3 = 10$.
%}

\begin{exercise}
    Find the row echelon form of the matrix $A$ given below. What does this tell you about the solutions to the equation $A\vec{x} = \vec{0}$?
    \begin{equation*}
        A = 
        \begin{bmatrix} 
        1 & 2 & 4 \\ 
        -2 & -3 & -6 \\ 
        1 & 1 & 2 
        \end{bmatrix}
    \end{equation*}
    \begin{multipleChoice}
        \choice{There is a solution.}
        \choice{There is no solution.}
        \choice[correct]{There are infinitely many solutions.}
    \end{multipleChoice}
\end{exercise}
%\comboSol
%{%
%There are infinitely many solutions. 
%}


\begin{exercise}
    Find the row echelon form of the matrix $A$ given below. What does this tell you about the solutions to the equation $A\vec{x} = \vec{0}$?
    \begin{equation*}
        A = 
        \begin{bmatrix} 
        1 & -1 & 4 & 1 \\ 
        2 & -3 & 10 & 2 \\ 
        -3 & 4 & -15 & -4 \\ 
        3 & -5 & 17 & 3 
        \end{bmatrix}
    \end{equation*}
    \begin{multipleChoice}
        \choice[correct]{There is a solution.}
        \choice{There is no solution.}
        \choice{There are infinitely many solutions.}
    \end{multipleChoice}
    \begin{problem}
        $\answer{1}\vec{x} = \vec{0}$
    \end{problem}
\end{exercise}
%\comboSol
%{%
%There is only one solution, $\vec{x} = \vec{0}$. 
%}
%
%
%\setcounter{exercise}{100}

\end{document}