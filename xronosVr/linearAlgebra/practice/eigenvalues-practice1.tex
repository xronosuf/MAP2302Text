\documentclass{ximera}

\title{Practice for Eigenvalues}

%\auor{Matthew Charnley and Jason Nowell}
\usepackage[margin=1.5cm]{geometry}
\usepackage{indentfirst}
\usepackage{sagetex}
\usepackage{lipsum}
\usepackage{amsmath}
\usepackage{mathrsfs}


%%% Random packages added without verifying what they are really doing - just to get initial compile to work.
\usepackage{tcolorbox}
\usepackage{hypcap}
\usepackage{booktabs}%% To get \toprule,\midrule,\bottomrule etc.
\usepackage{nicefrac}
\usepackage{caption}
\usepackage{units}

% This is my modified wrapfig that doesn't use intextsep
\usepackage{mywrapfig}
\usepackage{import}



%%% End to random added packages.


\graphicspath{
    {./figures/}
    {./../figures/}
    {./../../figures/}
}
\renewcommand{\log}{\ln}%%%%
\DeclareMathOperator{\arcsec}{arcsec}
%% New commands


%%%%%%%%%%%%%%%%%%%%
% New Conditionals %
%%%%%%%%%%%%%%%%%%%%


% referencing
\makeatletter
    \DeclareRobustCommand{\myvref}[2]{%
      \leavevmode%
      \begingroup
        \let\T@pageref\@pagerefstar
        \hyperref[{#2}]{%
	  #1~\ref*{#2}%
        }%
        \vpageref[\unskip]{#2}%
      \endgroup
    }%

    \DeclareRobustCommand{\myref}[2]{%
      \leavevmode%
      \begingroup
        \let\T@pageref\@pagerefstar
        \hyperref[{#2}]{%
	  #1~\ref*{#2}%
        }%
      \endgroup
    }%
\makeatother

\newcommand{\figurevref}[1]{\myvref{Figure}{#1}}
\newcommand{\figureref}[1]{\myref{Figure}{#1}}
\newcommand{\tablevref}[1]{\myvref{Table}{#1}}
\newcommand{\tableref}[1]{\myref{Table}{#1}}
\newcommand{\chapterref}[1]{\myref{chapter}{#1}}
\newcommand{\Chapterref}[1]{\myref{Chapter}{#1}}
\newcommand{\appendixref}[1]{\myref{appendix}{#1}}
\newcommand{\Appendixref}[1]{\myref{Appendix}{#1}}
\newcommand{\sectionref}[1]{\myref{\S}{#1}}
\newcommand{\subsectionref}[1]{\myref{subsection}{#1}}
\newcommand{\subsectionvref}[1]{\myvref{subsection}{#1}}
\newcommand{\exercisevref}[1]{\myvref{Exercise}{#1}}
\newcommand{\exerciseref}[1]{\myref{Exercise}{#1}}
\newcommand{\examplevref}[1]{\myvref{Example}{#1}}
\newcommand{\exampleref}[1]{\myref{Example}{#1}}
\newcommand{\thmvref}[1]{\myvref{Theorem}{#1}}
\newcommand{\thmref}[1]{\myref{Theorem}{#1}}


\renewcommand{\exampleref}[1]{ {\color{red} \bfseries Normally a reference to a previous example goes here.}}
\renewcommand{\figurevref}[1]{ {\color{red} \bfseries Normally a reference to a previous figure goes here.}}
\renewcommand{\tablevref}[1]{ {\color{red} \bfseries Normally a reference to a previous table goes here.}}
\renewcommand{\Appendixref}[1]{ {\color{red} \bfseries Normally a reference to an Appendix goes here.}}
\renewcommand{\exercisevref}[1]{ {\color{red} \bfseries Normally a reference to a previous exercise goes here.}}



\newcommand{\R}{\mathbb{R}}

%% Example Solution Env.
\def\beginSolclaim{\par\addvspace{\medskipamount}\noindent\hbox{\bf Solution:}\hspace{0.5em}\ignorespaces}
\def\endSolclaim{\par\addvspace{-1em}\hfill\rule{1em}{0.4pt}\hspace{-0.4pt}\rule{0.4pt}{1em}\par\addvspace{\medskipamount}}
\newenvironment{exampleSol}[1][]{\beginSolclaim}{\endSolclaim}

%% General figure formating from original book.
\newcommand{\mybeginframe}{%
\begin{tcolorbox}[colback=white,colframe=lightgray,left=5pt,right=5pt]%
}
\newcommand{\myendframe}{%
\end{tcolorbox}%
}

%%% Eventually return and fix this to make matlab code work correctly.
%% Define the matlab environment as another code environment
%\newenvironment{matlab}
%{% Begin Environment Code
%{ \centering \bfseries Matlab Code }
%\begin{code}
%}% End of Begin Environment Code
%{% Start of End Environment Code
%\end{code}
%}% End of End Environment Code


% this one should have a caption, first argument is the size
\newenvironment{mywrapfig}[2][]{
 \wrapfigure[#1]{r}{#2}
 \mybeginframe
 \centering
}{%
 \myendframe
 \endwrapfigure
}

% this one has no caption, first argument is size,
% the second argument is a larger size used for HTML (ignored by latex)
\newenvironment{mywrapfigsimp}[3][]{%
 \wrapfigure[#1]{r}{#2}%
 \centering%
}{%
 \endwrapfigure%
}
\newenvironment{myfig}
    {%
    \begin{figure}[h!t]
        \mybeginframe%
        \centering%
    }
    {%
        \myendframe
    \end{figure}%
    }


% graphics include
\newcommand{\diffyincludegraphics}[3]{\includegraphics[#1]{#3}}
\newcommand{\myincludegraphics}[3]{\includegraphics[#1]{#3}}
\newcommand{\inputpdft}[1]{\subimport*{../figures/}{#1.pdf_t}}


%% Not sure what these even do? They don't seem to actually work... fun!
%\newcommand{\mybxbg}[1]{\tcboxmath[colback=white,colframe=black,boxrule=0.5pt,top=1.5pt,bottom=1.5pt]{#1}}
%\newcommand{\mybxsm}[1]{\tcboxmath[colback=white,colframe=black,boxrule=0.5pt,left=0pt,right=0pt,top=0pt,bottom=0pt]{#1}}
\newcommand{\mybxsm}[1]{#1}
\newcommand{\mybxbg}[1]{#1}

%%% Something about tasks for practice/hw?
\usepackage{tasks}
\usepackage{footnote}
\makesavenoteenv{tasks}


%% For pdf only?
\newcommand{\diffypdfversion}[1]{#1}


%% Kill ``cite'' and go back later to fix it.
\renewcommand{\cite}[1]{}


%% Currently we can't really use index or its derivatives. So we are gonna kill them off.
\renewcommand{\index}[1]{}
\newcommand{\myindex}[1]{#1}







\begin{document}
\begin{abstract}
Why?
\end{abstract}
\maketitle


\begin{exercise}%
    Find the eigenvalues and corresponding eigenvectors of the matrix 
    \[ 
        \begin{bmatrix} 
        -8 & -18 \\ 
        4 & 10 
        \end{bmatrix} 
    \]
    $\lambda_1 = \answer{-2}$, $\vec{v}_1 = \left[\begin{smallmatrix} \answer{-3} \\ \answer{1} \end{smallmatrix}\right]$, $\lambda_2 = \answer{4}$, $\vec{v}_2 = \left[\begin{smallmatrix} \answer{3} \\ \answer{-2} \end{smallmatrix}\right]$
\end{exercise}
%\exsol{%
%$\lambda_1 = -2$, $\vec{v}_1 = \left[\begin{smallmatrix} -3 \\ 1 \end{smallmatrix}\right]$, $\lambda_2 = 4$, $\vec{v}_2 = \left[\begin{smallmatrix} 3 \\ -2 \end{smallmatrix}\right]$. 
%}

\begin{exercise}%
    Find the eigenvalues and corresponding eigenvectors of the matrix 
    \[ 
        \begin{bmatrix} 
        -2 & 0 \\ 
        8 & -4 
        \end{bmatrix} 
    \]
    $\lambda_1 = \answer{-2}$, $\vec{v}_1 = \left[\begin{smallmatrix} \answer{1} \\ \answer{4} \end{smallmatrix}\right]$, $\lambda_2 = \answer{-4}$, $\vec{v}_2 = \left[\begin{smallmatrix} \answer{0} \\ \answer{1} \end{smallmatrix}\right]$. 
\end{exercise}
%\exsol{%
%$\lambda_1 = -2$, $\vec{v}_1 = \left[\begin{smallmatrix} 1 \\ 4 \end{smallmatrix}\right]$, $\lambda_2 = -4$, $\vec{v}_2 = \left[\begin{smallmatrix} 0 \\ 1 \end{smallmatrix}\right]$. 
%}

\begin{exercise}%
    Find the eigenvalues and corresponding eigenvectors of the matrix 
    \[ 
        \begin{bmatrix} 
        -7 & 1 \\ 
        -12 & 0 
        \end{bmatrix} 
    \]
    $\lambda_1 = \answer{-4}$, $\vec{v}_1 = \left[\begin{smallmatrix} \answer{1} \\ \answer{3} \end{smallmatrix}\right]$, $\lambda_2 = \answer{-3}$, $\vec{v}_2 = \left[\begin{smallmatrix} \answer{1} \\ \answer{4} \end{smallmatrix}\right]$. 
\end{exercise}
%\exsol{%
%$\lambda_1 = -4$, $\vec{v}_1 = \left[\begin{smallmatrix} 1 \\ 3 \end{smallmatrix}\right]$, $\lambda_2 = -3$, $\vec{v}_2 = \left[\begin{smallmatrix} 1 \\ 4 \end{smallmatrix}\right]$. 
%}

\begin{exercise}%
    Find the eigenvalues and corresponding eigenvectors of the matrix 
    \[ 
        \begin{bmatrix} 
        -3 & 5 \\ 
        -8 & 9 
        \end{bmatrix} 
    \]
    $\lambda_1 = \answer{3+2i}$, $\vec{v}_1 = \left[\begin{smallmatrix} \answer{3-i} \\ \answer{4} \end{smallmatrix}\right]$, $\lambda_2 = \answer{3-2i}$, $\vec{v}_2 = \left[\begin{smallmatrix} \answer{3+i} \\ \answer{4} \end{smallmatrix}\right]$. 
\end{exercise}
%\exsol{%
%$\lambda_1 = 3+2i$, $\vec{v}_1 = \left[\begin{smallmatrix} 3-i \\ 4 \end{smallmatrix}\right]$, $\lambda_2 = 3-2i$, $\vec{v}_2 = \left[\begin{smallmatrix} 3+i \\ 4 \end{smallmatrix}\right]$. 
%}

\begin{exercise}%
    Find the eigenvalues and corresponding eigenvectors of the matrix 
    \[ 
        \begin{bmatrix} 
        0 & 2 \\ 
        -1 & -2 
        \end{bmatrix} 
    \]
    $\lambda_1 = \answer{-1+i}$, $\vec{v}_1 = \left[\begin{smallmatrix} \answer{2} \\ \answer{-1+i} \end{smallmatrix}\right]$, $\lambda_2 = \answer{-1-i}$, $\vec{v}_2 = \left[\begin{smallmatrix} \answer{2} \\ \answer{-1-i} \end{smallmatrix}\right]$. 
\end{exercise}
%\exsol{%
%$\lambda_1 = -1+i$, $\vec{v}_1 = \left[\begin{smallmatrix} 2 \\ -1+i \end{smallmatrix}\right]$, $\lambda_2 = -1-i$, $\vec{v}_2 = \left[\begin{smallmatrix} 2 \\ -1-i \end{smallmatrix}\right]$. 
%}

\begin{exercise}%
    Find the eigenvalues and corresponding eigenvectors of the matrix 
    \[ 
        \begin{bmatrix} 
        -4 & 1 \\ 
        -8 & 0 
        \end{bmatrix} 
    \]
    $\lambda_1 = \answer{-2+2i}$, $\vec{v}_1 = \left[\begin{smallmatrix} \answer{1-i} \\ \answer{4} \end{smallmatrix}\right]$, $\lambda_2 = \answer{-2-2i}$, $\vec{v}_2 = \left[\begin{smallmatrix} \answer{1+i}\\ \answer{4} \end{smallmatrix}\right]$. 
\end{exercise}
%\exsol{%
%$\lambda_1 = -2+2i$, $\vec{v}_1 = \left[\begin{smallmatrix} 1-i \\ 4 \end{smallmatrix}\right]$, $\lambda_2 = -2-2i$, $\vec{v}_2 = \left[\begin{smallmatrix} 1+i\\ 4 \end{smallmatrix}\right]$. 
%}

\begin{exercise}%
    Find the eigenvalues and corresponding eigenvectors of the matrix 
    \[ 
        \begin{bmatrix} 
        0 & -8 \\ 
        2 & 8 
        \end{bmatrix} 
    \]
    $\lambda_1 = \answer{4}$, $\vec{v}_1 = \left[\begin{smallmatrix} \answer{-2} \\ \answer{1} \end{smallmatrix}\right]$
\end{exercise}
%\exsol{%
%$\lambda_1 = 4$, $\vec{v}_1 = \left[\begin{smallmatrix} -2 \\ 1 \end{smallmatrix}\right]$
%}

\begin{exercise}%
    Find the eigenvalues and corresponding eigenvectors of the matrix 
    \[ 
        \begin{bmatrix} 
        1 & -2 \\ 
        8 & -7 
        \end{bmatrix} 
    \]
    $\lambda_1 = \answer{-3}$, $\vec{v}_1 = \left[\begin{smallmatrix} \answer{1} \\ \answer{2} \end{smallmatrix}\right]$
\end{exercise}
%\exsol{%
%$\lambda_1 = -3$, $\vec{v}_1 = \left[\begin{smallmatrix} 1 \\ 2 \end{smallmatrix}\right]$
%}

\begin{exercise}%
    Find the eigenvalues and corresponding eigenvectors of the matrix 
    \[ 
        \begin{bmatrix} 
        4 & 0 & 0 \\ 
        -4 & 2 & 1 \\ 
        -6 & 0 & 1 
        \end{bmatrix} 
    \]
    $\lambda_1 = \answer{2}$, $\vec{v}_1 = \left[\begin{smallmatrix} \answer{0}\\ \answer{1}\\ \answer{0} \end{smallmatrix}\right]$, $\lambda_2 = \answer{1}$, $\vec{v}_2 = \left[\begin{smallmatrix} \answer{0}\\ \answer{-1}\\ \answer{1} \end{smallmatrix}\right]$, $\lambda_3 = \answer{4}$, $\vec{v}_3 = \left[\begin{smallmatrix} \answer{1}\\ \answer{-3}\\ \answer{-2} \end{smallmatrix}\right]$
\end{exercise}
%\exsol{%
%$\lambda_1 = 2$, $\vec{v}_1 = \left[\begin{smallmatrix} 0\\ 1\\ 0 \end{smallmatrix}\right]$, $\lambda_2 = 1$, $\vec{v}_2 = \left[\begin{smallmatrix} 0\\ -1\\ 1 \end{smallmatrix}\right]$, $\lambda_3 = 4$, $\vec{v}_3 = \left[\begin{smallmatrix} 1\\ -3\\ -2 \end{smallmatrix}\right]$
%}

\begin{exercise}%
    Find the eigenvalues and corresponding eigenvectors of the matrix 
    \[ 
    \begin{bmatrix} 
    -4 & 9 & 9 \\ 
    -3 & 6 & 9 \\ 
    3 & -7 & -10 
    \end{bmatrix} 
    \]
    $\lambda_1 = \answer{-4}$, $\vec{v}_1 = \left[\begin{smallmatrix} \answer{1}\\ \answer{3}\\ \answer{-3} \end{smallmatrix}\right]$, $\lambda_2 = \answer{-3}$, $\vec{v}_2 = \left[\begin{smallmatrix} \answer{0}\\ \answer{-1}\\ \answer{1} \end{smallmatrix}\right]$, $\lambda_3 = \answer{-1}$, $\vec{v}_3 = \left[\begin{smallmatrix} \answer{3}\\ \answer{0}\\ \answer{1} \end{smallmatrix}\right]$
\end{exercise}
%\exsol{%
%$\lambda_1 = -4$, $\vec{v}_1 = \left[\begin{smallmatrix} 1\\ 3\\ -3 \end{smallmatrix}\right]$, $\lambda_2 = -3$, $\vec{v}_2 = \left[\begin{smallmatrix} 0\\ -1\\ 1 \end{smallmatrix}\right]$, $\lambda_3 = -1$, $\vec{v}_3 = \left[\begin{smallmatrix} 3\\ 0\\ 1 \end{smallmatrix}\right]$
%}

\begin{exercise}%
    Find the eigenvalues and corresponding eigenvectors of the matrix 
    \[ 
    \begin{bmatrix} 
    -2 & 0 & 0 \\ 
    0 & 4 & 6 \\ 
    6 & -3 & -2 
    \end{bmatrix} 
    \]
    $\lambda_1 = \answer{1+3i}$, $\vec{v}_1 = \left[\begin{smallmatrix} \answer{0}\\ \answer{2}\\ \answer{-1+i} \end{smallmatrix}\right]$, $\lambda_2 = \answer{1-3i}$, $\vec{v}_2 = \left[\begin{smallmatrix} \answer{0}\\ \answer{2}\\ \answer{-1-i} \end{smallmatrix}\right]$, $\lambda_3 = \answer{-2}$, $\vec{v}_3 = \left[\begin{smallmatrix} \answer{1}\\ \answer{2}\\ \answer{-2} \end{smallmatrix}\right]$
\end{exercise}
%\exsol{%
%$\lambda_1 = 1+3i$, $\vec{v}_1 = \left[\begin{smallmatrix} 0\\ 2\\ -1+i \end{smallmatrix}\right]$, $\lambda_2 = 1-3i$, $\vec{v}_2 = \left[\begin{smallmatrix} 0\\ 2\\ -1-i \end{smallmatrix}\right]$, $\lambda_3 = -2$, $\vec{v}_3 = \left[\begin{smallmatrix} 1\\ 2\\ -2 \end{smallmatrix}\right]$
%}

\begin{exercise}%
    Find the eigenvalues and corresponding eigenvectors of the matrix 
    \[ 
    \begin{bmatrix} 
    5 & 3 & 6 \\ 
    2 & 2 & 2 \\ 
    -3 & -2 & -3 
    \end{bmatrix} 
    \]
    $\lambda_1 = \answer{2}$, $\vec{v}_1 = \left[\begin{smallmatrix} \answer{1}\\ \answer{1}\\ \answer{-1} \end{smallmatrix}\right]$, $\lambda_2 = \answer{1}$, $\vec{v}_2 = \left[\begin{smallmatrix} \answer{0}\\ \answer{-2}\\ \answer{1} \end{smallmatrix}\right]$
\end{exercise}
%\exsol{%
%$\lambda_1 = 2$, $\vec{v}_1 = \left[\begin{smallmatrix} 1\\ 1\\ -1 \end{smallmatrix}\right]$, $\lambda_2 = 1$, $\vec{v}_2 = \left[\begin{smallmatrix} 0\\ -2\\ 1 \end{smallmatrix}\right]$ (double root)
%}

\begin{exercise}%
    Find the eigenvalues and eigenvectors for the matrix below. Compute generalized eigenvectors if needed to get to a total of two vectors. 
    \[ 
    \begin{bmatrix} 
    -11 & -9 \\ 
    12 & 10 
    \end{bmatrix} 
    \]
    $\lambda_1 = \answer{-2}$, $\vec{v}_1 = \left[\begin{smallmatrix} \answer{-1} \\ \answer{1} \end{smallmatrix}\right]$, $\lambda_2 = \answer{1}$, $\vec{v}_2 = \left[\begin{smallmatrix} \answer{3} \\ \answer{-4} \end{smallmatrix}\right]$. 
\end{exercise}
%\exsol{%
%$\lambda_1 = -2$, $\vec{v}_1 = \left[\begin{smallmatrix} -1 \\ 1 \end{smallmatrix}\right]$, $\lambda_2 = 1$, $\vec{v}_2 = \left[\begin{smallmatrix} 3 \\ -4 \end{smallmatrix}\right]$. 
%}

\begin{exercise}%
    Find the eigenvalues and eigenvectors for the matrix below. Compute generalized eigenvectors if needed to get to a total of two vectors. 
    \[ 
    \begin{bmatrix} 
    4 & -4 \\ 
    1 & 0 
    \end{bmatrix} 
    \]
    $\lambda_1 = \answer{2}$, $\vec{v}_1 = \left[\begin{smallmatrix} \answer{2} \\ \answer{1} \end{smallmatrix}\right]$, Generalized eigenvector $\vec{w} = \left[\begin{smallmatrix} \answer{1} \\ \answer{0} \end{smallmatrix}\right]$.  
\end{exercise}
%\exsol{%
%$\lambda_1 = 2$, $\vec{v}_1 = \left[\begin{smallmatrix} 2 \\ 1 \end{smallmatrix}\right]$, Generalized eigenvector $\vec{w} = \left[\begin{smallmatrix} 1 \\ 0 \end{smallmatrix}\right]$.  
%}

\begin{exercise}%
    This exercise will work through the process of finding the eigenvalues and corresponding eigenvectors of the matrix
    \begin{equation*}
    A = \begin{bmatrix}
        -2 & 0 & -3 \\ 
        12 & 5  & 12 \\ 
        0 &-1 & 1
    \end{bmatrix}.
    \end{equation*}
    Find the characteristic polynomial of this matrix by computing $\det(A - \lambda I)$ using any method from this section [Use $\lambda$ as your variable; enter an expression, not an equality]. $\answer{-\lambda^3 + 4\lambda^2  - 5\lambda + 2}$
    \begin{problem}
        This polynomial can be rewritten as $-(\lambda - r_1)^2(\lambda - r_2)$ where $r_1$ and $r_2$ are the eigenvalues of $A$. What are the eigenvalues? What is each of their algebraic multiplicity? (Hint: One of these roots is $2$)\\
        $r_1 = \answer{1}$ with algebraic multiplicity $2$, and $r_2 = \answer{2}$ with algebraic multiplicity $1$.
        \begin{problem}
            Find an eigenvector for eigenvalue $r_2$ above. $\left[\begin{smallmatrix} \answer{3} \\ \answer{4} \\ \answer{-4} \end{smallmatrix}\right]$ 
            \begin{problem}
                What is the geometric multiplicity of this eigenvalue? $\answer{1}$
            \end{problem}
        \end{problem}
        \begin{problem}
            Find an eigenvector for eigenvalue $r_1$. $\left[\begin{smallmatrix}1 \\ \answer{0} \\ \answer{-1} \end{smallmatrix}\right]$.
            \begin{problem}
                What is the geometric multiplicity of this eigenvalue? $\answer{1}$
            \end{problem}
        
            \begin{problem}
                There is only one possible eigenvector for $r_1$, which means it is defective. Find a solution to the equation $(A - r_1 I) \vec{w} = \vec{v}$, where $\vec{v}$ is the eigenvector you found in the previous part. \\
                $\left[\begin{smallmatrix} \answer{-\frac{1}{3}} \\ \answer{1} \\ \answer{0}\end{smallmatrix}\right]$
                
                This is the generalized eigenvector for $r_1$. 
            \end{problem}
        \end{problem}
    \end{problem}
\end{exercise}
%\exsol{%
%a)~$-\lambda^3 + 4\lambda^2  - 5\lambda + 2$ \quad b)~$r_1 = 1$ with algebraic multiplicity 2, and $r_2 = 2$ with algebraic multiplicity 1. \quad c)~$\left[\begin{smallmatrix} 3 \\ 4 \\ -4 \end{smallmatrix}\right]$. Geometric multiplicity is 1. \quad d)~ $\left[\begin{smallmatrix}1 \\ 0 \\ -1 \end{smallmatrix}\right]$. Geometric multiplicity is 1. \quad e)~
%$\left[\begin{smallmatrix} -1/3 \\ 1 \\ 0\end{smallmatrix}\right]$. There are many answers here, and they will satisfy $v_2 = 1$ and $v_1 + v_3 = -1/3$. 
%}

\begin{exercise} \label{ex:diagonalization}
    We say that a matrix $A$ is \emph{diagonalizable} if there exist matrices $D$ and $P$ so that $PDP^{-1} = A$. This really means that $A$ can be represented by a diagonal matrix in a different basis (as opposed to the standard basis). One way this can be done is with eigenvalues.
    \begin{itemize}
        \item Consider the matrix $A$ given by 
            \[ 
            A = 
            \begin{bmatrix} 
            -4 & 6 \\ 
            -1 & 1 
            \end{bmatrix}. 
            \]
            Find the eigenvalues and corresponding eigenvectors of this matrix.\\
            $\lambda = \answer{-1}$, $\left[\begin{smallmatrix} \answer{2} \\ \answer{1} \end{smallmatrix}\right]$\\
            $\lambda = \answer{-2}$, $\left[\begin{smallmatrix} \answer{3} \\ \answer{1} \end{smallmatrix}\right]$
        
        \item Consider, $D$, a diagonal matrix with the eigenvalues of $A$ on the diagonal, and form $E$, a matrix whose columns are the eigenvectors of $A$ in the \emph{same order} as the eigenvalues were put into $D$. Write out these matrices.\\
            $D = \left[\begin{smallmatrix} -2 & 0 \\ 0 & -1 \end{smallmatrix}\right]$, $E = \left[\begin{smallmatrix} \answer{3} & \answer{2} \\ \answer{1} & \answer{1} \end{smallmatrix}\right]$
        
        \item Compute $E^{-1}$. \\
            $E^{-1} = \left[\begin{smallmatrix} \answer{1} & \answer{-2} \\ \answer{-1} & \answer{3} \end{smallmatrix}\right]$
        
        \item Work out the products $EDE^{-1}$ and $E^{-1}AE$. %What do you notice here?
            $EDE^{-1} = \begin{bmatrix} \answer{-4} & \answer{6} \\ \answer{-1} & \answer{1} \end{bmatrix}$, $E^{-1}AE = \left[\begin{smallmatrix} \answer{-2} & \answer{0} \\ \answer{0} & \answer{-1} \end{smallmatrix}\right]$
            \begin{problem}
                Notice that $EDE^{-1}$ is equal to a Matrix we already established. Specifically, Matrix $\answer{A}$.\\
                Similarly,  $E^{-1}AE$ is equal to a Matrix we already established. Specifically, Matrix $\answer{D}$.\\
            \end{problem}
    \end{itemize}
    \begin{feedback}[correct]
        This shows that, in the case of a $2\times 2$ matrix, if we have two distinct real eigenvalues, that matrix is diagonalizable, using the eigenvectors.
    \end{feedback}
\end{exercise}
%\comboSol
%{%
%a)~ $\lambda =-1$, $\left[\begin{smallmatrix} 2 \\ 1 \end{smallmatrix}\right]$. $\lambda = -2$, $\left[\begin{smallmatrix} 3 \\ 1 \end{smallmatrix}\right]$ \quad b)~ $D = \left[\begin{smallmatrix} -2 & 0 \\ 0 & -1 \end{smallmatrix}\right]$, $E = \left[\begin{smallmatrix} 3 & 2 \\ 1 & 1 \end{smallmatrix}\right]$ \quad
%c)~$E^{-1} = \left[\begin{smallmatrix} 1 & -2 \\ -1 & 3 \end{smallmatrix}\right]$ \quad d)~ $EDE^{-1} = A$, $E^{-1}AE = D$
%}

\begin{exercise}
    Follow the process outlined in Exercise \ref{ex:diagonalization} to attempt to diagonalize the matrix
    \[ 
    \begin{bmatrix} 
    13 & 14 & 12 \\ 
    -6 & -4 & -6 \\ 
    -3 & -6 & -2 
    \end{bmatrix}
    \] 
    Hint: 1 is an eigenvalue. 
    $D = \left[\begin{smallmatrix} 1 & 0 & 0 \\ 0 & 2 & 0 \\ 0 & 0 & 4 \end{smallmatrix}\right]$, $E = \left[\begin{smallmatrix} \answer{1} & \answer{-2} & \answer{2} \\ \answer{0} & \answer{-1} & \answer{-3} \\\answer{-1} & \answer{3} & \answer{2} \end{smallmatrix}\right]$
\end{exercise}
%\comboSol
%{%
%$D = \left[\begin{smallmatrix} 1 & 0 & 0 \\ 0 & 2 & 0 \\ 0 & 0 & 4 \end{smallmatrix}\right]$, $E = \left[\begin{smallmatrix} 1 & -2 & 2 \\ 0 & -1 & -3 \\-1 & 3 & 2 \end{smallmatrix}\right]$
%}

\begin{exercise} \label{ex:jordanform}
    The diagonalization process decribed in Exercise \ref{ex:diagonalization} works for any case where there are real and distinct eigenvalues, as well as complex eigenvalues (but the algebra with the complex numbers gets complicated). It may or may not work in the case of repeated eigenvalues, and it fails whenever there are defective eigenvalues. Consider the matrix
    \[ 
    \begin{bmatrix} 
    4 & -1 \\ 
    1 & 2 
    \end{bmatrix} 
    \]
    \begin{itemize}
        \item Find the eigenvalue(s) of this matrix, and see that we have a repeated eigenvalue. The repeated Eigenvalue is $\answer{3}$
        \item Find the eigenvector for that eigenvalue, as well as a generalized eigenvector.\\
        Eigenvector $\left[\begin{smallmatrix}  \answer{1} \\ \answer{1} \end{smallmatrix}\right]$, Generalized $\left[\begin{smallmatrix} \answer{1} \\ \answer{0} \end{smallmatrix}\right]$
        \item Build a matrix $E$ like before, but this time put the eigenvector in the first column and the generalized eigenvector in the second. Compute $E^{-1}$. \\
        $E = \left[\begin{smallmatrix} \answer{1} & \answer{1} \\ \answer{1} & \answer{0} \end{smallmatrix}\right]$, $E^{-1} = \left[\begin{smallmatrix}  \answer{0} & \answer{1} \\ \answer{1} & \answer{-1} \end{smallmatrix}\right]$
        \item Find the product $E^{-1}AE$. Before, this gave us a diagonal matrix, but what do we get now? 
        $E^{-1}AE = \left[\begin{smallmatrix} \answer{3} & \answer{1} \\ \answer{0} & \answer{3} \end{smallmatrix}\right]$
    \end{itemize}
    \begin{feedback}[correct]
        The matrix we get here is almost diagonal, but not quite. It turns out that this is the best we can do for matrices with defective eigenvalues. This matrix is often called $J$ and is the \emph{Jordan Form} of the matrix $A$. 
    \end{feedback}
\end{exercise}
%\comboSol
%{%
%a)~ 3 repeated \quad b)~ Eigenvector $\left[\begin{smallmatrix}  1 \\ 1 \end{smallmatrix}\right]$, Generalized $\left[\begin{smallmatrix} 1 \\ 0 \end{smallmatrix}\right]$ \quad c)~$E = \left[\begin{smallmatrix} 1 & 1 \\ 1 & 0 \end{smallmatrix}\right]$, $E^{-1} = \left[\begin{smallmatrix}  0 & 1 \\ 1 & -1 \end{smallmatrix}\right]$ \quad d)~ $E^{-1}AE = \left[\begin{smallmatrix} 3 & 1 \\ 0 & 3 \end{smallmatrix}\right]$
%}

\begin{exercise}%
    Follow the process in Exercise \ref{ex:jordanform} to find the Jordan Form of the matrix
    \[ 
    \begin{bmatrix} 
    -7 & 5 & 5 \\ 
    -4 & 5 & 7 \\ 
    -6 & 3 & 1 
    \end{bmatrix}. 
    \]
    $\left[\begin{smallmatrix} \answer{3} & \answer{0} & \answer{0} \\ \answer{0} & \answer{-2} & \answer{1} \\ \answer{0} & \answer{0} & \answer{-2} \end{smallmatrix}\right]$
\end{exercise}
%\exsol{%
%$\left[\begin{smallmatrix} 3 & 0 & 0 \\ 0 & -2 & 1 \\ 0 & 0 & -2 \end{smallmatrix}\right]$
%}



\end{document}