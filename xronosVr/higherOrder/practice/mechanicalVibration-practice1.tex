\documentclass{ximera}

\title{Practice for Mechanical Vibrations}

%\auor{Matthew Charnley and Jason Nowell}
\usepackage[margin=1.5cm]{geometry}
\usepackage{indentfirst}
\usepackage{sagetex}
\usepackage{lipsum}
\usepackage{amsmath}
\usepackage{mathrsfs}
\usepackage{tikz}
\usetikzlibrary{matrix}

%%% Random packages added without verifying what they are really doing - just to get initial compile to work.
\usepackage{tcolorbox}
\usepackage{hypcap}
\usepackage{booktabs}%% To get \toprule,\midrule,\bottomrule etc.
\usepackage{caption}
\usepackage{units}
\usepackage{multicol}
\usepackage{hhline}


% This is my modified wrapfig that doesn't use intextsep
\usepackage{mywrapfig}
\usepackage{import}



%%% End to random added packages.


\graphicspath{
    {./}
    {./figures/}
    {./../figures/}
    {./../../figures/}
}
\renewcommand{\log}{\ln}%%%%
\DeclareMathOperator{\arcsec}{arcsec}
%% New commands


%%%%%%%%%%%%%%%%%%%%
% New Conditionals %
%%%%%%%%%%%%%%%%%%%%


% referencing
\makeatletter
    \DeclareRobustCommand{\myvref}[2]{%
      \leavevmode%
      \begingroup
        \let\T@pageref\@pagerefstar
        \hyperref[{#2}]{%
	  #1~\ref*{#2}%
        }%
        \vpageref[\unskip]{#2}%
      \endgroup
    }%

    \DeclareRobustCommand{\myref}[2]{%
      \leavevmode%
      \begingroup
        \let\T@pageref\@pagerefstar
        \hyperref[{#2}]{%
	  #1~\ref*{#2}%
        }%
      \endgroup
    }%
\makeatother

\newcommand{\figurevref}[1]{\myvref{Figure}{#1}}
\newcommand{\figureref}[1]{\myref{Figure}{#1}}
\newcommand{\tablevref}[1]{\myvref{Table}{#1}}
\newcommand{\tableref}[1]{\myref{Table}{#1}}
\newcommand{\chapterref}[1]{\myref{chapter}{#1}}
\newcommand{\Chapterref}[1]{\myref{Chapter}{#1}}
\newcommand{\appendixref}[1]{\myref{appendix}{#1}}
\newcommand{\Appendixref}[1]{\myref{Appendix}{#1}}
\newcommand{\sectionref}[1]{\myref{\S}{#1}}
\newcommand{\subsectionref}[1]{\myref{subsection}{#1}}
\newcommand{\subsectionvref}[1]{\myvref{subsection}{#1}}
\newcommand{\exercisevref}[1]{\myvref{Exercise}{#1}}
\newcommand{\exerciseref}[1]{\myref{Exercise}{#1}}
\newcommand{\examplevref}[1]{\myvref{Example}{#1}}
\newcommand{\exampleref}[1]{\myref{Example}{#1}}
\newcommand{\thmvref}[1]{\myvref{Theorem}{#1}}
\newcommand{\thmref}[1]{\myref{Theorem}{#1}}


\renewcommand{\exampleref}[1]{ {\color{red} \bfseries Normally a reference to a previous example goes here.}}
\renewcommand{\examplevref}[1]{ {\color{red} \bfseries Normally a reference to a previous example goes here.}}
\renewcommand{\figurevref}[1]{ {\color{red} \bfseries Normally a reference to a previous figure goes here.}}
\renewcommand{\tablevref}[1]{ {\color{red} \bfseries Normally a reference to a previous table goes here.}}
\renewcommand{\Appendixref}[1]{ {\color{red} \bfseries Normally a reference to an Appendix goes here.}}
\renewcommand{\exercisevref}[1]{ {\color{red} \bfseries Normally a reference to a previous exercise goes here.}}
\renewcommand{\thmvref}[1]{ {\color{red} \bfseries Normally a reference to a previous theorem goes here.}}
\renewcommand{\subsectionvref}[1]{ {\color{red} \bfseries Normally a reference to a previous subsection goes here.}}



\newcommand{\R}{\mathbb{R}}
\newcommand{\C}{\mathbb{C}}

%% Example Solution Env.
\def\beginSolclaim{\par\addvspace{\medskipamount}\noindent\hbox{\bf Solution:}\hspace{0.5em}\ignorespaces}
\def\endSolclaim{\par\addvspace{-1em}\hfill\rule{1em}{0.4pt}\hspace{-0.4pt}\rule{0.4pt}{1em}\par\addvspace{\medskipamount}}
\newenvironment{exampleSol}[1][]{\beginSolclaim}{\endSolclaim}

%% General figure formating from original book.
\newcommand{\mybeginframe}{%
\begin{tcolorbox}[colback=white,colframe=lightgray,left=5pt,right=5pt]%
}
\newcommand{\myendframe}{%
\end{tcolorbox}%
}

%%% Eventually return and fix this to make matlab code work correctly.
%% Define the matlab environment as another code environment
%\NewEnviron{matlab}{ {\centering\bfseries MATLAB Code} \\ \noexpand{\BODY} }
%\let\beginmatlab\begincode
%\let\endmatlab\endcode
%\newenvironment{matlab}{% Begin Environment Code
%\begin{minipage}{\linewidth}
%\begin{verbatim}
%}% End of Begin Environment Code
%{% Start of End Environment Code
%\end{verbatim}
%\end{minipage}
%}% End of End Environment Code


% this one should have a caption, first argument is the size
\newenvironment{mywrapfig}[2][]{
 \wrapfigure[#1]{r}{#2}
 \mybeginframe
 \centering
}{%
 \myendframe
 \endwrapfigure
}

% this one has no caption, first argument is size,
% the second argument is a larger size used for HTML (ignored by latex)
\newenvironment{mywrapfigsimp}[3][]{%
 \wrapfigure[#1]{r}{#2}%
 \centering%
}{%
 \endwrapfigure%
}
\newenvironment{myfig}
    {%
    \begin{figure}[h!t]
        \mybeginframe%
        \centering%
    }
    {%
        \myendframe
    \end{figure}%
    }


% graphics include
\newcommand{\diffyincludegraphics}[3]{\includegraphics[#1]{#3}}
\newcommand{\myincludegraphics}[3]{\includegraphics[#1]{#3}}
\newcommand{\inputpdft}[1]{\subimport*{../figures/}{#1.pdf_t}}


%% Not sure what these even do? They don't seem to actually work... fun!
%\newcommand{\mybxbg}[1]{\tcboxmath[colback=white,colframe=black,boxrule=0.5pt,top=1.5pt,bottom=1.5pt]{#1}}
%\newcommand{\mybxsm}[1]{\tcboxmath[colback=white,colframe=black,boxrule=0.5pt,left=0pt,right=0pt,top=0pt,bottom=0pt]{#1}}
\newcommand{\mybxsm}[1]{#1}
\newcommand{\mybxbg}[1]{#1}

%%% Something about tasks for practice/hw?
\usepackage{tasks}
\usepackage{footnote}
\makesavenoteenv{tasks}


%% For pdf only?
\newcommand{\diffypdfversion}[1]{#1}


%% Kill ``cite'' and go back later to fix it.
\renewcommand{\cite}[1]{}


%% Currently we can't really use index or its derivatives. So we are gonna kill them off.
\renewcommand{\index}[1]{}
\newcommand{\myindex}[1]{#1}







\begin{document}
\begin{abstract}
    Why?
\end{abstract}
\maketitle


\begin{exercise} \label{mv:ex1}
    Consider a mass and spring system with a mass $m=2$, spring constant $k=3$, and damping constant $\gamma=1$.
    
    Set up and find the general solution of the system. [Use $A$ and $B$ for your arbitrary constants]
    \[
        y = \answer{Ae^{-t/4}\cos\left(\frac{\sqrt{23}}{4}t\right) + Be^{-t/4}\sin\left(\frac{\sqrt{23}}{4}t\right)}
    \]
    \begin{problem}
        The system is...
        \begin{multipleChoice}
            \choice[correct]{Underdamped}
            \choice{Overdamped}
            \choice{Critically damped}
        \end{multipleChoice}
        \begin{problem}
            Find a $\gamma$ that makes the system critically damped: $\gamma = \answer{2\sqrt{6}}$.
        \end{problem}
    \end{problem}
\end{exercise}

%\comboSol
%{%
%a)~$C_1e^{-t/4}\cos\left(\frac{\sqrt{23}}{4}t\right) + C_2e^{-t/4}\sin\left(\frac{\sqrt{23}}{4}t\right)$ \\
%b)~ Underdamped \quad c)~ $\gamma = 2\sqrt{6}$
%}

\begin{exercise}
    Consider a mass and spring system with a mass $m=3$, spring constant $k=12$, and damping constant $\gamma=12$.
    
    Set up and find the general solution of the system. [Use $A$ and $B$ for your arbitrary constants]
    \[
        y = \answer{Ae^{-2t} + Bte^{-2t}}
    \]
    \begin{problem}
        The system is...
        \begin{multipleChoice}
            \choice{Underdamped}
            \choice{Overdamped}
            \choice[correct]{Critically damped}
        \end{multipleChoice}
%        \begin{problem}
%            Find a $\gamma$ that makes the system critically damped: $\gamma = \answer{2\sqrt{6}}$.
%        \end{problem}
    \end{problem}
\end{exercise}
%\comboSol
%{%
%a)~ $C_1e^{-2t} + C_2te^{-2t}$ \quad b)~ Critically Damped
%}

\begin{exercise} \label{mv:exwt1}
    Using the mks units (meters-kilograms-seconds), suppose you have a spring with spring constant \unitfrac[4]{N}{m}. You want to use it to weigh items.  Assume no friction.  You place the mass on the spring and put it in motion. 
    
    You count and find that the frequency is \unit[0.8]{Hz} (cycles per second).  What is the mass? (Be careful with the units here, the frequency is given in cycles per second, not radians per second.)
    $\answer{6.317}$kg [Enter to 3 decimal places]
    \begin{problem}
        Find a formula for the mass $m$ given the frequency $\omega$ in \unit{Hz}. $m = \answer{\frac{(2\pi \omega)^2}{4}}$
    \end{problem}
\end{exercise}
%\comboSol
%{%
%a)~ 6.317 kg \quad b)~ $m = \frac{(2\pi \omega)^2}{4}$.
%}

\begin{exercise}%
    A mass of $2$ kilograms is on a spring with spring constant $k$ newtons per meter with no damping.  Suppose the system is at rest and at time $t=0$ the mass is kicked and starts traveling at 2 meters per second.  What's the minimum that $k$ has to be so that the mass does not go further than 3 meters from the rest position? $k = \answer{\frac{8}{9}}$
\end{exercise}
%\exsol{%
%$k=\frac{8}{9}$ (and larger)
%}

\begin{exercise}
    Suppose we add possible friction to problem \ref{mv:exwt1} above. Further, suppose you do not know the spring constant, but you have two reference weights \unit[1]{kg} and \unit[2]{kg} to calibrate your setup. You put each in motion on your spring and measure the quasi-frequency.  For the \unit[1]{kg} weight you measured \unit[1.1]{Hz}, for the \unit[2]{kg} weight you measured \unit[0.8]{Hz}.
    
    Find $k$ (spring constant) and $\gamma$ (damping constant). $k = \answer{5.4\pi^2}$, $\gamma = \answer{\pi\sqrt{2.24}}$
    \begin{problem}
        Find a formula for the mass in terms of the frequency in Hz.  \emph{Note that there may be more than one possible mass for a given frequency.} [Use $H$ for the frequency in Hz] $m = \answer{\frac{21.6 \pm \sqrt{466.56 - 143.36H^2}}{32H^2}}$
        \begin{problem}
            For an unknown object you measured \unit[0.2]{Hz}, what is the mass of the object?  Suppose that you know that the mass of the unknown object is more than a kilogram. $\answer{33.65}$kg
        \end{problem}
    \end{problem}
\end{exercise}
%\comboSol
%{%
%a)~$k =5.4\pi^2$, $\gamma=\pi\sqrt{2.24}$ \quad
%b)~ $m = \frac{21.6 \pm \sqrt{466.56 - 143.36H^2}}{32H^2}$, frequency is $H$ Hz \quad
%c)~ 33.65 kg
%}

\begin{exercise}
    Suppose you wish to measure the friction a mass of \unit[0.1]{kg} experiences as it slides along a floor (you wish to find $\gamma$).  You have a spring with spring constant $k=\unitfrac[5]{N}{m}$.  You take the spring, you attach it to the mass and fix it to a wall.  Then you pull on the spring and let the mass go.  You find that the mass oscillates with quasi-frequency \unit[1]{Hz}. What is the friction? [Enter answer to 4 decimal digits] $\gamma = \answer{0.6487}$
\end{exercise}
%\comboSol
%{%
%$\gamma = .6487$
%}

\begin{exercise}%
    A \unit[5000]{kg} railcar hits a bumper (a spring) at \unitfrac[1]{m}{s}, and the spring compresses by \unit[0.1]{m}.  Assume no damping.
    
    Find $k$. $k=\answer{500000}$
    \begin{problem}
        How far does the spring compress when a \unit[10000]{kg} railcar hits the spring at the same speed? $\answer{\frac{1}{5\sqrt{2}}}$
        \begin{problem}
            If the spring would break if it compresses further than \unit[0.3]{m}, what is the maximum mass of a railcar that can hit it at \unitfrac[1]{m}{s}? $\answer{45000}$kg
            \begin{problem}
                What is the maximum mass of a railcar that can hit the spring without breaking at \unitfrac[2]{m}{s}? $\answer{11250}$kg
            \end{problem}
        \end{problem}
    \end{problem}
\end{exercise}
%\exsol{%
%a) $k=500000$
%\quad
%b) $\frac{1}{5\sqrt{2}} \approx 0.141$
%\quad
%c) \unit[45000]{kg}
%\quad
%d) \unit[11250]{kg}
%}

\begin{exercise}
    When attached to a spring, a \unit[2]{kg} mass stretches the spring by \unit[0.49]{m}. 
    
    What is the spring constant of this spring? Use \unitfrac[9.8]{m}{$s^2$} as the gravity constant. $k = \answer{40}$ N/m
    \begin{problem}
        This mass is allowed to come to rest, lifted up by \unit[0.4]{m} and then released. If there is no damping, set up and solve an initial value problem for the position of the mass as a function of time. $y = \answer{0.4 \cos(\sqrt{20}t)}$
        \begin{problem}
            For a next experiment, you attach a dampener of coefficient \unitfrac[16]{Ns}{m} to the system, and give the same initial condition. Set up and solve an initial value problem for the position of the mass. $y = \answer{0.4 e^{-4t}\cos(2t) + 0.8e^{-4t}\sin(2t)}$
            \begin{problem}
                What type of ``dampening'' would be used to characterize this situation?
                \begin{multipleChoice}
                    \choice[correct]{Underdamped}
                    \choice{Overdamped}
                    \choice{Critically damped}
                \end{multipleChoice}
            \end{problem}
        \end{problem}
    \end{problem}
\end{exercise}
%\comboSol
%{%
%a)~$k=40$ N/m \quad b)~$y = 0.4 \cos(\sqrt{20}t)$ \\
%c)~ $y = 0.4 e^{-4t}\cos(2t) + 0.8e^{-4t}\sin(2t)$, Underdamped
%}

\begin{exercise}%
    A mass of $m$ \unit{kg} is on a spring with $k=\unitfrac[3]{N}{m}$ and $c=\unitfrac[2]{Ns}{m}$.  Find the mass $m_0$ for which there is critical damping. $m_0 = \answer{\frac{1}{3}}$
    \begin{problem}
        If $m < m_0$, does the system oscillate or not, that is, is it underdamped or overdamped?
        \begin{multipleChoice}
            \choice{Underdamped}
            \choice[correct]{Overdamped}
        \end{multipleChoice}
    \end{problem}
\end{exercise}
%\exsol{%
%$m_0 = \frac{1}{3}$.  If $m < m_0$, then the system is overdamped and will
%not oscillate.
%}

\begin{exercise}%
    Suppose we have an RLC circuit with a resistor of 100 milliohms (0.1 ohms), inductor of inductance of 50 millihenries (0.05 henries), and a capacitor of 5 farads, with constant voltage.
    
    Set up the ODE equation for the current $I$. $\answer{0.05}I'' + \answer{0.1}I' + \answer{\frac{1}{5}} I = 0$
    \begin{problem}
        Find the general solution. [Use $A$ and $B$ as arbitrary constants]
        \[
            I = \answer{Ae^{-t}\cos(\sqrt{3} t) + Be^{-t}\sin(\sqrt{3} t)}
        \]
        \begin{problem}
            Solve for $I(0) = 10$ and $I'(0) = 0$. $I = \answer{10 e^{-t} \cos(\sqrt{3}  t) + \frac{10}{\sqrt{3}} e^{-t} \sin(\sqrt{3}  t)}$
        \end{problem}
    \end{problem}
\end{exercise}
%\exsol{%
%a) $0.05 I'' + 0.1 I' + (\frac{1}{5}) I = 0$
%\quad
%b) $I = C e^{-t} \cos(\sqrt{3}  t - \gamma)$ or $I = C_1e^{-t}\cos(\sqrt{3} t) + C_2e^{-t}\sin(\sqrt{3} t)$
%\quad
%c) $I = 10 e^{-t} \cos(\sqrt{3}  t) + \frac{10}{\sqrt{3}} e^{-t}
%\sin(\sqrt{3}  t)$
%}

\begin{exercise}
    For RLC circuits, we can use either charge or current to set up the equation. Let's see how the two of those compare.
    
    Assume that we have an RLC circuit with a 30 millihenry inductor, a 120 milliohm resistor, and a capacitor with capacitance $\frac{20}{3}$ F. Set up a differential equation for the charge on the capacitor as a function of time.
    $\answer{0.03}Q'' + \answer{0.12}Q' + \answer{\frac{3}{20}}Q = 0$
    \begin{problem} 
        Use the same circuit to set up a differential equation for the current through the circuit as a function of time. How do these equations relate? $\answer{0.03}I'' + \answer{0.12}I' + \answer{\frac{3}{20}}I = 0$
        \begin{problem}
            Find the general solution to each of these equations. [Use $A$ and $B$ for arbitrary constants] $Q(t) = \answer{Ae^{-2t}\cos(t) + Be^{-2t}\sin(t)}$
            \begin{problem}
                Solve the initial value problem for the charge with $Q(0) = \frac{1}{2} C$ and $Q'(0) = 0$. $Q(t) = \answer{\frac{1}{2}e^{-2t}\cos(t) + e^{-2t}\sin(t)}$
                \begin{problem}
                    Using the fact that $I = Q'$, determine the appropriate initial conditions needed for $I$ in order to solve for the current in this same setup (with those initial values for charge). $I(0) = \answer{0}$, $I'(0) = \answer{-\frac{5}{2}}$
                    \begin{problem}
                        Now, we'll do the same in the other direction. Solve the initial value problem for current with $I(0) = 2 A$ and $I'(0) = 1 \unitfrac{A}{s}$, and see what the initial conditions would be for $Q(t)$ for this setup.
                        $I(t) = \answer{2e^{-2t}\cos(t) + 5e^{-2t}\sin(t)}$, $Q(0) = \answer{-\frac{9}{5}}$, $Q'(0) = \answer{2}$
                    \end{problem}
                \end{problem}
            \end{problem}
        \end{problem}
    \end{problem}
\end{exercise}
%\comboSol
%{%
%a)~$0.03Q'' + 0.12Q' + \frac{3}{20}Q = 0$ \quad
%b)~$0.03I'' + 0.12I' + \frac{3}{20}I = 0$ \quad
%c)~$Q(t) = C_1e^{-2t}\cos(t) + C_2e^{-2t}\sin(t)$ \quad
%d)~$Q(t) = \frac{1}{2}e^{-2t}\cos(t) + e^{-2t}\sin(t)$ \quad
%e)~$I(0) = 0$, $I'(0) = -\frac{5}{2}$ \quad
%f)~$I(t) = 2e^{-2t}\cos(t) + 5e^{-2t}\sin(t)$, $Q(0) = -\frac{9}{5}$, $Q'(0) = 2$
%}

\begin{exercise}
    Assume that the system $my'' + \gamma y' + ky = 0$ is either critically or overdamped. Prove that the solution can pass through zero \emph{at most once}, regardless of initial conditions. \emph{Hint:} Try to find all values of $t$ for which $y(t) = 0$, given the form of the solution. [Use $A$ and $B$ for arbitrary constants]
    \begin{itemize}
        \item For overdamped system: $t = \answer{\frac{1}{r_2 - r_1}\ln\left(-\frac{A}{B}\right)}$
        \item For Critically damped system: $t = \answer{-\frac{A}{B}}$
    \end{itemize}
\end{exercise}
%\comboSol
%{%
%Overdamped: $t = \frac{1}{r_2 - r_1}\ln\left(-\frac{C_1}{C_2}\right)$, Critically Damped $t = -\frac{C_1}{C_2}$
%}
%
%\setcounter{exercise}{100}

\end{document}