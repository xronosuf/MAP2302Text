\documentclass{ximera}

\title{Practice for Second Order Linear ODEs}

%\auor{Matthew Charnley and Jason Nowell}
\usepackage[margin=1.5cm]{geometry}
\usepackage{indentfirst}
\usepackage{sagetex}
\usepackage{lipsum}
\usepackage{amsmath}
\usepackage{mathrsfs}
\usepackage{tikz}
\usetikzlibrary{matrix}

%%% Random packages added without verifying what they are really doing - just to get initial compile to work.
\usepackage{tcolorbox}
\usepackage{hypcap}
\usepackage{booktabs}%% To get \toprule,\midrule,\bottomrule etc.
\usepackage{caption}
\usepackage{units}
\usepackage{multicol}
\usepackage{hhline}


% This is my modified wrapfig that doesn't use intextsep
\usepackage{mywrapfig}
\usepackage{import}



%%% End to random added packages.


\graphicspath{
    {./}
    {./figures/}
    {./../figures/}
    {./../../figures/}
}
\renewcommand{\log}{\ln}%%%%
\DeclareMathOperator{\arcsec}{arcsec}
%% New commands


%%%%%%%%%%%%%%%%%%%%
% New Conditionals %
%%%%%%%%%%%%%%%%%%%%


% referencing
\makeatletter
    \DeclareRobustCommand{\myvref}[2]{%
      \leavevmode%
      \begingroup
        \let\T@pageref\@pagerefstar
        \hyperref[{#2}]{%
	  #1~\ref*{#2}%
        }%
        \vpageref[\unskip]{#2}%
      \endgroup
    }%

    \DeclareRobustCommand{\myref}[2]{%
      \leavevmode%
      \begingroup
        \let\T@pageref\@pagerefstar
        \hyperref[{#2}]{%
	  #1~\ref*{#2}%
        }%
      \endgroup
    }%
\makeatother

\newcommand{\figurevref}[1]{\myvref{Figure}{#1}}
\newcommand{\figureref}[1]{\myref{Figure}{#1}}
\newcommand{\tablevref}[1]{\myvref{Table}{#1}}
\newcommand{\tableref}[1]{\myref{Table}{#1}}
\newcommand{\chapterref}[1]{\myref{chapter}{#1}}
\newcommand{\Chapterref}[1]{\myref{Chapter}{#1}}
\newcommand{\appendixref}[1]{\myref{appendix}{#1}}
\newcommand{\Appendixref}[1]{\myref{Appendix}{#1}}
\newcommand{\sectionref}[1]{\myref{\S}{#1}}
\newcommand{\subsectionref}[1]{\myref{subsection}{#1}}
\newcommand{\subsectionvref}[1]{\myvref{subsection}{#1}}
\newcommand{\exercisevref}[1]{\myvref{Exercise}{#1}}
\newcommand{\exerciseref}[1]{\myref{Exercise}{#1}}
\newcommand{\examplevref}[1]{\myvref{Example}{#1}}
\newcommand{\exampleref}[1]{\myref{Example}{#1}}
\newcommand{\thmvref}[1]{\myvref{Theorem}{#1}}
\newcommand{\thmref}[1]{\myref{Theorem}{#1}}


\renewcommand{\exampleref}[1]{ {\color{red} \bfseries Normally a reference to a previous example goes here.}}
\renewcommand{\examplevref}[1]{ {\color{red} \bfseries Normally a reference to a previous example goes here.}}
\renewcommand{\figurevref}[1]{ {\color{red} \bfseries Normally a reference to a previous figure goes here.}}
\renewcommand{\tablevref}[1]{ {\color{red} \bfseries Normally a reference to a previous table goes here.}}
\renewcommand{\Appendixref}[1]{ {\color{red} \bfseries Normally a reference to an Appendix goes here.}}
\renewcommand{\exercisevref}[1]{ {\color{red} \bfseries Normally a reference to a previous exercise goes here.}}
\renewcommand{\thmvref}[1]{ {\color{red} \bfseries Normally a reference to a previous theorem goes here.}}
\renewcommand{\subsectionvref}[1]{ {\color{red} \bfseries Normally a reference to a previous subsection goes here.}}



\newcommand{\R}{\mathbb{R}}
\newcommand{\C}{\mathbb{C}}

%% Example Solution Env.
\def\beginSolclaim{\par\addvspace{\medskipamount}\noindent\hbox{\bf Solution:}\hspace{0.5em}\ignorespaces}
\def\endSolclaim{\par\addvspace{-1em}\hfill\rule{1em}{0.4pt}\hspace{-0.4pt}\rule{0.4pt}{1em}\par\addvspace{\medskipamount}}
\newenvironment{exampleSol}[1][]{\beginSolclaim}{\endSolclaim}

%% General figure formating from original book.
\newcommand{\mybeginframe}{%
\begin{tcolorbox}[colback=white,colframe=lightgray,left=5pt,right=5pt]%
}
\newcommand{\myendframe}{%
\end{tcolorbox}%
}

%%% Eventually return and fix this to make matlab code work correctly.
%% Define the matlab environment as another code environment
%\NewEnviron{matlab}{ {\centering\bfseries MATLAB Code} \\ \noexpand{\BODY} }
%\let\beginmatlab\begincode
%\let\endmatlab\endcode
%\newenvironment{matlab}{% Begin Environment Code
%\begin{minipage}{\linewidth}
%\begin{verbatim}
%}% End of Begin Environment Code
%{% Start of End Environment Code
%\end{verbatim}
%\end{minipage}
%}% End of End Environment Code


% this one should have a caption, first argument is the size
\newenvironment{mywrapfig}[2][]{
 \wrapfigure[#1]{r}{#2}
 \mybeginframe
 \centering
}{%
 \myendframe
 \endwrapfigure
}

% this one has no caption, first argument is size,
% the second argument is a larger size used for HTML (ignored by latex)
\newenvironment{mywrapfigsimp}[3][]{%
 \wrapfigure[#1]{r}{#2}%
 \centering%
}{%
 \endwrapfigure%
}
\newenvironment{myfig}
    {%
    \begin{figure}[h!t]
        \mybeginframe%
        \centering%
    }
    {%
        \myendframe
    \end{figure}%
    }


% graphics include
\newcommand{\diffyincludegraphics}[3]{\includegraphics[#1]{#3}}
\newcommand{\myincludegraphics}[3]{\includegraphics[#1]{#3}}
\newcommand{\inputpdft}[1]{\subimport*{../figures/}{#1.pdf_t}}


%% Not sure what these even do? They don't seem to actually work... fun!
%\newcommand{\mybxbg}[1]{\tcboxmath[colback=white,colframe=black,boxrule=0.5pt,top=1.5pt,bottom=1.5pt]{#1}}
%\newcommand{\mybxsm}[1]{\tcboxmath[colback=white,colframe=black,boxrule=0.5pt,left=0pt,right=0pt,top=0pt,bottom=0pt]{#1}}
\newcommand{\mybxsm}[1]{#1}
\newcommand{\mybxbg}[1]{#1}

%%% Something about tasks for practice/hw?
\usepackage{tasks}
\usepackage{footnote}
\makesavenoteenv{tasks}


%% For pdf only?
\newcommand{\diffypdfversion}[1]{#1}


%% Kill ``cite'' and go back later to fix it.
\renewcommand{\cite}[1]{}


%% Currently we can't really use index or its derivatives. So we are gonna kill them off.
\renewcommand{\index}[1]{}
\newcommand{\myindex}[1]{#1}







\begin{document}
\begin{abstract}
    Why?
\end{abstract}
\maketitle


\begin{exercise}
    How can you show that $y=e^x$ and $y=e^{2x}$ are linearly independent?
    \begin{multipleChoice}
        \choice{Take their derivative}
        \choice{Find their indefinite integral}
        \choice[correct]{Find their ratio and show it is non-constant.}
        \choice{Find their ratio and show it is constant.}
    \end{multipleChoice}
\end{exercise}
%\comboSol
%{%
%Ratio is non-constant.
%}

\begin{exercise}%
    Are $\sin(x)$ and $e^x$ linearly independent?  Justify.
    \begin{multipleChoice}
        \choice[correct]{Yes, since $e^x$ is monotonic and $\sin(x)$ is periodic.}
        \choice{Yes, since they are both transcendental functions.}
        \choice{No, since they are both transcendental functions.}
        \choice{No, since you can find a constant, $A$, such that $\sin(x)=Ae^x$}
    \end{multipleChoice}
\end{exercise}
%\exsol{%
%Yes.  To justify try to find a constant $A$ such that $\sin(x) = A e^x$
%for all $x$.
%}


\begin{exercise}%
    Are $e^x$ and $e^{x+2}$ linearly independent?  Justify.
    \begin{multipleChoice}
        \choice{Yes, since - if we plug in $x$ values, we get different numbers for each function.}
        \choice{Yes, since they are both transcendental functions.}
        \choice[correct]{No, since they are equal.}
        \choice{No, since they are both transcendental functions.}
        \choice{No, since you can find a constant, $A$, such that $e^{x+2}=A + e^x$}
    \end{multipleChoice}
\end{exercise}
%\exsol{%
%No.  $e^{x+2} = e^2 e^x$.
%}

\begin{exercise}%
    Guess a solution to $y'' + y' + y= 5$. $y = \answer{5}$
\end{exercise}
%\exsol{%
%$y=5$
%}

\begin{exercise}
    Take $y'' + 5 y = 10 x + 5$.  Find (guess!) a solution. [Make sure to enter the full equality] $\answer{y = 2x+1}$ 
\end{exercise}
%\comboSol
%{%
%$y = 2x+1$
%}

\begin{exercise}
    Verify that $y_1(t) = e^t \cos(2t)$ and $y_2(t) = e^t \sin(2t)$ both solve $y'' - 2y' + 5y = 0$. 
    
    \begin{itemize}
        \item $y_1' = \answer{-2e^t\sin(2t) + e^t\cos(2t)}$
        \item $y_2' = \answer{2e^t\cos(2t) + e^t\sin(2t)}$
        \item $y_1'' = \answer{-e^t (4 \sin(2t) + 3 \cos(2t))}$
        \item $y_2'' = \answer{e^t (4 \cos(2t) - 3 \sin(2t))}$
    \end{itemize}
    
    \begin{problem}
        Are these two solutions linearly independent? 
        \begin{multipleChoice}
            \choice[correct]{Yes.}
            \choice{No.}
        \end{multipleChoice}
        \begin{problem}
            Using the above, what is the general solution to $y'' - 2y' + 5y = 0$? [Use $A$ and $B$ as your arbitrary constants] $\answer{Ae^{t}\cos(2t) + B\sin(2t)}$
        \end{problem}
    \end{problem}
\end{exercise}
%\comboSol
%{%
%Yes. This means the general solution is $C_1e^{t}\cos(2t) + C_2\sin(2t)$.
%}

%% Not sure how to validate this question.
\begin{exercise}
    Prove the superposition principle for nonhomogeneous equations.  Suppose that $y_1$ is a solution to $L y_1 = f(x)$ and $y_2$ is a solution to $L y_2 = g(x)$ (same linear operator $L$).  Show that $y = y_1+y_2$ solves $Ly = f(x) + g(x)$.
\end{exercise}
%\comboSol
%{%
%Hint: Plug in $y_1 + y_2$ and see that it works correctly.
%}

\begin{exercise}
    Determine the maximal interval of existence of the solution to the differential equation
    \[ 
        (t - 5)y'' + \frac{1}{t+1}y' + e^t y = \frac{\cos(t)}{t^2 + 1} 
    \] 
    with initial condition $y(3) = 8$. $\left( \answer{-1},\answer{5} \right)$
    
    \begin{problem}
        What about if the initial condition is $y(-3) = 4$? $\left( \answer{-\infty},\answer{-1} \right)$
    \end{problem}
\end{exercise}
%\comboSol
%{%
%$(-1, 5)$ for $y(3) = 8$. $(-\infty, -1)$ for $y(-3) = 4$.
%}

\begin{exercise}
    For the equation $x^2 y'' - x y' = 0$, find two solutions, show that they are linearly independent and find the general solution.
    Hint: Try $y = x^r$.
    
    The general solution is [use $A$ and $B$ for arbitrary constants] $\answer{A + Bx^2}$
\end{exercise}
%\comboSol
%{%
%$y = C_1 + C_2x^2$
%}

\begin{exercise}%
    Find the general solution to $x y'' + y' = 0$.  Hint: It is a first order ODE in $y'$. [Use $A$ and $B$ for the arbitrary constants] $y = \answer{A\ln(x) + B}$.
\end{exercise}
%\exsol{%
%$y=C_1 \ln(x) + C_2$
%}

\begin{exercise}
    Find the general solution of $2y'' + 2y' -4 y = 0$. [Use $A$ and $B$ for the arbitrary constants] $y = \answer{Ae^{-2t} + Be^{t}}$.
\end{exercise}
%\comboSol
%{%
%$y = C_1e^{-2t} + C_2e^{t}$
%}

\begin{exercise}
    Solve $y'' + 9y' = 0$ with $y(0) = 1$, $y'(0) = 1$. $y = \answer{\frac{10}{9} - \frac{1}{9}e^{-9t}}$
\end{exercise}
%\comboSol
%{%
%$y = \frac{10}{9} - \frac{1}{9}e^{-9t}$
%}
\begin{exercise}
    Find the general solution of $y'' + 9y' - 10 y = 0$. [Use $A$ and $B$ for the arbitrary constants] $y = \answer{Ae^{-10t} + Be^t}$.
\end{exercise}
%\comboSol
%{%
%$y = C_1e^{-10t} + C_2e^t$
%}

\begin{exercise}
    Find the general solution to $y'' - 3y' - 4y = 0$. [Use $A$ and $B$ for the arbitrary constants] $y = \answer{Ae^{4t} + Be^{-t}}$.
\end{exercise}
%\comboSol
%{%
%$y = C_1e^{4t} + C_2e^{-t}$
%}

\begin{exercise}
    Find the general solution to $y'' + 6y' + 8y = 0$. [Use $A$ and $B$ for the arbitrary constants] $y = \answer{Ae^{-2t} + Be^{-4t}}$.
\end{exercise}
%\comboSol
%{%
%$y = C_1e^{-2t} + C_2e^{-4t}$
%}

\begin{exercise}
    Find the solution to $y'' - 3y' + 2y = 0$ with $y(0) = 3$ and $y'(0) = -1$. $y = \answer{7e^t - 4e^{2t}}$
\end{exercise}
%\comboSol
%{%
%$y = 7e^t - 4e^{2t}$
%}

\begin{exercise}
    Find the solution to $y'' + y' -12y = 0$ with $y(0) = 1$ and $y'(0) = -2$. $y = \answer{\frac{5}{7}e^{-4t} + \frac{2}{7}e^{3t}}$
\end{exercise}
%\comboSol
%{%
%$y = \frac{5}{7}e^{-4t} + \frac{2}{7}e^{3t}$
%}
\begin{exercise}%
    Find the general solution to $y''+4y'+2y=0$. [Use $A$ and $B$ for the arbitrary constants] $y = \answer{A e^{(-2+\sqrt{2}) x} + B e^{(-2-\sqrt{2}) x}}$.
\end{exercise}
%\exsol{%
%$y =
%C_1 e^{(-2+\sqrt{2}) x}
%+
%C_2 e^{(-2-\sqrt{2}) x}$
%}

\begin{exercise}%
    Find the solution to $2y''+y'-3y=0$, $y(0) = a$, $y'(0)=b$. $y = \answer{\frac{2(a-b)}{5} e^{-3x/2}+\frac{3 a+2 b}{5} e^x}$
\end{exercise}
%\exsol{%
%$y = \frac{2(a-b)}{5} \, e^{-3x/2}+\frac{3 a+2 b}{5} \, e^x$
%}

\begin{exercise}%
    Find the solution to $y''-(\alpha+\beta) y' + \alpha \beta y=0$, $y(0) = a$, $y'(0)=b$, where $\alpha$, $\beta$, $a$, and $b$ are real numbers, and $\alpha \neq \beta$. 
    
    \[
        y = \answer{\frac{a \beta-b}{\beta-\alpha} e^{\alpha x} + \frac{b-a \alpha}{\beta-\alpha} e^{\beta x}}
    \]
\end{exercise}
%\exsol{%
%$y =
%\frac{a \beta-b}{\beta-\alpha} e^{\alpha x} + 
%\frac{b-a \alpha}{\beta-\alpha} e^{\beta x}$
%}

\begin{exercise}%
    Write down an equation (guess) for which we have the solutions $e^x$ and $e^{2x}$.  Hint: Try an equation of the form $y''+Ay'+By = 0$ for constants $A$ and $B$, plug in both $e^x$ and $e^{2x}$ and solve for $A$ and $B$. [Use $\frac{d^2y}{dx^2}$ notation for $y''$ and $\frac{dy}{dx}$ notation for $y'$. Make sure to enter a full equality for your answer]. $\answer{\frac{d^2y}{dx^2} - 3\frac{dy}{dx} + 2y = 0}$
\end{exercise}
%\exsol{%
%$y''-3y'+2y = 0$
%}

\begin{exercise}%
    Construct an equation such that $y = C_1 e^{3x} + C_2 e^{-2x}$ is the general solution. [Use $\frac{d^2y}{dx^2}$ notation for $y''$ and $\frac{dy}{dx}$ notation for $y'$. Make sure to enter a full equality for your answer]. $\answer{\frac{d^2y}{dx^2} - \frac{dy}{dx} - 6y = 0}$
\end{exercise}
%\exsol{%
%$y'' -y'-6y=0$
%}

\begin{exercise}
    Give an example of a 2nd-order DE whose general solution is $y=c_1e^{-2t}+c_2e^{-4t}$. [Use $\frac{d^2y}{dx^2}$ notation for $y''$ and $\frac{dy}{dx}$ notation for $y'$. Make sure to enter a full equality for your answer]. $\answer{\frac{d^2y}{dx^2} + 6\frac{dy}{dx} + 8y = 0}$
\end{exercise}
%\comboSol
%{%
%$y'' + 6y' + 8y = 0$
%}


Equations of the form $a x^2 y'' + b x y' + c y = 0$ are called \emph{Euler's equations} or \emph{Cauchy--Euler equations}. They are solved by trying $y=x^r$ and solving for $r$ (assume that $x \geq 0$ for simplicity).

\begin{exercise} \label{sol:eulerex}
    Suppose that ${(b-a)}^2-4ac > 0$.
    Find a formula for the general solution of $a x^2 y'' + b x y' + c y = 0$.  Hint: Try $y=x^r$ and find a formula for $r$. 
    \begin{itemize}
        \item $r_1 = \answer{\frac{(a-b) + \sqrt{(b-a)^2 - 4ac}}{2a}}$,
        \item $r_2 = \answer{\frac{(a-b) - \sqrt{(b-a)^2 - 4ac}}{2a}}$
        \item General Solution [Use $A$ and $B$ as your arbitrary constants]: $y = \answer{Ax^{r_1} + Bx^{r_2}}$
    \end{itemize}
    \begin{problem}
        What happens when ${(b-a)}^2-4ac = 0$ 
        \begin{multipleChoice}
            \choice{There are two distinct real solutions.}
            \choice[correct]{There is one repeated real solution.}
            \choice{There are only complex solutions.}
        \end{multipleChoice}
        \begin{problem}
            What happens when ${(b-a)}^2-4ac < 0$?
            \begin{multipleChoice}
                \choice{There are two distinct real solutions.}
                \choice{There is one repeated real solution.}
                \choice[correct]{There are only complex solutions.}
            \end{multipleChoice}
            \begin{feedback}[correct]
                We will revisit the case when ${(b-a)}^2-4ac < 0$ later.
            \end{feedback}
        \end{problem}
    \end{problem}
\end{exercise}
%\comboSol
%{%
%a)~$r_1 = \frac{(a-b) + \sqrt{(b-a)^2 - 4ac}}{2a}$, $r_2 = \frac{(a-b) - \sqrt{(b-a)^2 - 4ac}}{2a}$, General Solution: $y = C_1x^{r_1} + C_2x^{r_2}$. \\
%b)~ Only one solution or complex solutions.
%}



\begin{exercise} \label{sol:eulerexln}
    Same equation as in the previous problem. Suppose ${(b-a)}^2-4ac = 0$.  Find a formula for the general solution of $a x^2 y'' + b x y' + c y = 0$.  Hint: Try $y=x^r \ln x$ for the second solution. [Use $A$ and $B$ for your arbitrary constants] $y = \answer{Ax^{\frac{a-b}{2a}} + Bx^{\frac{a-b}{2a}}ln(x)}$
\end{exercise}
%\comboSol
%{%
%$y = C_1x^{\frac{a-b}{2a}} + C_2x^{\frac{a-b}{2a}}ln(x)$
%}


%\setcounter{exercise}{100}



\end{document}