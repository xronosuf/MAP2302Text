\documentclass{ximera}
%\auor{Matthew Charnley and Jason Nowell}
\usepackage[margin=1.5cm]{geometry}
\usepackage{indentfirst}
\usepackage{sagetex}
\usepackage{lipsum}
\usepackage{amsmath}
\usepackage{mathrsfs}


%%% Random packages added without verifying what they are really doing - just to get initial compile to work.
\usepackage{tcolorbox}
\usepackage{hypcap}
\usepackage{booktabs}%% To get \toprule,\midrule,\bottomrule etc.
\usepackage{nicefrac}
\usepackage{caption}
\usepackage{units}

% This is my modified wrapfig that doesn't use intextsep
\usepackage{mywrapfig}
\usepackage{import}



%%% End to random added packages.


\graphicspath{
    {./figures/}
    {./../figures/}
    {./../../figures/}
}
\renewcommand{\log}{\ln}%%%%
\DeclareMathOperator{\arcsec}{arcsec}
%% New commands


%%%%%%%%%%%%%%%%%%%%
% New Conditionals %
%%%%%%%%%%%%%%%%%%%%


% referencing
\makeatletter
    \DeclareRobustCommand{\myvref}[2]{%
      \leavevmode%
      \begingroup
        \let\T@pageref\@pagerefstar
        \hyperref[{#2}]{%
	  #1~\ref*{#2}%
        }%
        \vpageref[\unskip]{#2}%
      \endgroup
    }%

    \DeclareRobustCommand{\myref}[2]{%
      \leavevmode%
      \begingroup
        \let\T@pageref\@pagerefstar
        \hyperref[{#2}]{%
	  #1~\ref*{#2}%
        }%
      \endgroup
    }%
\makeatother

\newcommand{\figurevref}[1]{\myvref{Figure}{#1}}
\newcommand{\figureref}[1]{\myref{Figure}{#1}}
\newcommand{\tablevref}[1]{\myvref{Table}{#1}}
\newcommand{\tableref}[1]{\myref{Table}{#1}}
\newcommand{\chapterref}[1]{\myref{chapter}{#1}}
\newcommand{\Chapterref}[1]{\myref{Chapter}{#1}}
\newcommand{\appendixref}[1]{\myref{appendix}{#1}}
\newcommand{\Appendixref}[1]{\myref{Appendix}{#1}}
\newcommand{\sectionref}[1]{\myref{\S}{#1}}
\newcommand{\subsectionref}[1]{\myref{subsection}{#1}}
\newcommand{\subsectionvref}[1]{\myvref{subsection}{#1}}
\newcommand{\exercisevref}[1]{\myvref{Exercise}{#1}}
\newcommand{\exerciseref}[1]{\myref{Exercise}{#1}}
\newcommand{\examplevref}[1]{\myvref{Example}{#1}}
\newcommand{\exampleref}[1]{\myref{Example}{#1}}
\newcommand{\thmvref}[1]{\myvref{Theorem}{#1}}
\newcommand{\thmref}[1]{\myref{Theorem}{#1}}


\renewcommand{\exampleref}[1]{ {\color{red} \bfseries Normally a reference to a previous example goes here.}}
\renewcommand{\figurevref}[1]{ {\color{red} \bfseries Normally a reference to a previous figure goes here.}}
\renewcommand{\tablevref}[1]{ {\color{red} \bfseries Normally a reference to a previous table goes here.}}
\renewcommand{\Appendixref}[1]{ {\color{red} \bfseries Normally a reference to an Appendix goes here.}}
\renewcommand{\exercisevref}[1]{ {\color{red} \bfseries Normally a reference to a previous exercise goes here.}}



\newcommand{\R}{\mathbb{R}}

%% Example Solution Env.
\def\beginSolclaim{\par\addvspace{\medskipamount}\noindent\hbox{\bf Solution:}\hspace{0.5em}\ignorespaces}
\def\endSolclaim{\par\addvspace{-1em}\hfill\rule{1em}{0.4pt}\hspace{-0.4pt}\rule{0.4pt}{1em}\par\addvspace{\medskipamount}}
\newenvironment{exampleSol}[1][]{\beginSolclaim}{\endSolclaim}

%% General figure formating from original book.
\newcommand{\mybeginframe}{%
\begin{tcolorbox}[colback=white,colframe=lightgray,left=5pt,right=5pt]%
}
\newcommand{\myendframe}{%
\end{tcolorbox}%
}

%%% Eventually return and fix this to make matlab code work correctly.
%% Define the matlab environment as another code environment
%\newenvironment{matlab}
%{% Begin Environment Code
%{ \centering \bfseries Matlab Code }
%\begin{code}
%}% End of Begin Environment Code
%{% Start of End Environment Code
%\end{code}
%}% End of End Environment Code


% this one should have a caption, first argument is the size
\newenvironment{mywrapfig}[2][]{
 \wrapfigure[#1]{r}{#2}
 \mybeginframe
 \centering
}{%
 \myendframe
 \endwrapfigure
}

% this one has no caption, first argument is size,
% the second argument is a larger size used for HTML (ignored by latex)
\newenvironment{mywrapfigsimp}[3][]{%
 \wrapfigure[#1]{r}{#2}%
 \centering%
}{%
 \endwrapfigure%
}
\newenvironment{myfig}
    {%
    \begin{figure}[h!t]
        \mybeginframe%
        \centering%
    }
    {%
        \myendframe
    \end{figure}%
    }


% graphics include
\newcommand{\diffyincludegraphics}[3]{\includegraphics[#1]{#3}}
\newcommand{\myincludegraphics}[3]{\includegraphics[#1]{#3}}
\newcommand{\inputpdft}[1]{\subimport*{../figures/}{#1.pdf_t}}


%% Not sure what these even do? They don't seem to actually work... fun!
%\newcommand{\mybxbg}[1]{\tcboxmath[colback=white,colframe=black,boxrule=0.5pt,top=1.5pt,bottom=1.5pt]{#1}}
%\newcommand{\mybxsm}[1]{\tcboxmath[colback=white,colframe=black,boxrule=0.5pt,left=0pt,right=0pt,top=0pt,bottom=0pt]{#1}}
\newcommand{\mybxsm}[1]{#1}
\newcommand{\mybxbg}[1]{#1}

%%% Something about tasks for practice/hw?
\usepackage{tasks}
\usepackage{footnote}
\makesavenoteenv{tasks}


%% For pdf only?
\newcommand{\diffypdfversion}[1]{#1}


%% Kill ``cite'' and go back later to fix it.
\renewcommand{\cite}[1]{}


%% Currently we can't really use index or its derivatives. So we are gonna kill them off.
\renewcommand{\index}[1]{}
\newcommand{\myindex}[1]{#1}






\title{Linearization, critical points, and stability}
\author{Matthew Charnley and Jason Nowell}


\outcome{Find critical points of a non-linear system of differential equations}
\outcome{Linearize a non-linear system around a critical point}
\outcome{Determine if a critical point of a non-linear system is isolated}
\outcome{Use the Jacobian matrix to classify the critical point of a non-linear system}
\outcome{Determine the stability of a critical point from the classification.}


\begin{document}
\begin{abstract}
    We discuss Linearization, critical points, and stability
\end{abstract}
\maketitle

\label{linearization:section}



%\subsection{Nonlinear equations}

Except for a few brief detours in \chapterref{fo:chapter}, we considered mostly linear equations.  Linear equations suffice in many applications, but in reality most phenomena require nonlinear equations.  Nonlinear equations, however, are notoriously more difficult to understand than linear ones, and many strange new phenomena appear when we allow our equations to be nonlinear.

Not to worry, we did not waste all this time studying linear equations. Nonlinear equations can often be approximated by linear ones if we only need a solution ``locally,'' for example, only for a short period of time, or only for certain parameters.  Understanding specific linear equations can also give us qualitative understanding about a more general nonlinear problem.  The idea is similar to what you did in calculus in trying to approximate a function by a line with the right slope.

\begin{mywrapfigsimp}{1.45in}{1.75in}
    \noindent
    \input{figures/mv-pend.pdf_t}
\end{mywrapfigsimp}
In \sectionref{sec:mv} we looked at the pendulum of %mass $m$ and
length $L$.  The goal was to solve for the angle $\theta(t)$ as a function of the time $t$.  The equation for the setup is the nonlinear equation
\begin{equation*}
    \theta'' + \frac{g}{L} \sin \theta = 0 .
\end{equation*}
Instead of solving this equation, we solved the rather easier linear equation
\begin{equation*}
    \theta'' + \frac{g}{L} \theta = 0 .
\end{equation*}
While the solution to the linear equation is not exactly what we were looking for, it is rather close to the original, as long as the angle $\theta$ is small and the time period involved is short.

You might ask: Why don't we just solve the nonlinear problem?  Well, it might be very difficult, impractical, or impossible to solve analytically, depending on the equation in question.  We may not even be interested in the actual solution, we might only be interested in some qualitative idea of what the solution is doing.  For example,
%we may be interested in
what happens as time goes to infinity?
%In the case
%of the pendulum we found that it oscillates and we can even approximate
%the period well if the swings are small.
%In other words, why do more work, when we can do less.
%The exact solution, even if found, might be harder to analyze.


\subsection{Autonomous systems and phase plane analysis}

We restrict our attention to a two-dimensional autonomous system
\begin{equation*}
    x' = f(x,y) , \qquad y' = g(x,y) ,
\end{equation*}
where $f(x,y)$ and $g(x,y)$ are functions of two variables, and the derivatives are taken with respect to time $t$.  Solutions are functions $x(t)$ and $y(t)$ such that
\begin{equation*}
    x'(t) = f\bigl(x(t),y(t)\bigr), \qquad y'(t) = g\bigl(x(t),y(t)\bigr) .
\end{equation*}
The way we will analyze the system is very similar to \sectionref{auteq:section}, where we studied a single autonomous equation.  The ideas in two dimensions are the same, but the behavior can be far more complicated.
%We will do the same sort of analysis.  We will
%look for the \emph{critical points} of the system and then we will analyze
%what happens when time goes to infinity.

It may be best to think of the system of equations as the single vector equation
\begin{equation} \label{eq:nlinautn2}
    \begin{bmatrix} 
        x \\ 
        y 
    \end{bmatrix} ' =
    \begin{bmatrix} 
        f(x,y) \\ 
        g(x,y) 
    \end{bmatrix} .
\end{equation}
As in \sectionref{sec:introtosys} we draw the \emph{\myindex{phase portrait}} (or \emph{\myindex{phase diagram}}), where each point $(x,y)$ corresponds to a specific state of the system. We draw the \emph{\myindex{vector field}} given at each point $(x,y)$ by the vector
$\left[ \begin{smallmatrix} f(x,y) \\ g(x,y) \end{smallmatrix} \right]$.
And as before if we find solutions, we draw the trajectories by plotting all points $\bigl(x(t),y(t)\bigr)$ for a certain range of $t$.

\begin{example} \label{example:nlin-1b-example}
    Consider the second order equation $x''=-x+x^2$. Write this equation as a first order nonlinear system
    \begin{equation*}
        x' = y , \qquad y' = -x+x^2 .
    \end{equation*}
    The phase portrait with some trajectories is drawn in \figurevref{fig:nlin-1b}.
    \begin{myfig}
        \capstart
        \diffyincludegraphics{width=3in}{width=4.5in}{nlin-1b}
        \caption{Phase portrait with some trajectories of $x' = y$, $y' = -x+x^2$. \label{fig:nlin-1b}}
    \end{myfig}
    
    From the phase portrait it should be clear that even this simple system has fairly complicated behavior.  Some trajectories keep oscillating around the origin, and some go off towards infinity.  We will return to this example often, and analyze it completely in this (and the next) section.
\end{example}

If we zoom into the diagram near a point where 
$\left[ \begin{smallmatrix} f(x,y) \\ g(x,y) \end{smallmatrix} \right]$ 
is not zero, then nearby the arrows point generally in essentially that same direction and have essentially the same magnitude. In other words the behavior is not that interesting near such a point. We are of course assuming that $f(x,y)$ and $g(x,y)$ are continuous.

Let us concentrate on those points in the phase diagram above where the trajectories seem to start, end, or go around.  We see two such points: $(0,0)$ and $(1,0)$.  The trajectories seem to go around the point $(0,0)$, and they seem to either go in or out of the point $(1,0)$.
%
These points are precisely those points where the derivatives of both $x$ and $y$ are zero.  

\begin{definition} 
    The \emph{critical points}\index{critical point} of a system of differential equations
    \begin{equation*}
        \begin{split}
            \frac{dx}{dt} &= f(x,y) \\
            \frac{dy}{dt} &= g(x,y)
        \end{split}
    \end{equation*}
    are the points $(x,y)$ such that
    \begin{equation*} 
        \begin{bmatrix} 
            f(x,y) \\ 
            g(x,y) 
        \end{bmatrix} = \vec{0} .
    \end{equation*}
    In other words, these are the points where both $f(x,y)=0$ and $g(x,y)=0$.
\end{definition}

The critical points are where the behavior of the system is in some sense the most complicated.  If
$\left[ \begin{smallmatrix} f(x,y) \\ g(x,y) \end{smallmatrix} \right]$
is zero, then nearby, the vector can point in any direction whatsoever. Also, the trajectories are either going towards, away from, or around these points, so if we are looking for long-term qualitative behavior of the system, we should look at what is happening near the critical points.

Critical points are also sometimes called \emph{equilibria}\index{equilibrium}, since we have so-called \emph{equilibrium solutions}\index{equilibrium solution} at critical points. If $(x_0,y_0)$ is a critical point, then we have the solutions
\begin{equation*}
    x(t) = x_0, \quad y(t) = y_0 .
\end{equation*}
In \examplevref{example:nlin-1b-example}, there are two equilibrium solutions:
\begin{equation*}
    x(t) = 0, \quad y(t) = 0, \qquad \text{and} \qquad x(t) = 1, \quad y(t) = 0.
\end{equation*}
The discussion here should seem a bit familiar; it is the same as how we formulated equilibrium solutions to autonomous differential equations in in \sectionref{auteq:section}.  
% The underlying concept is exactly the same.

\subsection{Linearization}

How do linear systems fit into this approach? For a linear, homogeneous system of two variables defined by
\begin{equation*}
    \vec{x}' = A\vec{x}
\end{equation*}
where $A$ is an invertible matrix, the only critical point is the origin $(0,0)$. Since $A$ is invertible, the only vector that satisfies $A\vec{x} = 0$ is $\vec{x} = 0$, see \sectionref{det:section}. (This also applies beyond two variables, but we'll stick to that for simplicity.) In \sectionref{sec:twodimaut} we studied the behavior of a homogeneous linear system of two equations near a critical point. Let us put the understanding we gained in that section to good use understanding what happens near critical points of nonlinear systems.

%Just as
In calculus we learned to estimate a function by taking its derivative and linearizing.  We work similarly with nonlinear systems of ODE.
%The idea is the following procedure.
Suppose $(x_0,y_0)$ is a critical point. In order to linearize the system of differential equations, we want to linearize the two functions $f(x,y)$ and $g(x,y)$ that define this system. To do so, we will replace $f$ and $g$ by the tangent plane approximation to the functions. That is, if we set $z = f(x,y)$, the tangent plane is given by
\[ 
    L_f(x,y) = f(x_0, y_0) + f_x(x_0, y_0)(x - x_0) + f_y(x_0, y_0)(y - y_0). 
\] 
Since $(x_0, y_0)$ is a critical point, we know that $f(x_0, y_0) = 0$, so the tangent plane is given by 
\[ 
    L_f(x,y) = f_x(x_0, y_0)(x - x_0) + f_y(x_0, y_0)(y - y_0). 
\]
Similarly, the tangent plane for $g(x,y)$ near the critical point $(x_0, y_0)$ is given by 
\[ 
    L_g(x,y) = g_x(x_0, y_0)(x - x_0) + g_y(x_0, y_0)(y - y_0). 
\]

The idea of linearization in calculus was that we could use the tangent line or tangent plane to approximate a function near to a given point. For systems of differential equations, the idea is that we can approximate the solutions to the system of differential equations by the solutions to the linearized systems as long as we stay near the critical point. That means that we can approximate the solution to
\[
    \begin{split}
        \frac{dx}{dt} &= f(x,y) \\
        \frac{dy}{dt} &= g(x,y)
    \end{split}
\]
near the critical point $(x_0, y_0)$ by the solution to the system
\[
    \begin{split}
        \frac{dx}{dt} &= f_x(x_0, y_0)(x - x_0) + f_y(x_0, y_0)(y - y_0) \\
        \frac{dy}{dt} &= g_x(x_0, y_0)(x - x_0) + g_y(x_0, y_0)(y - y_0)
    \end{split}
\]

Next, change variables to $(u,v)$, so that $(u,v)=(0,0)$ corresponds to $(x_0,y_0)$.  That is,
\begin{equation*}
    u=x-x_0, \qquad v=y-y_0,
\end{equation*}
which is not going to affect our differential equations because $x_0$ and $y_0$ are constant. 

Since $\frac{dx}{dt} = \frac{du}{dt}$ and $\frac{dy}{dt} = \frac{dv}{dt}$, we can rewrite the approximation system as 
\[
    \begin{split}
        \frac{du}{dt} &= f_x(x_0, y_0)u + f_y(x_0, y_0)v \\
        \frac{dv}{dt} &= g_x(x_0, y_0)u + g_y(x_0, y_0)v
    \end{split}
\]

In multivariable calculus you may have seen that the several variables version of the derivative is the \emph{\myindex{Jacobian matrix}}%
\footnote{
    Named for the German mathematician \href{https://en.wikipedia.org/wiki/Carl_Gustav_Jacob_Jacobi}{Carl Gustav Jacob Jacobi} (1804--1851).
    }.   
The Jacobian matrix of the vector-valued function
$\left[ \begin{smallmatrix} f(x,y) \\ g(x,y) \end{smallmatrix} \right]$
at $(x_0,y_0)$ is 
\begin{equation*}
    \begin{bmatrix}
        \frac{\partial f}{\partial x}(x_0,y_0) & \frac{\partial f}{\partial y}(x_0,y_0) \\
        \frac{\partial g}{\partial x}(x_0,y_0) & \frac{\partial g}{\partial y}(x_0,y_0)
    \end{bmatrix} .
\end{equation*}
This matrix gives the best linear approximation as $u$ and $v$ (and therefore $x$ and $y$) vary.  

\begin{definition}
    The \emph{\myindex{linearization}} of the equation \eqref{eq:nlinautn2} as the linear system
    \begin{equation*}
        \begin{bmatrix} 
            u \\ 
            v 
        \end{bmatrix} ' =
        \begin{bmatrix}
            \frac{\partial f}{\partial x}(x_0,y_0) & \frac{\partial f}{\partial y}(x_0,y_0) \\
            \frac{\partial g}{\partial x}(x_0,y_0) & \frac{\partial g}{\partial y}(x_0,y_0)
        \end{bmatrix} 
        \begin{bmatrix} 
            u \\ 
            v 
        \end{bmatrix} .
    \end{equation*}
\end{definition}

\begin{example} \label{example:nlin-1b-examplelin}
    Determine the linearization of the system of differential equations in \exampleref{example:nlin-1b-example}: $x' = y$, $y' = -x+x^2$ at all of its critical points.
\end{example}

\begin{exampleSol}
    There are two critical points, $(0,0)$ and $(1,0)$.  The Jacobian matrix at any point is
    \begin{equation*}
        \begin{bmatrix}
            \frac{\partial f}{\partial x}(x,y) & \frac{\partial f}{\partial y}(x,y) \\
            \frac{\partial g}{\partial x}(x,y) & \frac{\partial g}{\partial y}(x,y)
        \end{bmatrix} =
        \begin{bmatrix}
            0 & 1 \\
            -1+2x & 0
        \end{bmatrix}.
    \end{equation*}
    Therefore at $(0,0)$, we have $u=x$ and $v=y$, and the linearization is
    \begin{equation*}
        \begin{bmatrix} 
            u \\ 
            v 
        \end{bmatrix} ' =
        \begin{bmatrix}
            0 & 1 \\
            -1 & 0
        \end{bmatrix}
        \begin{bmatrix} 
            u \\ 
            v 
        \end{bmatrix} .
    \end{equation*}
    
    At the point $(1,0)$, we have $u=x-1$ and $v=y$, and the linearization is
    \begin{equation*}
        \begin{bmatrix} 
            u \\ 
            v 
        \end{bmatrix} ' =
        \begin{bmatrix}
            0 & 1 \\
            1 & 0
        \end{bmatrix}
        \begin{bmatrix} 
            u \\ 
            v 
        \end{bmatrix} .
    \end{equation*}
    
    The phase diagrams of the two linearizations at the point $(0,0)$ and $(1,0)$ are given in \figurevref{fig:nlin-1b-lin}.  Note that the variables are now $u$ and $v$.  Compare \figureref{fig:nlin-1b-lin} with \figurevref{fig:nlin-1b}, and look especially at the behavior near the critical points.
    
    \begin{myfig}
        \capstart
        %original files nlin-1b-lin-00 nlin-1b-lin-01
        \diffyincludegraphics{width=6.24in}{width=9in}{nlin-1b-lin-00-01}
        \caption{Phase diagram with some trajectories of linearizations at the critical points $(0,0)$ (left) and $(1,0)$ (right) of $x' = y$, $y' = -x+x^2$. \label{fig:nlin-1b-lin}}
    \end{myfig}
\end{exampleSol}

\subsection{Isolated critical points and almost linear systems}
The next step in this process is to try to figure out a way to analyze what is happening to a non-linear system of differential equations near equilibrium solutions \emph{without} using a slope field/phase portrait. We would like to be able to determine this from the equations alone, not any of the pictures that come from them. Thankfully, our ability to analyze linear systems helps us accomplish this goal. 

\begin{definition}
    A critical point is \emph{isolated}\index{isolated critical point} if it is the only critical point in some small ``neighborhood'' of the point. 
\end{definition}
That is, if we zoom in far enough it is the only critical point we see.  In the example above, the critical point was isolated.  If on the other hand there would be a whole curve of critical points, then it would not be isolated. For example, the system
\begin{equation*}
    x' = y(x-1) \qquad y' = (x-2)(x-1)
\end{equation*}
has the entire line $x=1$ as critical points. Therefore, these are not isolated.

\begin{definition}
    A system is called \emph{\myindex{almost linear}} at a critical point $(x_0,y_0)$, if the critical point is isolated and the Jacobian matrix at the point is invertible, or equivalently if the linearized system has an isolated critical point.
\end{definition}
This is also equivalent to zero not being an eigenvalue of the Jacobian matrix at the critical point. In such a case, the nonlinear terms are very small and the system behaves like its linearization, at least if we are close to the critical point.

For example, the system in Examples~\ref{example:nlin-1b-example} and \ref{example:nlin-1b-examplelin} has two isolated critical points $(0,0)$ and $(0,1)$, and is almost linear at both critical points as the Jacobian matrices at both points,
$\left[ \begin{smallmatrix} 0 & 1 \\ -1 & 0 \end{smallmatrix} \right]$ and
$\left[ \begin{smallmatrix} 0 & 1 \\ 1 & 0 \end{smallmatrix} \right]$,
are invertible.

On the other hand, the system $x' = x^2$, $y' = y^2$ has an isolated critical point at $(0,0)$, however the Jacobian matrix
\begin{equation*}
    \begin{bmatrix} 
        2x & 0 \\ 
        0 & 2y 
    \end{bmatrix}
\end{equation*}
is zero when $(x,y) = (0,0)$.  So the system is not almost linear. Even a worse example is the system $x' = x$, $y' = x^2$, which does not have isolated critical points; $x'$ and $y'$ are both zero whenever $x=0$, that is, the entire $y$-axis.

Fortunately, most often critical points are isolated, and the system is almost linear at the critical points.  So if we learn what happens there, we will have figured out the majority of situations that arise in applications.



\subsection{Stability and classification of isolated critical points}

Once we have an isolated critical point, the system is almost linear at that critical point, and we computed the associated linearized system, we can classify what happens to the solutions.  The classifications for linear two-variable systems from \sectionref{sec:twodimaut} are generally the same as what we use here, with one minor caveat. Let us list the behaviors depending on the eigenvalues of the Jacobian matrix at the critical point in \tablevref{pln:behtab2}. This table is very similar to \tablevref{pln:behtab}, with the exception of missing ``center'' points. The repeated eigenvalue cases are also missing. They behave similarly to the real eigenvalue descriptions in the table below, but similar to centers, the behavior can change slightly. It can behave like either a spiral or a node, but will be either a source or sink based on the sign of the repeated eigenvalue. 
%There is also a new column
%that we will discuss.
We will discuss centers later, as they are more complicated.

\begin{table}[h!t]
    \mybeginframe
    \capstart
    \begin{center}
        \begin{tabular}{@{}lll@{}}
            \toprule
            Eigenvalues of the Jacobian matrix & Behavior & Stability \\
            \midrule
            real and both positive & source / unstable node & unstable \\
            real and both negative & sink / stable node & asymptotically stable \\
            real and opposite signs & saddle & unstable \\
            complex with positive real part & spiral source & unstable \\
            complex with negative real part & spiral sink & asymptotically stable \\
            \bottomrule
        \end{tabular}
    \end{center}
    \caption{Behavior of an almost linear system near an isolated critical point.  \label{pln:behtab2}}
    \myendframe
\end{table}

In the third column, we mark points as \emph{asymptotically stable} or \emph{unstable}.  
\begin{definition}
    Let $(x_0, y_0)$ be a critical point for a non-linear system of two differential equations.
    \begin{enumerate}
        \item We say that the critical point is a \emph{\myindex{stable critical point}} if, given any small distance $\epsilon$ to $(x_0,y_0)$, and any initial condition within a perhaps smaller radius around $(x_0,y_0)$, the trajectory of the system never goes further away from $(x_0,y_0)$ than $\epsilon$.
        \item The critical point is an \emph{\myindex{unstable critical point}} if it is not stable; that is, there are trajectories that start within a distance $\epsilon$ of $(x_0, y_0)$ and end up farther than $\epsilon$ from that point.
        \item The critical point is called \emph{\myindex{asymptotically stable}} if given any initial condition sufficiently close to $(x_0,y_0)$ and any solution $\bigl( x(t), y(t) \bigr)$ satisfying that condition, then
        \begin{equation*}
            \lim_{t \to \infty} \bigl( x(t), y(t) \bigr) = (x_0,y_0) .
        \end{equation*}
    \end{enumerate}
\end{definition}

Informally, a point is stable if we start close to a critical point and follow a trajectory we either go towards, or at least not away from, this critical point. If the point is asymptotically stable, then any trajectory for a sufficiently close initial condition goes towards the critical point $(x_0,y_0)$, and unstable means that, in general, trajectories move away from the critical point. 

\begin{example} \label{example:nlin-xplusy}
    Find and analyze the critical points of $x'=-y-x^2$, $y'=-x+y^2$.
\end{example}

\begin{exampleSol}
    See \figurevref{fig:nlin-ex813-new} for the phase diagram. Let us find the critical points.  These are the points where $-y-x^2 = 0$ and $-x+y^2=0$.  The first equation means $y = -x^2$, and so $y^2 = x^4$.  Plugging into the second equation we obtain $-x+x^4 = 0$.  Factoring we obtain $x(1-x^3)=0$.  Since we are looking only for real solutions we get either $x=0$ or $x=1$.  Solving for the corresponding $y$ using $y = -x^2$, we get two critical points, one being $(0,0)$ and the other being $(1,-1)$.  Clearly the critical points are isolated.
    
    \begin{myfig}
        \capstart
        \diffyincludegraphics{width=3in}{width=4.5in}{nlin-ex813-new}
        \caption{The phase portrait with few sample trajectories of $x'=-y-x^2$, $y'=-x+y^2$.  \label{fig:nlin-ex813-new}}
    \end{myfig}
    
    
    Let us compute the Jacobian matrix:
    \begin{equation*}
        \begin{bmatrix}
            -2x & -1 \\
            -1 & 2y
        \end{bmatrix} .
    \end{equation*}
    At the point $(0,0)$ we get the matrix 
    $\left[ \begin{smallmatrix} 0 & -1 \\ -1 & 0 \end{smallmatrix} \right]$ 
    and so the two eigenvalues are $1$ and $-1$.  As the matrix is invertible, the system is almost linear at $(0,0)$.  As the eigenvalues are real and of opposite signs, we get a saddle point, which is an unstable equilibrium point. Looking at the phase portrait, we can see trajectories that would start near $(0,0)$ and end up farther away from $(0,0)$. These trajectories may end up at $(1,-1)$, but that is away from $(0,0)$. 
    
    At the point $(1,-1)$ we get the matrix 
    $\left[ \begin{smallmatrix} -2 & -1 \\ -1 & -2 \end{smallmatrix} \right]$ 
    and computing the eigenvalues we get $-1$, $-3$. The matrix is invertible, and so the system is almost linear at $(1,-1)$. As we have real eigenvalues and both negative, the critical point is a sink, and therefore an asymptotically stable equilibrium point. That is, if we start with any point $(x(0),y(0))$ close to $(1,-1)$ as an initial condition and plot a trajectory, it approaches $(1,-1)$. In other words,
    \begin{equation*}
        \lim_{t \to \infty} \bigl( x(t), y(t) \bigr) = (1,-1) .
    \end{equation*}
    As you can see from the diagram, this behavior is true even for some initial points quite far from $(1,-1)$, but it is definitely not true for all initial points.
\end{exampleSol}

\begin{example} \label{example:nlin-withexp}
    Find and analyze the critical points of $x'=y+y^2e^x$, $y'=x$.
\end{example}

\begin{exampleSol}
    First let us find the critical points.  These are the points where $y+y^2e^x = 0$ and $x=0$.  Simplifying we get $0=y+y^2 = y(y+1)$.  So the critical points are $(0,0)$ and $(0,-1)$, and hence are isolated.  Let us compute the Jacobian matrix:
    \begin{equation*}
        \begin{bmatrix}
            y^2e^x & 1+2ye^x \\
            1 & 0
        \end{bmatrix}.
    \end{equation*}
    
    At the point $(0,0)$ we get the matrix 
    $\left[ \begin{smallmatrix} 0 & 1 \\ 1 & 0 \end{smallmatrix} \right]$ 
    and so the two eigenvalues are $1$ and $-1$.  As the matrix is invertible, the system is almost linear at $(0,0)$.  And, as the eigenvalues are real and of opposite signs, we get a saddle point, which is an unstable equilibrium point.
    
    At the point $(0,-1)$ we get the matrix 
    $\left[ \begin{smallmatrix} 1 & -1 \\ 1 & 0 \end{smallmatrix} \right]$ 
    whose eigenvalues are $\frac{1}{2} \pm i \frac{\sqrt{3}}{2}$. The matrix is invertible, and so the system is almost linear at $(0,-1)$. As we have complex eigenvalues with positive real part, the critical point is a spiral source, and therefore an unstable equilibrium point.
    
    \begin{myfig}
        \capstart
        \diffyincludegraphics{width=3in}{width=4.5in}{nlin-ex813}
        \caption{The phase portrait with few sample trajectories of $x'=y+y^2e^x$, $y'=x$.  \label{fig:nlin-ex813}}
    \end{myfig}
    
    See \figurevref{fig:nlin-ex813} for the phase diagram.  Notice the two critical points, and the behavior of the arrows in the vector field around these points.
\end{exampleSol}

\subsection{The trouble with centers}

Recall, a linear system with a center means that trajectories travel in closed elliptical orbits in some direction around the critical point.  Such a critical point we call a \emph{\myindex{center}} or a \emph{\myindex{stable center}}.  It is not an asymptotically stable critical point, as the trajectories never approach the critical point, but at least if you start sufficiently close to the critical point, you stay close to the critical point.  The simplest example of such behavior is the linear system with a center.  Another example is the critical point $(0,0)$ in \examplevref{example:nlin-1b-example}.

The trouble with a center in a nonlinear system is that whether the trajectory goes towards or away from the critical point is governed by the sign of the real part of the eigenvalues of the Jacobian matrix, and the Jacobian matrix in a nonlinear system changes from point to point.  Since this real part is zero at the critical point itself, it can have either sign nearby, meaning the trajectory could be pulled towards or away from the critical point.

\begin{example}
    Find and analyze the critical point(s) of $x'=y, y' = -x+y^3$.  
\end{example}

\begin{exampleSol}
    The only critical point is the origin $(0,0)$.  The Jacobian matrix is 
    \begin{equation*}
        \begin{bmatrix}
            0 & 1 \\
            -1 & 3 y^2 \\
        \end{bmatrix} .
    \end{equation*}
    At $(0,0)$ the Jacobian matrix is
    $\left[ \begin{smallmatrix}
        0 & 1 \\
        -1 & 0 \\
    \end{smallmatrix} \right]$, 
    which has eigenvalues $\pm i$.  So the linearization has a center.
    
    Using the quadratic equation, the eigenvalues of the Jacobian matrix at any point $(x,y)$ are
    \begin{equation*}
        \lambda = \frac{3}{2}y^2 \pm i \frac{\sqrt{4-9y^4}}{2} .
    \end{equation*}
    At any point where $y \not= 0$ (so at most points near the origin), the eigenvalues have a positive real part ($y^2$ can never be negative).  This positive real part pulls the trajectory away from the origin.  A sample trajectory for an initial condition near the origin is given in \figurevref{fig:nlin-unstable-center}.
    \begin{myfig}
        \capstart
        \diffyincludegraphics{width=3in}{width=4.5in}{nlin-unstable-centerfig}
        \caption{An unstable critical point (spiral source) at the origin for $x'=y, y' = -x+y^3$, even if the linearization has a center.  \label{fig:nlin-unstable-center}}
    \end{myfig}
\end{exampleSol}

The same process could be carried out with the system $x'=y, y' = -x-y^3$. This one will also have a center as the linearization at the origin, but the non-linear system will have a spiral sink at the origin. The moral of the example is that further analysis is needed when the linearization has a center.  The analysis will in general be more complicated than in the example above, and is more likely to involve case-by-case consideration.  Such a complication should not be surprising to you.  By now in your mathematical career, you have seen many places where a simple test is inconclusive, recall for example the second derivative test for maxima or minima, and requires more careful, and perhaps ad hoc analysis of the situation.


\end{document}
