\documentclass{ximera}

\title{Practice for Advanced Nonlinear Analysis}

%\auor{Matthew Charnley and Jason Nowell}
\usepackage[margin=1.5cm]{geometry}
\usepackage{indentfirst}
\usepackage{sagetex}
\usepackage{lipsum}
\usepackage{amsmath}
\usepackage{mathrsfs}
\usepackage{tikz}
\usetikzlibrary{matrix}

%%% Random packages added without verifying what they are really doing - just to get initial compile to work.
\usepackage{tcolorbox}
\usepackage{hypcap}
\usepackage{booktabs}%% To get \toprule,\midrule,\bottomrule etc.
\usepackage{caption}
\usepackage{units}
\usepackage{multicol}
\usepackage{hhline}


% This is my modified wrapfig that doesn't use intextsep
\usepackage{mywrapfig}
\usepackage{import}



%%% End to random added packages.


\graphicspath{
    {./}
    {./figures/}
    {./../figures/}
    {./../../figures/}
}
\renewcommand{\log}{\ln}%%%%
\DeclareMathOperator{\arcsec}{arcsec}
%% New commands


%%%%%%%%%%%%%%%%%%%%
% New Conditionals %
%%%%%%%%%%%%%%%%%%%%


% referencing
\makeatletter
    \DeclareRobustCommand{\myvref}[2]{%
      \leavevmode%
      \begingroup
        \let\T@pageref\@pagerefstar
        \hyperref[{#2}]{%
	  #1~\ref*{#2}%
        }%
        \vpageref[\unskip]{#2}%
      \endgroup
    }%

    \DeclareRobustCommand{\myref}[2]{%
      \leavevmode%
      \begingroup
        \let\T@pageref\@pagerefstar
        \hyperref[{#2}]{%
	  #1~\ref*{#2}%
        }%
      \endgroup
    }%
\makeatother

\newcommand{\figurevref}[1]{\myvref{Figure}{#1}}
\newcommand{\figureref}[1]{\myref{Figure}{#1}}
\newcommand{\tablevref}[1]{\myvref{Table}{#1}}
\newcommand{\tableref}[1]{\myref{Table}{#1}}
\newcommand{\chapterref}[1]{\myref{chapter}{#1}}
\newcommand{\Chapterref}[1]{\myref{Chapter}{#1}}
\newcommand{\appendixref}[1]{\myref{appendix}{#1}}
\newcommand{\Appendixref}[1]{\myref{Appendix}{#1}}
\newcommand{\sectionref}[1]{\myref{\S}{#1}}
\newcommand{\subsectionref}[1]{\myref{subsection}{#1}}
\newcommand{\subsectionvref}[1]{\myvref{subsection}{#1}}
\newcommand{\exercisevref}[1]{\myvref{Exercise}{#1}}
\newcommand{\exerciseref}[1]{\myref{Exercise}{#1}}
\newcommand{\examplevref}[1]{\myvref{Example}{#1}}
\newcommand{\exampleref}[1]{\myref{Example}{#1}}
\newcommand{\thmvref}[1]{\myvref{Theorem}{#1}}
\newcommand{\thmref}[1]{\myref{Theorem}{#1}}


\renewcommand{\exampleref}[1]{ {\color{red} \bfseries Normally a reference to a previous example goes here.}}
\renewcommand{\examplevref}[1]{ {\color{red} \bfseries Normally a reference to a previous example goes here.}}
\renewcommand{\figurevref}[1]{ {\color{red} \bfseries Normally a reference to a previous figure goes here.}}
\renewcommand{\tablevref}[1]{ {\color{red} \bfseries Normally a reference to a previous table goes here.}}
\renewcommand{\Appendixref}[1]{ {\color{red} \bfseries Normally a reference to an Appendix goes here.}}
\renewcommand{\exercisevref}[1]{ {\color{red} \bfseries Normally a reference to a previous exercise goes here.}}
\renewcommand{\thmvref}[1]{ {\color{red} \bfseries Normally a reference to a previous theorem goes here.}}
\renewcommand{\subsectionvref}[1]{ {\color{red} \bfseries Normally a reference to a previous subsection goes here.}}



\newcommand{\R}{\mathbb{R}}
\newcommand{\C}{\mathbb{C}}

%% Example Solution Env.
\def\beginSolclaim{\par\addvspace{\medskipamount}\noindent\hbox{\bf Solution:}\hspace{0.5em}\ignorespaces}
\def\endSolclaim{\par\addvspace{-1em}\hfill\rule{1em}{0.4pt}\hspace{-0.4pt}\rule{0.4pt}{1em}\par\addvspace{\medskipamount}}
\newenvironment{exampleSol}[1][]{\beginSolclaim}{\endSolclaim}

%% General figure formating from original book.
\newcommand{\mybeginframe}{%
\begin{tcolorbox}[colback=white,colframe=lightgray,left=5pt,right=5pt]%
}
\newcommand{\myendframe}{%
\end{tcolorbox}%
}

%%% Eventually return and fix this to make matlab code work correctly.
%% Define the matlab environment as another code environment
%\NewEnviron{matlab}{ {\centering\bfseries MATLAB Code} \\ \noexpand{\BODY} }
%\let\beginmatlab\begincode
%\let\endmatlab\endcode
%\newenvironment{matlab}{% Begin Environment Code
%\begin{minipage}{\linewidth}
%\begin{verbatim}
%}% End of Begin Environment Code
%{% Start of End Environment Code
%\end{verbatim}
%\end{minipage}
%}% End of End Environment Code


% this one should have a caption, first argument is the size
\newenvironment{mywrapfig}[2][]{
 \wrapfigure[#1]{r}{#2}
 \mybeginframe
 \centering
}{%
 \myendframe
 \endwrapfigure
}

% this one has no caption, first argument is size,
% the second argument is a larger size used for HTML (ignored by latex)
\newenvironment{mywrapfigsimp}[3][]{%
 \wrapfigure[#1]{r}{#2}%
 \centering%
}{%
 \endwrapfigure%
}
\newenvironment{myfig}
    {%
    \begin{figure}[h!t]
        \mybeginframe%
        \centering%
    }
    {%
        \myendframe
    \end{figure}%
    }


% graphics include
\newcommand{\diffyincludegraphics}[3]{\includegraphics[#1]{#3}}
\newcommand{\myincludegraphics}[3]{\includegraphics[#1]{#3}}
\newcommand{\inputpdft}[1]{\subimport*{../figures/}{#1.pdf_t}}


%% Not sure what these even do? They don't seem to actually work... fun!
%\newcommand{\mybxbg}[1]{\tcboxmath[colback=white,colframe=black,boxrule=0.5pt,top=1.5pt,bottom=1.5pt]{#1}}
%\newcommand{\mybxsm}[1]{\tcboxmath[colback=white,colframe=black,boxrule=0.5pt,left=0pt,right=0pt,top=0pt,bottom=0pt]{#1}}
\newcommand{\mybxsm}[1]{#1}
\newcommand{\mybxbg}[1]{#1}

%%% Something about tasks for practice/hw?
\usepackage{tasks}
\usepackage{footnote}
\makesavenoteenv{tasks}


%% For pdf only?
\newcommand{\diffypdfversion}[1]{#1}


%% Kill ``cite'' and go back later to fix it.
\renewcommand{\cite}[1]{}


%% Currently we can't really use index or its derivatives. So we are gonna kill them off.
\renewcommand{\index}[1]{}
\newcommand{\myindex}[1]{#1}







\begin{document}
\begin{abstract}
Why?
\end{abstract}
\maketitle


\begin{exercise}
    Find the implicit equations of the trajectories of the following conservative systems.  Next find their critical points (if any) and classify them.
    \begin{tasks}(2)
        \task $x''+ x+x^3 = 0$
        \task $\theta''+\sin \theta = 0$
        \task $z''+ (z-1)(z+1) = 0$
        \task $x''+ x^2+1 = 0$
    \end{tasks}
\end{exercise}
%\comboSol
%{%
%a)~$\frac{1}{2}y^2 + \frac{x^4}{4} + \frac{x^2}{2} = C$. $(0,0)$ is a center. \\
%b)~$\frac{1}{2}y^2 - \cos(\theta) = C$. $(n\pi, 0)$ is a saddle if $n$ is even, and a center if $n$ is odd. \\
%c)~$\frac{1}{2}y^2 + \frac{z^3}{3} - z = C$, $(1,0)$ is a center, $(-1, 0)$ is a saddle. \\
%d)~$\frac{1}{2}y^2 + \frac{x^3}{3} + x = C$, no critical points.
%}

\begin{exercise}%
    Find the implicit equations of the trajectories of the following conservative systems.  Next find their critical points (if any) and classify them.
    \begin{tasks}(3)
        \task $x''+ x^2 = 4$
        \task $x''+ e^x = 0$
        \task $x''+ (x+1)e^x = 0$
    \end{tasks}
\end{exercise}
%\exsol{%
%a) $\frac{1}{2}y^2 + \frac{1}{3}x^3 -4x = C$, critical points:
%$(-2,0)$, an unstable saddle, and $(2,0)$, a stable center. \quad
%b) $\frac{1}{2}y^2 + e^x = C$, no critical points. \quad
%c) $\frac{1}{2}y^2 + xe^x = C$, critical point at $(-1,0)$ is a stable center.
%}


\begin{exercise}%
    The conservative system $x''+x^3 = 0$ is not almost linear.  Classify its critical point(s) nonetheless.
\end{exercise}
%\exsol{%
%Critical point at $(0,0)$.
%Trajectories are $y = \pm \sqrt{2C-(\nicefrac{1}{2})x^4}$, for $C > 0$, these give closed
%curves around the origin, so the critical point is a stable center.
%}

\begin{exercise}
    Determine if the following system is Hamiltonian. If it is, find the general solution in the form $H(x,y) = C$ and sketch some of the trajectories.
    \[ 
        \frac{dx}{dt} = x - 2y \qquad \frac{dy}{dt} = 3x - y.
    \]
\end{exercise}
%\comboSol
%{%
%$H(x,y) = -\frac{3}{2}x^2 + xy - y^2$
%}

\begin{exercise}
    Determine if the following system is Hamiltonian. If it is, find the general solution in the form $H(x,y) = C$ and sketch some of the trajectories.
    \[ 
        \frac{dx}{dt} = 4x - 2y + 2 \qquad \frac{dy}{dt} = -5x + y - 1.
    \]
\end{exercise}
%\comboSol
%{%
%No
%}


\begin{exercise}
    Determine if the following system is Hamiltonian. If it is, find the general solution in the form $H(x,y) = C$ and sketch some of the trajectories.
    \[ 
        \frac{dx}{dt} = x^2 - 2xy + 3y^2 \qquad \frac{dy}{dt} = y^2 - 2xy + e^x .
    \]
\end{exercise}
%\comboSol
%{%
%$H(x,y) = x^2y - xy^2 + y^3 - e^x$
%}


\begin{exercise}
    Determine if the following system is Hamiltonian. If it is, find the general solution in the form $H(x,y) = C$ and sketch some of the trajectories.
    \[ 
        \frac{dx}{dt} = 3x - 4xy \qquad \frac{dy}{dt} = 5xy - y.
    \]
    Afterwards, do the same but with the system 
    \[ 
        \frac{dx}{dt} = \frac{3}{y} - 4 \qquad \frac{dy}{dt} = 5 - \frac{3}{x}, 
    \] 
    noticing that this is the same as the first system with each equation divided by $xy$.
\end{exercise}
%\comboSol
%{%
%First is not Hamiltonian. Second is with $H(x,y) = 3\ln(|y|)  + \ln(|x|) - 5x - 4y$.
%}

\begin{exercise} 
    Consider a generic thing on a spring, with displacement $u$ and velocity $v$. Assume that $$mu''+ku=0,$$ where $m$ and $k$ are some positive constants.
    \begin{tasks}
        \task Rewrite this equation as a {\it first-order} system in $u$ and $v$.
        \task Find a Hamiltonian function for this system (in terms of $u$ and $v$).
        \task What shapes are the level curves of the Hamiltonian function? 
        \task Does this system have a {\it basin of attraction}? Explain briefly. 
    \end{tasks}
\end{exercise}
%\comboSol
%{%
%a)~$u' = v$, $v'= -\frac{k}{m}u$ \quad b)~ $\frac{v^2}{2} + \frac{k}{m}u^2$ \quad c)~ Ellipses \quad d)~ No
%}

\begin{exercise}
    Suppose $f$ is always positive. Find the trajectories of $x''+f(x') = 0$. Are there any critical points?
\end{exercise}
%\comboSol
%{%
%No.
%}

\begin{exercise}
    Suppose that $x' = f(x,y)$, $y' = g(x,y)$.  Suppose that $g(x,y) > 1$ for all $x$ and $y$.  Are there any critical points?  What can we say about the trajectories at $t$ goes to infinity?
\end{exercise}
%\comboSol
%{%
%All will have $y\rightarrow \infty$. No critical points.
%}

\begin{exercise}
    Here is the direction field for the system $\dfrac{dx}{dt}=y^2-3y, \dfrac{dy}{dt}=x^2-4x$. The critical points are $(0,0), (4,0), (0,3)$, and $(4,3)$. Draw the nullclines on the plot. What do the nullclines tell us about the critical points?
    
    \begin{center}
        \includegraphics[width=0.6\textwidth]{figures/NLVF_1.png}
    \end{center}
\end{exercise}
%\comboSol
%{%
%
%\includegraphics[width=1.5in]{Images/NLVF_1_Soln.png}
%}

\begin{exercise}
    Nullclines apply to linear systems as well, although since we can often solve those explicitly they're less necessary. Construct the nullcline diagram for the system $\dfrac{dx}{dt}=-3x+y$, $\dfrac{dy}{dt}=6x+2y$, and use it to classify (by type) the equilibrium point at the origin. What is the {\it linearization} of this system at $(0,0)$?
\end{exercise}
%\comboSol
%{%
%Saddle \hfill\raisebox{-0.5\height}{\includegraphics[height=1.2in]{Images/NullclineDiagramSoln1.png}}\hfill\hfill
%}

\begin{exercise}
    Consider the system $\displaystyle \frac{dx}{dt}= -2x+y, \frac{dy}{dt}=-y+x^2$.
    \begin{tasks}
        \task Find all equilibrium solutions.
        \task Sketch all nullclines for this system on a single diagram. Label each region, and use these results to classify each equilibrium point.
    \end{tasks}
\end{exercise}
%\comboSol
%{%
%$(0,0)$ and $(2, 4)$. \hfill\raisebox{-0.5\height}{\includegraphics[height=1.2in]{Images/NullclineDiagramSoln2.png}}\hfill\hfill
%}

\begin{exercise} Nullclines need not be lines. Consider the system
    \[
        \begin{bmatrix} 
            \displaystyle \frac{dx}{dt}=4-y^2\\ 
            \dfrac{dy}{dt}=8-x^2-y^2 
        \end{bmatrix}.
    \]
    
    \begin{tasks}
        \task Find all critical points of this system.
        \task Sketch the nullcline diagram and label all regions DL, DR, UL, or UR. Classify (according to type) any critical point(s) that can be classified using this analysis.
        \task Two critical points cannot be classified using the nullcline analysis. Classify these (again according to type) using the Jacobian. 
    \end{tasks}
\end{exercise}
%\comboSol
%{%
%\begin{multicols}{2}
%\noindent a)~$(2, 2)$, $(2, -2)$, $(-2, 2)$, $(-2, -2)$ \quad b)~$(2,2)$ is a saddle, $(-2, -2)$ is a saddle. \quad c)~ $(-2, 2)$ is a spiral sink, $(2, -2)$ is a spiral source.
%
%\includegraphics[height=1.2in]{Images/NullclineDiagramSoln3.png}
%\end{multicols}
%}

\begin{exercise}
    Consider the system
    \begin{equation}
        \begin{bmatrix}
            \frac{dx}{dt}=x-y^2+2\\[12pt]
            \frac{dy}{dt}=x^2-y^2
        \end{bmatrix}
        \label{eq:CriticalHamilEx}
    \end{equation}
    %5.1
    \begin{tasks}
        \task Find all critical points of \eqref{eq:CriticalHamilEx}.
        \task Create the nullcline diagram for the system, labelling each region as one of UL, UR, DL, or DR. Use this information to classify two critical points according to type.
        \task Use the Jacobian matrix to classify any remaining critical points.
        \task Is there a conserved quantity (Hamiltonian function) for this system? If so, find one. If not, explain why not.
    \end{tasks}
\end{exercise}
%\comboSol
%{%
%\begin{multicols}{2}
%\noindent a)~$(2, 2)$, $(-1, -1)$, $(2, -2)$, $(-1, 1)$ \quad b)~$(2, -2)$ is a saddle, $(-1, 1)$ is a saddle. \quad
%c)~$(2,2)$ is a spiral sink, $(-1, -1)$ is a spiral source. \quad d)~ No, can not have sources or sinks.
%
%\includegraphics[height=1.2in]{Images/NullclineDiagramSoln4.png}
%\end{multicols}
%}

\begin{exercise}
    For a conflict between two armies, Lanchester's Law asserts that $\dfrac{dx}{dt}=-\alpha y$ and $\dfrac{dy}{dt}=-\beta x$, %5.2 (Hamiltonian, separatrix) 
    where $x$ and $y$ are the two populations, and $\alpha$ and $\beta$ are some positive constants.
    
    \begin{tasks}
        \task Find a Hamiltonian function for this system satisfying $H(0,0)=0$. 
        \task Classify the critical point at the origin according to type {\bf and} stability.
        \task Assume that we are just looking at the first quadrant, since the populations are non-negative. Find the curve along which the Hamiltonian function is zero, and explain its significance in terms of who wins the conflict.
    \end{tasks}
\end{exercise}
%\comboSol
%{%
%a)~ $\beta\frac{x^2}{2} - \alpha \frac{y^2}{2} = 0$ \quad b)~Saddle, these are hyperbolas. \quad c)~ $y = \sqrt{\frac{\beta}{\alpha}}x$, dividing line deciding who wins
%}

\begin{exercise}
    Consider the non-linear system
    \begin{equation}
        \frac{dx}{dt}=4x-3y-x(x^2+y^2), \ \frac{dy}{dt}=3x+4y-y(x^2+y^2). \label{eq:NonLinStabLimitEx1} %5.1 or 2
    \end{equation}
    \begin{tasks}
        \task \eqref{eq:NonLinStabLimitEx1} has a critical point at the origin. What is the {\it linearization} of \eqref{eq:NonLinStabLimitEx1} at the origin?
        \task Demonstrate that \eqref{eq:NonLinStabLimitEx1} is {\it locally linear} in a neighborhood of the origin. 
        \task Classify the origin according to its {\it type} and {\it stability}.
    \end{tasks}
\end{exercise}
%\comboSol
%{%
%a)~Spiral Source \quad b)~Matrix is invertible \quad c)~Unstable spiral source
%}

\begin{exercise} 
    Consider the system of differential equations
    \begin{equation}
        \frac{dx}{dt} = (x^2 - 1)y \qquad \frac{dy}{dt} = (y-3)(y-1)x \label{eq:NonLinBoAExam1}
    \end{equation}
    which has slope field sketched below.
    \begin{center}
        \includegraphics[width=0.6\textwidth]{figures/NLBoA_Ex1.png}
    \end{center}
    \begin{tasks}
        \task Find and classify all critical points of the system \eqref{eq:NonLinBoAExam1}.
        \task Draw any separatrices that you can spot on the slope field.
        \task Do any of these critical points have a basin of attraction? If so, sketch out what regions of the plane correspond to a basin of attraction for those critical points. 
    \end{tasks}
\end{exercise}
%\comboSol
%{%
%\begin{multicols}{2}
%\noindent $(1, -1)$ Nodal sink, $(-1, -1)$ Nodal source, $(1, 3)$ Nodal source, $(-1, 3)$ Nodal sink, $(0,0)$ Saddle. 
% \quad c)~ Yes, bottom right goes to $(-1, 1)$, top left goes to $(-1, 3)$.
% 
% \includegraphics[width=1.2in]{Images/NLBoA_Ex1_Soln.png}
%\end{multicols}
%}

\begin{exercise} 
    Consider the system of differential equations
    \begin{equation}
        \frac{dx}{dt} = (2-y)(y+1)(x+1) \qquad \frac{dy}{dt} = -(x+2)(x-1)y \label{eq:NonLinBoAExam2}
    \end{equation}
    which has slope field sketched below.
    \begin{center}
        \includegraphics[width=0.6\textwidth]{figures/NLBoA_Ex2.png}
    \end{center}
    \begin{tasks}
        \task Find and classify all critical points of the system \eqref{eq:NonLinBoAExam2}.
        \task Draw any separatrices that you can spot on the slope field.
        \task Do any of these critical points have a basin of attraction? If so, sketch out what regions of the plane correspond to a basin of attraction for those critical points. 
    \end{tasks}
\end{exercise}
%\comboSol
%{%
%\begin{multicols}{2}
%\noindent $(-2, 2)$ saddle, $(-2, -1)$ saddle, $(1, 2)$ saddle, $(1, -1)$ saddle, $(-1, 0)$ nodal source.
% \quad c)~ No.
% 
% \includegraphics[width=1.2in]{Images/NLBoA_Ex2_Soln.png}
%\end{multicols}
%}


\end{document}