\documentclass{ximera}

\title{Practice for Limit Cycles}

%\auor{Matthew Charnley and Jason Nowell}
\usepackage[margin=1.5cm]{geometry}
\usepackage{indentfirst}
\usepackage{sagetex}
\usepackage{lipsum}
\usepackage{amsmath}
\usepackage{mathrsfs}


%%% Random packages added without verifying what they are really doing - just to get initial compile to work.
\usepackage{tcolorbox}
\usepackage{hypcap}
\usepackage{booktabs}%% To get \toprule,\midrule,\bottomrule etc.
\usepackage{nicefrac}
\usepackage{caption}
\usepackage{units}

% This is my modified wrapfig that doesn't use intextsep
\usepackage{mywrapfig}
\usepackage{import}



%%% End to random added packages.


\graphicspath{
    {./figures/}
    {./../figures/}
    {./../../figures/}
}
\renewcommand{\log}{\ln}%%%%
\DeclareMathOperator{\arcsec}{arcsec}
%% New commands


%%%%%%%%%%%%%%%%%%%%
% New Conditionals %
%%%%%%%%%%%%%%%%%%%%


% referencing
\makeatletter
    \DeclareRobustCommand{\myvref}[2]{%
      \leavevmode%
      \begingroup
        \let\T@pageref\@pagerefstar
        \hyperref[{#2}]{%
	  #1~\ref*{#2}%
        }%
        \vpageref[\unskip]{#2}%
      \endgroup
    }%

    \DeclareRobustCommand{\myref}[2]{%
      \leavevmode%
      \begingroup
        \let\T@pageref\@pagerefstar
        \hyperref[{#2}]{%
	  #1~\ref*{#2}%
        }%
      \endgroup
    }%
\makeatother

\newcommand{\figurevref}[1]{\myvref{Figure}{#1}}
\newcommand{\figureref}[1]{\myref{Figure}{#1}}
\newcommand{\tablevref}[1]{\myvref{Table}{#1}}
\newcommand{\tableref}[1]{\myref{Table}{#1}}
\newcommand{\chapterref}[1]{\myref{chapter}{#1}}
\newcommand{\Chapterref}[1]{\myref{Chapter}{#1}}
\newcommand{\appendixref}[1]{\myref{appendix}{#1}}
\newcommand{\Appendixref}[1]{\myref{Appendix}{#1}}
\newcommand{\sectionref}[1]{\myref{\S}{#1}}
\newcommand{\subsectionref}[1]{\myref{subsection}{#1}}
\newcommand{\subsectionvref}[1]{\myvref{subsection}{#1}}
\newcommand{\exercisevref}[1]{\myvref{Exercise}{#1}}
\newcommand{\exerciseref}[1]{\myref{Exercise}{#1}}
\newcommand{\examplevref}[1]{\myvref{Example}{#1}}
\newcommand{\exampleref}[1]{\myref{Example}{#1}}
\newcommand{\thmvref}[1]{\myvref{Theorem}{#1}}
\newcommand{\thmref}[1]{\myref{Theorem}{#1}}


\renewcommand{\exampleref}[1]{ {\color{red} \bfseries Normally a reference to a previous example goes here.}}
\renewcommand{\figurevref}[1]{ {\color{red} \bfseries Normally a reference to a previous figure goes here.}}
\renewcommand{\tablevref}[1]{ {\color{red} \bfseries Normally a reference to a previous table goes here.}}
\renewcommand{\Appendixref}[1]{ {\color{red} \bfseries Normally a reference to an Appendix goes here.}}
\renewcommand{\exercisevref}[1]{ {\color{red} \bfseries Normally a reference to a previous exercise goes here.}}



\newcommand{\R}{\mathbb{R}}

%% Example Solution Env.
\def\beginSolclaim{\par\addvspace{\medskipamount}\noindent\hbox{\bf Solution:}\hspace{0.5em}\ignorespaces}
\def\endSolclaim{\par\addvspace{-1em}\hfill\rule{1em}{0.4pt}\hspace{-0.4pt}\rule{0.4pt}{1em}\par\addvspace{\medskipamount}}
\newenvironment{exampleSol}[1][]{\beginSolclaim}{\endSolclaim}

%% General figure formating from original book.
\newcommand{\mybeginframe}{%
\begin{tcolorbox}[colback=white,colframe=lightgray,left=5pt,right=5pt]%
}
\newcommand{\myendframe}{%
\end{tcolorbox}%
}

%%% Eventually return and fix this to make matlab code work correctly.
%% Define the matlab environment as another code environment
%\newenvironment{matlab}
%{% Begin Environment Code
%{ \centering \bfseries Matlab Code }
%\begin{code}
%}% End of Begin Environment Code
%{% Start of End Environment Code
%\end{code}
%}% End of End Environment Code


% this one should have a caption, first argument is the size
\newenvironment{mywrapfig}[2][]{
 \wrapfigure[#1]{r}{#2}
 \mybeginframe
 \centering
}{%
 \myendframe
 \endwrapfigure
}

% this one has no caption, first argument is size,
% the second argument is a larger size used for HTML (ignored by latex)
\newenvironment{mywrapfigsimp}[3][]{%
 \wrapfigure[#1]{r}{#2}%
 \centering%
}{%
 \endwrapfigure%
}
\newenvironment{myfig}
    {%
    \begin{figure}[h!t]
        \mybeginframe%
        \centering%
    }
    {%
        \myendframe
    \end{figure}%
    }


% graphics include
\newcommand{\diffyincludegraphics}[3]{\includegraphics[#1]{#3}}
\newcommand{\myincludegraphics}[3]{\includegraphics[#1]{#3}}
\newcommand{\inputpdft}[1]{\subimport*{../figures/}{#1.pdf_t}}


%% Not sure what these even do? They don't seem to actually work... fun!
%\newcommand{\mybxbg}[1]{\tcboxmath[colback=white,colframe=black,boxrule=0.5pt,top=1.5pt,bottom=1.5pt]{#1}}
%\newcommand{\mybxsm}[1]{\tcboxmath[colback=white,colframe=black,boxrule=0.5pt,left=0pt,right=0pt,top=0pt,bottom=0pt]{#1}}
\newcommand{\mybxsm}[1]{#1}
\newcommand{\mybxbg}[1]{#1}

%%% Something about tasks for practice/hw?
\usepackage{tasks}
\usepackage{footnote}
\makesavenoteenv{tasks}


%% For pdf only?
\newcommand{\diffypdfversion}[1]{#1}


%% Kill ``cite'' and go back later to fix it.
\renewcommand{\cite}[1]{}


%% Currently we can't really use index or its derivatives. So we are gonna kill them off.
\renewcommand{\index}[1]{}
\newcommand{\myindex}[1]{#1}







\begin{document}
\begin{abstract}
Why?
\end{abstract}
\maketitle



\begin{exercise}
    Consider the two-dimensional system of differential equation written in polar coordinates as
    \[ 
        \frac{dr}{dt} = r(r-1)(r-4)^2 \qquad \frac{d\theta}{dt} = 1. 
    \] 
    Determine all limit cycles, periodic solutions, and classify the stability of each of these solutions. 
    $r = \answer{0}$ is asymptotically stable, $r = \answer{1}$ is a periodic solution, but not a limit cycle, is unstable, $r = \answer{4}$ is semistable, limit cycle from the inside.
\end{exercise}
%\comboSol
%{%
%$r=0$ is asymptotically stable, $r=1$ is a periodic solution, but not a limit cycle, is unstable, $r=4$ is semistable, limit cycle from the inside.
%}

\begin{exercise}
    Consider the two-dimensional system of differential equation written in polar coordinates as
    \[ 
        \frac{dr}{dt} = r^2(r-1)^2(r-3) \qquad \frac{d\theta}{dt} = -1. 
    \] 
    Determine all limit cycles, periodic solutions, and classify the stability of each of these solutions. 
    $r = \answer{0}$ asymptoticallys table, $r = \answer{1}$ semistable limit cycle, $r = \answer{3}$ periodic solution, unstable.
\end{exercise}
%\comboSol
%{%
%$r=0$ asymptoticallys table, $r=1$ semistable limit cycle, $r=3$ periodic solution, unstable.
%}

\begin{exercise}%
    Consider the system of differential equation given by
    \[ 
        \frac{dx}{dt} = x(3- 2y^2 - x^2) \qquad \frac{dy}{dt} = y(3-y^2) .
    \]
    Find and classify all limit cycles by converting to an autonomous equation in $r = \sqrt{x^2 + y^2}$ or $s = x^2 + y^2$. \\
    $\left(\answer{0},\answer{0}\right)$ is unstable, $r = \answer{\sqrt{3}}$ is asymptotically stable.
\end{exercise}
%\exsol{%
%$(0,0)$, unstable, $r = \sqrt{3}$, asymptotically stable.
%}

\begin{exercise}%
    Consider the system of differential equation given by
    \[ 
        \frac{dx}{dt} = -x(x^2 + y^2)^2 + 6x(x^2 + y^2) - 8x + 6y \qquad \frac{dy}{dt} = -y(x^2 + y^2)^2 + 6y(x^2 + y^2) - 8y - 6x .
    \]
    Find and classify all limit cycles by converting to an autonomous equation in $r = \sqrt{x^2 + y^2}$ or $s = x^2 + y^2$. \\
    $\left(\answer{0},\answer{0}\right)$ is asymptotically stable, $r = \answer{\sqrt{2}}$ is unstable, $r = \answer{2}$ is asymptotically stable. 
\end{exercise}
%\exsol{%
%$(0,0)$, asymptotically stable, $r = \sqrt{2}$, unstable, $r = 2$, asymptotically stable. 
%}

\begin{exercise}
    Consider the system %5.4
    \begin{equation}
        \begin{bmatrix}
            \dfrac{dx}{dt}=x+2y+x(x^2+y^2-2\sqrt{x^2+y^2})\\[6pt]
            \dfrac{dy}{dt}=-2x+y+y(x^2+y^2-2\sqrt{x^2+y^2})
        \end{bmatrix}. \label{eq:LimitCycleExercise1}
    \end{equation}
    \begin{itemize}
        \item Use polar coordinates to write $\dfrac{dr}{dt}$ as a function of $r$.\\
            $\frac{dr}{dt} = \answer{r(r-1)^2}$
        \item Draw the phase line of the DE $\dfrac{dr}{dt}=f(r)$, where $f(r)$ is the function from part a.
        \item Does the system \eqref{eq:LimitCycleExercise1} have a limit cycle? If so, find it. If not, explain why not. For each positive root of $f(r)$, decide whether the corresponding trajectory one is stable, unstable, or semistable.\\
        Limit cycle at $r = \answer{1}$, which is semistable.
    \end{itemize}
\end{exercise}
%\comboSol
%{%
%a)~$\frac{dr}{dt} = r(r-1)^2$ \quad c)~Limit cycle at $r=1$, it is semistable.
%}

\begin{exercise}
    Show that the following systems have no closed trajectories.
    \begin{itemize}
        \item $x'=x^3+y,\quad y'=y^3+x^2$, $\answer{f_x + g_y = 3(x^2 + y^2)} > \answer{0}$
        \item $x'=e^{x-y},\quad y'=e^{x+y}$, $f_x$, $g_y > \answer{0}$
        \item $x'=x+3y^2-y^3,\quad y'=y^3+x^2$. $f_x + g_y = \answer{1 + 3y^2} > 0$
    \end{itemize}
\end{exercise}
%\comboSol
%{%
%a)~$f_x + g_y = 3(x^2 + y^2) > 0$ \quad b)~$f_x$, $g_y > 0$ \quad c)~$f_x + g_y = 1+3y^2 > 0$
%}

\begin{exercise}%
    Show that the following systems have no closed trajectories.
    \begin{itemize}
        \item $x'=x+y^2,\quad y'=y+x^2$, $f_x + g_y = \answer{2} > 0$
        \item $x'=-x\sin^2(y),\quad y'=e^x$, $f_x+g_y = \answer{-(\sin(y))^2} < 0$
        \item $x'=xy,\quad y'=x+x^2$. $f_x + g_y = \answer{y} > 0$
    \end{itemize}
\end{exercise}
%\exsol{%
%Use Bendixson--Dulac Theorem.
%a) $f_x+g_y = 1+1 > 0$, so no closed trajectories.
%b) $f_x+g_y = -\sin^2(y)+0 < 0$ for all $x,y$ except the lines
%given by $y=k\pi$ (where we get zero), so no closed trajectories.
%c) $f_x+g_y = y + 0 > 0$ for all $x,y$ except the line
%given by $y=0$ (where we get zero), so no closed trajectories.
%}

\begin{exercise}%
    Suppose an autonomous system in the plane has a solution $x=\cos(t)+e^{-t}$, $y=\sin(t)+e^{-t}$.  What can you say about the system (in particular about limit cycles and periodic solutions)?
    \begin{multipleChoice}
        \choice{The system has no limit cycle.}
        \choice[correct]{The system has a limit cycle.}
    \end{multipleChoice}
    \begin{problem}
        The periodic solution is: $x = \answer{\cos(t)+e^{-t}}$, $y= \answer{\sin(t)+e^{-t}}$
    \end{problem}
\end{exercise}
%\exsol{%
%Using Poincar\'e--Bendixson Theorem,
%the system has a limit cycle, which is the unit circle centered at the origin as
%$x=\cos(t)+e^{-t}$, $y=\sin(t)+e^{-t}$ gets closer and closer to the unit
%circle.  Thus we also have that $x=\cos(t)$, $y=\sin(t)$ is the periodic
%solution.
%}

\begin{exercise}
    Formulate a condition for a 2-by-2 linear system ${\vec{x}}' = A \vec{x}$ to not be a center using the Bendixson--Dulac theorem. That is, the theorem says something about certain elements of $A$.\\
    $\answer{a+d} \neq \answer{0}$.
\end{exercise}
%\comboSol
%{%
%$a+d \neq 0$
%}

\begin{exercise}
    Explain why the Bendixson--Dulac Theorem does not apply for any conservative system $x''+h(x) = 0$.\\
    $f_x = \answer{0}$, $g_y = \answer{0}$
\end{exercise}
%\comboSol
%{%
%$f_x = 0$, $g_y = 0$, so it's always zero.
%}

\begin{exercise}
    A system such as $x'=x, y'=y$ has solutions that exist for all time $t$, yet there are no closed trajectories.  Explain why the Poincar\'e--Bendixson Theorem does not apply.
\end{exercise}
%\comboSol
%{%
%The solutions are not bounded.
%}

\begin{exercise}%
    Show that the limit cycle of the  Van der Pol oscillator (for $\mu > 0$) must not lie completely in the set where $-1 < x < 1$. Compare with figure \ref{fig:nlin-van-der-fig}.\\
    $f(x,y) = \answer{y}$, $g(x,y) = \mu\answer{(1 - x^2)y} + \answer{-x}$.  So $f_x + g_y = \mu\answer{(1-x^2)}$
\end{exercise}
%\exsol{%
%$f(x,y) = y$, $g(x,y) = \mu(1-x^2)y-x$.  So
%$f_x+g_y = \mu(1-x^2)$.  The Bendixson--Dulac Theorem
%says there is no closed trajectory lying entirely in the set $x^2 < 1$.
%}

\begin{exercise}
    Differential equations can also be given in different coordinate systems. Suppose we have the system $r' = 1-r^2$, $\theta' = 1$ given in polar coordinates.  Find all the closed trajectories and check if they are limit cycles and if so, if they are asymptotically stable or not. \\
    $r = \answer{0}$ is unstable, $r = \answer{1}$ is asymptotically stable limit cycle
\end{exercise}
%\comboSol
%{%
%$r=0$ is unstable, $r=1$ is asymptotically stable limit cycle
%}

\begin{exercise}%
    Suppose we have the system $r' = \sin(r)$, $\theta' = 1$ given in polar coordinates.  Find all the closed trajectories.
\end{exercise}
%\exsol{%
%The closed trajectories are those where $\sin(r) = 0$, therefore,
%all the circles centered at the origin with radius that
%is a multiple of $\pi$ are closed
%trajectories.
%}


%
%\setcounter{exercise}{100}

\end{document}