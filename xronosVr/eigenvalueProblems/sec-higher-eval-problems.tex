\documentclass{ximera}
%\auor{Matthew Charnley and Jason Nowell}
\usepackage[margin=1.5cm]{geometry}
\usepackage{indentfirst}
\usepackage{sagetex}
\usepackage{lipsum}
\usepackage{amsmath}
\usepackage{mathrsfs}


%%% Random packages added without verifying what they are really doing - just to get initial compile to work.
\usepackage{tcolorbox}
\usepackage{hypcap}
\usepackage{booktabs}%% To get \toprule,\midrule,\bottomrule etc.
\usepackage{nicefrac}
\usepackage{caption}
\usepackage{units}

% This is my modified wrapfig that doesn't use intextsep
\usepackage{mywrapfig}
\usepackage{import}



%%% End to random added packages.


\graphicspath{
    {./figures/}
    {./../figures/}
    {./../../figures/}
}
\renewcommand{\log}{\ln}%%%%
\DeclareMathOperator{\arcsec}{arcsec}
%% New commands


%%%%%%%%%%%%%%%%%%%%
% New Conditionals %
%%%%%%%%%%%%%%%%%%%%


% referencing
\makeatletter
    \DeclareRobustCommand{\myvref}[2]{%
      \leavevmode%
      \begingroup
        \let\T@pageref\@pagerefstar
        \hyperref[{#2}]{%
	  #1~\ref*{#2}%
        }%
        \vpageref[\unskip]{#2}%
      \endgroup
    }%

    \DeclareRobustCommand{\myref}[2]{%
      \leavevmode%
      \begingroup
        \let\T@pageref\@pagerefstar
        \hyperref[{#2}]{%
	  #1~\ref*{#2}%
        }%
      \endgroup
    }%
\makeatother

\newcommand{\figurevref}[1]{\myvref{Figure}{#1}}
\newcommand{\figureref}[1]{\myref{Figure}{#1}}
\newcommand{\tablevref}[1]{\myvref{Table}{#1}}
\newcommand{\tableref}[1]{\myref{Table}{#1}}
\newcommand{\chapterref}[1]{\myref{chapter}{#1}}
\newcommand{\Chapterref}[1]{\myref{Chapter}{#1}}
\newcommand{\appendixref}[1]{\myref{appendix}{#1}}
\newcommand{\Appendixref}[1]{\myref{Appendix}{#1}}
\newcommand{\sectionref}[1]{\myref{\S}{#1}}
\newcommand{\subsectionref}[1]{\myref{subsection}{#1}}
\newcommand{\subsectionvref}[1]{\myvref{subsection}{#1}}
\newcommand{\exercisevref}[1]{\myvref{Exercise}{#1}}
\newcommand{\exerciseref}[1]{\myref{Exercise}{#1}}
\newcommand{\examplevref}[1]{\myvref{Example}{#1}}
\newcommand{\exampleref}[1]{\myref{Example}{#1}}
\newcommand{\thmvref}[1]{\myvref{Theorem}{#1}}
\newcommand{\thmref}[1]{\myref{Theorem}{#1}}


\renewcommand{\exampleref}[1]{ {\color{red} \bfseries Normally a reference to a previous example goes here.}}
\renewcommand{\figurevref}[1]{ {\color{red} \bfseries Normally a reference to a previous figure goes here.}}
\renewcommand{\tablevref}[1]{ {\color{red} \bfseries Normally a reference to a previous table goes here.}}
\renewcommand{\Appendixref}[1]{ {\color{red} \bfseries Normally a reference to an Appendix goes here.}}
\renewcommand{\exercisevref}[1]{ {\color{red} \bfseries Normally a reference to a previous exercise goes here.}}



\newcommand{\R}{\mathbb{R}}

%% Example Solution Env.
\def\beginSolclaim{\par\addvspace{\medskipamount}\noindent\hbox{\bf Solution:}\hspace{0.5em}\ignorespaces}
\def\endSolclaim{\par\addvspace{-1em}\hfill\rule{1em}{0.4pt}\hspace{-0.4pt}\rule{0.4pt}{1em}\par\addvspace{\medskipamount}}
\newenvironment{exampleSol}[1][]{\beginSolclaim}{\endSolclaim}

%% General figure formating from original book.
\newcommand{\mybeginframe}{%
\begin{tcolorbox}[colback=white,colframe=lightgray,left=5pt,right=5pt]%
}
\newcommand{\myendframe}{%
\end{tcolorbox}%
}

%%% Eventually return and fix this to make matlab code work correctly.
%% Define the matlab environment as another code environment
%\newenvironment{matlab}
%{% Begin Environment Code
%{ \centering \bfseries Matlab Code }
%\begin{code}
%}% End of Begin Environment Code
%{% Start of End Environment Code
%\end{code}
%}% End of End Environment Code


% this one should have a caption, first argument is the size
\newenvironment{mywrapfig}[2][]{
 \wrapfigure[#1]{r}{#2}
 \mybeginframe
 \centering
}{%
 \myendframe
 \endwrapfigure
}

% this one has no caption, first argument is size,
% the second argument is a larger size used for HTML (ignored by latex)
\newenvironment{mywrapfigsimp}[3][]{%
 \wrapfigure[#1]{r}{#2}%
 \centering%
}{%
 \endwrapfigure%
}
\newenvironment{myfig}
    {%
    \begin{figure}[h!t]
        \mybeginframe%
        \centering%
    }
    {%
        \myendframe
    \end{figure}%
    }


% graphics include
\newcommand{\diffyincludegraphics}[3]{\includegraphics[#1]{#3}}
\newcommand{\myincludegraphics}[3]{\includegraphics[#1]{#3}}
\newcommand{\inputpdft}[1]{\subimport*{../figures/}{#1.pdf_t}}


%% Not sure what these even do? They don't seem to actually work... fun!
%\newcommand{\mybxbg}[1]{\tcboxmath[colback=white,colframe=black,boxrule=0.5pt,top=1.5pt,bottom=1.5pt]{#1}}
%\newcommand{\mybxsm}[1]{\tcboxmath[colback=white,colframe=black,boxrule=0.5pt,left=0pt,right=0pt,top=0pt,bottom=0pt]{#1}}
\newcommand{\mybxsm}[1]{#1}
\newcommand{\mybxbg}[1]{#1}

%%% Something about tasks for practice/hw?
\usepackage{tasks}
\usepackage{footnote}
\makesavenoteenv{tasks}


%% For pdf only?
\newcommand{\diffypdfversion}[1]{#1}


%% Kill ``cite'' and go back later to fix it.
\renewcommand{\cite}[1]{}


%% Currently we can't really use index or its derivatives. So we are gonna kill them off.
\renewcommand{\index}[1]{}
\newcommand{\myindex}[1]{#1}






\title{Higher order eigenvalue problems}
\author{Matthew Charnley and Jason Nowell}


%\outcome{Use power series methods to solve second order linear ODEs near ordinary points}
%\outcome{Write a recurrence relation for the coefficients in a power series solution to an ODE.}


\begin{document}
\begin{abstract}
    We discuss Higher order eigenvalue problems
\end{abstract}
\maketitle

\label{sec:appeig}

%\sectionnotes{Verbatim from Lebl}
%
%\sectionnotes{1 lecture\EPref{, \S10.2 in \cite{EP}}\BDref{,exercises in \S11.2 in \cite{BD}}}

The eigenfunction series can arise even from higher order equations. Consider an elastic beam (say made of steel).  We will study the transversal vibrations of the beam.  That is, suppose the beam lies along the $x$-axis and let $y(x,t)$ measure the displacement of the point $x$ on the beam at time $t$.  See \figurevref{appeig:transbeamfig}.

\begin{myfig}
    \capstart
    \input{figures/trans-beam.pdf_t}
    \caption{Transversal vibrations of a beam.\label{appeig:transbeamfig}}
\end{myfig}

The equation that governs this setup is
\begin{equation*}
    a^4 \frac{\partial^4 y}{\partial x^4} + \frac{\partial^2 y}{\partial t^2} = 0,
\end{equation*}
for some constant $a > 0$, let us not worry about the physics%
\footnote{%
    If you are interested, $a^4 = \frac{EI}{\rho}$, where $E$ is the elastic modulus, $I$ is the second moment of area of the cross section, and $\rho$ is linear density.%
    }.%

Suppose the beam is of length 1 simply supported (hinged) at the ends. The beam is displaced by some function $f(x)$ at time $t=0$ and then let go (initial velocity is 0).  Then $y$ satisfies:
\begin{equation} 
    \label{appeig:beameq}
    \begin{aligned}
        & a^4 y_{xxxx} + y_{tt} = 0 \qquad (0 < x < 1, \enspace t > 0), \\
        & y(0,t) = y_{xx}(0,t) = 0 , \\
        & y(1,t) = y_{xx}(1,t) = 0 , \\
        & y(x,0) = f(x), \qquad y_{t}(x,0) = 0 .
    \end{aligned}
\end{equation}

Again we try $y(x,t) = X(x)T(t)$ and plug in to get $a^4 X^{(4)}T + XT'' = 0$ or 
\begin{equation*}
    \frac{X^{(4)}}{X} = \frac{- T''}{a^4T} = \lambda .
\end{equation*}
The equations are
\begin{equation*}
    T'' + \lambda a^4 T = 0, \qquad X^{(4)} - \lambda X = 0 .
\end{equation*}
The boundary conditions $y(0,t) = y_{xx}(0,t) = 0$ and $y(1,t) = y_{xx}(1,t) = 0$ imply
\begin{equation*}
    X(0) =  X''(0) = 0, \qquad \text{and} \qquad X(1) =  X''(1) = 0 .
\end{equation*}
and the initial homogeneous condition $y_t(x,0) = 0$ implies
\begin{equation*}
    T'(0) = 0 .
\end{equation*}
As usual, we leave the nonhomogeneous $y(x,0) = f(x)$ for later.

Considering the equation for $T$, that is, $T'' + \lambda a^4 T = 0$, and physical intuition leads us to the fact that if $\lambda$ is an eigenvalue then $\lambda > 0$. This is because we expect vibration and not exponential growth nor decay in the $t$ direction (there is no friction in our model for instance). So there are no negative eigenvalues. Similarly $\lambda = 0$ is not an eigenvalue.

\begin{exercise}
    Justify $\lambda > 0$ just from the equation for $X$ and the boundary conditions.
\end{exercise}

Let $\omega = \sqrt[4]{\lambda}$, that is $\omega^4 = \lambda$, so that we do not need to write the fourth root all the time.  Notice $\omega > 0$. The equation $X^{(4)} - \omega^4 X = 0$ has the general solution is
\begin{equation*}
    X(x) = A e^{\omega x} + B e^{-\omega x} + C \sin (\omega x) + D \cos (\omega x) .
\end{equation*}
Now $0 = X(0) = A+B+D$, $0 = X''(0) = \omega^2 (A + B - D)$.  Hence, $D = 0$ and $A+B = 0$, or $B = - A$.  So we have
\begin{equation*}
    X(x) = A e^{\omega x} - A e^{-\omega x} + C \sin (\omega x) .
\end{equation*}
Also $0 = X(1) = A (e^{\omega} - e^{-\omega}) + C \sin \omega$, and $0 = X''(1) = A \omega^2 (e^{\omega} - e^{-\omega}) - C \omega^2 \sin \omega$. This means that $C \sin \omega  = 0$ and $A (e^{\omega} - e^{-\omega}) = 2 A \sinh \omega = 0$.  If $\omega > 0$, then $\sinh \omega \not= 0$ and so $A = 0$.  This means that $C \not=0$ otherwise $\lambda$ is not an eigenvalue.  Also $\omega$ must be an integer multiple of $\pi$.   Hence $\omega = n \pi$ and $n \geq 1$ (as $\omega > 0$).  We can take $C=1$.  So the eigenvalues are $\lambda_n = n^4 \pi^4$ and corresponding eigenfunctions are $\sin (n \pi x)$.

Now $T'' + n^4 \pi^4 a^4 T = 0$.  The general solution is $T(t) = A \sin (n^2 \pi^2 a^2 t) + B \cos (n^2 \pi^2 a^2 t)$.  But $T'(0) = 0$ and hence we must have $A=0$.  We can take $B=1$ to make $T(0) = 1$ for convenience. So our solutions are $T_n(t) = \cos (n^2 \pi^2 a^2 t)$.

As eigenfunctions are just sines again, we decompose the function $f(x)$ on $0 < x < 1$ using the sine series. We find numbers $b_n$ such that for $0 < x < 1$ we have
\begin{equation*}
    f(x) = \sum_{n=1}^\infty b_n \sin (n \pi x) .
\end{equation*}
Then the solution to \eqref{appeig:beameq} is
\begin{equation*}
    y(x,t) = \sum_{n=1}^\infty b_n X_n(x) T_n(t) = \sum_{n=1}^\infty b_n \sin (n \pi x)  \cos ( n^2 \pi^2 a^2 t ) .
\end{equation*}
The point is that $X_nT_n$ is a solution that satisfies all the homogeneous conditions (that is, all conditions except the initial position).  And since $T_n(0) = 1$, we have
\begin{equation*}
    y(x,0) = \sum_{n=1}^\infty b_n X_n(x) T_n(0) = \sum_{n=1}^\infty b_n X_n(x) = \sum_{n=1}^\infty b_n \sin (n \pi x) = f(x) .
\end{equation*}
So $y(x,t)$ solves \eqref{appeig:beameq}.

The natural (angular) frequencies of the system are $n^2 \pi^2 a^2$. These frequencies are all integer multiples of the fundamental frequency $\pi^2 a^2$, so we get a nice musical note.  The exact frequencies and their amplitude are what musicians call the \emph{\myindex{timbre}} of the note (outside of music it is called the spectrum).

The timbre of a beam is different than for a vibrating string where we get ``more'' of the lower frequencies since we get all integer multiples, $1,2,3,4,5,\ldots$.  For a steel beam we get only the square multiples $1,4,9,16,25,\ldots$.  That is why when you hit a steel beam you hear a very pure sound.  The sound of a xylophone or vibraphone is, therefore, very different from a guitar or piano.

\begin{example}
    Let us assume that $f(x) = \frac{x(x-1)}{10}$. On $0 < x < 1$ we have (you know how to do this by now)
    \begin{equation*}
    f(x) = \sum_{\substack{n=1\\n \text{~odd}}}^\infty \frac{4}{5\pi^3 n^3} \sin (n \pi x) .
    \end{equation*}
    Hence, the solution to \eqref{appeig:beameq} with the given initial position $f(x)$ is
    \begin{equation*}
    y(x,t) = \sum_{\substack{n=1\\n \text{~odd}}}^\infty \frac{4}{5\pi^3 n^3} \sin (n \pi x) \cos ( n^2 \pi^2 a^2 t ) .
    \end{equation*}
\end{example}

There are other boundary conditions than just hinged ends.  There are three basic possibilities: hinged, free, or fixed.  Let us consider the end at $x=0$.  For the other end, it is the same idea. If the end is \emph{hinged}\index{hinged end of beam}, then
\begin{equation*}
    u(0,t) = u_{xx}(0,t) = 0 .
\end{equation*}
If the end is \emph{free}\index{free end of beam}, that is, it is just floating in air, then
\begin{equation*}
    u_{xx}(0,t) = u_{xxx}(0,t) = 0 .
\end{equation*}
And finally, if the end is \emph{clamped}\index{clamped end of beam} or \emph{fixed}\index{fixed end of beam}, for example it is welded to a wall, then
\begin{equation*}
    u(0,t) = u_{x}(0,t) = 0 .
\end{equation*}


\end{document}