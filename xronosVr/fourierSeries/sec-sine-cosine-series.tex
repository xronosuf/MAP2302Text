\documentclass{ximera}
%\auor{Matthew Charnley and Jason Nowell}
\usepackage[margin=1.5cm]{geometry}
\usepackage{indentfirst}
\usepackage{sagetex}
\usepackage{lipsum}
\usepackage{amsmath}
\usepackage{mathrsfs}
\usepackage{tikz}
\usetikzlibrary{matrix}

%%% Random packages added without verifying what they are really doing - just to get initial compile to work.
\usepackage{tcolorbox}
\usepackage{hypcap}
\usepackage{booktabs}%% To get \toprule,\midrule,\bottomrule etc.
\usepackage{caption}
\usepackage{units}
\usepackage{multicol}
\usepackage{hhline}


% This is my modified wrapfig that doesn't use intextsep
\usepackage{mywrapfig}
\usepackage{import}



%%% End to random added packages.


\graphicspath{
    {./}
    {./figures/}
    {./../figures/}
    {./../../figures/}
}
\renewcommand{\log}{\ln}%%%%
\DeclareMathOperator{\arcsec}{arcsec}
%% New commands


%%%%%%%%%%%%%%%%%%%%
% New Conditionals %
%%%%%%%%%%%%%%%%%%%%


% referencing
\makeatletter
    \DeclareRobustCommand{\myvref}[2]{%
      \leavevmode%
      \begingroup
        \let\T@pageref\@pagerefstar
        \hyperref[{#2}]{%
	  #1~\ref*{#2}%
        }%
        \vpageref[\unskip]{#2}%
      \endgroup
    }%

    \DeclareRobustCommand{\myref}[2]{%
      \leavevmode%
      \begingroup
        \let\T@pageref\@pagerefstar
        \hyperref[{#2}]{%
	  #1~\ref*{#2}%
        }%
      \endgroup
    }%
\makeatother

\newcommand{\figurevref}[1]{\myvref{Figure}{#1}}
\newcommand{\figureref}[1]{\myref{Figure}{#1}}
\newcommand{\tablevref}[1]{\myvref{Table}{#1}}
\newcommand{\tableref}[1]{\myref{Table}{#1}}
\newcommand{\chapterref}[1]{\myref{chapter}{#1}}
\newcommand{\Chapterref}[1]{\myref{Chapter}{#1}}
\newcommand{\appendixref}[1]{\myref{appendix}{#1}}
\newcommand{\Appendixref}[1]{\myref{Appendix}{#1}}
\newcommand{\sectionref}[1]{\myref{\S}{#1}}
\newcommand{\subsectionref}[1]{\myref{subsection}{#1}}
\newcommand{\subsectionvref}[1]{\myvref{subsection}{#1}}
\newcommand{\exercisevref}[1]{\myvref{Exercise}{#1}}
\newcommand{\exerciseref}[1]{\myref{Exercise}{#1}}
\newcommand{\examplevref}[1]{\myvref{Example}{#1}}
\newcommand{\exampleref}[1]{\myref{Example}{#1}}
\newcommand{\thmvref}[1]{\myvref{Theorem}{#1}}
\newcommand{\thmref}[1]{\myref{Theorem}{#1}}


\renewcommand{\exampleref}[1]{ {\color{red} \bfseries Normally a reference to a previous example goes here.}}
\renewcommand{\examplevref}[1]{ {\color{red} \bfseries Normally a reference to a previous example goes here.}}
\renewcommand{\figurevref}[1]{ {\color{red} \bfseries Normally a reference to a previous figure goes here.}}
\renewcommand{\tablevref}[1]{ {\color{red} \bfseries Normally a reference to a previous table goes here.}}
\renewcommand{\Appendixref}[1]{ {\color{red} \bfseries Normally a reference to an Appendix goes here.}}
\renewcommand{\exercisevref}[1]{ {\color{red} \bfseries Normally a reference to a previous exercise goes here.}}
\renewcommand{\thmvref}[1]{ {\color{red} \bfseries Normally a reference to a previous theorem goes here.}}
\renewcommand{\subsectionvref}[1]{ {\color{red} \bfseries Normally a reference to a previous subsection goes here.}}



\newcommand{\R}{\mathbb{R}}
\newcommand{\C}{\mathbb{C}}

%% Example Solution Env.
\def\beginSolclaim{\par\addvspace{\medskipamount}\noindent\hbox{\bf Solution:}\hspace{0.5em}\ignorespaces}
\def\endSolclaim{\par\addvspace{-1em}\hfill\rule{1em}{0.4pt}\hspace{-0.4pt}\rule{0.4pt}{1em}\par\addvspace{\medskipamount}}
\newenvironment{exampleSol}[1][]{\beginSolclaim}{\endSolclaim}

%% General figure formating from original book.
\newcommand{\mybeginframe}{%
\begin{tcolorbox}[colback=white,colframe=lightgray,left=5pt,right=5pt]%
}
\newcommand{\myendframe}{%
\end{tcolorbox}%
}

%%% Eventually return and fix this to make matlab code work correctly.
%% Define the matlab environment as another code environment
%\NewEnviron{matlab}{ {\centering\bfseries MATLAB Code} \\ \noexpand{\BODY} }
%\let\beginmatlab\begincode
%\let\endmatlab\endcode
%\newenvironment{matlab}{% Begin Environment Code
%\begin{minipage}{\linewidth}
%\begin{verbatim}
%}% End of Begin Environment Code
%{% Start of End Environment Code
%\end{verbatim}
%\end{minipage}
%}% End of End Environment Code


% this one should have a caption, first argument is the size
\newenvironment{mywrapfig}[2][]{
 \wrapfigure[#1]{r}{#2}
 \mybeginframe
 \centering
}{%
 \myendframe
 \endwrapfigure
}

% this one has no caption, first argument is size,
% the second argument is a larger size used for HTML (ignored by latex)
\newenvironment{mywrapfigsimp}[3][]{%
 \wrapfigure[#1]{r}{#2}%
 \centering%
}{%
 \endwrapfigure%
}
\newenvironment{myfig}
    {%
    \begin{figure}[h!t]
        \mybeginframe%
        \centering%
    }
    {%
        \myendframe
    \end{figure}%
    }


% graphics include
\newcommand{\diffyincludegraphics}[3]{\includegraphics[#1]{#3}}
\newcommand{\myincludegraphics}[3]{\includegraphics[#1]{#3}}
\newcommand{\inputpdft}[1]{\subimport*{../figures/}{#1.pdf_t}}


%% Not sure what these even do? They don't seem to actually work... fun!
%\newcommand{\mybxbg}[1]{\tcboxmath[colback=white,colframe=black,boxrule=0.5pt,top=1.5pt,bottom=1.5pt]{#1}}
%\newcommand{\mybxsm}[1]{\tcboxmath[colback=white,colframe=black,boxrule=0.5pt,left=0pt,right=0pt,top=0pt,bottom=0pt]{#1}}
\newcommand{\mybxsm}[1]{#1}
\newcommand{\mybxbg}[1]{#1}

%%% Something about tasks for practice/hw?
\usepackage{tasks}
\usepackage{footnote}
\makesavenoteenv{tasks}


%% For pdf only?
\newcommand{\diffypdfversion}[1]{#1}


%% Kill ``cite'' and go back later to fix it.
\renewcommand{\cite}[1]{}


%% Currently we can't really use index or its derivatives. So we are gonna kill them off.
\renewcommand{\index}[1]{}
\newcommand{\myindex}[1]{#1}






\title{Sine and cosine series}
\author{Matthew Charnley and Jason Nowell}


\outcome{Use sine and cosine series to represent odd and even periodic extensions of functions}
\outcome{Understand the connection between Fourier series and boundary value problems.}


\begin{document}
\begin{abstract}
    We discuss Sine and cosine series
\end{abstract}
\maketitle


\label{sec:scs}


% \sectionnotes{Verbatim from Lebl}

% \sectionnotes{2 lectures\EPref{, \S9.3 in \cite{EP}}\BDref{,
% \S10.4 in \cite{BD}}}

\subsection{Odd and even periodic functions}

You may have noticed by now that an odd function has no cosine terms in the Fourier series and an even function has no sine terms in the Fourier series. This observation is not a coincidence.  Let us look at even and odd periodic function in more detail.

Recall that a function $f(t)$ is \emph{odd}\index{odd function} if $f(-t) = -f(t)$.  A function $f(t)$ is \emph{even}\index{even function} if $f(-t) = f(t)$.  For example, $\cos (n t)$ is even and $\sin (n t)$ is odd. Similarly the function $t^k$ is even if $k$ is even and odd when $k$ is odd.

\begin{exercise}
    Take two functions $f(t)$ and $g(t)$ and define their product $h(t) = f(t)g(t)$.
    \begin{tasks}
        \task Suppose both $f(t)$ and $g(t)$ are odd.  Is $h(t)$ odd or even?
        \task Suppose one is even and one is odd.  Is $h(t)$ odd or even?
        \task Suppose both are even.  Is $h(t)$ odd or even?
    \end{tasks}
\end{exercise}

If $f(t)$ and $g(t)$ are both odd, then $f(t)+g(t)$ is odd.  Similarly for even functions.  On the other hand, if $f(t)$ is odd and $g(t)$ even, then we cannot say anything about the sum $f(t) + g(t)$.  In fact, the Fourier series of any function is a sum of an odd (the sine terms) and an even (the cosine terms) function.

In this section we consider odd and even periodic functions.  We have previously defined the $2L$-periodic extension of a function defined on the interval $[-L,L]$.  Sometimes we are only interested in the function on the range $[0,L]$ and it would be convenient to have an odd (resp.\ even) function.  If the function is odd (resp.\ even), all the cosine (resp.\ sine) terms disappear. What we will do is take the odd (resp.\ even) extension of the function to $[-L,L]$ and then extend periodically to a $2L$-periodic function.

Take a function $f(t)$ defined on $[0,L]$.  On $(-L,L]$ define the functions
\begin{align*}
    F_{\text{odd}}(t) & \overset{\text{def}}{=}
    \begin{cases}
        f(t) & \text{if } \; \phantom{-}0 \leq t \leq L , \\
        -f(-t) & \text{if } \; {-L} < t < 0 ,
    \end{cases} \\
    F_{\text{even}}(t) & \overset{\text{def}}{=}
    \begin{cases}
        f(t) & \text{if } \; \phantom{-}0 \leq t \leq L , \\
        f(-t) & \text{if } \; {-L} < t < 0 .
    \end{cases}
\end{align*}
Extend $F_{\text{odd}}(t)$ and $F_{\text{even}}(t)$ to be $2L$-periodic. Then $F_{\text{odd}}(t)$ is called the \emph{\myindex{odd periodic extension}} of $f(t)$, and $F_{\text{even}}(t)$ is called the \emph{\myindex{even periodic extension}} of $f(t)$. For the odd extension we generally assume that $f(0) = f(L) = 0$.

\begin{exercise}
    Check that $F_{\text{odd}}(t)$ is odd and $F_{\text{even}}(t)$ is even. For $F_{\text{odd}}$, assume $f(0) = f(L) = 0$.
\end{exercise}

\begin{example}
    Take the function $f(t) = t\,(1-t)$ defined on $[0,1]$. \figurevref{scs:oddevenextfig} shows the plots of the odd and even periodic extensions of $f(t)$.
    
    \begin{myfig}
        \capstart
        %original files scs-oddext scs-evenext
        \diffyincludegraphics{width=6.24in}{width=9in}{scs-ext-odd-even}
        \caption{Odd and even 2-periodic extension of $f(t) = t\,(1-t)$, $0 \leq t \leq 1$.\label{scs:oddevenextfig}}
    \end{myfig}
\end{example}

\subsection{Sine and cosine series}

Let $f(t)$ be an odd $2L$-periodic function.  We write the Fourier series for $f(t)$.  First, we compute the coefficients $a_n$ (including $n=0$) and get
\begin{equation*}
    a_n = \frac{1}{L} \int_{-L}^L f(t) \cos \left( \frac{n \pi}{L} t \right) \, dt = 0 .
\end{equation*}
That is, there are no cosine terms in the Fourier series of an odd function. The integral is zero because $f(t) \cos \left( {n \pi}{L} t \right)$ is an odd function (product of an odd and an even function is odd) and the integral of an odd function over a symmetric interval is always zero. The integral of an even function over a symmetric interval $[-L,L]$ is twice the integral of the function over the interval $[0,L]$. The function $f(t) \sin \left( \frac{n \pi}{L} t \right)$ is the product of two odd functions and hence is even.
\begin{equation*}
    b_n = \frac{1}{L} \int_{-L}^L f(t) \sin \left( \frac{n \pi}{L} t \right) \, dt = \frac{2}{L} \int_{0}^L f(t) \sin \left( \frac{n \pi}{L} t \right) \, dt .
\end{equation*}
We now write the Fourier series of $f(t)$ as
\begin{equation*}
    \sum_{n=1}^\infty b_n \sin \left( \frac{n \pi}{L} t \right) .
\end{equation*}

Similarly, if $f(t)$ is an even $2L$-periodic function.  For the same exact reasons as above, we find that $b_n = 0$ and
\begin{equation*}
    a_n = \frac{2}{L} \int_{0}^L f(t) \cos \left( \frac{n \pi}{L} t \right) \, dt .
\end{equation*}
The formula still works for $n=0$, in which case it becomes
\begin{equation*}
    a_0 = \frac{2}{L} \int_{0}^L f(t) \, dt .
\end{equation*}
The Fourier series is then
\begin{equation*}
    \frac{a_0}{2} + \sum_{n=1}^\infty a_n \cos \left( \frac{n \pi}{L} t \right) .
\end{equation*}

An interesting consequence is that the coefficients of the Fourier series of an odd (or even) function can be computed by just integrating over the half interval $[0,L]$.  Therefore, we can compute the Fourier series of the odd (or even) extension of a function by computing certain integrals over the interval where the original function is defined.

\begin{theorem}{}
    Let $f(t)$ be a piecewise smooth function defined on $[0,L]$. Then the odd periodic extension of $f(t)$ has the Fourier series
    \begin{equation*}
        %\mybxbg{~~
        F_{\text{odd}}(t) = \sum_{n=1}^\infty b_n \sin \left( \frac{n \pi}{L} t \right) ,
        %~~}
    \end{equation*}
    where
    \begin{equation*}
        %\mybxbg{~~
        b_n = \frac{2}{L} \int_{0}^L f(t)\, \sin \left( \frac{n \pi}{L} t \right) \, dt .
        %~~}
    \end{equation*}
    The even periodic extension of $f(t)$ has the Fourier series
    \begin{equation*}
        %\mybxbg{~~
        F_{\text{even}}(t) = \frac{a_0}{2} + \sum_{n=1}^\infty a_n \cos \left(\frac{n \pi}{L} t \right) ,
        %~~}
    \end{equation*}
    where
    \begin{equation*}
        %\mybxbg{~~
        a_n = \frac{2}{L} \int_{0}^L f(t)\, \cos \left( \frac{n \pi}{L} t \right) \, dt .
        %~~}
    \end{equation*}
\end{theorem}

\begin{definition}
    We call the series $\sum_{n=1}^\infty b_n \sin \left( \frac{n \pi}{L} t\right)$ the \emph{\myindex{sine series}} of $f(t)$ and we call the series
    $\frac{a_0}{2} + \sum_{n=1}^\infty a_n \cos \left( \frac{n \pi}{L} t \right)$ 
    the \emph{\myindex{cosine series}} of $f(t)$.  
\end{definition}
We often do not actually care what happens outside of $[0,L]$.  In this case, we pick whichever series fits our problem better.

It is not necessary to start with the full Fourier series to obtain the sine and cosine series. The sine series is really the eigenfunction expansion of $f(t)$ using  eigenfunctions of the eigenvalue problem $x''+\lambda x = 0$, $x(0) = 0$, $x(L) = L$.  The cosine series is the eigenfunction expansion of $f(t)$ using eigenfunctions of the eigenvalue problem $x''+\lambda x = 0$, $x'(0) = 0$, $x'(L) = L$.  We could have, therefore, gotten the same formulas by defining the inner product
\begin{equation*}
    \langle f(t), g(t) \rangle = \int_0^L f(t) g(t) \, dt ,
\end{equation*}
and following the procedure of \sectionref{ts:section}.  This point of view is useful, as we commonly use a specific series that arose because our underlying question led to a certain eigenvalue problem.  If the eigenvalue problem is not one of the three we covered so far, you can still do an eigenfunction expansion, generalizing the results of this chapter.  %We will
%deal with such a generalization in \chapterref{SL:chapter}.

%f(t) = \frac{a_0}{2} + \sum_{n=1}^\infty a_n \cos \left( \frac{n \pi}{L} 
%t \right)
%+ b_n \sin \left( \frac{n \pi}{L} t \right) ,

\begin{example}
    Find the Fourier series of the even periodic extension of the function $f(t) = t^2$ for $0 \leq t \leq \pi$.
\end{example}

\begin{exampleSol}
    We want to write
    \begin{equation*}
        f(t) = \frac{a_0}{2} + \sum_{n=1}^\infty a_n \cos (n t) ,
    \end{equation*}
    where
    \begin{equation*}
        a_0 = \frac{2}{\pi} \int_0^\pi t^2 \, dt = \frac{2 \pi^2}{3} ,
    \end{equation*}
    and
    \begin{equation*}
        \begin{split}
            a_n & = \frac{2}{\pi} \int_0^\pi t^2 \cos (n t) \, dt
            = \frac{2}{\pi} \left[ t^2 \frac{1}{n} \sin (nt) \right]_0^\pi - \frac{4}{n\pi} \int_0^\pi t \sin (n t) \, dt \\
            & = \frac{4}{n^2\pi} \Bigl[ t \cos (n t) \Bigr]_0^\pi + \frac{4}{n^2\pi} \int_0^\pi \cos (n t) \, dt
            = \frac{4{(-1)}^n}{n^2} .
        \end{split}
    \end{equation*}
    Note that we have ``detected'' the continuity of the extension since the coefficients decay as $\frac{1}{n^2}$.  That is, the even periodic extension of $t^2$ has no jump discontinuities.  It does have corners, since the derivative, which is an odd function and a sine series, has jumps; it has a Fourier series whose coefficients decay only as $\frac{1}{n}$.
    
    Explicitly, the first few terms of the series are
    \begin{equation*}
        \frac{\pi^2}{3} - 4 \cos (t) + \cos (2t) - \frac{4}{9} \cos (3t) + \cdots
    \end{equation*}
\end{exampleSol}

\begin{exercise}
    \begin{tasks}
        \task Compute the derivative of the even periodic extension of $f(t)$ above and verify it has jump discontinuities.  Use the actual definition of $f(t)$, not its cosine series!
        \task Why is it that the derivative of the even periodic extension of $f(t)$ is the odd periodic extension of $f'(t)$?
    \end{tasks}
\end{exercise}

\subsection{Application}

Fourier series ties in to the boundary value problems we studied earlier.  Let us see this connection in an application.

Consider the boundary value problem for $0 < t < L$,
\begin{equation*}
    x''(t) + \lambda\, x(t) = f(t) ,
\end{equation*}
for the \emph{\myindex{Dirichlet boundary conditions}} $x(0) = 0$, $x(L) = 0$. The Fredholm alternative (\thmvref{thm:fredholmsimple}) says that as long as $\lambda$ is not an eigenvalue of the underlying homogeneous problem, there exists a unique solution. Eigenfunctions of this eigenvalue problem are the functions $\sin \left( \frac{n \pi}{L} t \right)$. Therefore, to find the solution, we first find the Fourier sine series for $f(t)$. We write $x$ also as a sine series, but with unknown coefficients. We substitute the series for $x$ into the equation and solve for the unknown coefficients. If we have the \emph{\myindex{Neumann boundary conditions}} $x'(0) = 0$, $x'(L) = 0$, we do the same procedure using the cosine series.

Let us see how this method works on examples.

\begin{example}
    Take the boundary value problem for $0 < t < 1$,
    \begin{equation*}
        x''(t) + 2 x(t) = f(t) ,
    \end{equation*}
    where $f(t) = t$ on $0 < t < 1$, and satisfying the Dirichlet boundary conditions $x(0) = 0$, $x(1)=0$.
\end{example}

\begin{exampleSol}
    We write $f(t)$ as a sine series
    \begin{equation*}
        f(t) = \sum_{n=1}^\infty c_n \sin (n \pi t) .
    \end{equation*}
    Compute
    \begin{equation*}
        c_n = 2 \int_0^1 t \sin (n \pi t) \,dt = \frac{2 \, {(-1)}^{n+1}}{n \pi} .
    \end{equation*}
    We write $x(t)$ as
    \begin{equation*}
        x(t) = \sum_{n=1}^\infty b_n \sin (n \pi t) .
    \end{equation*}
    We plug in to obtain 
    \begin{equation*}
        \begin{split}
            x''(t) + 2 x(t) 
            & = \underbrace{\sum_{n=1}^\infty - b_n n^2 \pi^2 \sin (n \pi t)}_{x''} + 2 \underbrace{\sum_{n=1}^\infty b_n \sin (n \pi t) }_{x} \\
            & = \sum_{n=1}^\infty b_n (2 - n^2 \pi^2 ) \sin (n \pi t) \\
            & = f(t) = \sum_{n=1}^\infty  \frac{2\, {(-1)}^{n+1}}{n \pi} \sin (n \pi t) .
        \end{split}
    \end{equation*}
    Therefore,
    \begin{equation*}
        b_n (2 - n^2 \pi^2) = \frac{2\,{(-1)}^{n+1}}{n \pi}
    \end{equation*}
    or
    \begin{equation*}
        b_n = \frac{2\,{(-1)}^{n+1}}{n \pi (2 - n^2 \pi^2)} .
    \end{equation*}
    That $2-n^2\pi^2$ is not zero for any $n$, and that we can solve for $b_n$, is precisely because $2$ is not an eigenvalue of the problem. We have thus obtained a Fourier series for the solution
    \begin{equation*}
        x(t) = \sum_{n=1}^\infty \frac{2\,{(-1)}^{n+1}}{n \pi \,(2 - n^2 \pi^2)} \sin (n \pi t) .
    \end{equation*}
    See \figurevref{bnd-dirich-graph:fig} for a graph of the solution. Notice that because the eigenfunctions satisfy the boundary conditions, and $x$ is written in terms of the boundary conditions, then $x$ satisfies the boundary conditions.
    \begin{myfig}
        \capstart
        \diffyincludegraphics{width=3in}{width=4.5in}{bnd-dirich-graph}
        \caption{Plot of the solution of $x''+2x=t$, $x(0)=0$, $x(1)=0$.%
        \label{bnd-dirich-graph:fig}}
    \end{myfig}
\end{exampleSol}

\begin{example}
    Similarly we handle the Neumann conditions. Take the boundary value problem for $0 < t < 1$,
    \begin{equation*}
        x''(t) + 2 x(t) = f(t) ,
    \end{equation*}
    where again $f(t) = t$ on $0 < t < 1$, but now satisfying the Neumann boundary conditions $x'(0) = 0$, $x'(1)=0$.
\end{example}

\begin{exampleSol}
    We write $f(t)$ as a cosine series
    \begin{equation*}
        f(t) = \frac{c_0}{2} + \sum_{n=1}^\infty c_n \cos (n \pi t) ,
    \end{equation*}
    where
    \begin{equation*}
        c_0 = 2 \int_0^1 t \,dt = 1 ,
    \end{equation*}
    and
    \begin{equation*}
        c_n = 2 \int_0^1 t \cos (n \pi t) \,dt = \frac{2\bigl({(-1)}^n-1\bigr)}{\pi^2 n^2} = 
        \begin{cases}
            \frac{-4}{\pi^2 n^2} & \text{if } n \text{ odd} , \\
            0 & \text{if } n \text{ even}.
        \end{cases}
    \end{equation*}
    We write $x(t)$ as a cosine series
    \begin{equation*}
        x(t) = \frac{a_0}{2} + \sum_{n=1}^\infty a_n \cos (n \pi t) .
    \end{equation*}
    We plug in to obtain 
    \begin{equation*}
        \begin{split}
            x''(t) + 2 x(t) 
            & = \sum_{n=1}^\infty \Bigl[ - a_n n^2 \pi^2 \cos (n \pi t) \Bigr] + a_0 + 2 \sum_{n=1}^\infty \Bigl[ a_n \cos (n \pi t) \Bigr] \\
            & = a_0 + \sum_{n=1}^\infty a_n (2 - n^2 \pi^2 ) \cos (n \pi t) \\
            & = f(t) = \frac{1}{2} + \sum_{\substack{n=1\\n~\text{odd}}}^\infty \frac{-4}{\pi^2 n^2} \cos (n \pi t) .
        \end{split}
    \end{equation*}
    Therefore, $a_0 = \frac{1}{2}$, $a_n = 0$ for $n$ even ($n \geq 2$) and for $n$ odd we have
    \begin{equation*}
        a_n (2 - n^2 \pi^2) = \frac{-4}{\pi^2 n^2} ,
    \end{equation*}
    or
    \begin{equation*}
        a_n = \frac{-4}{n^2 \pi^2 (2 - n^2 \pi^2)} .
    \end{equation*}
    The Fourier series for the solution $x(t)$ is
    \begin{equation*}
        x(t) = \frac{1}{4} + \sum_{\substack{n=1\\n~\text{odd}}}^\infty \frac{-4}{n^2 \pi^2 (2 - n^2 \pi^2)} \cos (n \pi t) .
    \end{equation*}
\end{exampleSol}


\end{document}