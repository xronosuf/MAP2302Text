\documentclass{ximera}

\title{Practice for Trig Series}

%\auor{Matthew Charnley and Jason Nowell}
\usepackage[margin=1.5cm]{geometry}
\usepackage{indentfirst}
\usepackage{sagetex}
\usepackage{lipsum}
\usepackage{amsmath}
\usepackage{mathrsfs}
\usepackage{tikz}
\usetikzlibrary{matrix}

%%% Random packages added without verifying what they are really doing - just to get initial compile to work.
\usepackage{tcolorbox}
\usepackage{hypcap}
\usepackage{booktabs}%% To get \toprule,\midrule,\bottomrule etc.
\usepackage{caption}
\usepackage{units}
\usepackage{multicol}
\usepackage{hhline}


% This is my modified wrapfig that doesn't use intextsep
\usepackage{mywrapfig}
\usepackage{import}



%%% End to random added packages.


\graphicspath{
    {./}
    {./figures/}
    {./../figures/}
    {./../../figures/}
}
\renewcommand{\log}{\ln}%%%%
\DeclareMathOperator{\arcsec}{arcsec}
%% New commands


%%%%%%%%%%%%%%%%%%%%
% New Conditionals %
%%%%%%%%%%%%%%%%%%%%


% referencing
\makeatletter
    \DeclareRobustCommand{\myvref}[2]{%
      \leavevmode%
      \begingroup
        \let\T@pageref\@pagerefstar
        \hyperref[{#2}]{%
	  #1~\ref*{#2}%
        }%
        \vpageref[\unskip]{#2}%
      \endgroup
    }%

    \DeclareRobustCommand{\myref}[2]{%
      \leavevmode%
      \begingroup
        \let\T@pageref\@pagerefstar
        \hyperref[{#2}]{%
	  #1~\ref*{#2}%
        }%
      \endgroup
    }%
\makeatother

\newcommand{\figurevref}[1]{\myvref{Figure}{#1}}
\newcommand{\figureref}[1]{\myref{Figure}{#1}}
\newcommand{\tablevref}[1]{\myvref{Table}{#1}}
\newcommand{\tableref}[1]{\myref{Table}{#1}}
\newcommand{\chapterref}[1]{\myref{chapter}{#1}}
\newcommand{\Chapterref}[1]{\myref{Chapter}{#1}}
\newcommand{\appendixref}[1]{\myref{appendix}{#1}}
\newcommand{\Appendixref}[1]{\myref{Appendix}{#1}}
\newcommand{\sectionref}[1]{\myref{\S}{#1}}
\newcommand{\subsectionref}[1]{\myref{subsection}{#1}}
\newcommand{\subsectionvref}[1]{\myvref{subsection}{#1}}
\newcommand{\exercisevref}[1]{\myvref{Exercise}{#1}}
\newcommand{\exerciseref}[1]{\myref{Exercise}{#1}}
\newcommand{\examplevref}[1]{\myvref{Example}{#1}}
\newcommand{\exampleref}[1]{\myref{Example}{#1}}
\newcommand{\thmvref}[1]{\myvref{Theorem}{#1}}
\newcommand{\thmref}[1]{\myref{Theorem}{#1}}


\renewcommand{\exampleref}[1]{ {\color{red} \bfseries Normally a reference to a previous example goes here.}}
\renewcommand{\examplevref}[1]{ {\color{red} \bfseries Normally a reference to a previous example goes here.}}
\renewcommand{\figurevref}[1]{ {\color{red} \bfseries Normally a reference to a previous figure goes here.}}
\renewcommand{\tablevref}[1]{ {\color{red} \bfseries Normally a reference to a previous table goes here.}}
\renewcommand{\Appendixref}[1]{ {\color{red} \bfseries Normally a reference to an Appendix goes here.}}
\renewcommand{\exercisevref}[1]{ {\color{red} \bfseries Normally a reference to a previous exercise goes here.}}
\renewcommand{\thmvref}[1]{ {\color{red} \bfseries Normally a reference to a previous theorem goes here.}}
\renewcommand{\subsectionvref}[1]{ {\color{red} \bfseries Normally a reference to a previous subsection goes here.}}



\newcommand{\R}{\mathbb{R}}
\newcommand{\C}{\mathbb{C}}

%% Example Solution Env.
\def\beginSolclaim{\par\addvspace{\medskipamount}\noindent\hbox{\bf Solution:}\hspace{0.5em}\ignorespaces}
\def\endSolclaim{\par\addvspace{-1em}\hfill\rule{1em}{0.4pt}\hspace{-0.4pt}\rule{0.4pt}{1em}\par\addvspace{\medskipamount}}
\newenvironment{exampleSol}[1][]{\beginSolclaim}{\endSolclaim}

%% General figure formating from original book.
\newcommand{\mybeginframe}{%
\begin{tcolorbox}[colback=white,colframe=lightgray,left=5pt,right=5pt]%
}
\newcommand{\myendframe}{%
\end{tcolorbox}%
}

%%% Eventually return and fix this to make matlab code work correctly.
%% Define the matlab environment as another code environment
%\NewEnviron{matlab}{ {\centering\bfseries MATLAB Code} \\ \noexpand{\BODY} }
%\let\beginmatlab\begincode
%\let\endmatlab\endcode
%\newenvironment{matlab}{% Begin Environment Code
%\begin{minipage}{\linewidth}
%\begin{verbatim}
%}% End of Begin Environment Code
%{% Start of End Environment Code
%\end{verbatim}
%\end{minipage}
%}% End of End Environment Code


% this one should have a caption, first argument is the size
\newenvironment{mywrapfig}[2][]{
 \wrapfigure[#1]{r}{#2}
 \mybeginframe
 \centering
}{%
 \myendframe
 \endwrapfigure
}

% this one has no caption, first argument is size,
% the second argument is a larger size used for HTML (ignored by latex)
\newenvironment{mywrapfigsimp}[3][]{%
 \wrapfigure[#1]{r}{#2}%
 \centering%
}{%
 \endwrapfigure%
}
\newenvironment{myfig}
    {%
    \begin{figure}[h!t]
        \mybeginframe%
        \centering%
    }
    {%
        \myendframe
    \end{figure}%
    }


% graphics include
\newcommand{\diffyincludegraphics}[3]{\includegraphics[#1]{#3}}
\newcommand{\myincludegraphics}[3]{\includegraphics[#1]{#3}}
\newcommand{\inputpdft}[1]{\subimport*{../figures/}{#1.pdf_t}}


%% Not sure what these even do? They don't seem to actually work... fun!
%\newcommand{\mybxbg}[1]{\tcboxmath[colback=white,colframe=black,boxrule=0.5pt,top=1.5pt,bottom=1.5pt]{#1}}
%\newcommand{\mybxsm}[1]{\tcboxmath[colback=white,colframe=black,boxrule=0.5pt,left=0pt,right=0pt,top=0pt,bottom=0pt]{#1}}
\newcommand{\mybxsm}[1]{#1}
\newcommand{\mybxbg}[1]{#1}

%%% Something about tasks for practice/hw?
\usepackage{tasks}
\usepackage{footnote}
\makesavenoteenv{tasks}


%% For pdf only?
\newcommand{\diffypdfversion}[1]{#1}


%% Kill ``cite'' and go back later to fix it.
\renewcommand{\cite}[1]{}


%% Currently we can't really use index or its derivatives. So we are gonna kill them off.
\renewcommand{\index}[1]{}
\newcommand{\myindex}[1]{#1}







\begin{document}
\begin{abstract}
Why?
\end{abstract}
\maketitle



\begin{exercise}\%
    Suppose $f(t)$ is defined on $[-\pi,\pi]$ as $f(t) = \sin(t)$.  Extend periodically and compute the Fourier series.
\end{exercise}
%\exsol{%
%$\sin(t)$
%}

\begin{exercise}
    Suppose $f(t)$ is defined on $[-\pi,\pi]$ as $\sin (5t) + \cos (3t)$.  Extend periodically and compute the Fourier series of $f(t)$.
\end{exercise}

\begin{exercise}
    Suppose $f(t)$ is defined on $[-\pi,\pi]$ as $\lvert t \rvert$. Extend periodically and compute the Fourier series of $f(t)$.
\end{exercise}

\begin{exercise}
    Suppose $f(t)$ is defined on $[-\pi,\pi]$ as $\lvert t \rvert^3$. Extend periodically and compute the Fourier series of $f(t)$.
\end{exercise}

\begin{exercise}\%
    Suppose $f(t)$ is defined on $(-\pi,\pi]$ as $f(t) = \sin(\pi t)$.  Extend periodically and compute the Fourier series.
\end{exercise}
%\exsol{%
%$\sum\limits_{n=1}^\infty
%\frac{(\pi-n) \sin( \pi n+{\pi}^{2})
%+(\pi+n)\sin(\pi n-{\pi}^{2}) }{\pi {n}^{2}-{\pi}^{3}}
%\sin(nt)$
%%$a_0 = \frac{1}{\pi} \int_{-\pi}^\pi \sin(\pi t) \, dt$
%%\\
%%$a_n =
%%\frac{1}{\pi} \int_{-\pi}^\pi \sin(\pi t) \cos (nt) \, dt$
%%\\
%%$b_n =
%%\frac{1}{\pi} \int_{-\pi}^\pi f(t) \sin (nt) \, dt$
%}

\begin{exercise}
    Suppose $f(t)$ is defined on $(-\pi,\pi]$ as
    \begin{equation*}
        f(t) =
        \begin{cases}
            -1 & \text{if } \; {-\pi} < t \leq 0 , \\
            1 & \text{if } \; \phantom{-}0 < t \leq \pi .
        \end{cases}
    \end{equation*}
    Extend periodically and compute the Fourier series of $f(t)$.
\end{exercise}

\begin{exercise}
    Suppose $f(t)$ is defined on $(-\pi,\pi]$ as $t^3$. Extend periodically and compute the Fourier series of $f(t)$.
\end{exercise}

\begin{exercise}\%
    Suppose $f(t)$ is defined on $(-\pi,\pi]$ as $f(t) = \sin^2(t)$. Extend periodically and compute the Fourier series.
\end{exercise}
%\exsol{%
%$\frac{1}{2}-\frac{1}{2}\cos(2t)$
%}

\begin{exercise}
    Suppose $f(t)$ is defined on $[-\pi,\pi]$ as $t^2$. Extend periodically and compute the Fourier series of $f(t)$.
\end{exercise}

\begin{exercise}\%
    Suppose $f(t)$ is defined on $(-\pi,\pi]$ as $f(t) = t^4$. Extend periodically and compute the Fourier series.
\end{exercise}
%\exsol{%
%$\frac{\pi^4}{5} + \sum\limits_{n=1}^\infty
%\frac{{(-1)}^{n} (8{\pi}^{2}{n}^{2}-48) }{{n}^{4}}
%\cos(nt)$
%}

There is another form of the Fourier series using complex exponentials $e^{nt}$ for $n=\ldots,-2,-1,0,1,2,\ldots$ instead of $\cos(nt)$ and $\sin(nt)$ for positive $n$.  This form may be easier to work with sometimes.  It is certainly more compact to write, and there is only one formula for the coefficients.  On the downside, the coefficients are complex numbers.

\begin{exercise}
    Let 
    \begin{equation*}
        f(t) = \frac{a_0}{2} + \sum_{n=1}^\infty a_n \cos (n t) + b_n \sin (n t) .
    \end{equation*}
    Use Euler's formula $e^{i\theta} = \cos (\theta) + i \sin (\theta)$ to show that there exist complex numbers $c_m$ such that
    \begin{equation*}
    f(t) = \sum_{m=-\infty}^\infty c_m e^{imt} .
    \end{equation*}
    Note that the sum now ranges over all the integers including negative ones. Do not worry about convergence in this calculation. Hint: It may be better to start from the complex exponential form and write the series as
    \begin{equation*}
        c_0 + \sum_{m=1}^\infty \Bigl( c_m e^{imt} + c_{-m} e^{-imt}  \Bigr).
    \end{equation*}
\end{exercise}

%\setcounter{exercise}{100}

\end{document}