\documentclass{ximera}

\title{Practice for Sine and Cosine Series}

%\auor{Matthew Charnley and Jason Nowell}
\usepackage[margin=1.5cm]{geometry}
\usepackage{indentfirst}
\usepackage{sagetex}
\usepackage{lipsum}
\usepackage{amsmath}
\usepackage{mathrsfs}


%%% Random packages added without verifying what they are really doing - just to get initial compile to work.
\usepackage{tcolorbox}
\usepackage{hypcap}
\usepackage{booktabs}%% To get \toprule,\midrule,\bottomrule etc.
\usepackage{nicefrac}
\usepackage{caption}
\usepackage{units}

% This is my modified wrapfig that doesn't use intextsep
\usepackage{mywrapfig}
\usepackage{import}



%%% End to random added packages.


\graphicspath{
    {./figures/}
    {./../figures/}
    {./../../figures/}
}
\renewcommand{\log}{\ln}%%%%
\DeclareMathOperator{\arcsec}{arcsec}
%% New commands


%%%%%%%%%%%%%%%%%%%%
% New Conditionals %
%%%%%%%%%%%%%%%%%%%%


% referencing
\makeatletter
    \DeclareRobustCommand{\myvref}[2]{%
      \leavevmode%
      \begingroup
        \let\T@pageref\@pagerefstar
        \hyperref[{#2}]{%
	  #1~\ref*{#2}%
        }%
        \vpageref[\unskip]{#2}%
      \endgroup
    }%

    \DeclareRobustCommand{\myref}[2]{%
      \leavevmode%
      \begingroup
        \let\T@pageref\@pagerefstar
        \hyperref[{#2}]{%
	  #1~\ref*{#2}%
        }%
      \endgroup
    }%
\makeatother

\newcommand{\figurevref}[1]{\myvref{Figure}{#1}}
\newcommand{\figureref}[1]{\myref{Figure}{#1}}
\newcommand{\tablevref}[1]{\myvref{Table}{#1}}
\newcommand{\tableref}[1]{\myref{Table}{#1}}
\newcommand{\chapterref}[1]{\myref{chapter}{#1}}
\newcommand{\Chapterref}[1]{\myref{Chapter}{#1}}
\newcommand{\appendixref}[1]{\myref{appendix}{#1}}
\newcommand{\Appendixref}[1]{\myref{Appendix}{#1}}
\newcommand{\sectionref}[1]{\myref{\S}{#1}}
\newcommand{\subsectionref}[1]{\myref{subsection}{#1}}
\newcommand{\subsectionvref}[1]{\myvref{subsection}{#1}}
\newcommand{\exercisevref}[1]{\myvref{Exercise}{#1}}
\newcommand{\exerciseref}[1]{\myref{Exercise}{#1}}
\newcommand{\examplevref}[1]{\myvref{Example}{#1}}
\newcommand{\exampleref}[1]{\myref{Example}{#1}}
\newcommand{\thmvref}[1]{\myvref{Theorem}{#1}}
\newcommand{\thmref}[1]{\myref{Theorem}{#1}}


\renewcommand{\exampleref}[1]{ {\color{red} \bfseries Normally a reference to a previous example goes here.}}
\renewcommand{\figurevref}[1]{ {\color{red} \bfseries Normally a reference to a previous figure goes here.}}
\renewcommand{\tablevref}[1]{ {\color{red} \bfseries Normally a reference to a previous table goes here.}}
\renewcommand{\Appendixref}[1]{ {\color{red} \bfseries Normally a reference to an Appendix goes here.}}
\renewcommand{\exercisevref}[1]{ {\color{red} \bfseries Normally a reference to a previous exercise goes here.}}



\newcommand{\R}{\mathbb{R}}

%% Example Solution Env.
\def\beginSolclaim{\par\addvspace{\medskipamount}\noindent\hbox{\bf Solution:}\hspace{0.5em}\ignorespaces}
\def\endSolclaim{\par\addvspace{-1em}\hfill\rule{1em}{0.4pt}\hspace{-0.4pt}\rule{0.4pt}{1em}\par\addvspace{\medskipamount}}
\newenvironment{exampleSol}[1][]{\beginSolclaim}{\endSolclaim}

%% General figure formating from original book.
\newcommand{\mybeginframe}{%
\begin{tcolorbox}[colback=white,colframe=lightgray,left=5pt,right=5pt]%
}
\newcommand{\myendframe}{%
\end{tcolorbox}%
}

%%% Eventually return and fix this to make matlab code work correctly.
%% Define the matlab environment as another code environment
%\newenvironment{matlab}
%{% Begin Environment Code
%{ \centering \bfseries Matlab Code }
%\begin{code}
%}% End of Begin Environment Code
%{% Start of End Environment Code
%\end{code}
%}% End of End Environment Code


% this one should have a caption, first argument is the size
\newenvironment{mywrapfig}[2][]{
 \wrapfigure[#1]{r}{#2}
 \mybeginframe
 \centering
}{%
 \myendframe
 \endwrapfigure
}

% this one has no caption, first argument is size,
% the second argument is a larger size used for HTML (ignored by latex)
\newenvironment{mywrapfigsimp}[3][]{%
 \wrapfigure[#1]{r}{#2}%
 \centering%
}{%
 \endwrapfigure%
}
\newenvironment{myfig}
    {%
    \begin{figure}[h!t]
        \mybeginframe%
        \centering%
    }
    {%
        \myendframe
    \end{figure}%
    }


% graphics include
\newcommand{\diffyincludegraphics}[3]{\includegraphics[#1]{#3}}
\newcommand{\myincludegraphics}[3]{\includegraphics[#1]{#3}}
\newcommand{\inputpdft}[1]{\subimport*{../figures/}{#1.pdf_t}}


%% Not sure what these even do? They don't seem to actually work... fun!
%\newcommand{\mybxbg}[1]{\tcboxmath[colback=white,colframe=black,boxrule=0.5pt,top=1.5pt,bottom=1.5pt]{#1}}
%\newcommand{\mybxsm}[1]{\tcboxmath[colback=white,colframe=black,boxrule=0.5pt,left=0pt,right=0pt,top=0pt,bottom=0pt]{#1}}
\newcommand{\mybxsm}[1]{#1}
\newcommand{\mybxbg}[1]{#1}

%%% Something about tasks for practice/hw?
\usepackage{tasks}
\usepackage{footnote}
\makesavenoteenv{tasks}


%% For pdf only?
\newcommand{\diffypdfversion}[1]{#1}


%% Kill ``cite'' and go back later to fix it.
\renewcommand{\cite}[1]{}


%% Currently we can't really use index or its derivatives. So we are gonna kill them off.
\renewcommand{\index}[1]{}
\newcommand{\myindex}[1]{#1}







\begin{document}
\begin{abstract}
Why?
\end{abstract}
\maketitle


\begin{exercise}
    Take $f(t) = {(t-1)}^2$ defined on $0 \leq t \leq 1$.
    \begin{itemize}
        \item Sketch the plot of the even periodic extension of $f$.
        \item Sketch the plot of the odd periodic extension of $f$.
    \end{itemize}
\end{exercise}

\begin{exercise}
    Find the Fourier series of both the odd and even periodic extension of the function $f(t) = {(t-1)}^2$ for $0 \leq t \leq 1$. Can you tell which extension is continuous from the Fourier series coefficients?
\end{exercise}

\begin{exercise}
    Find the Fourier series of both the odd and even periodic extension of the function $f(t) = t$ for $0 \leq t \leq \pi$.
\end{exercise}

\begin{exercise}\%
    Let $f(t) = \frac{t}{3}$ on $0 \leq t < 3$.
    \begin{itemize}
        \item Find the Fourier series of the even periodic extension.
        \item Find the Fourier series of the odd periodic extension.
    \end{itemize}
\end{exercise}
%\exsol{%
%a)
%$\frac{1}{2}
%+
%\sum\limits_{\substack{n=1\\n\text{ odd}}}^\infty
%\frac{-4}{\pi^2 n^2}
%\cos\bigl(\frac{n\pi}{3} t \bigr)$
%\qquad
%b) 
%$\sum\limits_{n=1}^\infty
%\frac{2{(-1)}^{n+1}}{\pi n}
%\sin\bigl(\frac{n\pi}{3} t \bigr)$
%}

\begin{exercise}
    Find the Fourier series of the even periodic extension of the function $f(t) = \sin t$ for $0 \leq t \leq \pi$.
\end{exercise}

\begin{exercise}\%
    Let $f(t) = \cos(2t)$ on $0 \leq t < \pi$.
    \begin{itemize}
        \item Find the Fourier series of the even periodic extension.
        \item Find the Fourier series of the odd periodic extension.
    \end{itemize}
\end{exercise}
%\exsol{%
%a)
%$\cos(2t)$
%\qquad
%b) 
%$\sum\limits_{\substack{n=1 \\n \text{ odd}}}^\infty
%\frac{-4n}{\pi n^2 - 4 \pi}
%\sin(n t)$
%}

\begin{exercise}\%
    Let $f(t)$ be defined on $0 \leq t < 1$.  Now take the average of the two extensions $g(t) = \frac{F_{\text{odd}}(t)+ F_{\text{even}}(t)}{2}$.
    \begin{itemize}
        \item What is $g(t)$ if $0 \leq t < 1$ (Justify!)
        \item What is $g(t)$ if $-1 < t < 0$ (Justify!)
    \end{itemize}
\end{exercise}
%\exsol{%
%a) $f(t)$
%\qquad
%b) $0$
%}


\begin{exercise}
    Consider
    \begin{equation*}
        x''(t) + 4 x(t) = f(t) ,
    \end{equation*}
    where $f(t) = 1$ on $0 < t < 1$.
    \begin{itemize}
        \item Solve for the Dirichlet conditions $x(0)=0, x(1) = 0$.
        \item Solve for the Neumann conditions $x'(0)=0, x'(1) = 0$.
    \end{itemize}
\end{exercise}

\begin{exercise}
    Consider
    \begin{equation*}
        x''(t) + 9 x(t) = f(t) ,
    \end{equation*}
    for $f(t) = \sin (2\pi t)$ on $0 < t < 1$.
    \begin{itemize}
        \item Solve for the Dirichlet conditions $x(0)=0, x(1) = 0$.
        \item Solve for the Neumann conditions $x'(0)=0, x'(1) = 0$.
    \end{itemize}
\end{exercise}

\begin{exercise}\%
    Let $f(t) = \sum_{n=1}^\infty \frac{1}{n^2} \sin(nt)$.  Solve $x''- x = f(t)$ for the Dirichlet conditions $x(0) = 0$ and $x(\pi) = 0$.
\end{exercise}
%\exsol{%
%$\sum\limits_{n=1}^\infty \frac{-1}{n^2(1+n^2)} \sin(nt)$
%}

\begin{exercise}
    Consider
    \begin{equation*}
        x''(t) + 3 x(t) = f(t) , \quad x(0) = 0, \quad x(1) = 0,
    \end{equation*}
    where $f(t) = \sum_{n=1}^\infty b_n \sin (n \pi t)$.  Write the solution $x(t)$ as a Fourier series, where the coefficients are given in terms of $b_n$.
\end{exercise}

\begin{exercise}
    Let $f(t) = t^2(2-t)$ for $0 \leq t \leq 2$.  Let $F(t)$ be the odd periodic extension.  Compute $F(1)$, $F(2)$, $F(3)$, $F(-1)$, $F(\frac{9}{2})$, $F(101)$, $F(103)$.  Note: Do \textbf{not} compute using the sine series.
\end{exercise}

\begin{exercise}%[challenging]\%
    Let $f(t) = t + \sum_{n=1}^\infty \frac{1}{2^n} \sin(nt)$.  Solve $x'' + \pi x = f(t)$ for the Dirichlet conditions $x(0) = 0$ and $x(\pi) = 1$.  Hint:  Note that $\frac{t}{\pi}$ satisfies the given Dirichlet conditions.
\end{exercise}
%\exsol{%
%$\frac{t}{\pi} + \sum\limits_{n=1}^\infty \frac{1}{2^n(\pi-n^2)} \sin(nt)$
%}

%\setcounter{exercise}{100}

\end{document}