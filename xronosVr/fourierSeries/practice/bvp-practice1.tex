\documentclass{ximera}

\title{Practice for Boundary Value Problems}

%\auor{Matthew Charnley and Jason Nowell}
\usepackage[margin=1.5cm]{geometry}
\usepackage{indentfirst}
\usepackage{sagetex}
\usepackage{lipsum}
\usepackage{amsmath}
\usepackage{mathrsfs}


%%% Random packages added without verifying what they are really doing - just to get initial compile to work.
\usepackage{tcolorbox}
\usepackage{hypcap}
\usepackage{booktabs}%% To get \toprule,\midrule,\bottomrule etc.
\usepackage{nicefrac}
\usepackage{caption}
\usepackage{units}

% This is my modified wrapfig that doesn't use intextsep
\usepackage{mywrapfig}
\usepackage{import}



%%% End to random added packages.


\graphicspath{
    {./figures/}
    {./../figures/}
    {./../../figures/}
}
\renewcommand{\log}{\ln}%%%%
\DeclareMathOperator{\arcsec}{arcsec}
%% New commands


%%%%%%%%%%%%%%%%%%%%
% New Conditionals %
%%%%%%%%%%%%%%%%%%%%


% referencing
\makeatletter
    \DeclareRobustCommand{\myvref}[2]{%
      \leavevmode%
      \begingroup
        \let\T@pageref\@pagerefstar
        \hyperref[{#2}]{%
	  #1~\ref*{#2}%
        }%
        \vpageref[\unskip]{#2}%
      \endgroup
    }%

    \DeclareRobustCommand{\myref}[2]{%
      \leavevmode%
      \begingroup
        \let\T@pageref\@pagerefstar
        \hyperref[{#2}]{%
	  #1~\ref*{#2}%
        }%
      \endgroup
    }%
\makeatother

\newcommand{\figurevref}[1]{\myvref{Figure}{#1}}
\newcommand{\figureref}[1]{\myref{Figure}{#1}}
\newcommand{\tablevref}[1]{\myvref{Table}{#1}}
\newcommand{\tableref}[1]{\myref{Table}{#1}}
\newcommand{\chapterref}[1]{\myref{chapter}{#1}}
\newcommand{\Chapterref}[1]{\myref{Chapter}{#1}}
\newcommand{\appendixref}[1]{\myref{appendix}{#1}}
\newcommand{\Appendixref}[1]{\myref{Appendix}{#1}}
\newcommand{\sectionref}[1]{\myref{\S}{#1}}
\newcommand{\subsectionref}[1]{\myref{subsection}{#1}}
\newcommand{\subsectionvref}[1]{\myvref{subsection}{#1}}
\newcommand{\exercisevref}[1]{\myvref{Exercise}{#1}}
\newcommand{\exerciseref}[1]{\myref{Exercise}{#1}}
\newcommand{\examplevref}[1]{\myvref{Example}{#1}}
\newcommand{\exampleref}[1]{\myref{Example}{#1}}
\newcommand{\thmvref}[1]{\myvref{Theorem}{#1}}
\newcommand{\thmref}[1]{\myref{Theorem}{#1}}


\renewcommand{\exampleref}[1]{ {\color{red} \bfseries Normally a reference to a previous example goes here.}}
\renewcommand{\figurevref}[1]{ {\color{red} \bfseries Normally a reference to a previous figure goes here.}}
\renewcommand{\tablevref}[1]{ {\color{red} \bfseries Normally a reference to a previous table goes here.}}
\renewcommand{\Appendixref}[1]{ {\color{red} \bfseries Normally a reference to an Appendix goes here.}}
\renewcommand{\exercisevref}[1]{ {\color{red} \bfseries Normally a reference to a previous exercise goes here.}}



\newcommand{\R}{\mathbb{R}}

%% Example Solution Env.
\def\beginSolclaim{\par\addvspace{\medskipamount}\noindent\hbox{\bf Solution:}\hspace{0.5em}\ignorespaces}
\def\endSolclaim{\par\addvspace{-1em}\hfill\rule{1em}{0.4pt}\hspace{-0.4pt}\rule{0.4pt}{1em}\par\addvspace{\medskipamount}}
\newenvironment{exampleSol}[1][]{\beginSolclaim}{\endSolclaim}

%% General figure formating from original book.
\newcommand{\mybeginframe}{%
\begin{tcolorbox}[colback=white,colframe=lightgray,left=5pt,right=5pt]%
}
\newcommand{\myendframe}{%
\end{tcolorbox}%
}

%%% Eventually return and fix this to make matlab code work correctly.
%% Define the matlab environment as another code environment
%\newenvironment{matlab}
%{% Begin Environment Code
%{ \centering \bfseries Matlab Code }
%\begin{code}
%}% End of Begin Environment Code
%{% Start of End Environment Code
%\end{code}
%}% End of End Environment Code


% this one should have a caption, first argument is the size
\newenvironment{mywrapfig}[2][]{
 \wrapfigure[#1]{r}{#2}
 \mybeginframe
 \centering
}{%
 \myendframe
 \endwrapfigure
}

% this one has no caption, first argument is size,
% the second argument is a larger size used for HTML (ignored by latex)
\newenvironment{mywrapfigsimp}[3][]{%
 \wrapfigure[#1]{r}{#2}%
 \centering%
}{%
 \endwrapfigure%
}
\newenvironment{myfig}
    {%
    \begin{figure}[h!t]
        \mybeginframe%
        \centering%
    }
    {%
        \myendframe
    \end{figure}%
    }


% graphics include
\newcommand{\diffyincludegraphics}[3]{\includegraphics[#1]{#3}}
\newcommand{\myincludegraphics}[3]{\includegraphics[#1]{#3}}
\newcommand{\inputpdft}[1]{\subimport*{../figures/}{#1.pdf_t}}


%% Not sure what these even do? They don't seem to actually work... fun!
%\newcommand{\mybxbg}[1]{\tcboxmath[colback=white,colframe=black,boxrule=0.5pt,top=1.5pt,bottom=1.5pt]{#1}}
%\newcommand{\mybxsm}[1]{\tcboxmath[colback=white,colframe=black,boxrule=0.5pt,left=0pt,right=0pt,top=0pt,bottom=0pt]{#1}}
\newcommand{\mybxsm}[1]{#1}
\newcommand{\mybxbg}[1]{#1}

%%% Something about tasks for practice/hw?
\usepackage{tasks}
\usepackage{footnote}
\makesavenoteenv{tasks}


%% For pdf only?
\newcommand{\diffypdfversion}[1]{#1}


%% Kill ``cite'' and go back later to fix it.
\renewcommand{\cite}[1]{}


%% Currently we can't really use index or its derivatives. So we are gonna kill them off.
\renewcommand{\index}[1]{}
\newcommand{\myindex}[1]{#1}







\begin{document}
\begin{abstract}
Why?
\end{abstract}
\maketitle


\begin{exercise}\%
    Consider a spinning string of length 2 and linear density 0.1 and tension 3. Find smallest angular velocity when the string pops out.
\end{exercise}
%\exsol{%
%$\omega = \pi \sqrt{\frac{15}{2}}$
%}


Hint for the following exercises:  Note that when $\lambda > 0$, then $\cos \bigl( \sqrt{\lambda}\, (t - a) \bigr)$ and $\sin  \bigl( \sqrt{\lambda}\, (t - a) \bigr)$ are also solutions of the homogeneous equation.

\begin{exercise}
    Compute all eigenvalues and eigenfunctions of $x'' + \lambda x = 0, ~ x(a) = 0, ~ x(b) = 0$ (assume $a < b$).
\end{exercise}

\begin{exercise}
    Compute all eigenvalues and eigenfunctions of $x'' + \lambda x = 0, ~ x'(a) = 0, ~ x'(b) = 0$ (assume $a < b$).
\end{exercise}

\begin{exercise}\%
    Suppose $x'' + \lambda x = 0$ and $x(0)=1$, $x(1) = 1$. Find all $\lambda$ for which there is more than one solution.  Also find the corresponding solutions (only for the eigenvalues).
\end{exercise}
%\exsol{%
%$\lambda_k = 4 k^2 \pi^2$ for $k = 1,2,3,\ldots$
%\quad
%$x_k =  \cos (2k\pi t) + B \sin (2k\pi t)$ \quad (for any $B$)
%}


\begin{exercise}
    Compute all eigenvalues and eigenfunctions of $x'' + \lambda x = 0, ~ x'(a) = 0, ~ x(b) = 0$ (assume $a < b$).
\end{exercise}

\begin{exercise}
    Compute all eigenvalues and eigenfunctions of $x'' + \lambda x = 0, ~ x(a) = x(b), ~ x'(a) = x'(b)$ (assume $a < b$).
\end{exercise}

\begin{exercise}%
    Suppose $x'' + x = 0$ and $x(0)=0$, $x'(\pi) = 1$. Find all the solution(s) if any exist.
\end{exercise}
%\exsol{%
%$x(t) = - \sin(t)$
%}


\begin{exercise}
    We skipped the case of $\lambda < 0$ for the boundary value problem $x'' + \lambda x = 0, ~ x(-\pi) = x(\pi), ~ x'(-\pi) = x'(\pi)$. Finish the calculation and show that there are no negative eigenvalues.
\end{exercise}

\begin{exercise}\%
    Consider $x' + \lambda x = 0$ and $x(0)=0$, $x(1) = 0$.  Why does it not have any eigenvalues?  Why does any first order equation with two endpoint conditions such as above have no eigenvalues?
\end{exercise}
%\exsol{%
%General solution is $x = C e^{-\lambda t}$.  Since $x(0) = 0$ then $C=0$, and so $x(t) = 0$.
%Therefore,
%the solution is always identically zero.  One condition is always
%enough to guarantee a unique solution for a first order equation.
%}

\begin{exercise}%[challenging]\%
    Suppose $x''' + \lambda x = 0$ and $x(0)=0$, $x'(0) = 0$, $x(1) = 0$. Suppose that $\lambda > 0$.  Find an equation that all such eigenvalues must satisfy. Hint: Note that $-\sqrt[3]{\lambda}$ is a root of $r^3+\lambda = 0$.
\end{exercise}
%\exsol{%
%$\frac{\sqrt{3}}{3} e^{\frac{-3}{2}\sqrt[3]{\lambda}}
%- \frac{\sqrt{3}}{3} \cos \bigl( \frac{\sqrt{3}\, \sqrt[3]{\lambda}}{2} \bigr)
%+ \sin \bigl( \frac{\sqrt{3}\, \sqrt[3]{\lambda}}{2}\bigr) = 0$
%}


%\setcounter{exercise}{100}

\end{document}