\documentclass{ximera}
%\auor{Matthew Charnley and Jason Nowell}
\usepackage[margin=1.5cm]{geometry}
\usepackage{indentfirst}
\usepackage{sagetex}
\usepackage{lipsum}
\usepackage{amsmath}
\usepackage{mathrsfs}
\usepackage{tikz}
\usetikzlibrary{matrix}

%%% Random packages added without verifying what they are really doing - just to get initial compile to work.
\usepackage{tcolorbox}
\usepackage{hypcap}
\usepackage{booktabs}%% To get \toprule,\midrule,\bottomrule etc.
\usepackage{caption}
\usepackage{units}
\usepackage{multicol}
\usepackage{hhline}


% This is my modified wrapfig that doesn't use intextsep
\usepackage{mywrapfig}
\usepackage{import}



%%% End to random added packages.


\graphicspath{
    {./}
    {./figures/}
    {./../figures/}
    {./../../figures/}
}
\renewcommand{\log}{\ln}%%%%
\DeclareMathOperator{\arcsec}{arcsec}
%% New commands


%%%%%%%%%%%%%%%%%%%%
% New Conditionals %
%%%%%%%%%%%%%%%%%%%%


% referencing
\makeatletter
    \DeclareRobustCommand{\myvref}[2]{%
      \leavevmode%
      \begingroup
        \let\T@pageref\@pagerefstar
        \hyperref[{#2}]{%
	  #1~\ref*{#2}%
        }%
        \vpageref[\unskip]{#2}%
      \endgroup
    }%

    \DeclareRobustCommand{\myref}[2]{%
      \leavevmode%
      \begingroup
        \let\T@pageref\@pagerefstar
        \hyperref[{#2}]{%
	  #1~\ref*{#2}%
        }%
      \endgroup
    }%
\makeatother

\newcommand{\figurevref}[1]{\myvref{Figure}{#1}}
\newcommand{\figureref}[1]{\myref{Figure}{#1}}
\newcommand{\tablevref}[1]{\myvref{Table}{#1}}
\newcommand{\tableref}[1]{\myref{Table}{#1}}
\newcommand{\chapterref}[1]{\myref{chapter}{#1}}
\newcommand{\Chapterref}[1]{\myref{Chapter}{#1}}
\newcommand{\appendixref}[1]{\myref{appendix}{#1}}
\newcommand{\Appendixref}[1]{\myref{Appendix}{#1}}
\newcommand{\sectionref}[1]{\myref{\S}{#1}}
\newcommand{\subsectionref}[1]{\myref{subsection}{#1}}
\newcommand{\subsectionvref}[1]{\myvref{subsection}{#1}}
\newcommand{\exercisevref}[1]{\myvref{Exercise}{#1}}
\newcommand{\exerciseref}[1]{\myref{Exercise}{#1}}
\newcommand{\examplevref}[1]{\myvref{Example}{#1}}
\newcommand{\exampleref}[1]{\myref{Example}{#1}}
\newcommand{\thmvref}[1]{\myvref{Theorem}{#1}}
\newcommand{\thmref}[1]{\myref{Theorem}{#1}}


\renewcommand{\exampleref}[1]{ {\color{red} \bfseries Normally a reference to a previous example goes here.}}
\renewcommand{\examplevref}[1]{ {\color{red} \bfseries Normally a reference to a previous example goes here.}}
\renewcommand{\figurevref}[1]{ {\color{red} \bfseries Normally a reference to a previous figure goes here.}}
\renewcommand{\tablevref}[1]{ {\color{red} \bfseries Normally a reference to a previous table goes here.}}
\renewcommand{\Appendixref}[1]{ {\color{red} \bfseries Normally a reference to an Appendix goes here.}}
\renewcommand{\exercisevref}[1]{ {\color{red} \bfseries Normally a reference to a previous exercise goes here.}}
\renewcommand{\thmvref}[1]{ {\color{red} \bfseries Normally a reference to a previous theorem goes here.}}
\renewcommand{\subsectionvref}[1]{ {\color{red} \bfseries Normally a reference to a previous subsection goes here.}}



\newcommand{\R}{\mathbb{R}}
\newcommand{\C}{\mathbb{C}}

%% Example Solution Env.
\def\beginSolclaim{\par\addvspace{\medskipamount}\noindent\hbox{\bf Solution:}\hspace{0.5em}\ignorespaces}
\def\endSolclaim{\par\addvspace{-1em}\hfill\rule{1em}{0.4pt}\hspace{-0.4pt}\rule{0.4pt}{1em}\par\addvspace{\medskipamount}}
\newenvironment{exampleSol}[1][]{\beginSolclaim}{\endSolclaim}

%% General figure formating from original book.
\newcommand{\mybeginframe}{%
\begin{tcolorbox}[colback=white,colframe=lightgray,left=5pt,right=5pt]%
}
\newcommand{\myendframe}{%
\end{tcolorbox}%
}

%%% Eventually return and fix this to make matlab code work correctly.
%% Define the matlab environment as another code environment
%\NewEnviron{matlab}{ {\centering\bfseries MATLAB Code} \\ \noexpand{\BODY} }
%\let\beginmatlab\begincode
%\let\endmatlab\endcode
%\newenvironment{matlab}{% Begin Environment Code
%\begin{minipage}{\linewidth}
%\begin{verbatim}
%}% End of Begin Environment Code
%{% Start of End Environment Code
%\end{verbatim}
%\end{minipage}
%}% End of End Environment Code


% this one should have a caption, first argument is the size
\newenvironment{mywrapfig}[2][]{
 \wrapfigure[#1]{r}{#2}
 \mybeginframe
 \centering
}{%
 \myendframe
 \endwrapfigure
}

% this one has no caption, first argument is size,
% the second argument is a larger size used for HTML (ignored by latex)
\newenvironment{mywrapfigsimp}[3][]{%
 \wrapfigure[#1]{r}{#2}%
 \centering%
}{%
 \endwrapfigure%
}
\newenvironment{myfig}
    {%
    \begin{figure}[h!t]
        \mybeginframe%
        \centering%
    }
    {%
        \myendframe
    \end{figure}%
    }


% graphics include
\newcommand{\diffyincludegraphics}[3]{\includegraphics[#1]{#3}}
\newcommand{\myincludegraphics}[3]{\includegraphics[#1]{#3}}
\newcommand{\inputpdft}[1]{\subimport*{../figures/}{#1.pdf_t}}


%% Not sure what these even do? They don't seem to actually work... fun!
%\newcommand{\mybxbg}[1]{\tcboxmath[colback=white,colframe=black,boxrule=0.5pt,top=1.5pt,bottom=1.5pt]{#1}}
%\newcommand{\mybxsm}[1]{\tcboxmath[colback=white,colframe=black,boxrule=0.5pt,left=0pt,right=0pt,top=0pt,bottom=0pt]{#1}}
\newcommand{\mybxsm}[1]{#1}
\newcommand{\mybxbg}[1]{#1}

%%% Something about tasks for practice/hw?
\usepackage{tasks}
\usepackage{footnote}
\makesavenoteenv{tasks}


%% For pdf only?
\newcommand{\diffypdfversion}[1]{#1}


%% Kill ``cite'' and go back later to fix it.
\renewcommand{\cite}[1]{}


%% Currently we can't really use index or its derivatives. So we are gonna kill them off.
\renewcommand{\index}[1]{}
\newcommand{\myindex}[1]{#1}






\title{Classification of differential equations}
\author{Matthew Charnley and Jason Nowell}

\outcome{Classify equation as ordinary or partial differential equations}
\outcome{Identify whether an equation is linear or non-linear}
\outcome{Classify linear equations as homogenoeus, non-homogenoeus, or constant coefficient, as appropriate}

\begin{document}
\begin{abstract}
    We introduce how to classify various properties of differential equations.
\end{abstract}
\maketitle

\label{classification:section}

% \sectionnotes{less than 1 lecture or left as reading\BDref{, \S1.3 in \cite{BD}}}

There are many types of differential equations, and we classify them into different categories based on their properties.  Let us quickly go over the most basic classification.  We already saw the distinction between ordinary and partial differential equations:

\begin{definition}
    \begin{itemize}
        \item \emph{Ordinary differential equations} or (ODE) are equations where the derivatives are taken with respect to only one variable. That is, there is only one independent variable.
        \item \emph{Partial differential equations} or (PDE) are equations that depend on partial derivatives of several variables. That is, there are several independent variables.
    \end{itemize}
\end{definition}

Let us see some examples of ordinary differential equations:
\begin{align*}
    & \frac{d y}{dt} = ky , & & \text{(Exponential growth\index{exponential growth})} \\
    & \frac{d y}{dt} = k(A-y) , & & \text{(Newton's law of cooling)} \\
    & m \frac{d^2 x}{dt^2} + c \frac{dx}{dt} + kx = f(t) . & &
    \text{(Mechanical vibrations\index{mechanical vibrations})}
\end{align*}
And of partial differential equations:
\begin{align*}
    & \frac{\partial y}{\partial t} + c \frac{\partial y}{\partial x} = 0 , & & 
    \text{(Transport equation\index{transport equation})} \\
    & \frac{\partial u}{\partial t} = \frac{\partial^2 u}{\partial x^2} , & & 
    \text{(Heat equation\index{heat equation})} \\
    & \frac{\partial^2 u}{\partial t^2} = \frac{\partial^2 u}{\partial x^2} +
    \frac{\partial^2 u}{\partial y^2} . & & 
    \text{(Wave equation in 2 dimensions\index{wave equation in 2 dimensions})}
\end{align*}

If there are several equations working together, we have a so-called \emph{system of differential equations}.  For example,
\begin{equation*}
    y' = x , \qquad x' = y
\end{equation*}
is a simple system of ordinary differential equations. Maxwell's equations for electromagnetics,
\begin{align*}
    & \nabla \cdot \vec{D} = \rho, & & \nabla \cdot \vec{B} = 0 , \\
    & \nabla \times \vec{E} = - \frac{\partial \vec{B}}{\partial t}, &
    & \nabla \times \vec{H} = \vec{J} + \frac{\partial \vec{D}}{\partial t} ,
\end{align*}
are a system of partial differential equations. The divergence operator $\nabla \cdot$ and the curl operator $\nabla \times$ can be written out in partial derivatives of the functions involved in the $x$, $y$, and $z$ variables.

%The next bit of information is the \emph{\myindex{order}} of the
%equation (or system).  The order is simply the order of the largest
%derivative that appears.  If the highest derivative that appears is
%the first derivative, the equation is of first order.  If the highest
%derivative that appears is the second derivative, then the equation is of second
%order.  For example, Newton's law of cooling above is a first order
%equation, while the mechanical vibrations equation is a second order equation.
%The equation governing transversal vibrations in a beam,
%\begin{equation*}
%a^4 \frac{\partial^4 y}{\partial x^4} + \frac{\partial^2 y}{\partial t^2} = 0,
%\end{equation*}
%is a fourth order partial differential equation.  It is
%fourth order as at least one derivative is the fourth derivative.  It
%does not matter that the derivative in $t$ is only of second order.

In the first chapter, we will start attacking first order ordinary differential equations, that is, equations of the form $\frac{dy}{dx} = f(x,y)$. In general, lower order equations are easier to work with and have simpler behavior, which is why we start with them.

We also distinguish how the dependent variables appear in the equation (or system).  
\begin{definition}
    We say an equation is \emph{linear} if the dependent variable (or variables) and their derivatives appear linearly, that is only as first powers, they are not multiplied together, and no other functions of the dependent variables appear. Otherwise, the equation is called \emph{nonlinear}.
\end{definition}

Another way to determine if a differential equation is linear is if the equation is a sum of terms, where each term is some function of the independent variables or some function of the independent variables multiplied by a dependent variable or its derivative. That is, an ordinary differential equation is linear if it can be put into the form
\begin{equation} \label{classification:eqlingen}
    a_n(x) \frac{d^n y}{dx^n} + a_{n-1}(x) \frac{d^{n-1} y}{dx^{n-1}} + \cdots + a_{1}(x) \frac{dy}{dx} + a_{0}(x) y = b(x) .
\end{equation}
The functions $a_0$, $a_1$, \ldots, $a_n$ are called the \emph{coefficients}. The equation is allowed to depend arbitrarily on the independent variable. So 
\begin{equation} \label{classification:eqlinex}
    e^x \frac{d^2 y}{dx^2} + \sin(x) \frac{d y}{dx} + x^2 y = \frac{1}{x}
\end{equation}
is still a linear equation as $y$ and its derivatives only appear linearly. The equation
\[ 
    \cos(x) \frac{d^2y}{dx^2} - xy + \frac{e^x}{x} = 0 
\] 
is also linear, even though it is not initially in the correct form. From this equation, we can move the last term over to the right-hand side as a $-\frac{e^x}{x}$, and then it is in the correct form, with the $\frac{dy}{dx}$ term missing (or has coefficient zero).

All the equations and systems above as examples are linear. It may not be immediately obvious for Maxwell's equations unless you write out the divergence and curl in terms of partial derivatives.  Let us see some nonlinear equations.  For example Burger's equation,
\begin{equation*}
    \frac{\partial y}{\partial t} + y \frac{\partial y}{\partial x} = \nu \frac{\partial^2 y}{\partial x^2} ,
\end{equation*}
is a nonlinear second order partial differential equation.  It is nonlinear because $y$ and $\frac{\partial y}{\partial x}$ are multiplied together. The equation
\begin{equation} \label{classification:eqnonlinode}
    \frac{dx}{dt} = x^2
\end{equation}
is a nonlinear first order differential equation as there is a second power of the dependent variable $x$.

\begin{definition}
    A linear equation may further be called \emph{homogeneous} if all terms depend on the dependent variable.  That is, if no term is a function of the independent variables alone.  Otherwise, the equation is called \emph{nonhomogeneous} or \emph{inhomogeneous}.
\end{definition}

For example, the exponential growth equation, the wave equation, or the transport equation above are homogeneous. The mechanical vibrations equation above is nonhomogeneous as long as $f(t)$ is not the zero function.  Similarly, if the ambient temperature $A$ is nonzero, Newton's law of cooling is nonhomogeneous. A homogeneous linear ODE can be put into the form
\begin{equation*}
    a_n(x) \frac{d^n y}{dx^n} + a_{n-1}(x) \frac{d^{n-1} y}{dx^{n-1}} +  \cdots + a_{1}(x) \frac{dy}{dx} + a_{0}(x) y = 0 .
\end{equation*}
Compare to \eqref{classification:eqlingen} and notice there is no function $b(x)$.


If the coefficients of a linear equation are actually constant functions, then the equation is said to have \emph{constant coefficients}. The coefficients are the functions multiplying the dependent variable(s) or one of its derivatives, not the function $b(x)$ standing alone. A constant coefficient nonhomogeneous ODE is an equation of the form
\begin{equation*}
    a_n \frac{d^n y}{dx^n} + a_{n-1} \frac{d^{n-1} y}{dx^{n-1}} +  \cdots + a_{1} \frac{dy}{dx} + a_{0} y = b(x) ,
\end{equation*}
where $a_0, a_1, \ldots, a_n$ are all constants, but $b$ may depend on the independent variable $x$. The mechanical vibrations equation above is a constant coefficient nonhomogeneous second order ODE\@. The same nomenclature applies to PDEs, so the transport equation, heat equation and wave equation are all examples of constant coefficient linear PDEs.


Finally, an equation (or system) is called \emph{autonomous} if the equation does not explicitly depend on the independent variable. For autonomous ordinary differential equations, the independent variable is then thought of as time.  Autonomous equation means an equation that does not change with time. For example, Newton's law of cooling is autonomous, so is equation \eqref{classification:eqnonlinode}.  On the other hand, mechanical vibrations or \eqref{classification:eqlinex} are not autonomous.

\end{document}




