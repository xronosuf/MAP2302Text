\documentclass{ximera}

\title{Practice for Classifying Equations}

%\auor{Matthew Charnley and Jason Nowell}
\usepackage[margin=1.5cm]{geometry}
\usepackage{indentfirst}
\usepackage{sagetex}
\usepackage{lipsum}
\usepackage{amsmath}
\usepackage{mathrsfs}


%%% Random packages added without verifying what they are really doing - just to get initial compile to work.
\usepackage{tcolorbox}
\usepackage{hypcap}
\usepackage{booktabs}%% To get \toprule,\midrule,\bottomrule etc.
\usepackage{nicefrac}
\usepackage{caption}
\usepackage{units}

% This is my modified wrapfig that doesn't use intextsep
\usepackage{mywrapfig}
\usepackage{import}



%%% End to random added packages.


\graphicspath{
    {./figures/}
    {./../figures/}
    {./../../figures/}
}
\renewcommand{\log}{\ln}%%%%
\DeclareMathOperator{\arcsec}{arcsec}
%% New commands


%%%%%%%%%%%%%%%%%%%%
% New Conditionals %
%%%%%%%%%%%%%%%%%%%%


% referencing
\makeatletter
    \DeclareRobustCommand{\myvref}[2]{%
      \leavevmode%
      \begingroup
        \let\T@pageref\@pagerefstar
        \hyperref[{#2}]{%
	  #1~\ref*{#2}%
        }%
        \vpageref[\unskip]{#2}%
      \endgroup
    }%

    \DeclareRobustCommand{\myref}[2]{%
      \leavevmode%
      \begingroup
        \let\T@pageref\@pagerefstar
        \hyperref[{#2}]{%
	  #1~\ref*{#2}%
        }%
      \endgroup
    }%
\makeatother

\newcommand{\figurevref}[1]{\myvref{Figure}{#1}}
\newcommand{\figureref}[1]{\myref{Figure}{#1}}
\newcommand{\tablevref}[1]{\myvref{Table}{#1}}
\newcommand{\tableref}[1]{\myref{Table}{#1}}
\newcommand{\chapterref}[1]{\myref{chapter}{#1}}
\newcommand{\Chapterref}[1]{\myref{Chapter}{#1}}
\newcommand{\appendixref}[1]{\myref{appendix}{#1}}
\newcommand{\Appendixref}[1]{\myref{Appendix}{#1}}
\newcommand{\sectionref}[1]{\myref{\S}{#1}}
\newcommand{\subsectionref}[1]{\myref{subsection}{#1}}
\newcommand{\subsectionvref}[1]{\myvref{subsection}{#1}}
\newcommand{\exercisevref}[1]{\myvref{Exercise}{#1}}
\newcommand{\exerciseref}[1]{\myref{Exercise}{#1}}
\newcommand{\examplevref}[1]{\myvref{Example}{#1}}
\newcommand{\exampleref}[1]{\myref{Example}{#1}}
\newcommand{\thmvref}[1]{\myvref{Theorem}{#1}}
\newcommand{\thmref}[1]{\myref{Theorem}{#1}}


\renewcommand{\exampleref}[1]{ {\color{red} \bfseries Normally a reference to a previous example goes here.}}
\renewcommand{\figurevref}[1]{ {\color{red} \bfseries Normally a reference to a previous figure goes here.}}
\renewcommand{\tablevref}[1]{ {\color{red} \bfseries Normally a reference to a previous table goes here.}}
\renewcommand{\Appendixref}[1]{ {\color{red} \bfseries Normally a reference to an Appendix goes here.}}
\renewcommand{\exercisevref}[1]{ {\color{red} \bfseries Normally a reference to a previous exercise goes here.}}



\newcommand{\R}{\mathbb{R}}

%% Example Solution Env.
\def\beginSolclaim{\par\addvspace{\medskipamount}\noindent\hbox{\bf Solution:}\hspace{0.5em}\ignorespaces}
\def\endSolclaim{\par\addvspace{-1em}\hfill\rule{1em}{0.4pt}\hspace{-0.4pt}\rule{0.4pt}{1em}\par\addvspace{\medskipamount}}
\newenvironment{exampleSol}[1][]{\beginSolclaim}{\endSolclaim}

%% General figure formating from original book.
\newcommand{\mybeginframe}{%
\begin{tcolorbox}[colback=white,colframe=lightgray,left=5pt,right=5pt]%
}
\newcommand{\myendframe}{%
\end{tcolorbox}%
}

%%% Eventually return and fix this to make matlab code work correctly.
%% Define the matlab environment as another code environment
%\newenvironment{matlab}
%{% Begin Environment Code
%{ \centering \bfseries Matlab Code }
%\begin{code}
%}% End of Begin Environment Code
%{% Start of End Environment Code
%\end{code}
%}% End of End Environment Code


% this one should have a caption, first argument is the size
\newenvironment{mywrapfig}[2][]{
 \wrapfigure[#1]{r}{#2}
 \mybeginframe
 \centering
}{%
 \myendframe
 \endwrapfigure
}

% this one has no caption, first argument is size,
% the second argument is a larger size used for HTML (ignored by latex)
\newenvironment{mywrapfigsimp}[3][]{%
 \wrapfigure[#1]{r}{#2}%
 \centering%
}{%
 \endwrapfigure%
}
\newenvironment{myfig}
    {%
    \begin{figure}[h!t]
        \mybeginframe%
        \centering%
    }
    {%
        \myendframe
    \end{figure}%
    }


% graphics include
\newcommand{\diffyincludegraphics}[3]{\includegraphics[#1]{#3}}
\newcommand{\myincludegraphics}[3]{\includegraphics[#1]{#3}}
\newcommand{\inputpdft}[1]{\subimport*{../figures/}{#1.pdf_t}}


%% Not sure what these even do? They don't seem to actually work... fun!
%\newcommand{\mybxbg}[1]{\tcboxmath[colback=white,colframe=black,boxrule=0.5pt,top=1.5pt,bottom=1.5pt]{#1}}
%\newcommand{\mybxsm}[1]{\tcboxmath[colback=white,colframe=black,boxrule=0.5pt,left=0pt,right=0pt,top=0pt,bottom=0pt]{#1}}
\newcommand{\mybxsm}[1]{#1}
\newcommand{\mybxbg}[1]{#1}

%%% Something about tasks for practice/hw?
\usepackage{tasks}
\usepackage{footnote}
\makesavenoteenv{tasks}


%% For pdf only?
\newcommand{\diffypdfversion}[1]{#1}


%% Kill ``cite'' and go back later to fix it.
\renewcommand{\cite}[1]{}


%% Currently we can't really use index or its derivatives. So we are gonna kill them off.
\renewcommand{\index}[1]{}
\newcommand{\myindex}[1]{#1}







\begin{document}
\begin{abstract}
    Why?
\end{abstract}
\maketitle


\begin{exercise}
    Classify the following equations.  Are they ODE or PDE\@?  Is it an equation or a system?  What is the order?  Is it linear or nonlinear, and if it is linear, is it homogeneous, constant coefficient?  If it is an ODE\@, is it autonomous?
    \begin{tasks}(2)
        \task $\displaystyle \sin(t) \frac{d^2 x}{dt^2} + \cos(t) x = t^2$
        \task $\displaystyle \frac{\partial u}{\partial x} + 3 \frac{\partial u}{\partial y} = xy$
        \task $\displaystyle y''+3y+5x=0, \quad x''+x-y=0$
        \task $\displaystyle \frac{\partial^2 u}{\partial t^2} + u\frac{\partial^2 u}{\partial s^2} = 0$
        \task $\displaystyle x''+tx^2=t$
        \task $\displaystyle \frac{d^4 x}{dt^4} = 0$
    \end{tasks}
\end{exercise}
%\comboSol
%{%
%a)~ODE, equation, second order, linear, non-homogeneous, not constant coefficient, not autonomous.\\
%b)~PDE, equation, first order, linear, constant coefficient, non-homogeneous.\\
%c)~ODE, system, second order, linear, constant coefficient, homogeneous, autonomous.\\
%d)~PDE, equation, second order, non-linear.\\
%e)~ODE, equation, second order, non-linear, not autonomous.\\
%f)~ODE, equation, fourth order, linear, constant coefficient, homogeneous, autonomous.
%}

\begin{exercise}%
    Classify the following equations.  Are they ODE or PDE\@?  Is it an equation or a system?  What is the order?  Is it linear or nonlinear, and if it is linear, is it homogeneous, constant coefficient?  If it is an ODE\@, is it autonomous?
    \begin{tasks}(2)
        \task $\displaystyle \frac{\partial^2 v}{\partial x^2} + 3 \frac{\partial^2 v}{\partial y^2} = \sin(x)$
        \task $\displaystyle \frac{d x}{dt} + \cos(t) x = t^2+t+1$
        \task $\displaystyle \frac{d^7 F}{dx^7} = 3F(x)$
        \task $\displaystyle y''+8y'=1$
        \task $\displaystyle x''+tyx'=0, \quad y''+txy = 0$
        \task $\displaystyle \frac{\partial u}{\partial t} = \frac{\partial^2 u}{\partial s^2} + u^2$
    \end{tasks}
\end{exercise}
%\exsol{%
%a) 
%PDE\@, equation, second order, linear, nonhomogeneous, constant coefficient.\\
%b) 
%ODE\@, equation, first order, linear, nonhomogeneous, not constant coefficient, not autonomous.\\
%c) 
%ODE\@, equation, seventh order, linear, homogeneous, constant coefficient, autonomous.\\
%d) 
%ODE\@, equation, second order, linear, nonhomogeneous, constant coefficient, autonomous.\\
%e) 
%ODE\@, system, second order, nonlinear.\\
%f) 
%PDE\@, equation, second order, nonlinear.
%}

\begin{exercise}
    If $\vec{u} = (u_1,u_2,u_3)$ is a vector, we have the divergence $\nabla \cdot \vec{u} = \frac{\partial u_1}{\partial x} + \frac{\partial u_2}{\partial y} + \frac{\partial u_3}{\partial z}$ and curl $\nabla \times \vec{u} = \Bigl( \frac{\partial u_3}{\partial y} - \frac{\partial u_2}{\partial z} , ~ \frac{\partial u_1}{\partial z} - \frac{\partial u_3}{\partial x} , ~ \frac{\partial u_2}{\partial x} - \frac{\partial u_1}{\partial y} \Bigr)$. Notice that curl of a vector is still a vector.  Write out Maxwell's equations in terms of partial derivatives and classify the system.
\end{exercise}
%\comboSol
%{%
%Order 1, linear PDEs, with constant coefficients.
%}

\begin{exercise}
    Suppose $F$ is a linear function, that is, $F(x,y) = ax+by$ for constants $a$ and $b$.  What is the classification of equations of the form $F(y',y) = 0$.
\end{exercise}
%\comboSol
%{%
%First order, linear, homogeneous, constant coefficients.
%}

\begin{exercise}
    Write down an explicit example of a third order, linear, nonconstant coefficient, nonautonomous, nonhomogeneous system of two ODE such that every derivative that could appear, does appear.
\end{exercise}
%\comboSol
%{%
%One example:
%$x''' + e^t x'' + tx' + x = t^2 + 1$ \quad 
%$y''' + y'' + \sin(t)y' + y = e^{-t} - 2$
%}

\begin{exercise}%
    Write down the general \emph{zero}th order linear ordinary differential equation.  Write down the general solution.
\end{exercise}
%\exsol{%
%equation: $a(x) y = b(x)$, solution: $y = \frac{b(x)}{a(x)}$.
%}

\begin{exercise}%
    For which $k$ is $\frac{dx}{dt}+x^k = t^{k+2}$ linear.  Hint: there are two answers.
\end{exercise}
%\exsol{%
%$k=0$ or $k=1$
%}

\begin{exercise}
    Write out an explicit example of a non-homogeneous fourth order, linear, constant coefficient differential equation. where all possible derivatives of the unknown function $y$ appear. 
\end{exercise}
%\comboSol
%{%
%One example: $y'''' + 8y''' + 3y'' + 2y' + 7y = e^{3t}$
%}

\begin{exercise}%
    Let $x$, $y$, and $z$ be three functions of $t$ defined by the system of differential equations
    \begin{equation*} 
        x' = y \quad y' = z \quad z' = 3x - 2y + 5z + e^t
    \end{equation*}
    
    with initial conditions $x(0) = 3$, $y(0) = -2$ and $z(0) = 1$, and let $u(t)$ be the function defined by the solution to
    \begin{equation*}
        u''' - 5u'' + 2u' - 3u = e^t
    \end{equation*}
    with initial conditions $u(0) = 3$, $u'(0) = -2$, and $u''(0) = 1$. 
    \begin{tasks}
        \task Use the substitution $u=x$, $u' = y$, and $u'' = z$ to verify that $x(t) = u(t)$ because they solve the same initial value problem.
        \task What is the order of the system defining $x$, $y$, and $z$ and how many components does it have?
        \task What is the order of the equation defining $u$? How many components does that have?
        \task How do these numbers relate to each other?
    \end{tasks}
\end{exercise}
%\exsol{%
%b) First order with three components.\\
%c) Third order with one component. \\
%d) The product is three in both cases. ($1 \times 3 = 3 \times 1$).
%}
%
%\setcounter{exercise}{100}





\end{document}