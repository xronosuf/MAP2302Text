\documentclass{ximera}
%\auor{Matthew Charnley and Jason Nowell}
\usepackage[margin=1.5cm]{geometry}
\usepackage{indentfirst}
\usepackage{sagetex}
\usepackage{lipsum}
\usepackage{amsmath}
\usepackage{mathrsfs}


%%% Random packages added without verifying what they are really doing - just to get initial compile to work.
\usepackage{tcolorbox}
\usepackage{hypcap}
\usepackage{booktabs}%% To get \toprule,\midrule,\bottomrule etc.
\usepackage{nicefrac}
\usepackage{caption}
\usepackage{units}

% This is my modified wrapfig that doesn't use intextsep
\usepackage{mywrapfig}
\usepackage{import}



%%% End to random added packages.


\graphicspath{
    {./figures/}
    {./../figures/}
    {./../../figures/}
}
\renewcommand{\log}{\ln}%%%%
\DeclareMathOperator{\arcsec}{arcsec}
%% New commands


%%%%%%%%%%%%%%%%%%%%
% New Conditionals %
%%%%%%%%%%%%%%%%%%%%


% referencing
\makeatletter
    \DeclareRobustCommand{\myvref}[2]{%
      \leavevmode%
      \begingroup
        \let\T@pageref\@pagerefstar
        \hyperref[{#2}]{%
	  #1~\ref*{#2}%
        }%
        \vpageref[\unskip]{#2}%
      \endgroup
    }%

    \DeclareRobustCommand{\myref}[2]{%
      \leavevmode%
      \begingroup
        \let\T@pageref\@pagerefstar
        \hyperref[{#2}]{%
	  #1~\ref*{#2}%
        }%
      \endgroup
    }%
\makeatother

\newcommand{\figurevref}[1]{\myvref{Figure}{#1}}
\newcommand{\figureref}[1]{\myref{Figure}{#1}}
\newcommand{\tablevref}[1]{\myvref{Table}{#1}}
\newcommand{\tableref}[1]{\myref{Table}{#1}}
\newcommand{\chapterref}[1]{\myref{chapter}{#1}}
\newcommand{\Chapterref}[1]{\myref{Chapter}{#1}}
\newcommand{\appendixref}[1]{\myref{appendix}{#1}}
\newcommand{\Appendixref}[1]{\myref{Appendix}{#1}}
\newcommand{\sectionref}[1]{\myref{\S}{#1}}
\newcommand{\subsectionref}[1]{\myref{subsection}{#1}}
\newcommand{\subsectionvref}[1]{\myvref{subsection}{#1}}
\newcommand{\exercisevref}[1]{\myvref{Exercise}{#1}}
\newcommand{\exerciseref}[1]{\myref{Exercise}{#1}}
\newcommand{\examplevref}[1]{\myvref{Example}{#1}}
\newcommand{\exampleref}[1]{\myref{Example}{#1}}
\newcommand{\thmvref}[1]{\myvref{Theorem}{#1}}
\newcommand{\thmref}[1]{\myref{Theorem}{#1}}


\renewcommand{\exampleref}[1]{ {\color{red} \bfseries Normally a reference to a previous example goes here.}}
\renewcommand{\figurevref}[1]{ {\color{red} \bfseries Normally a reference to a previous figure goes here.}}
\renewcommand{\tablevref}[1]{ {\color{red} \bfseries Normally a reference to a previous table goes here.}}
\renewcommand{\Appendixref}[1]{ {\color{red} \bfseries Normally a reference to an Appendix goes here.}}
\renewcommand{\exercisevref}[1]{ {\color{red} \bfseries Normally a reference to a previous exercise goes here.}}



\newcommand{\R}{\mathbb{R}}

%% Example Solution Env.
\def\beginSolclaim{\par\addvspace{\medskipamount}\noindent\hbox{\bf Solution:}\hspace{0.5em}\ignorespaces}
\def\endSolclaim{\par\addvspace{-1em}\hfill\rule{1em}{0.4pt}\hspace{-0.4pt}\rule{0.4pt}{1em}\par\addvspace{\medskipamount}}
\newenvironment{exampleSol}[1][]{\beginSolclaim}{\endSolclaim}

%% General figure formating from original book.
\newcommand{\mybeginframe}{%
\begin{tcolorbox}[colback=white,colframe=lightgray,left=5pt,right=5pt]%
}
\newcommand{\myendframe}{%
\end{tcolorbox}%
}

%%% Eventually return and fix this to make matlab code work correctly.
%% Define the matlab environment as another code environment
%\newenvironment{matlab}
%{% Begin Environment Code
%{ \centering \bfseries Matlab Code }
%\begin{code}
%}% End of Begin Environment Code
%{% Start of End Environment Code
%\end{code}
%}% End of End Environment Code


% this one should have a caption, first argument is the size
\newenvironment{mywrapfig}[2][]{
 \wrapfigure[#1]{r}{#2}
 \mybeginframe
 \centering
}{%
 \myendframe
 \endwrapfigure
}

% this one has no caption, first argument is size,
% the second argument is a larger size used for HTML (ignored by latex)
\newenvironment{mywrapfigsimp}[3][]{%
 \wrapfigure[#1]{r}{#2}%
 \centering%
}{%
 \endwrapfigure%
}
\newenvironment{myfig}
    {%
    \begin{figure}[h!t]
        \mybeginframe%
        \centering%
    }
    {%
        \myendframe
    \end{figure}%
    }


% graphics include
\newcommand{\diffyincludegraphics}[3]{\includegraphics[#1]{#3}}
\newcommand{\myincludegraphics}[3]{\includegraphics[#1]{#3}}
\newcommand{\inputpdft}[1]{\subimport*{../figures/}{#1.pdf_t}}


%% Not sure what these even do? They don't seem to actually work... fun!
%\newcommand{\mybxbg}[1]{\tcboxmath[colback=white,colframe=black,boxrule=0.5pt,top=1.5pt,bottom=1.5pt]{#1}}
%\newcommand{\mybxsm}[1]{\tcboxmath[colback=white,colframe=black,boxrule=0.5pt,left=0pt,right=0pt,top=0pt,bottom=0pt]{#1}}
\newcommand{\mybxsm}[1]{#1}
\newcommand{\mybxbg}[1]{#1}

%%% Something about tasks for practice/hw?
\usepackage{tasks}
\usepackage{footnote}
\makesavenoteenv{tasks}


%% For pdf only?
\newcommand{\diffypdfversion}[1]{#1}


%% Kill ``cite'' and go back later to fix it.
\renewcommand{\cite}[1]{}


%% Currently we can't really use index or its derivatives. So we are gonna kill them off.
\renewcommand{\index}[1]{}
\newcommand{\myindex}[1]{#1}






\title{Substitution}
\author{Matthew Charnley and Jason Nowell}


\outcome{Use substitution to solve more complicated first order equations}
\outcome{Use a Bernoulli substitution to solve appropriate first order equations}
\outcome{Use a homogeneity transformation to solve appropriate first order equations.}


\begin{document}
\begin{abstract}
    Stuff about Substitution
\end{abstract}
\maketitle

\label{substitution:section}


% \sectionnotes{1 lecture, can safely be skipped\EPref{, \S1.6 in \cite{EP}}\BDref{, not in
% \cite{BD}}}

The equation
\begin{equation*}
    y' = {(x-y+1)}^2 
\end{equation*}
is neither separable nor linear.  What can we do? One technique that worked for helping us in evaluating integrals was substitution, or change of variables. For example, in order to evaluate the integral
\begin{equation*}
    \int 2x(x^2 + 4)^5 \ dx
\end{equation*} we can do so by defining $u = x^2 + 4$ so that $du = 2x\ dx$, and then evaluate the integral as 
\begin{equation*}
    \int u^5\ du = \frac{u^6}{6} + C = \frac{(x^2 + 4)^6}{6} + C
\end{equation*} after we have plugged our original function back in.

We can try to do the same thing here, and be careful with how we set things up. Our general strategy will be to pick a new dependent variable, find a differential equation that this new variable solves (which will come from the old equation), solve that equation, then convert back to the original variable. We will take $v$ as our new dependent variable here, which is as function $v(x)$. Let us try
\begin{equation*}
    v = x-y+1,
\end{equation*}
which is the ``inside'' function (it's inside the square) of our example. In order to get to a differential equation that $v$ satisfies, we need to figure out $y'$ in terms of $v'$, $v$ and $x$.  We differentiate (in $x$) to obtain $v' = 1 - y'$.  So $y' = 1-v'$.  We plug this into the equation to get
\begin{equation*}
    1-v' = v^2 .
\end{equation*}
In other words, $v' = 1-v^2$.  Such an equation we know how to solve by separating variables:
\begin{equation*}
    \frac{1}{1-v^2} \,dv = dx .
\end{equation*}
So
\begin{equation*}
    \frac{1}{2} \ln \left\lvert  \frac{v+1}{v-1} \right\rvert = x + C ,\qquad \text{or} \qquad \left\lvert \frac{v+1}{v-1} \right\rvert = e^{2x + 2C} , \qquad \text{or} \qquad \frac{v+1}{v-1} = D e^{2x} ,
\end{equation*}
for some constant $D$. Note that $v=1$ and $v=-1$ are also solutions; they are the \emph{singular solutions} in this problem. (This solution method requires partial fractions; for a review of that topic, see \sectionref{sec:derivInt}.)

Now we need to ``unsubstitute'' to obtain
\begin{equation*}
    \frac{x-y+2}{x-y} = D e^{2x} ,
\end{equation*}
and also the two solutions $x-y+1=1$ or $y=x$, and $x-y+1=-1$ or $y=x+2$. We solve the first equation for $y$.
\begin{align*}
    x-y+2 &= (x-y)D e^{2x} , \\
    x-y+2 &= Dx e^{2x}-yD e^{2x} , \\
    -y + yD e^{2x} &= Dx e^{2x} - x - 2 , \displaybreak[0]\\
    y\,(-1+ D e^{2x}) &= Dx e^{2x} - x - 2 , \displaybreak[0]\\
    y  &= \frac{Dx e^{2x} - x - 2}{D e^{2x}-1} .
\end{align*}
Note that $D=0$ gives $y=x+2$, but no value of $D$ gives the solution $y=x$.


Substitution in differential equations is applied in much the same way that it is applied in calculus.  You guess.  Several different substitutions might work. There are some general patterns to look for.  We summarize a few of these in a table.

\begin{center}
    \begin{tabular}{@{}ll@{}}
        \toprule
        When you see & Try substituting \\
        \midrule
        $yy'$ & $v=y^2$ \\
        $y^2y'$ & $v=y^3$ \\
        $(\cos y)y'$ & $v=\sin y$ \\
        $(\sin y)y'$ & $v=\cos y$ \\
        $y'e^y$ & $v=e^y$ \\ \bottomrule
    \end{tabular}
\end{center}

Usually you try to substitute in the ``most complicated'' part of the equation with the hopes of simplifying it.  The table above is just a rule of thumb.  You might have to modify your guesses.  If a substitution does not work (it does not make the equation any simpler), try a different one.

\subsection{Bernoulli equations}

There are some forms of equations where there is a general rule for substitution that always works.
One such example is the so-called \emph{Bernoulli equation}%
\footnote{There are several things called Bernoulli equations, this is just one of them.  The Bernoullis were a prominent Swiss family of mathematicians.  These particular equations are named for \href{https://en.wikipedia.org/wiki/Jacob_Bernoulli}{Jacob Bernoulli} (1654--1705).}:
\begin{equation*}
    y' + p(x)y = q(x)y^n .
\end{equation*}
This equation looks a lot like a linear equation except for the $y^n$.  If $n=0$ or $n=1$, then the equation is linear and we can solve it.  Otherwise, the substitution $v=y^{1-n}$ transforms the Bernoulli equation into a linear equation.  Note that $n$ need not be an integer.

\begin{example}
    Find the general solution of
    \begin{equation*}
        y' - \frac{4}{3x}y = -\frac{2}{3}y^4 
    \end{equation*}
\end{example}

\begin{exampleSol}
    This equation fits the Bernoulli equation structure with $p(x) = -\frac{4}{3x}$ and $q(x) = -\frac{2}{3}$. Since there is a $y^4$ on the right-hand side, we take $n=4$ and make the substitution $v = y^{1-4} = y^{-3}$. With this, we see that
    \begin{equation*}
        v' = -3 y^{-4} y'
    \end{equation*}
    or $y' = -\nicefrac{1}{3}y^4 v'$. Plugging this into the equation gives
    \begin{equation*}
        \begin{split}
            -\frac{1}{3} y^4 v' - \frac{4}{3x}y &= -\frac{2}{3}y^4 \\
            -\frac{1}{3} v' - \frac{4}{3x} y^{-3} &= -\frac{2}{3} \\
            v' + \frac{4}{x} v &= 2
        \end{split}
    \end{equation*}
    This last equation is now a first order linear equation, so we can solve it. The integrating factor we are looking for is 
    \begin{equation*}
        \mu(x) = e^{\int p(x)\ dx} = e^{\int \frac{4}{x}\ dx} = e^{4\ln{x}} = x^4, 
    \end{equation*}
    which results in the euation
    \begin{equation*}
        x^4 v' + 4x^3 v = 2x^4.
    \end{equation*}
    The left-hand side is $(x^4v)'$, so we can integrate both sides to get
    \begin{equation*}
        x^4 v = \frac{2}{5}x^5 + C,
    \end{equation*}
    or, solving for $v$, 
    \begin{equation*}
        v(x) = \frac{2}{5} x + \frac{C}{x^4}.
    \end{equation*}
    However, our original equation was for $y$, not $v$. Using the fact that $v = y^{-3}$, we can solve for $y$ as $y = v^{-1/3}$, giving
    \begin{equation*}
        y(x) = \left(\frac{2}{5}x + \frac{C}{x^4}\right)^{-1/3} = \frac{1}{\sqrt[3]{\frac{2}{5}x + \frac{C}{x^4}}}
    \end{equation*}
    as the general solution to this equation. 
\end{exampleSol}

Even if we need to use integrals to write out the solution to these Bernoulli equations, we can still use the substitution method and solve back out for the desired solution at the end.

\begin{example}
    Solve
    \begin{equation*}
        xy'+ y(x+1)+xy^5 = 0, \qquad y(1)=1 .
    \end{equation*}
\end{example}

\begin{exampleSol}
    First, the equation is Bernoulli ($p(x) = (x+1)/x$ and $q(x) = -1$). We substitute
    \begin{equation*}
        v=y^{1-5} = y^{-4}, \qquad v' = -4 y^{-5} y' .
    \end{equation*}
    In other words, $\left( \nicefrac{-1}{4} \right) y^5 v' = y'$.  So
    \begin{align*}
        xy'+ y(x+1)+xy^5 & = 0 , \\
        \frac{-xy^5}{4} v'+ y(x+1)+xy^5 & = 0 , \displaybreak[0]\\
        \frac{-x}{4} v'+ y^{-4}(x+1)+x & = 0 , \displaybreak[0]\\
        \frac{-x}{4} v'+ v(x+1)+x & = 0 ,
    \end{align*}
    and finally
    \begin{equation*}
        v'- \frac{4(x+1)}{x} v  = 4 .
    \end{equation*}
    The equation is now linear. We can use the integrating factor method.  In particular, we use formula \eqref{lei:defsol}.  Let us assume that $x > 0$ so $\lvert x \rvert = x$.  This assumption is OK\@, as our initial condition is $x=1$.  Let us compute the integrating factor.  Here $p(s)$ from formula \eqref{lei:defsol} is $\frac{-4(s+1)}{s}$.
    \begin{align*}
        e^{\int_1^x p(s)\,ds} & = \exp \left( \int_1^x \frac{-4(s+1)}{s} \,ds \right) =
        e^{-4x-4\ln(x)+4} = 
        e^{-4x+4} x^{-4}
        =
        \frac{e^{-4x+4}}{x^4} , \\
        e^{-\int_1^x p(s)\,ds} & =
        e^{4x+4\ln(x)-4} = 
        e^{4x-4} x^4 .
    \end{align*}
    We now plug in to \eqref{lei:defsol}
    \begin{equation*}
        \begin{split}
            v(x) & =
            e^{-\int_{1}^x p(s)\, ds} \left( \int_{1}^x e^{\int_{1}^t p(s)\, ds} 4 \,dt
            + 1 \right) \\
            & =
            e^{4x-4} x^4
            \left( \int_{1}^x 4 \frac{e^{-4t+4}}{t^4} \,dt
            + 1 \right) .
        \end{split}
    \end{equation*}
    The integral in this expression is not possible to find in closed form.  As we said before, it is perfectly fine to have a definite integral in our solution.  Now ``unsubstitute''
    \begin{align*}
         y^{-4} &= e^{4x-4}x^4 \left( 4 \int_1^x \frac{e^{-4t+4}}{t^4} \,dt + 1\right) , \\
         y &= \frac{e^{-x+1}}{x {\left( 4 \int_1^x \frac{e^{-4t+4}}{t^4} \,dt +
        1\right)}^{1/4}} .
    \end{align*}
\end{exampleSol}

\subsection{Homogeneous equations}

Another type of equations we can solve by substitution are the so-called \emph{homogeneous equations\index{homogeneous equation}}. Note that this is \emph{not} the same as a homogeneous linear equation. These equations do not have to be linear, and are solved in a very different way. Suppose that we can write the differential equation as
\begin{equation*}
    y' = F\left(\frac{y}{x}\right) .
\end{equation*}
Here we try the substitutions
\begin{equation*}
    v = \frac{y}{x} \qquad \text{and therefore} \qquad y' = v + x v' .
\end{equation*}
We note that the equation is transformed into
\begin{equation*}
    v+ xv' = F(v) \qquad \text{or} \qquad xv' = F(v)-v \qquad \text{or} \qquad \frac{v'}{F(v)-v} = \frac{1}{x} .
\end{equation*}
Hence an implicit solution is
\begin{equation*}
    \int \frac{1}{F(v)-v} \,dv = \ln \, \lvert x \rvert + C .
\end{equation*}

\begin{example}
    Solve 
    \begin{equation*}
        x^2y' = y^2+xy, \qquad y(1)=1.
    \end{equation*}
\end{example}

\begin{exampleSol}
    We put the equation into the form $y'= {\left(\nicefrac{y}{x}\right)}^2+\nicefrac{y}{x}$.  We substitute $v=\nicefrac{y}{x}$ to get the separable equation
    \begin{equation*}
        xv' = v^2+v-v = v^2 ,
    \end{equation*}
    which has a solution
    \begin{align*}
        \int \frac{1}{v^2} \,dv &= \ln \, \lvert x \rvert + C , \\
        \frac{-1}{v} &= \ln \, \lvert x \rvert + C , \\
        v &= \frac{-1}{\ln \, \lvert x \rvert + C} .
    \end{align*}
    We unsubstitute
    \begin{align*}
        \frac{y}{x} &= \frac{-1}{\ln \, \lvert x \rvert + C} , \\
        y &= \frac{-x}{\ln \, \lvert x \rvert + C} .
    \end{align*}
    We want $y(1)=1$, so 
    \begin{equation*}
        1 = y(1) = \frac{-1}{\ln \, \lvert 1 \rvert + C} = \frac{-1}{C} .
    \end{equation*}
    Thus $C = -1$ and the solution we are looking for is
    \begin{equation*}
        y = \frac{-x}{\ln \, \lvert x \rvert -1} .
    \end{equation*}
\end{exampleSol}

\end{document}
